\documentclass[10pt]{article}
\usepackage[usenames]{color} %used for font color
\usepackage{amssymb} %maths
\usepackage{amsmath} %maths
\usepackage[utf8]{inputenc} %useful to type directly diacritic characters
\begin{document}
\[\renewcommand{\thechapter}{9}
\chapter{Results}
\label{chap:Results}

\section{Water Model Results}
\label{sec:Water Model Results}

%Figures \ref{fig:DUSRFlow} and \ref{fig:USRFlow} depict the final results for the USR water balance model.  They show the calculated average daily water flow rate between the aquifer and the river channel for the deterministic and stochastic models, respectively.  The blue band in figure \ref{fig:} depicts the 95\% CIR for the calculated time steps.  Positive values indicate that water is moving into the river channel.

%\begin{figure}[htbp]
%\centering
%	\begin{subfigure}{0.5\textwidth}
%		\centering
%		\includegraphics[width=0.9\linewidth]{"Figures/Results_DUSR/Balance Water"}
%		\caption{Deterministic Model.}
%		\label{sub:USRWaterD}
%	\end{subfigure}%
%	\begin{subfigure}{0.5\textwidth}
%		\centering
%		\includegraphics[width=0.9\linewidth]{"Figures/Results_USR/Balance Water"}
%		\caption{Stochastic Model.}
%		\label{sub:USRWaterS}
%	\end{subfigure}
%	\caption[Time series of the USR Arkansas R. unaccounted for water flow.]{Time series of the USR Arkansas R. unaccounted for water flow.  Positive values indicate a gain to the river reach.}
%	\label{fig:USRWater}
%\end{figure}

%\begin{figure}[htbp]
%\centering
%	\begin{subfigure}{0.5\textwidth}
%		\centering
%		\includegraphics[width=0.9\linewidth]{"Figures/Results_DDSR/Balance Water"}
%		\caption{Deterministic Model.}
%		\label{sub:DSRWaterD}
%	\end{subfigure}%
%	\begin{subfigure}{0.5\textwidth}
%		\centering
%		\includegraphics[width=0.9\linewidth]{"Figures/Results_DSR/Balance Water"}
%		\caption{Stochastic Model.}
%		\label{sub:DSRWaterS}
%	\end{subfigure}
%	\caption[Time series of the DSR Arkansas R. unaccounted for water flow.]{Time series of the DSR Arkansas R. unaccounted for water flow.  Positive values indicate a gain to the river reach.}
%	\label{fig:DSRWater}
%\end{figure}

%As anticipated with the study reach intermediate results, the model results indicate that there is seasonable variability with the transport of selenium.  The figures show that the USR and DSR receive a significant quantity of water from unaccounted for sources.  There is a very short period during midyear where the unaccounted for flows are a sink for water being lost from the Arkansas R.

%The distribution of all realizations within each time step was analyzed to determine a distribution type.  This analysis was performed to determine if the assumption that the deterministic model results were representative of the stochastic model.   Testing was performed by comparing K-S statistics for the best fit normal, log-normal, logistic, exponential, gamma, and Weibull distributions.  In the USR, 92\% of all storage component time steps best fit a normal distribution and 8\% best fit a gamma distribution.  In the DSR, 66\% best fit a normal distribution and 33\% best fit a gamma distribution.  This indicates that for both the USR and DSR, the distributions across the realizations are normal.  The DSR water balance model does show some significant skewness due to the relatively large percentage of time steps that best fit a gamma distribution.

%Tables \ref{tab:USRFlow} and \ref{tab:DSRFlow} presents summary statistics of the deterministic and stochastic storage component changes.  Values are presented in units of \si{\cubic\meter\per\second\per\kilo\meter}.   These tables show the mean, 2.5th, and 97.5th percentile of the deterministic model and the same statistics applied to the three 1-D stochastic models.  This additional set of statistics from the 1-D stochastic 2.5th and 95.5th percentile models provide a better understanding of the extremes of the calculated values.  As before, comparison between the deterministic and stochastic models should be limited to the comparing the deterministic model to the 1-D stochastic mean model.  Also included in these two tables is the percent difference between the deterministic model and the 1-D stochastic mean model.  The mean, 2.5th, and 97.5th percentile values were calculated from the percent difference of the 1-D stochastic mean model from the deterministic model at each time step.  These values show the range of the variance from the deterministic model.

%\begin{table}[htbp]
%\centering
%\caption[USR river section unaccounted for flow.]{USR river section unaccounted for flow.  All values are in \si{\cubic\meter\per\second\per\kilo\meter}.  Positive values indicate a gain to the river reach.}
%\label{tab:USRFlow}
%\begin{tabular}{c|ccc}
%	\toprule
%	Model& 2.5\% & Mean & 97.5\% \\
%	\midrule
%	\midrule
%	Deterministic&		-0.04002&	0.04944&	0.1762\\
%	\midrule			                               
%	Stochastic 2.5\%&	-0.1122&	-0.008835&	0.09072\\
%	Stochastic Mean&	-0.04047&	0.05&	0.1794\\     
%	Stochastic 97.5\%&	0.02927&	0.109&	0.272\\      
%	\midrule                                           
%	\% Diff. Means &	-23.26&	-5.656&	11.55 \\
%	\bottomrule
%\end{tabular}
%\end{table}

%\begin{table}[htbp]
%\centering
%\caption[DSR river section unaccounted for flow.]{DSR river section unaccounted for flow.  All values are in \si{\cubic\meter\per\second\per\kilo\meter}.  Positive values indicate a gain to the river reach.}
%\label{tab:DSRFlow}
%\begin{tabular}{c|ccc}
%	\toprule
%	Model& 2.5\% & Mean & 97.5\% \\
%	\midrule
%	\midrule
%	Deterministic&		-0.01974&	0.02993&	0.06909\\
%	\midrule			                                
%	Stochastic 2.5\%&	-0.06068&	0.01223&	0.04919\\
%	Stochastic Mean&	-0.01746&	0.03012&	0.06729\\
%	Stochastic 97.5\%&	0.02265&	0.04817&	0.08884\\
%	\midrule                                            
%	\% Diff. Means &	-8.243&	-0.3074&	7.571\\
%	\bottomrule
%\end{tabular}
%\end{table}

%The mean of the percent difference between the deterministic and 1-D stochastic mean models is very low.  This indicates that the deterministic model is representative of the stochastic model expected value.  The moderate percent differences at the 2.5th and 97.5th percentile indicate that there is still a significant range of uncertainty contained within the stochastic model that the deterministic model cannot replicate.  The deterministic model can be used to determine how changes can affect a reach over a span of time, but using it to estimate values for specific time steps is only acceptable if the tolerance for this uncertainty is acceptable.
\clearpage

\section{Selenium Transport Model Results}
\label{sec:Selenium Transport Model Results}

%Figures \ref{fig:USRMass} and \ref{fig:DSRMass} depict the final results for the DSR mass balance model.  They show the calculated average daily selenium transport rate between the aquifer and the river channel for the deterministic and stochastic models, respectively.  The blue band in figure \ref{fig:} depicts the 95\% CIR for the calculated time steps.  Positive values indicate that water is moving into the river channel from the aquifer.

%\begin{figure}[htbp]
%\centering
%	\begin{subfigure}{0.5\textwidth}
%		\centering
%		\includegraphics[width=0.9\linewidth]{"Figures/Results_DUSR/Balance Mass"}
%		\caption{Deterministic Model.}
%		\label{sub:USRMassD}
%	\end{subfigure}%
%	\begin{subfigure}{0.5\textwidth}
%		\centering
%		\includegraphics[width=0.9\linewidth]{"Figures/Results_USR/Balance Mass"}
%		\caption{Stochastic Model.}
%		\label{sub:USRMassS}
%	\end{subfigure}
%	\caption[Time series of the USR Arkansas R. unaccounted for selenium mass transport.]{Time series of the USR Arkansas R. unaccounted for selenium mass transport.  Positive values indicate mass is moving into the river reach.}
%	\label{fig:USRMass}
%\end{figure}

%\begin{figure}[htbp]
%\centering
%	\begin{subfigure}{0.5\textwidth}
%		\centering
%		\includegraphics[width=0.9\linewidth]{"Figures/Results_DDSR/Balance Mass"}
%		\caption{Deterministic Model.}
%		\label{sub:DSRMassD}
%	\end{subfigure}%
%	\begin{subfigure}{0.5\textwidth}
%		\centering
%		\includegraphics[width=0.9\linewidth]{"Figures/Results_DSR/Balance Mass"}
%		\caption{Stochastic Model.}
%		\label{sub:DSRMassS}
%	\end{subfigure}
%	\caption[Time series of the DSR Arkansas R. unaccounted for selenium mass transport.]{Time series of the DSR Arkansas R. unaccounted for selenium mass transport.  Positive values indicate mass is moving into the river reach.}
%	\label{fig:DSRMass}
%\end{figure}

%As anticipated with the study reach intermediate results, the model results indicate that there is seasonable variability with the transport of selenium.  The figures show that the USR and DSR receive a significant quantity of selenium from unaccounted for sources.  There is a very short period during midyear where the unaccounted for selenium transport flows are a sink for mass being lost from the Arkansas R.

%The distribution of all realizations within each time step was analyzed to determine a distribution type.  This analysis was performed to determine if the assumption that the deterministic model results were representative of the stochastic model.   Testing was performed by comparing K-S statistics for the best fit normal, log-normal, logistic, exponential, gamma, and Weibull distributions.  In the USR, 98\% of all storage component time steps best fit a normal distribution and 2\% best fit a gamma distribution.  In the DSR, 99.7\% best fit a normal distribution and 0.3\% best fit a gamma distribution.  This indicates that for both the USR and DSR, the distributions across the realizations are normal.  

%Tables \ref{tab:USRSe} and \ref{tab:DSRSe} presents summary statistics of the deterministic and stochastic storage component changes.  Values are presented in units of \si{\kilo\gram\per\day\per\kilo\meter}.   These tables show the mean, 2.5th, and 97.5th percentile of the deterministic model and the same statistics applied to the three 1-D stochastic models.  This additional set of statistics from the 1-D stochastic 2.5th and 95.5th percentile models provide a better understanding of the extremes of the calculated values.  As before, comparison between the deterministic and stochastic models should be limited to the comparing the deterministic model to the 1-D stochastic mean model.  Also included in these two tables is the percent difference between the deterministic model and the 1-D stochastic mean model.  The mean, 2.5th, and 97.5th percentile values were calculated from the percent difference of the 1-D stochastic mean model from the deterministic model at each time step.  These values show the range of the variance from the deterministic model.

%\begin{table}[htbp]
%\centering
%\caption[USR river section unaccounted for selenium mass transport.]{USR river section unaccounted for selenium mass transport.  Stochastic mean values are calculated as the mean of the realizations for each time step.  Positive values indicate mass is moving into the river reach.}
%\label{tab:USRSe}
%\begin{tabular}{c|ccc}
%	\toprule
%	Model& 2.5\% & Mean & 97.5\% \\
%	\midrule
%	\midrule
%	Deterministic&		-0.04191&	0.05627&	0.1424\\
%	\midrule			                               
%	Stochastic 2.5\%&	-0.1096&	0.002481&	0.06902\\
%	Stochastic Mean&	-0.03797&	0.0555&	0.1375\\     
%	Stochastic 97.5\%&	0.03027&	0.1107&	0.2174\\     
%	\midrule                                           
%	\% Diff. Means &	-11.14&	9.36&	42.13\\
%	\bottomrule
%\end{tabular}
%\end{table}

%\begin{table}[htbp]
%\centering
%\caption[DSR river section unaccounted for selenium mass transport.]{DSR river section unaccounted for selenium mass transport.  Stochastic mean values are calculated as the mean of the realizations for each time step.  Positive values indicate mass is moving into the river reach.}
%\label{tab:DSRSe}
%\begin{tabular}{c|ccc}
%	\toprule
%	Model& 2.5\% & Mean & 97.5\% \\
%	\midrule
%	\midrule
%	Deterministic&		0.02053&	0.05162&	0.1308\\
%	\midrule			                                
%	Stochastic 2.5\%&	-0.03585&	0.008662&	0.0297\\
%	Stochastic Mean&	0.0207&	0.04935&	0.117\\     
%	Stochastic 97.5\%&	0.04185&	0.09204&	0.2407\\
%	\midrule                                            
%	\% Diff. Means &	-5.229&	1.999&	20.44\\
%	\bottomrule
%\end{tabular}
%\end{table}

%The mean of the percent difference between the deterministic and 1-D stochastic mean models is very low.  This indicates that the deterministic model is representative of the stochastic model expected value.  The moderate percent differences at the 2.5th and 97.5th percentile indicate that there is still a significant range of uncertainty contained within the stochastic model that the deterministic model cannot replicate.  The deterministic model can be used to determine how changes can affect a reach over a span of time, but using it to estimate values for specific time steps is only acceptable if the tolerance for this uncertainty is acceptable.
\clearpage

\section{Sensitivity Analysis Results}

%The sensitivity analysis serves as a means to determine the degree to which each input variable affects the model.  Sensitivity analyses were only performed on the deterministic models. Table~\ref{tab:USRSA} lists the results from the sensitivity analyses for both the water balance and selenium loading models in the USR.  The trial variables are explained in table~\ref{tab:USRvars}.  The results are given as the mean of all the calculated results in each of the two models and as the percent difference from the base model.  The base model is calculated without perturbing any of the input variables.  Those variables with "0" are either not significant to the model or are not part of the model.

%\begin{spacing}{1.0}
%\begin{center}
%\begin{longtable}{c|cc|cc}
%    \caption[USR Sensitivity Analysis Results]{USR Sensitivity Analysis Results} \label{tab:USRSA}\\ 
%	\hline
%    \multirow{3}[1]{*}{Trial} & Water Balance & Percent & Se Loading & Percent \\
%          & Results & Change from & Results & Change from \\ 
%          & \si{\cubic\meter\per\second\per\kilo\meter} & 'base' Trial & \si{\kilo\gram\per\day\per\kilo\meter} & 'base' Trial \\
%    \hline
%    \hline
%    \endfirsthead
%    \caption[]{(continued)}\\
%    \hline
%    \multirow{3}[1]{*}{Trial} & Water Balance & Percent & Se Loading & Percent \\
%          & Results & Change from & Results & Change from \\ 
%          & \si{\cubic\meter\per\second\per\kilo\meter} & 'base' Trial & \si{\kilo\gram\per\day\per\kilo\meter} & 'base' Trial \\
%    \hline
%    \hline
%    \endhead
%	Base (no change)&						0.04944&	--&	0.05627&	--\\               
%	$Q_{ARKLASCO}$ +10\%&					0.0289&	-41.55&	0.04594&	-18.36\\          
%	$Q_{ARKLASCO}$ -10\%&					0.06998&	41.55&	0.0666&	18.36\\           
%	$EC_{ARKJMRCO}$ +10\%&					0.04944&	0&	0.05627&	0\\               
%	$EC_{ARKJMRCO}$ -10\%&					0.04944&	0&	0.05627&	0\\               
%	$T_{ARKJMRCO}$ +5\si{\degreeCelsius}&	0.04944&	0&	0.05627&	0\\               
%	$T_{ARKJMRCO}$ -5\si{\degreeCelsius}&	0.04944&	0&	0.05627&	0\\               
%	$Q_{ARKCOOKS}$ +10\%&					0.06374&	28.92&	0.06553&	16.46\\       
%	$Q_{ARKCOOKS}$ -10\%&					0.03515&	-28.9&	0.04701&	-16.46\\      
%	$EC_{ARKCOOKS}$ +10\%&					0.04944&	0&	0.05627&	0\\               
%	$EC_{ARKCOOKS}$ -10\%&					0.04944&	0&	0.05627&	0\\               
%	$T_{ARKCOOKS}$ +5\si{\degreeCelsius}&	0.04944&	0&	0.05627&	0\\               
%	$T_{ARKCOOKS}$ -5\si{\degreeCelsius}&	0.04944&	0&	0.05627&	0\\               
%	$Q_{HOLCANCO}$ +20\%&					0.05564&	12.54&	0.0594&	5.562\\           
%	$Q_{HOLCANCO}$ -20\%&					0.04324&	-12.54&	0.05314&	-5.562\\      
%	$Q_{RFDMANCO}$ +20\%&					0.05202&	5.218&	0.05801&	3.092\\       
%	$Q_{RFDMANCO}$ -20\%&					0.04686&	-5.218&	0.05453&	-3.092\\      
%	$Q_{FLSCANCO}$ +20\%&					0.05127&	3.701&	0.05746&	2.115\\       
%	$Q_{FLSCANCO}$ -20\%&					0.04762&	-3.681&	0.05508&	-2.115\\      
%	$Q_{RFDRETCO}$ +20\%&					0.04822&	-2.468&	0.05545&	-1.457\\      
%	$Q_{RFDRETCO}$ -20\%&					0.05066&	2.468&	0.05709&	1.457\\       
%	$Q_{TIMSWICO}$ +15\%&					0.04577&	-7.423&	0.05436&	-3.394\\      
%	$Q_{TIMSWICO}$ -15\%&					0.05312&	7.443&	0.05818&	3.394\\       
%	$Q_{FLYCANCO}$ +20\%&					0.07468&	51.05&	0.07273&	29.25\\       
%	$Q_{FLYCANCO}$ -20\%&					0.02421&	-51.03&	0.03981&	-29.25\\      
%	$Q_{CANSWICO}$ +15\%&					0.04854&	-1.82&	0.05549&	-1.386\\      
%	$Q_{CANSWICO}$ -15\%&					0.05034&	1.82&	0.05706&	1.404\\       
%	$Q_{CONDITCO}$ +20\%&					0.05306&	7.322&	0.05892&	4.709\\       
%	$Q_{CONDITCO}$ -20\%&					0.04583&	-7.302&	0.05362&	-4.709\\      
%	$Q_{HRC194CO}$ +20\%&					0.04869&	-1.517&	0.05553&	-1.315\\      
%	$Q_{HRC194CO}$ -20\%&					0.0502&	1.537&	0.05701&	1.315\\           
%	$Q_{La Junta WWTP}$ +15\%&				0.04931&	-0.2629&	0.05602&	-0.4443\\ 
%	$Q_{La Junta WWTP}$ -15\%&				0.04958&	0.2832&	0.05652&	0.4443\\      
%	$P_{}$ +25\%&							0.04936&	-0.1618&	0.05627&	0\\       
%	$P_{}$ -25\%&							0.04953&	0.182&	0.05627&	0\\           
%	$ET_{}$ +0.98 mm&						0.05001&	1.153&	0.05627&	0\\           
%	$ET_{}$ -0.98 mm&						0.04887&	-1.153&	0.05627&	0\\           
%	$u_{2}$ +0.5 \si{\meter\per\second}&	0.04942&	-0.04045&	0.05627&	0\\       
%	$u_{2}$ -0.5 \si{\meter\per\second}&	0.04947&	0.06068&	0.05627&	0\\       
%	$RH_{min}$ +2\%&						0.04944&	0&	0.05627&	0\\               
%	$RH_{min}$ -2\%&						0.04944&	0&	0.05627&	0\\               
%	$\beta_{1}$ +10\%&						0.04979&	0.7079&	0.05621&	-0.1066\\     
%	$\beta_{1}$ -10\%&						0.04909&	-0.7079&	0.05633&	0.1066\\  
%	$\beta_{2}$ +10\%&						0.04939&	-0.1011&	0.05628&	0.01777\\ 
%	$\beta_{2}$ -10\%&						0.04949&	0.1011&	0.05627&	0\\           
%	$h_{A}$ +0.01 ft&						0.04944&	0&	0.05627&	0\\               
%	$h_{A}$ -0.01 ft&						0.04944&	0&	0.05627&	0\\               
%	$h_{B}$ +0.01 ft&						0.04944&	0&	0.05627&	0\\               
%	$h_{B}$ -0.01 ft&						0.04944&	0&	0.05627&	0\\               
%	$h_{C}$ +0.01 ft&						0.04945&	0.02023&	0.05627&	0\\       
%	$h_{C}$ -0.01 ft&						0.04944&	0&	0.05627&	0\\               
%	$h_{D}$ +0.01 ft&						0.04945&	0.02023&	0.05627&	0\\       
%	$h_{D}$ -0.01 ft&						0.04944&	0&	0.05627&	0\\               
%	$h_{E}$ +0.01 ft&						0.04944&	0&	0.05627&	0\\               
%	$h_{E}$ -0.01 ft&						0.04944&	0&	0.05627&	0\\               
%	$h_{A}$ +0.05 ft&						0.04945&	0.02023&	0.05627&	0\\       
%	$h_{A}$ -0.05 ft&						0.04944&	0&	0.05627&	0\\               
%	$h_{B}$ +0.05 ft&						0.04944&	0&	0.05627&	0\\               
%	$h_{B}$ -0.05 ft&						0.04944&	0&	0.05627&	0\\               
%	$h_{C}$ +0.05 ft&						0.04947&	0.06068&	0.05627&	0\\       
%	$h_{C}$ -0.05 ft&						0.04941&	-0.06068&	0.05627&	0\\       
%	$h_{D}$ +0.05 ft&						0.04947&	0.06068&	0.05626&	-0.01777\\
%	$h_{D}$ -0.05 ft&						0.04941&	-0.06068&	0.05628&	0.01777\\ 
%	$h_{E}$ +0.05 ft&						0.04945&	0.02023&	0.05627&	0\\       
%	$h_{E}$ -0.05 ft&						0.04943&	-0.02023&	0.05627&	0\\       
%	$h_{A}$ +0.1 ft	&						0.04945&	0.02023&	0.05627&	0\\       
%	$h_{A}$ -0.1 ft	&						0.04943&	-0.02023&	0.05627&	0\\       
%	$h_{B}$ +0.1 ft	&						0.04945&	0.02023&	0.05627&	0\\       
%	$h_{B}$ -0.1 ft	&						0.04944&	0&	0.05627&	0\\               
%	$h_{C}$ +0.1 ft	&						0.04949&	0.1011&	0.05627&	0\\           
%	$h_{C}$ -0.1 ft	&						0.04938&	-0.1214&	0.05627&	0\\       
%	$h_{D}$ +0.1 ft	&						0.0495&	0.1214&	0.05626&	-0.01777\\        
%	$h_{D}$ -0.1 ft	&						0.04936&	-0.1618&	0.05629&	0.03554\\ 
%	$h_{E}$ +0.1 ft	&						0.04947&	0.06068&	0.05627&	0\\       
%	$h_{E}$ -0.1 ft	&						0.04941&	-0.06068&	0.05627&	0\\       
%	$h_{A}$ +0.25 ft&						0.04947&	0.06068&	0.05627&	0\\       
%	$h_{A}$ -0.25 ft&						0.04963&	0.3843&	0.05714&	1.546\\       
%	$h_{B}$ +0.25 ft&						0.04945&	0.02023&	0.05627&	0\\       
%	$h_{B}$ -0.25 ft&						0.05026&	1.659&	0.05788&	2.861\\       
%	$h_{C}$ +0.25 ft&						0.04955&	0.2225&	0.05627&	0\\           
%	$h_{C}$ -0.25 ft&						0.05356&	8.333&	0.06222&	10.57\\       
%	$h_{D}$ +0.25 ft&						0.04957&	0.2629&	0.05624&	-0.05331\\    
%	$h_{D}$ -0.25 ft&						0.05197&	5.117&	0.05623&	-0.07109\\    
%	$h_{E}$ +0.25 ft&						0.0495&	0.1214&	0.05627&	0\\               
%	$h_{E}$ -0.25 ft&						0.04953&	0.182&	0.05657&	0.5331\\      
%    \hline
%\end{longtable}%
%\end{center}
%\end{spacing}

%Table~\ref{tab:DSRSA} lists the results from the sensitivity analyses for both the water balance and selenium loading models in the DSR.  The trial variables are explained in Table~\ref{tab:DSRvars}.  The results are given as the mean of all the calculated results in each of the two models and as the percent difference from the base model.  The base model is calculated without perturbing any of the input variables.  Those variables with "0" are either not significant to the model or are not part of the model.

%\begin{spacing}{1.0}
%\begin{center}
%\begin{longtable}{c|cc|cc}
%    \caption[DSR Sensitivity Analysis Results]{DSR Sensitivity Analysis Results} \label{tab:DSRSA}\\ 
%	\hline
%    \multirow{3}[1]{*}{Trial} & Water Balance & Percent & Se Loading & Percent \\
%          & Results & Change from & Results & Change from \\ 
%          & \si{\cubic\meter\per\second\per\kilo\meter} & 'base' Trial & \si{\kilo\gram\per\day\per\kilo\meter} & 'base' Trial \\
%    \hline
%    \hline
%    \endfirsthead
%    \caption[]{(continued)}\\
%    \hline
%    \multirow{3}[1]{*}{Trial} & Water Balance & Percent & Se Loading & Percent \\
%          & Results & Change from & Results & Change from \\ 
%          & \si{\cubic\meter\per\second\per\kilo\meter} & 'base' Trial & \si{\kilo\gram\per\day\per\kilo\meter} & 'base' Trial \\
%    \hline
%    \hline
%    \endhead
%    Base (no change)&					0.02993&	--&	0.05162&	--\\               
%	$Q_{ARKLASCO}$ +10\%&				0.02662&	-11.06&	0.05006&	-3.022\\      
%	$Q_{ARKLASCO}$ -10\%&				0.03325&	11.09&	0.05319&	3.041\\       
%	$EC_{ARKJMRCO}$ +10\%&				0.02993&	0&	0.05162&	0\\               
%	$EC_{ARKJMRCO}$ -10\%&				0.02993&	0&	0.05162&	0\\               
%	$T_{ARKJMRCO}$ +5C&					0.02993&	0&	0.05162&	0\\               
%	$T_{ARKJMRCO}$ -5C&					0.02993&	0&	0.05162&	0\\               
%	$Q_{ARKCOOKS}$ +10\%&				0.0378&	26.29&	0.0608&	17.78\\               
%	$Q_{ARKCOOKS}$ -10\%&				0.02207&	-26.26&	0.04244&	-17.78\\      
%	$EC_{ARKCOOKS}$ +10\%&				0.02993&	0&	0.05162&	0\\               
%	$EC_{ARKCOOKS}$ -10\%&				0.02993&	0&	0.05162&	0\\               
%	$T_{ARKCOOKS}$ +5C&					0.02993&	0&	0.05162&	0\\               
%	$T_{ARKCOOKS}$ -5C&					0.02993&	0&	0.05162&	0\\               
%	$Q_{BIGLAMCO}$ +20\%&				0.02921&	-2.406&	0.05051&	-2.15\\       
%	$Q_{BIGLAMCO}$ -20\%&				0.03066&	2.439&	0.05273&	2.15\\        
%	$Q_{WILDHOCO}$ +20\%&				0.02954&	-1.303&	0.05098&	-1.24\\       
%	$Q_{WILDHOCO}$ -20\%&				0.03033&	1.336&	0.05227&	1.259\\       
%	$Q_{BUFDITCO}$ +20\%&				0.03225&	7.751&	0.05374&	4.107\\       
%	$Q_{BUFDITCO}$ -20\%&				0.02762&	-7.718&	0.0495&	-4.107\\          
%	$Q_{FRODITKS}$ +20\%&				0.03069&	2.539&	0.05251&	1.724\\       
%	$Q_{FRODITKS}$ -20\%&				0.02918&	-2.506&	0.05073&	-1.724\\      
%	$P_{}$ +25\%&						0.02992&	-0.03341&	0.05162&	0\\       
%	$P_{}$ -25\%&						0.02995&	0.06682&	0.05162&	0\\       
%	$ET_{}$ +0.98 mm&					0.03006&	0.4343&	0.05162&	0\\           
%	$ET_{}$ -0.98 mm&					0.02981&	-0.4009&	0.05162&	0\\       
%	$u_{2}$ +0.5 \si{\meter\per\second}	0.02993&	0&	0.05162&	0\\               
%	$u_{2}$ -0.5 \si{\meter\per\second}	0.02994&	0.03341&	0.05162&	0\\       
%	$RH_{min}$ +2\%&					0.02994&	0.03341&	0.05162&	0\\       
%	$RH_{min}$ -2\%&					0.02993&	0&	0.05162&	0\\               
%	$\beta_{1}$ +10\%&					0.03002&	0.3007&	0.0516&	-0.03874\\        
%	$\beta_{1}$ -10\%&					0.02985&	-0.2673&	0.05164&	0.03874\\ 
%	$\beta_{2}$ +10\%&					0.02992&	-0.03341&	0.05163&	0.01937\\ 
%	$\beta_{2}$ -10\%&					0.02995&	0.06682&	0.05161&	-0.01937\\
%	$h_{F}$ +0.01 ft&					0.02994&	0.03341&	0.05162&	0\\       
%	$h_{F}$ -0.01 ft&					0.02993&	0&	0.05163&	0.01937\\         
%	$h_{G}$ +0.01 ft&					0.02994&	0.03341&	0.05162&	0\\       
%	$h_{G}$ -0.01 ft&					0.02993&	0&	0.05162&	0\\               
%	$h_{F}$ +0.05 ft&					0.02996&	0.1002&	0.0516&	-0.03874\\        
%	$h_{F}$ -0.05 ft&					0.0299&	-0.1002&	0.05164&	0.03874\\     
%	$h_{G}$ +0.05 ft&					0.02996&	0.1002&	0.05163&	0.01937\\     
%	$h_{G}$ -0.05 ft&					0.02991&	-0.06682&	0.05162&	0\\       
%	$h_{F}$ +0.1 ft	&					0.02999&	0.2005&	0.05159&	-0.05812\\    
%	$h_{F}$ -0.1 ft	&					0.02987&	-0.2005&	0.05166&	0.07749\\ 
%	$h_{G}$ +0.1 ft	&					0.02998&	0.1671&	0.05163&	0.01937\\     
%	$h_{G}$ -0.1 ft	&					0.02989&	-0.1336&	0.05161&	-0.01937\\
%	$h_{F}$ +0.25 ft&					0.03006&	0.4343&	0.05154&	-0.155\\      
%	$h_{F}$ -0.25 ft&					0.02203&	-26.39&	0.07928&	53.58\\       
%	$h_{G}$ +0.25 ft&					0.03004&	0.3675&	0.05164&	0.03874\\     
%	$h_{G}$ -0.25 ft&					0.02982&	-0.3675&	0.0516&	-0.03874\\    
%    \hline
%\end{longtable}%
%\end{center}
%\end{spacing}

%The large changes in both the water balance and selenium mass balance are due to changes in the flow input variables.  The measured flow depth in reach segments C, D, and F are also significant contributors to changes in the deterministic models.  

%Refining reported flow rate uncertainty will improve the models.  This is not an acceptable solution to refining the model.  Flow rate uncertainty is not solely based on measuring equipment or methodology, but also on the channel being measured.  The Arkansas R. in the LARB is a variable channel where the only method to improve measurements is to perform more frequent gauge calibration.  This is a cost that is borne by the CDWR and USGS, both of which are perpetually struggling to justify the existing stream gauge system.

%Model improvement can be realistically performed by improving the characterization of the river channel.  River cross section surveys should be performed at a more frequent interval along the river with each cross section tied to a benchmark.  Surveys should also be performed at the same location to determine a temporal relationship.  While more complicated to set up, this effort should be easy to routinely re-accomplish with GPS survey equipment.
\clearpage\]
\end{document}