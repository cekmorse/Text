\documentclass[10pt]{article}
\usepackage[usenames]{color} %used for font color
\usepackage{amssymb} %maths
\usepackage{amsmath} %maths
\usepackage[utf8]{inputenc} %useful to type directly diacritic characters
\begin{document}
\[\renewcommand{\thechapter}{7}

\chapter{Dissolved Selenium Concentration}
\label{chap:Dissolved Selenium Concentration}

\section{Dissolved Selenium Concentrations Models.}
\label{sec:Dissolved Selenium Concentration Models}
%The selenium concentration field data collection effort provided excellent data for specific locations at specific times.  This data needed to be expanded to include as many intermediate locations and times as possible to provide a more complete set of results.  The methodology for determining the dissolved selenium concentration at various locations in both study region river sections is the same with only one exception.  The set of starting independent variables, or starting terms, changes depending on the specific location, but the method of reducing the equation to the final equation is the same.

%Estimating equations were calculated using multivariate linear ordinary least squares regression.  The 'lm' function in the [R] 'stats' package was used to fit linear models of the general form response $~$ terms where response is the numeric response vector, or dependent variable, and terms is a series of terms, or independent variables, which specify a linear predictor.  In all cases, the response vector was the field selenium concentration data for specific locations.  The starting terms included the reported average daily stream flow, EC, and water temperature values for the same day as the selenium concentration sample was collected.  The stream flow, EC, and water temperature values were extracted from the same set of data used to generate the stochastic input variables described in Section \ref{sec:InStreamData}.  

%All data points were included in the linear regression even if some terms were missing.  The 'lm' function has an argument that allows the user to determine what should happen if missing data is encountered.  All analyses were performed such that data points with missing values were excluded from the analysis.  As the number of terms was reduced during the equation refinement process, these excluded data points may or may not be included in the analysis.  This allowed for the maximum number of data points without reforming the data set with each equation refinement iteration.

%Determining which terms to include in each initial regression analysis began with visual analysis of an enhanced scatter-plot matrix of the selenium concentration response vector and the independent variable terms.  Figure~\ref{fig:ExampleFullPairs} is an example.  This particular figure shows the selenium concentration from sampling point U163 compared to the full set of initial terms.  The diagonal contains the variable names for the row and column.  The lower triangle shows the individual variables when plotted against each other.  Individual graphs are in appropriate unites for the investigated variables.  Flow is in units of \si{\cubic\meter\per\second}, EC is in units of \si{\deci\siemens\per\meter}, and temperature is in units of \si{\degreeCelsius}The upper triangle presents the Pearson correlation value for the respective variable pair.  Similar figures for the other regression analyses are included in appendix \ref{App:Scatterplot}

%\begin{figure}[htbp]
%	\begin{center}
%		\includegraphics[width=6in]{"Figures/Results_USR/Conc Model Full PairsU163"}
%		\caption[USR upstream boundary dissolved selenium estimating scatter plot matrix.]{USR upstream boundary dissolved selenium estimating scatter plot matrix.  scatter plot matrix of the variables used to estimate dissolved selenium concentration at the USR upstream boundary.  Variable names are plotted down the diagonal.  Values in the upper triangle are correlation values for the intersecting variables.  Scatterplots for the intersecting variables are plotted in the lower triangle.  Scales are in the units for the given variable.  $C_{Se}$ is in \si{\micro\gram\per\liter}.  Q values are in \si{\cubic\meter\per\second}.  EC values are in \si{\deci\siemens\per\meter}.  T values are in \si{\degreeCelsius}.  Similar figures for the other regression analyses are included in appendix \ref{App:Scatterplot}.}
%	\label{fig:ExampleFullPairs}
%	\end{center}
%\end{figure}

%In this case and as with most other regression analyses, there are strong correlations between terms which would indicate that removing all but the EC value would give an adequate estimation.  The EC, which is related to the total salt concentration, is proportional to $C_{Se}$.  However, temperature and flow are negatively correlated, compared with EC's positive correlation with $C_{Se}$.  An increase in flow volume dilutes the total salt load and therefore $C_{Se}$.  Temperature affects the solubility of all salts.  Including temperature allows the difference in solubility constants to be expressed.  Also, flow has opposing correlations with EC and temperature.  These trends show that although EC alone can be used to estimate $C_{Se}$, it is not sufficient to completely describe how $C_{Se}$ reacts to the environment.

%An initial regression form was used and individual terms were removed based on their individual $p$ value.  Only one variable was removed at a time and the new, reduced form was re-analyzed.  This step-wise method was performed until the $p$ value for the individual variables was less than an $\alpha$ of 0.05.  Residual standard error values, r-squared values, adjusted r-squared values, F-statistics, and p-values were collected for each analysis.  These were provided by [R] as the summary of the regressed linear model.  If the statistics did not appear to support the conclusion that the final, reduced form linear mode was representative of the data, then the process was re-started with a different, more complex, initial regression form.  

%Tables \ref{tab:USRInitialRegression} and \ref{tab:DSRInitialRegression} present the initial regression forms for the river sections in the USR and DSR respectively. These initial regression forms presented do not include the resulting coefficients for the respective variables as they are insignificant to the results of this particular analysis and the complete model results.  Tables \ref{tab:USRFinalRegression} and \ref{tab:DSRFinalRegression} present the final, reduced regression equations with coefficients resulting from the regression analysis.

%\begin{table}[htbp]
%\centering
%\caption{USR Selenium Concentration Initial Regression Models.}
%\label{tab:USRInitialRegression}
%\begin{tabular}{ll}
%	\toprule
%	Location		& Initial regression model. $C_{Se}=$\\
%	\midrule
%	\midrule
%	Inlet		& $Q_{ARKCATCO} + EC_{ARKCATCO} + t_{ARKCATCO}$\\
%	\\
%	Outlet		& $Q_{ARKLASCO} + EC_{ARKLASCO} + t_{ARKLASCO}$\\
%	\\	
%	RFDMANCO		& $Q_{ARKCATCO} + EC_{ARKCATCO} + Q_{ARKLASCO}$\\
%					& $+ EC_{ARKLASCO} + t_{ARKLASCO}$\\
%	\\
%	CONDITCO		& $Q_{ARKCATCO} + EC_{ARKCATCO} + Q_{ARKLASCO}$\\
%					& $+ EC_{ARKLASCO} + t_{ARKLASCO}$\\
%	\\	
%	CANSWKCO		& $Q_{ARKCATCO} + EC_{ARKCATCO} + Q_{ARKLASCO}$\\ 
%					& $+ EC_{ARKLASCO} + t_{ARKLASCO} + Q_{CANSWKCO}$\\
%					& $+ Q_{ARKCATCO}Q_{CANSWKCO} + EC_{ARKCATCO}Q_{CANSWKCO}$\\
%					& $+ Q_{ARKLASCO}Q_{CANSWKCO} + EC_{ARKLASCO}Q_{CANSWKCO}$\\
%					& $+ t_{ARKLASCO}Q_{CANSWKCO}$\\
%	\\
%	TIMSWICO		& $Q_{ARKCATCO} + EC_{ARKCATCO} + Q_{ARKLASCO}$\\
%					& $+ EC_{ARKLASCO} + t_{ARKLASCO} + Q_{TIMSWICO}$\\
%	\\
%	HRC194CO		& $Q_{ARKCATCO} + EC_{ARKCATCO} + Q_{ARKLASCO}$\\
%					& $+ EC_{ARKLASCO} + t_{ARKLASCO} + Q_{HRC194CO}$\\
%	\\	
%	Diversions		& $Q_{ARKCATCO} + EC_{ARKCATCO} + Q_{ARKLASCO}$\\
%					& $+ EC_{ARKLASCO} + t_{ARKLASCO}$\\
%	\\	
%	La Junta WWTP	& $\beta_{1} \cdot Q_{WTP}^{\beta_{2}}$\\
%	\bottomrule \\
%\end{tabular}
%\end{table}

%\begin{table}[htbp]
%\centering
%\caption{USR Selenium Concentration Final Regression Models.}
%\label{tab:USRFinalRegression}
%\begin{tabular}{ll}
%	\toprule
%	Location		& Final regression model. $C_{Se}=$\\
%	\midrule
%	\midrule
%	Inlet		& $-0.05106 \cdot Q_{ARKCATCO} + 4.69 \cdot EC_{ARKCATCO}$\\
%					& $ -0.3063 \cdot t_{ARKCATCO} + 10.12$ \\
%	\\
%	Outlet		& $-0.3138 \cdot Q_{ARKLASCO} -0.1348 \cdot t_{ARKLASCO} + 14.91$\\
%	\\	
%	RFDMANCO		& $-0.3538 \cdot Q_{ARKLASCO} -2.021 \cdot EC_{ARKLASCO}$\\
%					& $ -0.3306 \cdot t_{ARKLASCO} + 21.15$\\
%	\\
%	CONDITCO		& No model due to insufficient data.\\
%	\\	
%	CANSWKCO		& $ -13.01 \cdot EC_{ARKCATCO} -1.022 \cdot Q_{ARKLASCO}-0.4132 \cdot t_{ARKLASCO}$\\ 
%					& $ -16.40 \cdot Q_{CANSWKCO}-0.3138 \cdot Q_{ARKCATCO}Q_{CANSWKCO}$\\
%					& $+2.0730 Q_{ARKLASCO}Q_{CANSWKCO} + 39.40$\\
%	\\
%	TIMSWICO		& $11.42 \cdot EC_{ARKCATCO} -3.160 \cdot Q_{TIMSWICO} + 4.605$\\
%	\\
%	HRC194CO		& $-17.78 \cdot Q_{HRC194CO} + 20.41$\\
%	\\	
%	Diversions		& $4.710 \cdot EC_{ARKCATCO} -0.1568 \cdot Q_{ARKLASCO}-0.1491 \cdot t_{ARKLASCO}$\\
%					& $+0.0203 \cdot d + 8.0831$\\
%	\\	
%	La Junta WWTP	& $19.18 \cdot Q_{WTP}^{0.07664}$\\
%	\bottomrule \\
%\end{tabular}
%\end{table}

%\begin{table}[htbp]
%\centering
%\caption{USR Selenium Concentration Initial Regression Models.}
%\label{tab:DSRInitialRegression}
%\begin{tabular}{ll}
%	\toprule
%	Location		& Initial regression model. $C_{Se}=$\\
%	\midrule
%	\midrule
%	Inlet		& $Q_{ARKLAMCO} + EC_{ARKJMRCO} + t_{ARKJMRCO}$\\
%					& $+Q_{ARKLAMCO}^2 + Q_{ARKLAMCO}EC_{ARKJMRCO}$\\
%					& $+ Q_{ARKLAMCO}t_{ARKJMRCO} + EC_{ARKJMRCO}^2$\\
%					& $+ EC_{ARKJMRCO}t_{ARKJMRCO} + t_{ARKJMRCO}^2$\\
%	\\
%	Outlet		& $Q_{ARKCOOKS} + EC_{ARKCOOKS} + t_{ARKCOOKS}$\\
%					& $+Q_{ARKCOOKS}^2 + Q_{ARKCOOKS}EC_{ARKCOOKS}$\\
%					& $+ Q_{ARKCOOKS}t_{ARKCOOKS} + EC_{ARKCOOKS}^2$\\
%					& $+ EC_{ARKCOOKS}t_{ARKCOOKS} + t_{ARKCOOKS}^2$\\
%	\\
%	BIGLAMCO		& $Q_{ARKLAMCO} + EC_{ARKJMRCO} + Q_{ARKCOOKS}$\\ 
%					& $+ EC_{ARKCOOKS} + t_{ARKCOOKS} + Q_{BIGLAMCO}$\\
%					& $+ Q_{ARKLAMCO}Q_{BIGLAMCO} + EC_{ARKJMRCO}Q_{BIGLAMCO}$\\
%					& $+ Q_{ARKCOOKS}Q_{BIGLAMCO} + EC_{ARKCOOKS}Q_{BIGLAMCO}$\\
%					& $+ t_{ARKCOOKS}Q_{BIGLAMCO}$\\
%	\\
%	WILDHOCO		& $Q_{ARKLAMCO} + EC_{ARKJMRCO} + t_{ARKJMRCO} + Q_{ARKCOOKS}$\\
%					& $+ EC_{ARKCOOKS} + Q_{WILDHOCO} + Q_{ARKLAMCO}^2 $\\
%					& $+ Q_{ARKLAMCO} EC_{ARKJMRCO}+ Q_{ARKLAMCO} t_{ARKJMRCO} $\\
%					& $+ Q_{ARKLAMCO} Q_{ARKCOOKS}+ Q_{ARKLAMCO} EC_{ARKCOOKS} $\\
%					& $+ Q_{ARKLAMCO} Q_{WILDHOCO}+ EC_{ARKJMRCO}^2 $\\
%					& $+ EC_{ARKJMRCO} t_{ARKJMRCO}+ EC_{ARKJMRCO} Q_{ARKCOOKS} $\\
%					& $+ EC_{ARKJMRCO} EC_{ARKCOOKS}+ EC_{ARKJMRCO} Q_{WILDHOCO} $\\
%					& $+ t_{ARKJMRCO}^2+ t_{ARKJMRCO} Q_{ARKCOOKS} $\\
%					& $+ t_{ARKJMRCO} EC_{ARKCOOKS}+ t_{ARKJMRCO} Q_{WILDHOCO} $\\
%					& $+ Q_{ARKCOOKS}^2+ Q_{ARKCOOKS} EC_{ARKCOOKS} $\\
%					& $+ Q_{ARKCOOKS} Q_{WILDHOCO}+ EC_{ARKCOOKS}^2 $\\
%					& $+ EC_{ARKCOOKS} Q_{WILDHOCO}+ Q_{WILDHOCO}^2$\\
%	\\
%	Diversions		& $Q_{ARKLAMCO} + EC_{ARKJMRCO} + Q_{ARKCOOKS}$ \\
%					& $+ EC_{ARKCOOKS} + t_{ARKCOOKS} + d$\\
%	\bottomrule \\
%\end{tabular}
%\end{table}

%\begin{table}[htbp]
%\centering
%\caption{DSR Selenium Concentration Final Regression Models.}
%\label{tab:DSRFinalRegression}
%\begin{tabular}{ll}
%	\toprule
%	Location	& Final regression model.  $C_{Se}=$\\
%	\midrule
%	\midrule
%	Inlet		& $3.429 \cdot EC_{ARKJMRCO} - 0.005623 \cdot Q_{ARKLAMCO}EC_{ARKJMRCO}$\\
%					& $-0.07581 \cdot EC_{ARKJMRCO}t_{ARKJMRCO} + 6.355$\\
%	\\
%	Outlet		& $-0.37 \cdot t_{ARKCOOKS}+ 0.0627 \cdot EC_{ARKCOOKS}t_{ARKCOOKS} + 16.56$\\
%	\\
%	BIGLAMCO		& $0.08951 \cdot Q_{ARKLAMCO} -0.9925 \cdot Q_{ARKCOOKS}$\\
%					& $-1.376 \cdot Q_{BIGLAMCO} -0.007387 \cdot Q_{ARKLAMCO}Q_{BIGLAMCO}$\\
%					& $+0.006882 \cdot Q_{ARKCOOKS}Q_{BIGLAMCO} + 36.96$\\ 
%	\\
%	WILDHOCO		& $-42.64 \cdot EC_{ARKJMRCO} -3.309 \cdot t_{ARKJMRCO}$\\
%					& $-14.18 \cdot EC_{ARKCOOKS} -0.006969 \cdot Q_{ARKLAMCO}Q_{WILDHOCO}$\\
%					& $+0.005895 \cdot Q_{ARKCOOKS}Q_{WILDHOCO}$\\
%					& $-0.2522 \cdot Q_{WILDHOCO}EC_{ARKJMRCO}$\\
%					& $+5.779 \cdot EC_{ARKJMRCO}EC_{ARKCOOKS}$\\
%					& $+1.029 \cdot EC_{ARKJMRCO}t_{ARKJMRCO}$\\
%					& $+0.2643 \cdot EC_{ARKCOOKS}t_{ARKJMRCO} -112.7$\\
%	\\
%	Diversions		& $-0.006576 \cdot Q_{ARKLAMCO} +1.322 \cdot EC_{ARKJMRCO}$\\
%					& $+0.003794 \cdot Q_{ARKCOOKS} +0.4512 \cdot EC_{ARKCOOKS}$\\
%					& $-0.07653 \cdot t_{ARKCOOKS} +0.1066 \cdot d + 5.709$\\
%	\bottomrule \\
%\end{tabular}
%\end{table}

%Selenium concentration at the La Junta WWTP was not able to be sufficiently estimated using linear models.  The La Junta water distribution system receives most of its water from wells.  It was not known which aquifer supplied the city.  It was assumed that the shallow river riparian aquifer was the city's water source as there are no known deep aquifers in the area.  It was also assumed that the water treatment plant processes and waste water treatment plant processes could change the selenium concentration.  The average daily flow produced by the plant is the only variable available to estimate selenium concentrations.  Visual analysis of the scatter plot similar to Figure \ref{fig:ExampleFullPairs} found in appendix \ref{App:Scatterplot} showed that the relationship between the plant discharge rate and selenium concentration could not be easily defined.  A non-linear model was used with the power function described in Table \ref{tab:USRInitialRegression}.  This model produced fairly reasonable results.  Better results could be obtained if average daily EC values of the plant discharge were available.

%The initial regression models for the inlet and outlet in the USR are the simplest forms used in this study.  Similar models were used for the other locations, but they did not produce final models that had acceptable model summary statistics.  More complex initial models were used to include the product of input variables.  Summary statistics are presented in Figures \ref{tab:USRSumStat} and \ref{tab:DSRSumStat} for the USR and DSR, respectively.  The desired summary statistics included r-squared values greater than 0.5 and a p-value less than or equal to 0.05.  Additionally, when different models of the same data were compared, the residual standard error (RSE) and multiple r-squared values were compared.  Lower RSE and higher multiple r-squared values were desired.

\section{Analysis of Dissolved Selenium Models}
\label{sec:Analysis of Dissolved Selenium Models}
%\begin{table}[htbp]
%\centering
%\caption{USR Selenium Concentration Regression Models Summary Statistics.}
%\label{tab:USRSumStat}
%\begin{tabular}{lccccc}
%	\toprule
%	\multirow{2}{*}{Location}	& \multirow{2}{*}{RSE, DoF}	& Multiple 	& Adjusted 	& F-statistic, 		& \multirow{2}{*}{p-value}\\
%				& 			& R-squared	& R-squared	& DoF&\\
%	\midrule
%	\midrule
%	Inlet			&0.5923, 11	&0.9818	&0.9769	&198.3, 3 and 11	&7.4\e{-10}\\
%	Outlet			&0.9467, 14	&0.9145	&0.9022	&74.84, 2 and 14	&3.3\e{-8}\\
%	RFDMANCO		&0.7016, 10	&0.9684	&0.9589	&102.1, 3 and 10	&8.4\e{-8}\\
%	CANSWKCO		&0.9868, 10	&0.9501	&0.9202	&31.74, 6 and 10	&6.0\e{-6}\\
%	TIMSWICO		&1.749, 15	&0.8633	&0.845	&47.35, 2 and 15	&3.3\e{-7}\\
%	HRC194CO		&1.301, 2	&0.9342	&0.9014	&28.41, 1 and 2		&3.3\e{-2}\\
%	Diversions		&1.196, 130	&0.8808	&0.8772	&240.2, 4 and 130	&<2.2\e{-16}\\
%	La Junta WWTP	&4.153, 85	&NA&NA&NA&NA\\
%	\bottomrule \\
%	\multicolumn{6}{l}{\footnotesize RSE = Residual standard error}\\
%	\multicolumn{6}{l}{\footnotesize DoF = Degrees of Freedom}\\
%\end{tabular}
%\end{table}

%\begin{table}[htbp]
%\centering
%\caption{DSR Selenium Concentration Regression Models Summary Statistics.}
%\label{tab:DSRSumStat}
%\begin{tabular}{lccccc}
%	\toprule
%	\multirow{2}{*}{Location}	& \multirow{2}{*}{RSE, DoF}	& Multiple 	& Adjusted 	& F-statistic, 		& \multirow{2}{*}{p-value}\\
%				& 			& R-squared	& R-squared	& DoF&\\
%	\midrule
%	\midrule
%	Inlet		& 2.553, 38		&0.5472	&0.5114	&15.31, 3 and 38	&1.1\e{-6}\\
%	Outlet		& 3.534, 39		&0.3404	&0.3065	&10.06, 2 and 36	&3.0\e{-4}\\
%	BIGLAMCO	& 5.701, 37		&0.5819	&0.5254	&10.3, 5 and 37		&3.5\e{-6}\\
%	WILDHOCO	& 2.337, 21		&0.7278	&0.6111	&3.238, 9 and 21	&2.7\e{-4}\\
%	Diversions	& 1.724, 156	&0.7104	&0.6993	&63.78, 6 and 156	&<2.2\e{-16}\\
%	\bottomrule \\
%	\multicolumn{6}{l}{\footnotesize RSE = Residual standard error}\\
%	\multicolumn{6}{l}{\footnotesize DoF = Degrees of Freedom}\\
%\end{tabular}
%\end{table}

%Individual regression models were analyzed to determine if they were representative of the data.  Figure \ref{fig:ExampleLmFit} shows an example of the graphs used.  This particular figures shows the results for the USR river section inlet.  Similar plots for the rest of the linear model analyses is included in appendix \ref{App:LmFit}. The top left panel shows the residuals plotted against the fitted values.  All points should be evenly distributed in both directions.  If the points are not symmetrical along the fitted values axis, then heteroscedasticity should be suspected.  If the points are not symmetrical about either axis, then the regression model is missing an estimating term.  The bottom left panel shows a scale-location plot of the data.  This panel shows the same information as the first panel with the exception that the square-root of the standardized residuals is used.  This reduces the effect of skewness on the analysis.  Like the first panel, the points should be symmetrical along both axes.  The top right panel shows a normal quantile-quantile (Q-Q) plot of the residuals.  It is a reasonable assumption that residuals of a linear model are normally distributed.  If the residuals are normally distributed, then the points should fit closely to the line at $y=x$.  The bottom right panel shows the residuals plotted against the leverage of the residuals.  The dashed red lines show the Cook's distance which is a measure of the influence a particular data point has on the regression model.  Points with a higher leverage have a higher influence on the model.  Points with leverage values higher than the majority of the data may be outliers.

%In all panels, potential outliers are indicated by having the index number of the data printed next to the point.  If the same data points are indicated as outliers on most of the panels, then it is most likely true that they are outliers.  None of the models had the outliers removed to improve the model.  In fact, outliers were expected.  Most of the dissolved selenium samples were collected during similar times in the year.  Bias towards certain flow regimes and other factors are most likely present in the data.  The outliers represent the samples taken outside of the normal sampling season and are more representative of the extremes.  These outliers are necessary to the complete analysis and were not removed from the linear regression analyses.

%\begin{figure}[htbp]
%	\begin{center}
%		\includegraphics[width=6in]{"Figures/Results_USR/Conc Model lm-fit U163"}
%		\caption[Example selenium concentration linear model analysis graphs.]{Example selenium concentration linear model analysis graphs.  This particular set of graphs shows the results for the USR river section inlet.  The description of the individual panels found in the text.  Similar plots for the rest of the linear model analyses are included in appendix \ref{App:LmFit}.}
%	\label{fig:ExampleLmFit}
%	\end{center}
%\end{figure}

%The measured concentration values were compared to the predicted concentration values for all regression analyses.  Figure \ref{fig:ExamplePredVMeas} shows an example of one of the graphs used in this comparison.  This particular figures presents data for the USR river section inlet.  An $y=x$ line was plotted .  Points below the line show that the estimating equation under-estimated the selenium concentration.  The vertical distance between the point and the line corresponds to the estimate error, or residual.  Similar figures for the other regression analyses are presented in appendix \ref{App:PredVMeas}.

%\begin{figure}[htbp]
%	\begin{center}
%		\includegraphics[width=6in]{"Figures/Results_USR/Conc Model pred v meas U163"}
%		\caption[Example measured vs. estimated selenium concentration comparison.]{Example measured vs. estimated selenium concentration comparison.  This particular graph shows the comparison for the upstream inlet to the USR river section.  The predicted values are calculated using the equation in Table \ref{tab:USRFinalRegression}.  The diagonal line has a slope of 1 passing through the origin to show the over or under estimation.  All concentrations are presented in \si{\micro\gram\per\liter}.  Similar figures for the other regression analyses are presented in appendix. \ref{App:PredVMeas}.}
%	\label{fig:ExamplePredVMeas}
%	\end{center}
%\end{figure}

%Selenium estimation calculation error, $\varepsilon_{1}$, was analyzed to determine the best fit distribution for each selenium estimating equation.  Normal and logistic distributions were fit to the regression model residuals.  Both of these distributions are unbounded and are simple to apply.  They also fit the assumption that the linear model residuals are normally distributed.  Logistic distributions were included because they are very similar to normal distributions, but with heavier tails.  The best-fit normal and logistic distributions were compared to the regression model residuals by using Kolmogorov-Smirnov, Cramer von Mises, and Anderson-Darling goodness-of-fit tests.  The results of these test determined which of either the best fit normal distribution or best fit logistic distribution described the regression model residuals.  Results from the goodness-of-fit tests are presented in Tables~\ref{tab:USRGoF} and \ref{tab:DSRGof} for the USR and DSR, respectively

%\begin{table}[htbp]
%  \centering
%  \caption[USR selenium concentration residuals goodness-of-fit test results.]{USR selenium concentration residuals goodness-of-fit test results.  Kolmogorov-Smirnov (K-S), Cramer von Mieses (CvM), and Anderson-Darling (AD) test statistics are presented for each regression model.}
%    \begin{tabular}{lcccc}
%    \toprule
%    \multirow{2}{*}{Location}&Tested & \multicolumn{3}{c}{Test Statistics} \\ \cline{3-5}
%    &Distribution  & K-S   & CvM   & A-D \\
%    \midrule
%    \midrule
%    \multirow{2}{*}{Inlet}			&Logistic*	&0.107	&0.022	&0.151 \\
%    								&Normal		&0.116	&0.028	&0.197 \\
%    \midrule
%    \multirow{2}{*}{Outlet}			&Logistic*	&0.180	&0.129	&0.766	\\
%    								&Normal		&0.178	&0.140	&0.792	\\
%    \midrule
%    \multirow{2}{*}{RFDMANCO}		&Logistic*	&0.091	&0.020	&0.165	\\
%    								&Normal		&0.100	&0.025	&0.192	\\
%    \midrule
%    \multirow{2}{*}{CANSWKCO}		&Logistic*	&0.136	&0.053	&0.315	\\
%    								&Normal		&0.140	&0.076	&0.419	\\
%    \midrule
%    \multirow{2}{*}{TIMSWICO}		&Logistic*	&0.128	&0.076	&0.476	\\
%    								&Normal		&0.161	&0.086	&0.548	\\
%    \midrule
%    HRC194CO						&Logistic*	&\multicolumn{3}{c}{Insufficent sample size}\\
%    \midrule
%    \multirow{2}{*}{Diversions}		&Logistic*	&0.041	&0.032	&0.352	\\
%    								&Normal		&0.109	&0.354	&2.47	\\
%    \midrule
%    \multirow{2}{*}{La Junta WWTP}	&Logistic*	&0.090	&0.163	&1.22	\\
%    								&Normal		&0.130	&0.281	&1.57	\\
%    \bottomrule
%    \multicolumn{5}{l}{\footnotesize * = best fit distribution}\\
%    \end{tabular}%
%  \label{tab:USRGoF}%
%\end{table}%

%\begin{table}[htbp]
%  \centering
%  \caption[DSR selenium concentration residuals goodness-of-fit test results.]{DSR selenium concentration residuals goodness-of-fit test results.  Kolmogorov-Smirnov (K-S), Cramer von Mieses (CvM), and Anderson-Darling (AD) test statistics are presented for each regression model.}
%    \begin{tabular}{lcccc}
%    \toprule
%    \multirow{2}{*}{Location}&Tested & \multicolumn{3}{c}{Test Statistics} \\ \cline{3-5}
%    &Distribution  & K-S   & CvM   & A-D \\
%    \midrule
%    \midrule
%    \multirow{2}{*}{Inlet}			&Logistic	&0.0923	&0.038	&0.263	\\
%    								&Normal*	&0.090	&0.378	&0.259	\\
%    \midrule
%    \multirow{2}{*}{Outlet}			&Logistic*	&0.082	&0.289	&0.199	\\
%    								&Normal		&0.110	&0.060	&0.384	\\
%    \midrule
%    \multirow{2}{*}{BIGLAMCO}		&Logistic	&0.103	&0.082	&0.508	\\
%    								&Normal*	&0.092	&0.073	&0.426	\\
%    \midrule
%    \multirow{2}{*}{WILDHOCO}		&Logistic*	&0.117	&0.058	&0.340	\\
%    								&Normal		&0.140	&0.093	&0.518	\\
%	\midrule
%    \multirow{2}{*}{Diversions}		&Logistic*	&0.041	&0.020	&0.112	\\
%    								&Normal		&0.043	&0.050	&0.319	\\
%    \midrule
%    \bottomrule
%    \multicolumn{5}{l}{\footnotesize * = best fit distribution}\\
%    \end{tabular}%
%  \label{tab:DSRGof}%
%\end{table}%

%The summaries of the fitted distributions are presented in Tables \ref{tab:USRResStat} and \ref{tab:DSRResStat}, for the USR and DSR respectively.  In all but one case, the location parameters are near zero.  The location parameter for logistic distributions and the mean parameter for normal distributions provide the same information; they describe the central tendency of the distribution.  For distributions that describe model error, the goal is to have this value near zero.  The location parameter for the La Junta WWTP selenium concentration error distribution is a significant distance from zero.  The selenium concentration lab results minimum detection level is \SI{0.4}{\micro\gram\per\liter} and the La Junta WWTP location parameter approaches this value.  This indicates that the selenium concentration estimating model for the La Junta WWTP is missing an estimating parameter.  Unfortunately, no other parameters were available for the collected data.

%\begin{table}
%  \caption[USR selenium concentration residual distribution summary statistics.]{USR selenium concentration residual distribution summary statistics.}
%  \label{tab:USRRes-Fit}
%  \centering
%    \begin{tabular}{lcccc}
%    \toprule
%    \multirow{2}{*}{Location}&Best Fit&\multirow{2}{*}{n}	&\multicolumn{2}{c}{Parameter Estimate}\\ \cline{4-5}
%    &Distribution&&Param. $1^{1}$&Param. $2^{2}$\\    
%    \midrule
%    \midrule
%    Inlet			&Logistic	&15	&5.7\e{-3}	&0.2810\\
%    Outlet			&Logistic	&17	&-3.0\e{-2}	&0.5365\\
%    RFDMANCO		&Logistic	&14	&1.5\e{-2} 	&0.3383\\
%	CANSWKCO		&Logistic	&17	&1.7\e{-3} 	&0.4049\\
%    TIMSWICO		&Logistic	&18	&-6.3\e{-2} &0.8743\\
%    HRC194CO		&Logistic	&4	&-1.4\e{-2} &0.5508\\
%    Diversions		&Logistic	&135&-5.2\e{-2} &0.5615\\
%	La Junta WWTP	&Logistic	&87	&-3.8\e{-1} &2.313\\
%    \bottomrule
%    \multicolumn{5}{l}{\footnotesize $^{1}$  For normal distributions, mean.  For logistic distributions, location.}\\
%    \multicolumn{5}{l}{\footnotesize $^{2}$  For normal distributions, standard deviation.  For logistic distributions, scale.}\\
%    \end{tabular}%
%\end{table}%

%\begin{table}
%  \caption[DSR selenium concentration residual distribution summary statistics.]{DSR selenium concentration residual distribution summary statistics.}
%  \label{tab:DSRRes-Fit}
%  \centering
%    \begin{tabular}{lcccc}
%    \toprule
%    \multirow{2}{*}{Location}&Best Fit&\multirow{2}{*}{n}	&\multicolumn{2}{c}{Parameter Estimate}\\ \cline{4-5}
%    &Distribution&&Param. $1^{1}$&Param. $2^{2}$\\    
%    \midrule
%    \midrule
%    Inlet			&Normal		&42	&-1.7\e{-17}	&2.429\\
%    Outlet			&Logistic	&42	&7.5\e{-2} 		&1.834\\
%    BIGLAMCO		&Normal		&43	&-1.2\e{-16}	&5.289\\
%    WILDHOCO		&Logistic	&31	&-8.4\e{-3} 	&1.026\\
%    Diversions		&Logistic	&163&1.2\e{-2} 		&0.9364\\
%    \bottomrule
%    \multicolumn{5}{l}{\footnotesize $^{1}$ = For normal distributions, mean.  For logistic distributions, location.}\\
%    \multicolumn{5}{l}{\footnotesize $^{2}$ = For normal distributions, standard deviation.  For logistic distributions, scale.}\\
%    \end{tabular}%
%\end{table}%

%Statistical plots of the residual distributions, similar to \ref{fig:ExampleRes-Fit}, were created to visually analyze the data distribution and determine if the chosen distribution represented the data.  These figures are diagnostic plots that are produced by [R].  They were not altered or customized.  The top left panel shows a histogram of the residuals.  The red curve is the chosen best-fit distribution.  The bottom left panel shows the cumulative distribution function of the chosen distribution in red.  The points represent the actual collected data.  In a well fit distribution, the points should lie on or near the fitted distribution cumulative distribution function line.  The top right panel shows the well known quantile-quantile plot.  This is a visual test for normalcy.  The points should lie on or near the line $y=x$ if the distribution is normal.  The focus on this panel is to see how well the tails fit a normal distribution.  The bottom right panel is a probability-probability plot.  This is also a visual test for normalcy and the points should lie on or near the line $y=x$ if the distribution is normal.  The focus on this panel is to see how well the center of the data fits a normal distribution.

%\begin{figure}[htbp]
%	\begin{center}
%		\includegraphics[width=6in]{"Figures/Results_USR/Conc Model res-fit U163"}
%		\caption[USR upstream boundary selenium estimate residual distribution analysis.]{USR upstream boundary selenium estimate residual distribution analysis.  The top left plot is a histogram of the residuals with the estimated logistic distribution plotted over top.  The top right plot is a quantile-quantile (Q-Q).  The bottom left is a plot of the theoretical cumulative distribution function (CDF) against the empirical CDF.  The bottom right is a probability-probability plot.  Data is the $C_{Se}$ estimation residuals from Eq.\ref{eq:USRUS} in units of $\mu g \cdot L^{-1}$.}
%	\label{fig:ExampleRes-Fit}
%	\end{center}
%\end{figure}

%A number of the chosen residual distributions do not appear to fit the data.  The residuals for the USR outlet appear to not be normally distributed.  The distributions for CANSWKCO and TIMSWICO do not appear to be good fits.  The data and distribution for HRC194CO does not have enough data to allow for a conclusive analysis.  For the most part, the lack of a good fit can be traced back to data collection.  First, for the poorly fit distributions, there was insufficient data collected.  Second, there was a tendency for the samples to be taken during the same time frame each year.  If selenium concentration is seasonal, then the samples should have been taken relatively equally spaced throughout the year to capture the seasonal variation.

%These selenium concentration error distributions were not discarded due to these findings.  It was assumed that the error could be best described by the best fit distributions determined in this analysis.  Future data should be included in future analyses to develop more accurate estimation models and error distribution.

%To test [R]'s ability to generate the required distribution, residuals were plotted as a histogram overlain with a kernel density estimate as shown in the example in Figure~\ref{fig:ExampleResHist}.  This particular figures shows the results for the USR river section inlet.  Similar plots for the rest of the linear model analyses are included in appendix \ref{App:ResHist}.  The black line is the kernel density estimate of the residuals.  The red line is the kernel density estimate of 5000 draws from the best fit regression model.  In spite of the earlier findings that some of the best fit selenium concentration error distributions are not good fits, all graphs visually indicated that the fitted distributions are not as poor as expected.

%\begin{figure}[htbp]
%	\begin{center}
%		\includegraphics[width=6in]{"Figures/Results_USR/Conc Model ResDist U163"}
%		\caption[USR upstream boundary selenium estimate residual histogram.]{USR upstream boundary selenium estimate residual histogram.  The kernel density of the residuals is plotted over the histogram.  Similar plots for the rest of the linear model analyses are included in appendix \ref{App:ResHist}.}
%	\label{fig:ExampleResHist}
%	\end{center}
%\end{figure}

\section{Dissolved Selenium Lab and Field Sampling Error Analysis}
\label{sec:Dissolved Selenium Lab and Field Sampling Error Analysis}
%Any sampling methodology is subject to error from a multitude of sources.  The additional combined selenium concentration estimating error, $\varepsilon_{2}$, includes error due to variations in field sampling technique, environmental variations, and lab analysis variations.  The samples collected in this study were also subject to an additional unknown error due to environmental conditions during transport from the field to the lab.  In some cases, samples reached the lab seven days after being taken in the field.  Field technicians took great efforts to keep the samples chilled throughout the field collection process.  At the end of a field sampling trip, samples were sent in a chilled insulated container by mail to the lab.  The environmental conditions during this transport phase were not and could not be monitored.

%Upon receipt at the lab, samples were stored in a refrigerator until they were analyzed.  The temperature upon receipt was not recorded by any of the labs.  The labs stored the samples in a refrigerator for a maximum of four days.  It is not known what, if any, chemistry changes within the samples from the time the samples are taken to the time they are analyzed at the lab.  It is also unknown if there is any difference due to minor variations in sampling technique.

%Preferably, these error sources could be accounted for on an individual basis.  There were a number of factors that determined that this methodology would not work.  Error analysis would have to be performed for each field technician.  The total project data collection time frame spanned 10 years and included an unknown number of field technicians.  The data collection methodology previously described was not entirely adhered for the entire data collection time frame.  Not all field technicians recorded what was later considered valuable information such as date and time of sample collection, field technician name, and sampling variances.

%The travel distance between the field locations and the lab is also a factor that cannot be overcome.  Preferably, the lab would be located fairly close to either the university or the study region.  At the start of the sampling time frame, there was no lab in Colorado capable of handling the required analysis with the additional physical, schedule, and fiscal requirements imposed by the project supervisor.  The additional distance made determining error due to transport time difficult to determine.

%All dissolved selenium samples were treated with nitric acid to stabilize the sample for transport.  The stabilization method and acceptable sample storage duration was discussed at length with the lab before any sampling was undertaken.  The samples were additionally preserved by storing and transporting them on ice.  We were assured by the sampling lab that this additional preservation step would only serve to lengthen the time the sample would be considered viable.

%The temperature variations experienced by the samples was not considered a factor during the sample collect time frame.  On hind-sight, this could have easily been performed by adding a temperature transponder to the sample container before shipping.  This technique might have brought transport temperature control issues to the technician's or the project supervisor's attention if the existed.  Unfortunately, this information is not available and we are left to assume that even though significant temperature variations may have existed, those variations did not significantly change the sample chemistry due to the applied sample preservation.

%Field and lab blanks were used to determine if the samples were subjected to contamination from the environment or cross contamination from other samples.  No lab or field blanks exhibited any evidence of contamination.  Since the blanks contained only de-ionized water, they did not have any chemical or physical markers to show whether they experienced unacceptable environmental conditions.

%Lab analysis errors are known to exist and the lab states these ranges.  Since the lab was USEPA certified and subscribed to USEPA proficiency testing, we can assume that the lab errors are as stated by the labs.  Verification through a different lab was not performed at any time.  Although the lab error range was known,  it was not known if that error range could be influence by variances in the sampling technique or transport environment.  

%Combining all individual unaccountable errors into a single error term seemed to be the most pragmatic means to estimate the total error.  The only data available to analyze was the set of duplicate samples.  As previously discussed at least two samples per sample trip were taken as duplicates.  Duplicate samples were taken near the beginning and the end of the sampling trip.  Only the 'A' sample was used for concentration estimation.  The 'B' samples were taken to monitor for equipment malfunction, significant deviation in sampling methodology, and significant lab error.  The 'A' and 'B' samples were taken using the same equipment and transported in the same container from the sample location to the lab.  Since they experienced the same environmental conditions, it was unreasonable to assume that this method could be used to estimate the error due to extreme transport environmental conditions.  The samples were well preserved and it was assumed that temperature variations did not significantly affect the samples.  

%Lab results for the 'A' and 'B' samples from both the USR and DSR were complied into a single data set.  Date, location, and all other identifying markers were removed from the data set to reduce potential bias due to prior knowledge of the individual samples.  'A' samples were assumed to be the expected value for the following samples.  

%Figure \ref{fig:CSeError} shows the results of the following calculations.  All four panels show histograms.  Kernel density estimates of the calculation results as black lines.  Red and blue lines are kernel density estimates from 5000 draws from the best fit normal and logistic distributions, respectively.  Kernel density estimates of the best fit distributions were used instead of the predicted distribution as a test to verify that [R] drew from the distribution appropriately.

%\begin{figure}[htbp]
%\begin{center}
%		\includegraphics[width=6in]{"Figures/Results_USR/CSe Error"}
%\caption[Dissolved selenium sample error analysis.]{Dissolved selenium sample error analysis.  A histogram is presented for each calculation.  The black line is the kernel density estimate of the calculation.  The red and blue lines are kernel density estimates from 5000 draws from the best fit normal and logistic distributions, respectively.}
%\label{fig:CSeError}
%\end{center}
%\end{figure}

%The first calculation, with results in the upper left panel of Figure \ref{fig:CSeError}, shows the difference of the 'B' sample from the 'A' sample.  These results appear centrally located near zero, but significantly large outliers are present.  The top right panel shows the absolute value of the difference calculated for the first panel.  Here, as expected, the best fit distributions do not fit the calculation results.  Again, a significant number of outliers are present.  The bottom left panel show the percent difference of the 'B' sample from the 'A' sample.  The outliers from the previous calculations are closer to the main body of data.  The best fit normal distribution incorporates more of the outliers within its span and the logistic distribution more closely resembles the kernel density estimate of the calculation results.  The bottom right panel shows the absolute value of the percent difference.  Again, neither the normal or logistic distributions fit the data well.

%Visual analysis seems to indicate that using the percent difference distribution would lead to the best characterization of the selenium sample errors.  This hypothesis was tested by using Kolmogorov-Smirnov, Cramer von Mises, and Anderson-Darling goodness-of-fit tests.  Results from these tests are presented in Table \ref{tab:CSeGoF}.  The logistic distribution of the percent difference calculation was shown to have the best fit calculation.

%\begin{table}
%  \caption[Selenium combined error analysis goodness-of-fit test results.]{Selenium combined error analysis goodness-of-fit test results.}
%  \label{tab:CSeGoF}
%  \centering
%    \begin{tabular}{lcccc}
%    \toprule
%    \multirow{2}{*}{Calculation} & \multirow{2}{*}{Distribution} & \multicolumn{3}{c}{Goodness-of-Fit Test Result}\\ \cline{3-5}
%     & & K-S & CvM & A-D\\
%    \midrule
%    \midrule
%    \multirow{2}{*}{Difference} & normal & 0.2876 & 2.598 & 13.48\\
%     & logistic & 0.1644 & 0.8538 & 5.225\\
%    \midrule
%    \multirow{2}{*}{Absolute Difference} & normal & 0.3114 & 3.703 & 18.80\\
%     & logistic & 0.2734 & 1.592 & 10.21\\
%    \midrule    
%    \multirow{2}{*}{Percent Difference} & normal & 0.173 & 1.150 & 6.365\\
%     & logistic & 0.1195 & 0.4139 & 2.729\\
%    \midrule     
%    \multirow{2}{*}{Absolute Percent Difference} & normal & 0.246 & 2.391 & 12.83\\
%     & logistic & 0.2260 & 1.010 & 7.698\\
%    \bottomrule
%    \end{tabular}%
%\end{table}%

%Given these results, the total field sampling and lab error distribution is described by a logistic distribution with a location parameter of -0.06693 and a scale parameter of 1.807.  The combined field sampling and lab error is bounded such that 95\% of the distribution lies in the range of approximately $\pm$6.6\% of the expected value.

%The combined field sampling and lab error is calculated independently from the selenium concentration estimation error previously described.  The estimated selenium concentration, without the estimation estimation, is taken as the expected value for the combined field sampling and lab error.  
\clearpage{}
\]
\end{document}