% This is the preamble document that contains the author editable preferences for creating a Thesis.  The thesis class document (csuthesis.cls) is required for this to work properly.  Make sure you make a backup and have reference documents before making changes to the class document.  Minor changes can make the final document ususable.

% Packages %
% package description in comments
% refer to package documentation for more info
% comment out (don't delete) packages you don't want to use or don't need

% Specifies file encoding, specifically for the bibliography.
%\usepackage[utf8]{inputenc}

% specify the main language of the document
\usepackage[american]{babel}  

% allows for landscape orientation of figures
\usepackage{pdflscape} 

% enhanced support for graphics.  Image scaling, etc.
\usepackage{graphicx} 

% create tabular cells spanning multiple rows (row-wise merge in a column)
\usepackage{multirow}

% produces figures which text can flow around floats.
%\usepackage{wrapfig} 

% adds line numbers.  Sometimes helps reviewers.
\usepackage[pagewise, modulo]{lineno}

% extends math symbols.  includes registered and copyright symbols.
\usepackage{amssymb} 

% attach files to document.  Easier to manage multiple chapter, sections, etc.
\usepackage{attachfile} 

% helps with spacing.  Need this for Table of Contents.
\usepackage[parfill]{parskip} 

% fairly simple method to add units to values.  Consistent standardized useage
\usepackage{siunitx} 

% publication quality tables.  Longtable allows for tables to span multiple pages
\usepackage{booktabs, longtable} 

% add appendicies to document.
\usepackage[toc,page]{appendix} 

% extends implementation of the array and tabular environments
\usepackage{array} 

% customize captions in float environments
\usepackage{caption} 

% caption sub figures in a figure
\usepackage{subcaption} 

% support for setting spacing between lines in a document
\usepackage{setspace} 

% allows for starting a command (especially a float) after the next page break.
\usepackage{afterpage} 

% fancier tables.  allows text wrapping in a cell
\usepackage{tabularx}

% better smart quotes.  Especially usefull for humanities
\usepackage{csquotes}

% Bibliography

% Old bibliography package. only use if you have to
%\usepackage{natbib} 

%  Highly recommended that the newer biblatex package is used.
\usepackage[style=authoryear-comp, url=true, backend=biber]{biblatex} %bibencoding=utf8
\addbibresource{BibRef.bib}

%  -----  End packages  -----  

%  -----  Margins and spacing  -----  

% use if you want more room for rh margin notes.
\usepackage[top=1in,bottom=1in,left=0.5in,right=1.5in]{geometry}  

% Use this when you're ready for final(ish) production
%\usepackage[top=1in,bottom=1in,left=1in,right=1in]{geometry}  
\makeatletter

% highlighted margin notes for 'todo' items
\usepackage[textwidth=1in]{todonotes}

% make todo notes single spaced
\renewcommand{\todo}[2][]{\@todo[caption={#2}, #1]{\begin{spacing}{0.5}#2\end{spacing}}}
\makeatother

%  'draft' header and today's date.  Remove for production
%\usepackage{fancyhdr}
%\pagestyle{fancy}
%\renewcommand{\headrulewidth}{0pt}
%\lhead{}
%\chead{--- DRAFT ---} 
%\rhead{\today}

% page number location
\footskip 0.5in 

% line spacing.  1.95 works better than 2
\setstretch{1.95} 

%spacing above+below a figure/table that has text above and below it
\setlength{\intextsep}{12pt plus 2pt minus 2pt} 

%spacing below/above a figure or table at the top/bottom of a page.
\setlength{\textfloatsep}{12pt plus 2pt minus 2pt}

% change the bibliography title.  Works Cited, Bibliography,  'Stuff I Read'
\renewcommand{\bibname}{References} 

% This is to typeset the bibliography title in the older bibtex packages.  Don't mess with this.
%\renewcommand{\bibsection}{  
%	\ifsmallcapsok
%	\centering {\large \scshape \bibname}
%	\else
%	\centering {\large \MakeUppercase{\bibname}}
%	\fi
%	\addcontentsline{toc}{chapter}{\bibname}}
%\setlength{\bibsep}{0mm plus .5\baselineskip}

%  -----  End margins and spacing  -----  

%  -----  Preferences  -----  

% location of the graphics files.  Recommend a single folder for all images
\graphicspath{ {./Figures/} }

% the file extensions for graphics used in this document
\DeclareGraphicsExtensions{.pdf,.png,.jpg,.jpeg}  
\captionsetup{margin=10pt, font=small, labelfont=bf} % figure and table captions.
%%% end preferences

%%% Custom commands
\providecommand{\e}[1]{\ensuremath{\times 10^{#1}}}  % simplify scientific notation
\DeclareSIUnit\inch{in}  %units not included in package{siunitx}
\DeclareSIUnit\foot{ft}
\DeclareSIUnit\mile{mi}
\DeclareSIUnit\acre{ac}
\DeclareSIUnit\gallon{gal}
\DeclareSIUnit\year{yr}
\DeclareSIUnit\micron{micron}
\def\fese2{$FeSe_2$}  % macro definitions.  I used these to simplify frequently used chemical compounds.
\def\dox{$O_{2}\,$}
\def\elemental{$Se^0$}
\def\nitrate{$NO_3^{2-}$}
\def\phosphate{$PO_4^{3-}$}
\def\selenate{$SeO_4^{2-}$}
\def\selenite{$SeO_3^{2-}$}
\def\sulfate{$SO_4^{2-}$}
\def\Qnps{$ Q_{UNPS} $}
\def\Wstorage{$ \frac{\Delta S}{\Delta t} $}
\def\todoc{\todo{cite}}
\def\todoe{\todo{eq}}
\newcolumntype{L}[1]{>{\raggedright\let\newline\\\arraybackslash\hspace{0pt}}m{#1}} % help define table column widths when required
\newcolumntype{C}[1]{>{\centering\let\newline\\\arraybackslash\hspace{0pt}}m{#1}}
\newcolumntype{R}[1]{>{\raggedleft\let\newline\\\arraybackslash\hspace{0pt}}m{#1}}

%  -----  End Custom Commands  -----  
