% This is the preamble document that contains the author editable preferences for creating a Thesis.  The thesis class document (csuthesis.cls) is required for this to work properly.  Make sure you make a backup and have reference documents before making changes to the class document.  Minor changes can make the final document ususable.


%% Packages %%
% package description in comments
% refer to package documentation for more info
% comment out (don't delete) packages you don't want to use or don't need
\usepackage{pdflscape} % allows for landscape orientation of figures
\usepackage{graphicx} % enhanced support for graphics.  Image scaling, etc.
\usepackage{multirow} % create tabular cells spanning multiple rows (row-wise merge in a column)
%\usepackage{wrapfig} % produces figures which text can flow around.
\usepackage[pagewise, modulo]{lineno} % adds line numbers.  Sometimes helps reviewers.
\usepackage{amssymb} % extends math symbols.  includes registered and copyright symbols.
\usepackage{attachfile} % attach files to document.  Easier to manage multiple chapter, sections, etc.
\usepackage[parfill]{parskip} % helps with spacing.  Need this for Table of Contents.
\usepackage{siunitx} % fairly simple method to add units to values.  Consistent standardized useage
\usepackage{booktabs, longtable} % publication quality tables.  Longtable allows for tables to span multiple pages
\usepackage[toc,page]{appendix} % add appendicies to document.
\usepackage{array} % extends implementation of the array and tabular environments
\usepackage{caption} % customize captions in float environments
\usepackage{subcaption} % caption sub figures in a figure
\usepackage{setspace} % support for setting spacing between lines in a document
\usepackage{afterpage} % allows for starting a command (especially a float) after the next page break.
\usepackage{tabularx}
\usepackage{csquotes} % better smart quotes.  Especially usefull for humanities
% BIBLIOGRAPY PACKAGE AND SETUP
%\usepackage{natbib} % Old bibliography package. only use if you have to
%  Highly recommended that the newer biblatex package is used. 
\usepackage[style=authoryear-comp, url=true, backend=biber]{biblatex}
\addbibresource{BibRef.bib}
%%% End packages

%%% margins and spacing
\usepackage[top=1in,bottom=1in,left=0.5in,right=1.5in]{geometry}  % use if you want more room for rh margin notes.
%\usepackage[top=1in,bottom=1in,left=1in,right=1in]{geometry}  % Use this when you're ready for final(ish) production
\makeatletter
\usepackage[textwidth=1in]{todonotes} % highlighted margin notes for 'todo' items
\renewcommand{\todo}[2][]{\@todo[caption={#2}, #1]{\begin{spacing}{0.5}#2\end{spacing}}} % make todo notes single spaced
\makeatother

%\usepackage{fancyhdr}  %  'draft' header and today's date.  Remove for production
%\pagestyle{fancy} % allow use of fancyhdr
%\renewcommand{\headrulewidth}{0pt}  % defines the width of the line between the header and text.
%\lhead{} % left header contents
%\chead{--- DRAFT ---} % center header contents
%\rhead{\today} % right header contents.  This is the current date.

\footskip 0.5in % page number location
\setstretch{1.95} % line spacing.  1.95 works better than 2
\setlength{\intextsep}{12pt plus 2pt minus 2pt} %spacing above+below a figure/table that has text above and below it
\setlength{\textfloatsep}{12pt plus 2pt minus 2pt} %spacing below/above a figure or table at the top/bottom of a page.
\renewcommand{\bibname}{References} % change the bibliography title.  Works Cited, Bibliography,  'Stuff I Read'
%\renewcommand{\bibsection}{  % Typeset the bibliography title.  Don't mess with this.
%	\ifsmallcapsok
%	\centering {\large \scshape \bibname}
%	\else
%	\centering {\large \MakeUppercase{\bibname}}
%	\fi
%	\addcontentsline{toc}{chapter}{\bibname}}
%\setlength{\bibsep}{0mm plus .5\baselineskip}
%%% end margins and spacing

%%% preferences
\graphicspath{ {./Figures/} }  % location of the graphics files.  Recommend a single folder for all images
\DeclareGraphicsExtensions{.pdf,.png,.jpg,.jpeg}  % the file extensions for graphics used in this document
\captionsetup{margin=10pt, font=small, labelfont=bf} % figure and table captions.
%%% end preferences

%%% Custom commands
\providecommand{\e}[1]{\ensuremath{\times 10^{#1}}}  % simplify scientific notation
\DeclareSIUnit\inch{in}  %units not included in package{siunitx}
\DeclareSIUnit\foot{ft}
\DeclareSIUnit\mile{mi}
\DeclareSIUnit\acre{ac}
\DeclareSIUnit\gallon{gal}
\DeclareSIUnit\year{yr}
\DeclareSIUnit\micron{micron}
\def\fese2{$FeSe_2$}  % macro definitions.  I used these to simplify frequently used chemical compounds.
\def\dox{$O_{2}\,$}
\def\elemental{$Se^0$}
\def\nitrate{$NO_3^{2-}$}
\def\phosphate{$PO_4^{3-}$}
\def\selenate{$SeO_4^{2-}$}
\def\selenite{$SeO_3^{2-}$}
\def\sulfate{$SO_4^{2-}$}
\def\Qnps{$ Q_{NPS} $}
\def\Wstorage{$ \frac{\Delta S}{\Delta t} $}
\def\todoc{\todo{cite}}
\def\todoe{\todo{eq}}
\newcolumntype{L}[1]{>{\raggedright\let\newline\\\arraybackslash\hspace{0pt}}m{#1}} % help define table column widths when required
\newcolumntype{C}[1]{>{\centering\let\newline\\\arraybackslash\hspace{0pt}}m{#1}}
\newcolumntype{R}[1]{>{\raggedleft\let\newline\\\arraybackslash\hspace{0pt}}m{#1}}
%%% End Custom Commands
