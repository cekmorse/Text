%Chapter 6
\renewcommand{\thechapter}{6}
\chapter{Discussion of Results}
\label{c-analysis}
\begin{linenumbers}[1]
The results were first analyzed to determine if there were significant errors in the models.  We initially needed to determine if incorrect assumptions could cause any of the discrepancies noted in the previous chapter.  The assumptions made are summarized as follows:
\begin{enumerate}
\item The geometry and existing condition of the river reaches is relatively unchanged for the study time period.
\item Selenium is a conservative element.
\item Flow from non-gauged minor tributaries and agricultural drains draining directly to the river are considered relatively minor and intermittent, and therefore insignificant.
\item The flow data collected from outside sources are as accurate as described by the provider.
\item The flow data collected from outside sources without described accuracies have poor accuracies.
\item Temperature and EC data collected from outside sources are considered to have good accuracies.
\item Flow depth values reported by the USGS and CDWR are accurate to within $\pm$ 0.01 ft.
\item The flow depth in reach B is the average of the depths of reaches A and C.
\item Temperature and EC data reported from the ARKJMRCO gauge have poor accuracy for the upstream inlet of the DSR.
\item The river cross section geometry for a specific sub-reach can be taken as the average of the geometries of multiple cross sections within that sub-reach.
\item Channel bottom elevations for all surveyed cross sections within a sub-reach can be given the same base elevation without significantly degrading calculations.
\item The river top width within a sub-reach is a power function of the river depth.
\item The river depth as recorded by flow gauges is the same depth along the entire sub-reach.
\item Sub-reach lengths are static.  They have no uncertainty or error associated with their calculation.
\item Unaccounted for flows are entering the river.  Positive results indicate gains to the river and negative values indicate losses from the river.  This assumption is made for all models.
\item All uncertainty and error distributions reported from data collection agencies is considered to be normal.
\item All uncertainty and error distributions calculated in the models have both positive and negative values.
\item All uncertainty and error distributions calculated in the models have extreme values of unknown magnitude.
\item Stochastic model results represent a set of possible realities, but not necessarily reality itself.
\item The input variables, intermediate parameter estimating equations, uncertainty distributions, and error distributions used in the models are the best possible with the given, available data.
\end{enumerate}

These assumptions have varying degrees of validity.  Each of the assumptions will be given a subjective grade of 'good', 'fair', or 'poor'.  
Assumption 1 is 'good'.  The study time period was adjusted to not include any flood periods.  Drought periods were included.  Flood periods have higher flow depths, which would be outside the calculation range for the depth to top width calculations.  During flood events, flow occurs outside of the primary flow channel.  The river length could be significantly shortened during these events.

Assumption 2 is considered 'fair'.  Dissolved selenium is conservative in the sense that it doesn't convert to some other element or become trapped in a compound.  There are soil pore water processes that can cause selenium to be sorbed to soil particles.  There are also process by which selenium can be volatilized directly from the water's surface and through biological pathways.  Volatilization is a valid pathway by which dissolved selenium can leave a water body.  It has not been included in these models because the volatiliztion rate is unknown.  It also unknown if there is a correlation between the volatilization rate and water quality parameters such as temperature, pH, and EC.

Assumption 3 is considered 'good' for the USR and 'fair' for the DSR.  There are a large number of gauged flows in the USR.  The volume of water accounted for in the known flows is much larger than the minor, ungauged flows.  This preponderance of known flow volume is not as pronounced in the DSR.  There are at least two known locations that are not gauged and yet have significant flows returning to the main stem of the river.

Assumption 4 is considered 'good'.  The USGS has multiple studies regarding gauge accuracies.  There is no reason to doubt the accuracy of this data.  Data from the CDWR does not share the same open level of scrutiny as the USGS data.  This does not change the reported accuracy of the provided data.  The CDWR follows procedures set forth by the USGS when maintaining, operating, and calibrating flow gauges.

Assumption 5 is considered 'good'.  Data that does not have a reported accuracy is usually at locations where the gauge is owned, operated, and maintained by a private organization or corporate entity.  It is not in the organization's best interest to spend money to improve the accuracy of their gauges unless required.  The CDWR has oversight on these gauges, but does not have the associated maintenance and calibration data.  They only have the flow data, as required by law.  The owning organization is required to routinely calibrate their gauge, but the calibration frequency is not as often as the frequency used by the CDWR and USGS.  Therefore, the data accuracies from these gauges is considered 'poor'.

Assumption 6 is considered 'good'.  Temperature and EC data is collected by modern, digital instrumentation that is calibrated and maintained on a regular basis by the CDWR and USGS.  The equipment accuracy ranges as reported by the equipment manufacturers are better than those used in this study.  We assume that the calibration will begin to degrade and drift between maintenance and calibration cycles.  Calibration drift has been observed with similar instruments used by CSU during the collection of water quality samples.  We assume the drift experience by CSU staff is much larger due to equipment transport conditions.

Assumption 7 is considered 'fair'.  The USGS has a policy to maintain their gauges to be within this range of uncertainty.  There is no known report determining if this goal has been met at all gauges.  There is also no known reason to doubt the accuracy of the provided data.  The accuracy of their flow rate data is somewhat indicative of the accuracy of the flow depth data.  The degree of correlation between the flow rate accuracy and the flow depth accuracy is unknown.

Assumption 8 is considered 'poor'.  Sub-reach B does not have a flow gauge within its boundaries.  It is a short reach with variable depth due to the existence of an irrigation diversion structure within its boundaries.  This essential gives two sub-reaches; subreach B1 and B2.  The flow depths in both portions is dependent on whether or not the diversion structure is operating and the flow rate being diverted.  There may be times when the diversion structure is operating at full capacity and the flow in the river is being completely diverted for irrigation.  The opposite may also be true at other times.  There is no flow depth data to develop a correlation between the depths in sub-reaches A and/or C.

Assumption 9 is considered 'good'.  Water quality samples were collected at the ARKJMRCO and ARKLAMCO gauge sites to determine the correlation between water quality data recorded at the ARKJMRCO gauge and the dissolved selenium concentration sampled at ARKLAMCO.  Dissolved selenium samples were also taken at ARKJMRCO to aid in this determination.

Assumption 10 is considered 'poor'.  Cross section locations were determined more by accessibility than by whether they were representative of the sub-reach.

Assumption 11 is considered 'poor'.  There is no evidence to prove this assumption.  The assumption is not invalid.  Previous experience with HEC-RAS analysis of natural stream systems was used to make this assumption.  It was observed that stream depth for the majority of the stream reach would be relatively constant unless the flow rate was changed or the stream slope was changed.  It was determined through observation and by analysis of topographic maps that the river channel slope is relatively constant for the majority of the studied river.  The obvious deviations occur at diversion structures where the channel slope is much higher.

Assumption 12 is considered 'good'.  This assumption is only valid for the main channel of a stream.  It is not valid during flood events when the flow breaches the primary channel bank and runs across a flood plain or flood channel.  Model calculations were not performed for days during the study time period when the flow depth exceeded an assumed flood depth of 5 feet.  During cross section surveying, it was observed that the main channel bottom at most locations was approximately 5 feet below the bank.

Assumption 13 is considered 'poor'.  The justification for this rating is the same as discussed for assumption 11.

Assumption 14 is considered 'good'.  Sub-reach lengths were taken as the thalweg length.  The thalweg is the line along the deepest part of the stream or river channel.  It is assumed to be the line where the flow is the fastest.  It is not the shortest nor the longest pathway along a sub-reach.  Thalweg lengths are fairly constant as they either change slowly over the span of many years or quickly with major flood events.  The study time frame is not sufficiently long enough for the thalweg to significantly change through erosion and deposition processes.  There is evidence that the thalweg has changed significantly in recent history.  Aerial imagery shows old river channels from flood events.  There is no major flood event during the study time frame, and therefore the thalweg did not change due to flood events.

Assumption 15 is considered 'good'.  This assumption was necessary to standardize the models.  There is no significant change in results by assuming the opposite is true.  The only change would be that the results would have the opposite sign.  Gains to the river would be negative and losses would be positive.  The assumption used allowed for easier comprehension of results.

Assumption 16 is considered 'good'.  The literature supports this assumption %\citep{USGS2007}.

Assumption 17 is considered 'good'.  This assumption is necessary to reduce the number of possible distribution types for goodness-of-fit testing.  Making this assumption allows for the analysis of only those distributions that allow for both positive and negative values.

Assumption 18 is considered 'good'.  Like assumption 17, this is necessary to reduce the number of possible distribution types for goodness-of-fit testing.  Making this assumption allows for the analysis of only those distributions that are not limited to some discrete minimum and/or maximum values.

Assumption 19 is considered 'good'.  This is a basic premise of Monte Carlo Simulations and is closely related to assumption 20.

Assumption 20 is considered 'fair'.  The accuracy of the input data and model structure is tightly correlated to the actual accuracy of Monte Carlo simulation results.  The models presented in this paper are approximations.  Yet, there are, undoubtedly, more accurate models possible.

The models presented in this paper allow us to come to the following series of conclusions:
\begin{enumerate}
\item The Arkansas R. in the USR is gaining water from unaccounted for sources.%1
\item The Arkansas R. in the USR is losing water to irrigation diversions.%2
\item The Arkansas R. in the USR has a net storage change of approximately zero.%3
\item The Arkansas R. in the DSR is gaining water from unaccounted for sources.%4
\item The Arkansas R. in the DSR is losing water to irrigation diversions.%5
\item The Arkansas R. in the DSR has a nest storage change of approximately zero.%6
\item The Arkansas R. in the USR is gaining selenium from unaccounted for sources.%7
\item The Arkansas R. in the USR is losing selenium to irrigation diversion.%8
\item The Arkansas R. in the USR has a net storage change of approximately zero.%9
\item The Arkansas R. in the DSR is gaining selenium from unaccounted for sources.%10
\item The Arkansas R. in the DSR is losing selenium to irrigation diversions.%11
\item The Arkansas R. in the DSR has a net storage change of approximately zero.%12
\item Average daily river water temperature and EC are significant components to estimating selenium loading%13
\end{enumerate}

Conclusions 1 through 12 are easily supported through direct observation of the results presented in sections~\label{sec:IRStoredWater} and \label{sec:Results} .

The sensitivity analysis results support conclusion 13.  Various flow and ET values are also significant, but this conclusion was expected since flow is a primary component in mass transport calculations and ET has been shown to be related to dissolved selenium concentrations.  This conclusion indicates one of two possible scenarios.  The first is that water temperature has a significant impact on selenium solubility.  While this is true, the degree of significance due to this one scenario is not large.  The second is that water temperature is functioning as a surrogate for time in the calculations.  This scenario is more likely to cause temperature to be a significant variable.

The average selenium concentration returned from unaccounted for sources was calculated for both study regions and compared to the average of the reported lab results for the respective regions.  In both cases, the model reported significantly higher concentrations than the lab results.  The average USR lab result for samples collected in the main channel of the river was approximately 9.3$\mu g \cdot L$ and the average calculated concentration from unaccounted sources was 12.5$\mu g \cdot L$.  This is an increase of 35\%  The average DSR lab result was 11.3$\mu g \cdot L$ and the calculated value was 17.3.  This is an increase of 54\%.

Both of these values are significant and may indicate that there is a need for more even temporal distribution of sampling.  It may also indicate that there is some process that is retaining or diverting selenium from the main channel of the river.  One of these processes could be volitalization within the riparian zone.  If this is the case, then the existing riparian zones are reducing selenium concentrations by a significant value.
\end{linenumbers}