\chapter{Literature Review and Research Overview}
\label{chap:litreview}

\begin{linenumbers}[1]

\section{The Environmental Selenium Problem}
\label{sec:the selenium problem}

Selenium (Se) is a non-metalic element closely related to suflur (S) and tellurium (Te) that was discovered over 200 years ago by the Swedish chemist Berzelius \citep{scott1973}.  It has beneficial uses in the glass industry, as an additive in brass and stainless steel to improve machinability, and in anti-dandruf shampoo.  

Se is an essential micro-nutrient in humans whose essentiality in mammals was discovered in 1957 \citep{schwarz1957}.   It is necessary in some proteins, participates in antioxidant defense, is involved in thyroid hormone metabolism, and others.  Not all human biological processes involving Se have been completely identified or understood \citep{rayman2000,navarro2000,roman2014}.  Clinical trials are being performed based on the evidence that Se might be used as a cancer treatment or preventative or even to counteract the progression of HIV to AIDS \citep{rayman2000,roman2014}.

\subsection*{Se Toxicity.}
The US Department of Agriculture (USDA) and the World Health Organization (WHO) both have recommended daily allowances (RDA) of approximately \SIrange{20}{55}{\micro\gram\per\day} (age and gender dependant) which is the generally accepted minimum needed to prevent the onset of Keshan's disease, a type of conjestive heart disease \citep{world1996,USDA2010}.  Large Se doses are toxic \citep{roman2014,navarro2000} with EPA human health limits set at \SI{4200}{\micro\gram\per\liter} \citep{EPA-Se}.  Less than toxic conentrations are known to cause adverse health effects on the endocrine system, immune system, skin, nails, and hair.  There are scattered places in the word where selenosis, or Se poisoning, affects mammals.  The Hubei Province in China is the only location where natural human selenium toxicity occured \citep{2002Spallholz}.

Se toxicity is due partially to its ability to replace sulfur in many organic and inorganic compounds \citep{Besser1989}.  Se accumulating plants, such as \textit{Astragalus}, prince's plume, and some woody asters, may accumulate Se in concentrations up to \SI{3000}{\milli\gram\per\kilo\gram} of plant mass.  These plants are not palatable by most grazing animals.  The presence of these plants may promote nearby forage plants to accumulate more Se \citep{2006USDA}.  As early as 1934, Se was found to cause deaths and illnesses in livestock that eat high Se \citep{scott1973,Rohwer1931,Besser1989,2006USDA}.  Cows are more resistant to Se toxicity with fatal doses at \SI{\pm11.0}{\milli\gram\per\kilo\gram} of body weight, while the fatal dose for horses is less than \SI{4.4}{\milli\gram\per\kilo\gram} of body weight \citep{Painter1940}.  Se in fodder with doses beween \SIrange{5}{40}{\milli\gram\per\kilo\gram} of fodder weight for long periods, can cause chronic poisoning in cows, also known as alkali disease \citep{2006USDA}.

Se is toxic to aquatic birds, primarily affecting reproduction.  A study of aquatic birds and nests at Kesterson National Wildlife Refuge in California found ebryo death rates at approximately 50\% for some species.  Of the eggs that hatched, many of the chicks suffered from major abnormalities including skeletal and major organ defects.  Selenium analysis of affected eggs found concentrations between \SIrange{2.2}{110}{\milli\gram\per\kilo\gram} dry weight.  Compared to the same species living in an area not affected by Se, birds living on Kesterson had Se concentrations 20 times higher and food organizims had Se concentrations about 12 to 130 times greater \citep{Ohlendorf1986}.  In one controlled study it was found that Se concentrations in feed as low as \SI{10}{\milli\gram\per\kilo\gram} dry weight caused embry malformations to increase by approximately 1\% and concentrations at  \SI{8}{\milli\gram\per\kilo\gram} causing a 68\% increase \citep{2002Spallholz}.

Se toxicity is also found in fish \citep{gillespie1986,Lemly1988}, but the specific level is dependent on the species of fish and habitat.  \citet{Besser1989} found that Selenomethionine, an organic Se compound, is preferentially bioaccumulted compared to selenite (\selenite) or  (\selenate).  This lead to their hypothesis that organic Se compounds may contribute disproportionatly to toxicity in aquatic organisms.  At \SIrange{1}{5}{\micro\gram\per\liter}, Se can bioaccumulate in aquatic food-chains and become a concentrated, toxic, dietary source for fish.  The additional strain of cold winters increases the mortality of Se contaminated fish \citep{Lemly1993}.  

While not a widespread problem, there are areas in the world that have to deal with either Se deficiency or excess.  Australia introduced Se supplements to improve livestock health in the late 1970's and at the same time had some regions where Se toxicity has been reported as the result of livestock feeding on Se accumulative plants.  New Zealand has had to import Se-enriched Australian wheat to combat Se deficiencies in their population \citep{1996Thomson,Tinggi2003}.  In 1969, Finland had to enrich animal feeds with Se and in the early 1980's they introduce high-Se wheat to increase Se uptake in the diet of their population.  China has seen the worst Se poisoning between 1961 and 1964 in the Hubei Province.  These were drought years that forced villagers to eat more vegetables and corn grown in high Se soil and less protien.  During these years, the morbidity in the affected villages was almost 50\% \citep{yang1983}.

Issues with Se defficiency in the U.S. at the commercial level were first noted in turkey flocks in Ohio where there was a high mortality rate in chicks at 5-6 weeks of age \citep{scott1967}.  Lambs and calves in the eastern U.S. suffered from nutritional muscular dystrophy .  Supplementing the diet of the parent animal with Se concentrations as low as \SIrange{0.1}{0.2}{\milli\gram\per\kilo\gram} of total diet prevented these defects .  Even in 1973, the U.S. Food and Drug Administration (F.D.A.) was resistant to requests from livestock producers to add Se to the diets of their livestock due to the claim that Se was a carcinogen \citep{scott1973}.  This claim has since been refuted and since 1974 Se is a required additive to the feed for livestock in the U.S \citep{1999Jensen}.

\subsection*{Se Toxicity in the Environment.}
Se toxicity in the U.S. became a national issue at Kesterson Reservoir in the San Joaquin Valley, California.  While not the first incedence of Se toxicity in the U.S. it was the most publicized.  Here, the inflows changed from primarily fresh water in 1978 to primarily irrigation runoff by 1981.  This highly saline water carried the mobile \selenate which bio-accumulated in the fish and birds \citep{Besser1989}.  In the years after the change in water source, the ecology of the reservoir changed such that the only fish and birds that survived were those that were highly salt-tolerant.  This was evident by frequent mass fish kills, disappearing waterfowl, and disfigured waterfowl \citep{Clifton1989,Saiki1993,Hamilton1999,Lemly2002}.

Various lakes and still water bodies have high Se concentrations.  The Chesapeake Bay and estuary has recorded high Se concentrations due to irrigation runoff.  While their concentrations are low at approximately \SI{0.15}{\micro\gram\per\liter}, the values are higher than the ocean water outside of the estuary \citep{Takayanagi1984}.  The San Francisco Bay and Delta receive approximately 15,000 to 45,000 pounds of Se per year from various sources.  These sources are primarily outflows from the California Central Valley which is one of the largest agricultural centers in the United States \citep{2000luoma}.  Rainwater runoff was found to be a significant source of Se in two watersheds in Maryland \citep{Lawson2001}.  The Great Salt Lake, among it's other issues, has Se concerns with concentrations up to 1.68 \si{\micro\gram\per\liter}.  The Great Salt Lake does remove Se through sedimentation and volitalization at rates greater than 1,900 kg/year \citep{2008naftz,oliver2009}

The Arkansas River is not the only river contaminated with Se.  Belews River in North Carolina was contaminated with Se from a coal fired power plant disposing of ash waste into the river.  Se concentrations in the river are still high 10 years after the source was removed.  This shows that Se sources are persistent and decay slowly \citep{Lemly2002}.   The San Joaquine River in California has Se concentrations at 286 to 869 \si{\milli\gram\per\liter} \citep{Clifton1989}.  Multiple studies have been performed in the San Joaquine River Valley to find efficient methods for removing Se from agricultural waters.  A mass balance study of the Imperial and Brawley constructed wetlands have shown removal rates of 56-70\% with volitalization estimated between 17-50\% \citep{Gersberg2006}.

High Se levels in Colorado are of primary concern in the Lower Colorado River, Fountain Creek, Segments of the Upper Yampa River, North Fork of the Gunnison River, Lower Gunnison River, Upper South Platte River, and Lower Arkansas River.  The largest impacts are seen on the Colorado, Gunnison, and Arkansas Rivers \citep{5CCR1002}.  The Gunnsion River in Colorado has Se concentrations up to \SI{25}{\micro\gram\per\liter} \citep{2008USBR}.  \citet{2005Donnelly} estimated Se returns to the Arkansas River in the LARV at 15.6 \si{\kilo\gram\per\year\per\kilo\meter}.  The sources in the Donnelly study included groundwater and surfacewater returns.

\subsection*{Se Regulation in the United States.}
The U.S. Federal and several state governments have moved to study and reduce the effects of Se toxicity.  Shortly after the change in the Kesterson Reservoir ecology was evident, studies were initiated by the EPA to determine the cause.  Se was considered one of the major toxic factors.  The National Irrigation Water Quality Program (NIWQP) was initiated shortly thereafter to determine the concentrations of potentially toxic constituents, especially Se, in water, bottom sediment, and biota due to irrigation water runoff at multiple sites in the U.S. \citep{Hamilton1999}.  The current EPA aquatic life criteria chronic level for Se was set in 1990 at \SI{5.0}{\micro\gram\per\liter} \citep{EPA-Se}.  The new draft criteria are under consideration for Se that will redefine the criteria for still and moving water, lentic and lotic respectively, not acute and chronic as currently defined.  The draft criteria are proposed to be set at \SI{1.3}{\micro\gram\per\liter} and \SI{4.8}{\micro\gram\per\liter} for still (lentic) and moving (lotic) water, respectively \citep{2014USEPA,Hamilton1999,EPA-Se}.

Colorado and Kansas has set limits on Se concentrations that meet or exceed those currently set by the EPA.  Colorado's state wide limits for Se are \SI{18.4}{\micro\gram\per\liter} and \SI{4.6}{\micro\gram\per\liter} for acute and chronic conditions, respectively \citep{5CCR1002-31}.  Each major river basin in Colorado has an additional regulation modifying the standards for specific stream segments.  The chronic and acute condition standards for the entire Arkansas River are \SI{5}{\micro\gram\per\liter} and \SI{20}{\micro\gram\per\liter}, respectively, which reflects the standards set in 1995 to account for the higher than average groundwater Se concentration.  The USR and DSR have ammended chronic condition standards of \SI{16}{\micro\gram\per\liter} and \SI{19}{\micro\gram\per\liter}, respectively.  These standards were emplaced to reflect existing Se concentrations exceeding the Arkansas River chronic standard of \SI{5}{\micro\gram\per\liter} \citep{5CCR1002-32}.  Colorado and Kansas have identified Se pollution as an issue of concern and have included it on their 303(d) lists which identify which water bodies are contaminated and the level of impact \citep{5CCR1002-93,2014Kansas303d}.

\clearpage{}

\section{Major Selenium Cycle Processes in the Environment.}
\label{sec:major processes}

\subsection*{Selenium Sources.}
As usual with most environmental contaminants, in the eyes of the public, the sources of greatest concern are industrial.  Industrial sources include coal mining and combustion.  Coal-fired power plants are of special concern because they concentrate the Se in the ash.  Power plants that do not collect or or incorrectly dispose of ash run the risk of contaminating the environment with high does of Se \todoc.  In fact, the same technology that was implemented to reduce power plant emissions has increased the volume of Se enriched ash.  Some of this ash is used in the concrete manufacturing industry.  The EPA has reported the release of Se from fly ash concrete is unmeasureable \citep{EPA2014}.  Other sources include oil refinery waste, mining various precious and semi-precious minerals, and agricultural drainage.

Se is naturally occuring, and as such contamination of soils and waterways can be from completely natural causes.  Such was the case with the Se poisoning in China.  The Se came from coal deposits containg Se concentrations up to \SI{80000}{\micro\gram\per\gram} (ppm).  These deposits were weathered and deposited large quantities of Se into the soil.   Conditions were exacerbated by the villagers overuse of lime as a fertilizer \citep{yang1983}.

Agricultural drainage is the Se source that is the least regulated.  Power plants are required to meet requirements and are monitored by various regulatory agencies.  This is because they have control over the concentration of Se from fly ash and its eventual disposal.  Agricultural sources are not regulated by the quantity of Se discharged from fields because, in most cases in the U.S., Se is not applied to the fields.  In these cases, Se is naturally occuring and is released from the parent rock through oxidation reactions.  

\subsection*{Major Environmental Processes.}
With agricultural drainage, there are four major processes involved in the Se cycle: reduction-oxidation reactions, sorption-desorption, biological uptake-decomposition, and volatilization.  All processes are bi-directional, meaning, given the right conditions, a given process is reversable.  While not listed volatilization has a contrary method occuring through Se contaminated rainfall \todoc and may occur through dust deposition \todoc, but these methods are not well known and not documented in the LARV.  Figure \ref{fig:SeRedOx} is a simple depiction of the Se cycle in a groundwater system.

\begin{figure}[htbp]
	\centering
	\includegraphics[scale=.6]{"Figures/SeRedOx"}
	\caption[Oxidation-reduction transformations of Se species in a soil and groundwater system.]{Oxidation-reduction transformations of Se species in a soil and groundwater system  \citep{Bailey2012}.  DMSe, dimethyl-selenide.}
	\label{fig:SeRedOx}
\end{figure}

This diagram become much more complicated once the \nitrate cycle is included as shown in Figure \ref{fig:fateAndTransport}.  This figure does not include volatilization pathways.  \nitrate is primarily added to the system through fertilizers.  The three primary commercial fertilizer compents; nitrogen ($N$), \phosphate, and potash ($K_2O$) are found in various mixtures.  Nitrogen, which includes \nitrate, ammonia, and other nitrogen bearing compounds, promotes plant stem and leaf growth. \phosphate promotes plant root growth.  $K_2O$ promotes overall plant health.  $N$ is also introduced through the decomposition of plant roots and stems after harvest and through the application of manures \citep{Bailey2012}. 

\begin{figure}[htbp]
	\centering
	\includegraphics[scale=1]{"Figures/fateAndTransport"}
	\caption[Conceptual model of the fate and transport of O2, NO3, and SeO4 in an irrigated stream-aquifer system subject to agricultural activities.]{Conceptual model of the fate and transport of O2, NO3, and SeO4 in an irrigated stream-aquifer system subject to agricultural activities (e.g., irrigation and fertilize loading) \citep{Bailey2012}.  Volatilization is not included in this diagram.  The blue line denotes the water table, the green line is the ground surface.}
	\label{fig:fateAndTransport}
\end{figure}

\subsection*{Reduction-Oxidation Reactions}
Reduction-oxidation (redox) reaction are the primary reaction method in groundwater and surface waters.  Dissolved oxygen (\dox) and \nitrate from fertilized fields leach into the groundwater table.   \dox is introduced into surface and groundwaters from the partial pressure of \dox in the atmosphere.  \dox is consumed first in all oxidation reactions and usually does not last long enough to leach to the bedrock.  The \nitrate persists longer in the groundwater and eventually moves, through advection and dispersion processes, to come in contact with the Se bearing parent bedrock.  The \nitrate oxidizes the Se, causing it to transform to the soluble and weekly adsorbing \selenate species.

Table \ref{tab:SeOxidationStates} shows selenium's common oxidation states and forms present in soils.  Other oxidation states and forms exist, but are not common.  Iron selenite is the most common form with very little elemental Se is found in soil and a wide range of organic Se compounds present \citep{Painter1940}.

\begin{table}[htbp]
	\centering	
	\caption[Selenium Oxidation States and Common Forms Present in Soil.]{Selenium Oxidation States and Common Forms Present in Soil.}
	\label{tab:SeOxidationStates}
	\begin{tabular}{ccc}
		\toprule
		Oxidation & \multicolumn{2}{c}{Form}\\\cmidrule{2-3}
		State & Name & Formula \\
		\midrule
		6 & selenate & $SeO_4^{2-}$\\
		4 & selenite & $SeO_3^{2-}$\\
		0 & elemental & $Se$\\
		multiple & organic & multiple\\
		\bottomrule
	\end{tabular}
\end{table}

For Se to move out of the \selenate species, all \dox, \nitrate, and other strong oxidizing agents must be consumed from the water and the pH must be fairly high \todo{check}.  Figure \todo{fig ref} is a Pourbaix diagram of Se which maps out the possible stable phases of Se in an aqueous system.  The bottom scale is the pH range, the left side scale (pe) is the concentration of the standard reducing agent, the electron ($e^-$).  The right side scale (Eh (V)) is the oxidation-reduction potential expressed in units of volts (V).  The diagram shows that elemental Se converts to the \selenate species in low to moderate oxidizing environments along the spectrum of pH values that are common for surface and groundwaters.  For \selenate to convert to the \selenite species, the pH will have to be higher and the oxidation potential fairly low.  These conditions are not common in the grounwaters in the LARV.

\begin{figure}
\missingfigure{Pourbaix diagram}
\end{figure}

Based on a Pourbaix diagram, in any given aqueous system, any number of the species combinations should exist based on chemical kinetics and reaction rates.  The Pourbaix diagaram does not include rate or kinetic limits between species.  \selenate is kinetically limited for reduction to \selenite.  This transformation is mediated by microbial processes \todo{cite Lalvani}.  Some of these microbial processes have been recommended for Se remediation at mining sites \todo{cite MSE tech}.  The magnitude of \selenate reduction via microabial processes in the LARV has not been studied.

\subsection*{Sorption Processes.}
The second primary fertilizer, \phosphate, is a significant contributor to preventing the sorption of \selenite to soil particles.  \phosphate is preferentially adsorbed due to it's affinity for iron and aluminum at lower pH values and calcium at higher pH values, all three of which are present in clays common in the LARV.  \citet{Besser1989} noted that, Se was more rapidly sorbed in fine-textured, highly organic pond sediments than sandy riverine sediments.  Since sandy sediments are primarily silica, which is a very week adsorber, it follows that river sediments are not good adsorbers of either \phosphate or \selenite \citep{Oram2008}.

\subsection*{Biological Uptake of Selenium.}
Biological uptake by plants is a major contributor to Se cycling.  Se taken up as \selenate by plants is stored in the stems and leaves.  There is very little reduction of \selenate to \selenite or organic Se within the plant.  The \selenate is then re-cycled into the system with the decomposition of the leaves that fall every year.  This temporary storage of Se may serve as a buffer to the system.  However, Se taken up as \selenite is not stored.  It is converted to organic selenium which is volatilized through plant transpiration processes.  \citet{Besser1989} noted that organic Se compounds were lost from the water column more rapidly than other Se species.  The presence of suflates in concentrations less than \SI{80}{\milli\gram\per\liter} promotes an increase in Se bioaccumulation and concentrations greater than \SI{180}{\milli\gram\per\liter} decreased Se uptake by organisms.   

\subsection*{Selenium Volitalization}
Volatilization of Se is through two possible pathways: chemical and biological.  Chemical volitilization is very slow, as noted by \citet{Besser1989}, when analysis of their sterile control groups reported no loss of Se. Biological pathways are again the prefered path for reduction of \selenate and \selenite to volatile organic Se species.  A Se volatiliation study performed in California's Imperial Valley constructed wetlands showed that most of the Se was retained in the sediments.  These sediments are the fine-textured, highly organic type noted by \citet{Besser1989}.  Of the remaining Se, less than 1\% was accumulated in plant tissues.  This left 33-50\% of the remaining Se unaccounted for.  \citet{Gersberg2006} reasoned that this unaccounted for Se was lost through volatilization.  Other constructed wetland studies have reported up to 69\%.  The constructed wetlands assessed in this study retained Se bearing water for approximately 18 days before discharging it back to the New River.  The wetlands were planted with bulrush (\textit{Schoenoplectus californicus}), tamarisk (\textit{Tamarix spp.}), and wild grasses \citep{Gersberg2006,johnson2009}.  

The use of tamarisk is of special note with the study in this thesis because this plant, along with Russian olive (\todo{species?}) is an invasive species that has overtaken much of the riparian area along the Arkansas River in Colorado \todoc.  Tamarisk has also been shown to perform as an effect pollutant accumulator \citep{Sorensen2009}. Table \ref{tab:SeSpeciesCharacteristics} is a summary of the common Se species and their characteristics in an aqueous environment.

\begin{table}[htbp]
\centering
\caption[Characteristics of Se Species.]{Characteristics of Se Species.}
\label{tab:SeSpeciesCharacteristics}
\begin{tabular}{ccccccc} 
\toprule
	\multirow{2}{*}{Se Species} & Oxidation & \multirow{2}{*}{Soluble} & \multirow{2}{*}{Adsorption} & Oxidizing & Reducing & \multirow{2}{*}{Toxic} \\
		& State &  & & Conditions & Conditions & \\ 
\midrule
	Selenate & +6 & yes & weak & Present & Absent & yes \\
	Selenite & +4 & yes & strong & Present & Absent & no \\
	Selenium & 0 & no & none & Absent & Present & no \\
	Selenide & -2 & no & none & Absent & Present & yes \\
\bottomrule
\end{tabular}
\end{table}

\clearpage{}
\section{Water Balance Methods for Estimating NPS Return Flows to Streams}
\label{sec:water balance methods}

\clearpage{}
\section{Mass Balance Methods for Estimating NPS Solute Loading to Streams}
\label{sec:mass balance methods}
\begin{itemize}
	\item Summarize \citep{Gersberg2006}-- still water body
	\item Summarize \citep{Takayanagi1984}
	\item Summarize \citep{Mueller2008}
	\item Summarize \citep{Miller2010}
\end{itemize}

This US Bureau of Reclamation Study mass balance study used complex logarithmic equations to estimate Se concentrations from specific conductivity values.  They also found a correlation between stream flow rates and Se concentrations.  Lower concentrations were found during high flow rates and high concentrations were found during low flow rates.  \citep{2008USBR}

A recent study has created a groundwater model to determine the effect of policy changes to groundwater Se concentrations \citep{2012Bailey}.  They found that by reducing nitrate (\nitrate) loads to the aquifer, Se loads to the Arkansas River could be greatly reduced.  Another study found correlations between Se and specific conductivity in the two regions of the LARB discussed in this thesis.  Non-linear estimating equations using power functions were developed for Se concentration in ground and surface water from specific conductivity \citep{2010Cody}.

\clearpage{}
\section{Previous Related Studies in Colorado's Lower Arkansas River Valley}
\label{sec:previous studies}

The studdies performed in the Lower Arkansas River Valley (LARV)

\clearpage{}
\section{Goals and Objectives of this Study}
The overall goal of this study is to estimate the magnitude, variability, and uncertainty of volumetric rates and dissolved Se loads in non-point source (NPS) groundwater return flows to two representative reaches of Colorado's Lower Arkansas River.  The study is conducted at regional scales, i.e. along river reach lenghts of tens of kilometers fed by irrigated alluvial lands with areas of thousands of hectares, and addresses daily average return flow rates and Se loads from these lands.  The aim is to describe current conditions in the irrigated stream-aquifer system and to provide support for the development of models for predicting the prospects for reducing return flow rates and Se loads through the use of alternative land and water best management practices.

The specific objectives of this study are as follows:
\begin{enumerate}
	\item start list here
\end{enumerate}

This study does not address all of the major selenium cycle processes.  Chemical and biologically assisted Se volitalization, Se storage and chemistry within soil pore water, and Se transport with suspended and bed sediments is not covered.  It is our hope that the results of this study may justify studying some or all of these processes within river systems such as the Lower Arkansas River Valley.

\section{--Keep for other chapters--}
Flow depth and width relationship
\begin{itemize}
	\item Summarize \citep{Gates1996}
	\item Summarize \citep{Buhman2002}
\end{itemize}

Sensitivity analysis
\begin{itemize}
	\item Summarize \citep{Saltelli2004}
\end{itemize} 

Monte Carlo simulations and other simulation methods.
\begin{itemize}
	\item \citet{Spanou2001}
\end{itemize}

\end{linenumbers}
\clearpage{}