\chapter{Literature Review and Research Overview}
\label{chap:litreview}

\begin{linenumbers}[1]

\section{The Environmental Selenium Problem}
\label{sec:the selenium problem}

Selenium ($ Se $) is a non-metallic element closely related to sulfur ($ S $) and tellurium ($ Te $) that was discovered over 200 years ago by the Swedish chemist Berzelius \parencite{scott1973}.  It has beneficial uses in the glass industry, as an additive in brass and stainless steel to improve machine-ability, and in anti-dandruff shampoo.  

\Se is an essential micro-nutrient in humans whose essentiality in mammals was discovered in 1957 \parencite{schwarz1957}.   It is necessary in some proteins, participates in antioxidant defense, is involved in thyroid hormone metabolism, and others.  Not all human biological processes involving \Se have been completely identified or understood \parencite{rayman2000,navarro2000,roman2014}.  Clinical trials are being performed based on the evidence that \Se might be used as a cancer treatment or preventative or even to counteract the progression of HIV to AIDS \parencite{rayman2000,roman2014}.

\subsection*{\Se Toxicity.}
The US Department of Agriculture (USDA) and the World Health Organization (WHO) both have recommended daily allowances (RDA) of approximately \SIrange{20}{55}{\micro\g\per\day} (age and gender dependent) which is the generally accepted minimum needed to prevent the onset of Keshan's disease, a type of congestive heart disease \parencite{world1996,USDA2010}.  Large \Se doses are toxic \parencite{roman2014,navarro2000} with EPA human health limits set at \SI{4200}{\micro\g\per\liter} \parencite{EPA-Se}.  Less than toxic concentrations are known to cause adverse health effects on the endocrine system, immune system, skin, nails, and hair.  There are scattered places in the word where selenosis, or \Se poisoning, affects mammals.  The Hubei Province in China is the only location where natural human \Se toxicity occured \parencite{2002Spallholz}.

\Se toxicity is due partially to its ability to replace sulfur in many organic and inorganic compounds \parencite{Besser1989}.  \Se accumulating plants, such as \textit{Astragalus}, prince's plume, and some woody asters, may accumulate \Se in concentrations up to \SI{3000}{\milli\g\per\kilo\g} of plant mass.  These plants are not palatable by most grazing animals.  The presence of these plants may promote nearby forage plants to accumulate more \Se \parencite{2006USDA}.  As early as 1934, \Se was found to cause deaths and illnesses in livestock that eat high \Se \parencite{scott1973,Rohwer1931,Besser1989,2006USDA}.  Cows are more resistant to \Se toxicity with fatal doses at \SI{11.0}{\milli\g\per\kilo\g} of body weight, while the fatal dose for horses is less than \SI{4.4}{\milli\g\per\kilo\g} of body weight \parencite{Painter1940}.  \Se in fodder with doses between \SIrange{5}{40}{\milli\g\per\kilo\g} of fodder weight for long periods, can cause chronic poisoning in cows, also known as alkali disease \parencite{2006USDA}.

\Se is toxic to aquatic birds, primarily affecting reproduction.  A study of aquatic birds and nests at Kesterson National Wildlife Refuge in California found embryo death rates at approximately 50\% for some species.  Of the eggs that hatched, many of the chicks suffered from major abnormalities including skeletal and major organ defects.  Selenium analysis of affected eggs found concentrations between \SIrange{2.2}{110}{\milli\g\per\kilo\g} dry weight.  Compared to the same species living in an area not affected by Se, birds living on Kesterson had \Se concentrations 20 times higher and food organisms had \Se concentrations about 12 to 130 times greater \parencite{Ohlendorf1986}.  In one controlled study it was found that \Se concentrations in feed as low as \SI{8}{\milli\g\per\kilo\g} dry weight caused embryo malformations to increase by approximately 1\% and concentrations at  \SI{10}{\milli\g\per\kilo\g} causing a 68\% increase \parencite{2002Spallholz}.

\Se toxicity is also found in fish \parencite{gillespie1986,Lemly1988}, but the specific level is dependent on the species of fish and habitat.  \textcite{Besser1989} found that Selenomethionine, an organic \Se compound, is preferentially bio-accumulated compared to selenite (\selenite) or  (\selenate).  This lead to their hypothesis that organic \Se compounds may contribute disproportionately to toxicity in aquatic organisms.  At \SIrange{1}{5}{\micro\g\per\liter}, \Se can bio-accumulate in aquatic food-chains and become a concentrated, toxic, dietary source for fish.  The additional strain of cold winters increases the mortality of \Se contaminated fish \parencite{Lemly1993}.  

While not a widespread problem, there are areas in the world that have to deal with either \Se deficiency or excess.  Australia introduced \Se supplements to improve livestock health in the late 1970's and at the same time had some regions where \Se toxicity has been reported as the result of livestock feeding on \Se accumulative plants.  New Zealand has had to import Se-enriched Australian wheat to combat \Se deficiencies in their population \parencite{1996Thomson,Tinggi2003}.  In 1969, Finland had to enrich animal feeds with \Se and in the early 1980's they introduce high-Se wheat to increase \Se uptake in the diet of their population.  China has seen the worst \Se poisoning between 1961 and 1964 in the Hubei Province.  These were drought years that forced villagers to eat more vegetables and corn grown in high \Se soil and less protein.  During these years, the morbidity in the affected villages was almost 50\% \parencite{yang1983}.

Issues with \Se deficiency in the U.S. at the commercial level were first noted in turkey flocks in Ohio where there was a high mortality rate in chicks at 5-6 weeks of age \parencite{scott1967}.  Lambs and calves in the eastern U.S. suffered from nutritional muscular dystrophy.  Supplementing the diet of the parent animal with \Se concentrations as low as \SIrange{0.1}{0.2}{\milli\g\per\kilo\g} of total diet prevented these defects .  Even in 1973, the U.S. Food and Drug Administration (F.D.A.) was resistant to requests from livestock producers to add \Se to the diets of their livestock due to the claim that \Se was a carcinogen \parencite{scott1973}.  This claim has since been refuted and since 1974 \Se is a required additive to the feed for livestock in the U.S \parencite{1999Jensen}.

\subsection*{\Se Toxicity in the Environment.}
\Se toxicity in the U.S. became a national issue at Kesterson Reservoir in the San Joaquin Valley, California.  While not the first incidence of \Se toxicity in the U.S. it was the most publicized.  Here, the inflows changed from primarily fresh water in 1978 to primarily irrigation runoff by 1981.  This highly saline water carried the mobile \selenate which bio-accumulated in the fish and birds \parencite{Besser1989}.  In the years after the change in water source, the ecology of the reservoir changed such that the only fish and birds that survived were those that were highly salt-tolerant.  This was evident by frequent mass fish kills, disappearing waterfowl, and disfigured waterfowl \parencite{Clifton1989,Saiki1993,Hamilton1999,Lemly2002}.

Various lakes and still water bodies have high \Se concentrations.  The Chesapeake Bay and estuary has recorded high \Se concentrations due to irrigation runoff.  While their concentrations are low at approximately \SI{0.15}{\micro\g\per\liter}, the values are higher than the ocean water outside of the estuary \parencite{Takayanagi1984}.  The San Francisco Bay and Delta receive approximately 15,000 to 45,000 pounds of \Se per year from various sources.  These sources are primarily outflows from the California Central Valley which is one of the largest agricultural centers in the United States \parencite{2000luoma}.  Rainwater runoff was found to be a significant source of \Se in two watersheds in Maryland \parencite{Lawson2001}.  The Great Salt Lake, among it's other issues, has \Se concerns with concentrations up to 1.68 \si{\micro\g\per\liter}.  The Great Salt Lake does remove \Se through sedimentation and volatilization at rates greater than 1,900 kg/year \parencite{2008naftz,oliver2009}

The Arkansas River is not the only river contaminated with Se.  Belews River in North Carolina was contaminated with \Se from a coal fired power plant disposing of ash waste into the river.  \Se concentrations in the river are still high 10 years after the source was removed.  This shows that \Se sources are persistent and decay slowly \parencite{Lemly2002}.   The San Joaquine River in California has \Se concentrations at 286 to 869 \si{\milli\g\per\liter} \parencite{Clifton1989}.  Multiple studies have been performed in the San Joaquine River Valley to find efficient methods for removing \Se from agricultural waters.  A mass balance study of the Imperial and Brawley constructed wetlands have shown removal rates of 56-70\% with volatilization estimated between 17-50\% \parencite{Gersberg2006}.

High \Se levels in Colorado are of primary concern in the Lower Colorado River, Fountain Creek, Segments of the Upper Yampa River, North Fork of the Gunnison River, Lower Gunnison River, Upper South Platte River, and Lower Arkansas River.  The largest impacts are seen on the Colorado, Gunnison, and Arkansas Rivers \parencite{5CCR1002-93}.  The Gunnsion River in Colorado has \Se concentrations up to \SI{25}{\micro\g\per\liter} \parencite{2008USBR}.  \textcite{2005Donnelly} estimated \Se returns to the Arkansas River in the LARV at 15.6 \si{\kilo\g\per\year\per\kilo\meter}.  The sources in the Donnelly study included groundwater and surfacewater returns.

\subsection*{\Se Regulation in the United States.}
The U.S. Federal and several state governments have moved to study and reduce the effects of \Se toxicity.  Shortly after the change in the Kesterson Reservoir ecology was evident, studies were initiated by the EPA to determine the cause.  \Se was considered one of the major toxic factors.  The National Irrigation Water Quality Prog (NIWQP) was initiated shortly thereafter to determine the concentrations of potentially toxic constituents, especially Se, in water, bottom sediment, and biota due to irrigation water runoff at multiple sites in the U.S. \parencite{Hamilton1999}.  The current EPA aquatic life criteria chronic level for \Se was set in 1990 at \SI{5.0}{\micro\g\per\liter} \parencite{EPA-Se}.  The new draft criteria are under consideration for \Se that will redefine the criteria for still and moving water, lentic and lotic respectively, not acute and chronic as currently defined.  The draft criteria are proposed to be set at \SI{1.3}{\micro\g\per\liter} and \SI{4.8}{\micro\g\per\liter} for still (lentic) and moving (lotic) water, respectively \parencite{2014USEPA,Hamilton1999,EPA-Se}.

Colorado and Kansas has set limits on \Se concentrations that meet or exceed those currently set by the EPA.  Colorado's state wide limits for \Se are \SI{18.4}{\micro\g\per\liter} and \SI{4.6}{\micro\g\per\liter} for acute and chronic conditions, respectively \parencite{5CCR1002-31}.  Each major river basin in Colorado has an additional regulation modifying the standards for specific stream segments.  The chronic and acute condition standards for the entire Arkansas River are \SI{5}{\micro\g\per\liter} and \SI{20}{\micro\g\per\liter}, respectively, which reflects the standards set in 1995 to account for the higher than average groundwater \Se concentration.  The USR and DSR have amended chronic condition standards of \SI{16}{\micro\g\per\liter} and \SI{19}{\micro\g\per\liter}, respectively.  These standards were emplaced to reflect existing \Se concentrations exceeding the Arkansas River chronic standard of \SI{5}{\micro\g\per\liter} \parencite{5CCR1002-32}.  Colorado and Kansas have identified \Se pollution as an issue of concern and have included it on their 303(d) lists which identify which water bodies are contaminated and the level of impact \parencite{5CCR1002-93,2014Kansas303d}.

\section{Major \textit{Se} Cycle Processes in the Environment.}
\label{sec:major processes}

\subsection*{\Se Sources.}
As usual with most environmental contaminants, in the eyes of the public, the sources of greatest concern are industrial.  Industrial sources include coal mining and combustion.  Coal-fired power plants are of special concern because they concentrate the \Se in the ash.  Power plants that do not collect or or incorrectly dispose of ash run the risk of contaminating the environment with high does of \Se \parencite{Lemly2002}.  In fact, the same technology that was implemented to reduce power plant emissions has increased the volume of \Se enriched ash.  Some of this ash is used in the concrete manufacturing industry.  The EPA has reported the release of \Se from fly ash concrete is immeasurable \parencite{EPA2014}.  Other sources include oil refinery waste, mining various precious and semi-precious minerals, and agricultural drainage.

\Se is naturally occurring, and as such contamination of soils and waterways can be from completely natural causes.  Such was the case with the \Se poisoning in China.  The \Se came from coal deposits containing \Se concentrations up to \SI{80000}{\micro\g\per\g} (ppm).  These deposits were weathered and deposited large quantities of \Se into the soil.   Conditions were exacerbated by the villagers overuse of lime as a fertilizer \parencite{yang1983}.

Agricultural drainage is the \Se source that is the least regulated.  Power plants are required to meet requirements and are monitored by various regulatory agencies.  This is because they have control over the concentration of \Se from fly ash and its eventual disposal.  Agricultural sources are not regulated by the quantity of \Se discharged from fields because, in most cases in the U.S., \Se is not applied to the fields.  In these cases, \Se is naturally occurring and is released from the parent rock through oxidation reactions.  

\subsection*{Major Environmental Processes.}
With agricultural drainage, there are four major processes involved in the \Se cycle: reduction-oxidation reactions, sorption-desorption, biological uptake-decomposition, and volatilization.  All processes are bi-directional, meaning, given the right conditions, a given process is reversible.  Volatilization is reversible through \Se dust deposition or \Se contaminated rainfall, but these pathways have not been studied in the LARV \parencite{Lawson2001}.  Figure \ref{fig:SeRedOx} is a simple depiction of the \Se cycle in a groundwater system.

\begin{figure}[!htbp]
	\centering
	\includegraphics[scale=.6]{"Figures/SeRedOx"}
	\caption[Oxidation-reduction transformations of \Se species in a soil and groundwater system.]{Oxidation-reduction transformations of \Se species in a soil and groundwater system  \parencite{Bailey2012}.  DMSe, dimethyl-selenide.}
	\label{fig:SeRedOx}
\end{figure}

This diagram become much more complicated once the nitrate ($ NO_3^{2-} $) cycle is included as shown in Figure \ref{fig:fateAndTransport}.  This figure does not include volatilization pathways.  \nitrate is primarily added to the system through fertilizers.  The three primary commercial fertilizer compents; nitrogen ($N$), \phosphate, and potash ($K_2O$) are found in various mixtures.  Nitrogen, which includes \nitrate, ammonia, and other nitrogen bearing compounds, promotes plant stem and leaf growth. \phosphate promotes plant root growth.  $K_2O$ promotes overall plant health.  $N$ is also introduced through the decomposition of plant roots and stems after harvest and through the application of manures \parencite{Bailey2012}. 

\begin{figure}[!htbp]
	\centering
	\includegraphics[scale=1]{"Figures/fateAndTransport"}
	\caption[Conceptual model of the fate and transport of O2, NO3, and SeO4 in an irrigated stream-aquifer system subject to agricultural activities.]{Conceptual model of the fate and transport of \dox, \nitrate, and \sulfate in an irrigated stream-aquifer system subject to agricultural activities (e.g., irrigation and fertilize loading) \parencite{Bailey2012}.  Volatilization is not included in this diagram.  The blue line denotes the water table, the green line is the ground surface.}
	\label{fig:fateAndTransport}
\end{figure}

\subsection*{Reduction-Oxidation Reactions}
Reduction-oxidation (redox) reaction are the primary reaction method in groundwater and surface waters.  Dissolved oxygen (\dox) and \nitrate from fertilized fields leach into the groundwater table.   \dox is introduced into surface and groundwaters from the partial pressure of \dox in the atmosphere.  \dox is consumed first in all oxidation reactions and usually does not last long enough to leach to the bedrock.  The \nitrate persists longer in the groundwater and eventually moves, through advection and dispersion processes, to come in contact with the \Se bearing parent bedrock.  The \nitrate oxidizes the Se, causing it to transform to the soluble and weekly adsorbing \selenate species.

Table \ref{tab:SeOxidationStates} shows Se's common oxidation states and forms present in soils.  Other oxidation states and forms exist, but are not common.  Iron selenite is the most common form with very little elemental \Se is found in soil and a wide range of organic \Se compounds present \parencite{Painter1940}.

\begin{table}[!htbp]
	\centering	
	\caption[Selenium Oxidation States and Common Forms Present in Soil.]{Selenium Oxidation States and Common Forms Present in Soil.}
	\label{tab:SeOxidationStates}
	\begin{tabular}{ccc}
		\toprule
		Oxidation & \multicolumn{2}{c}{Form}\\\cmidrule{2-3}
		State & Name & Formula \\
		\midrule
		6 & selenate & $SeO_4^{2-}$\\
		4 & selenite & $SeO_3^{2-}$\\
		0 & elemental & $Se$\\
		multiple & organic & multiple\\
		\bottomrule
	\end{tabular}
\end{table}

For \Se to move out of the \selenate species, all \dox, \nitrate, and other strong oxidizing agents must be consumed from the water and the pH must be fairly high.  Figure \ref{fig:Pourbaix} is a Pourbaix diagram of \Se which maps out the possible stable phases of \Se in an aqueous system.  The bottom scale is the pH range, the left side scale (pe) is the concentration of the standard reducing agent, the electron ($e^-$).  The right side scale (Eh (V)) is the oxidation-reduction potential expressed in units of volts (V).  The diagram shows that elemental \Se converts to the \selenate species in low to moderate oxidizing environments along the spectrum of pH values that are common for surface and groundwaters.  For \selenate to convert to the \selenite species, the pH will have to be higher and the oxidation potential fairly low.  These conditions are not common in the grounwaters in the LARV.

\begin{figure}[!htbp]
%\missingfigure{Pourbaix diagram}
	\centering
	\includegraphics[scale=0.75]{"Figures/SeleniumPourbaix"}
	\caption[Selenium Pourbaix diagram.]{Selenium Pourbaix diagram.}
	\label{fig:Pourbaix}
\end{figure}

Based on a Pourbaix diagram, in any given aqueous system, any number of the species combinations should exist based on chemical kinetics and reaction rates.  The Pourbaix diagram does not include rate or kinetic limits between species.  \selenate is kinetically limited for reduction to \selenite.  This transformation is mediated by microbial processes \parencite{Lalvani2004}.  Some of these microbial processes have been recommended for \Se remediation at mining sites \parencite{MSE2001}.  The magnitude of \selenate reduction via microbial processes in the LARV has not been studied.

\subsection*{Sorption Processes.}
The second primary fertilizer, \phosphate, is a significant contributor to preventing the sorption of \selenite to soil particles.  \phosphate is preferentially adsorbed due to it's affinity for iron and aluminum at lower pH values and calcium at higher pH values, all three of which are present in clays common in the LARV.  \textcite{Besser1989} noted that, \Se was more rapidly sorbed in fine-textured, highly organic pond sediments than sandy riverine sediments.  Since sandy sediments are primarily silica, which is a very week adsorber, it follows that river sediments are not good adsorbers of either \phosphate or \selenite \parencite{Oram2008}.

\subsection*{Biological Uptake of \Se.}
Biological uptake by plants is a major contributor to \Se cycling.  \Se taken up as \selenate by plants is stored in the stems and leaves.  There is very little reduction of \selenate to \selenite or organic \Se within the plant.  The \selenate is then re-cycled into the system with the decomposition of the leaves that fall every year.  This temporary storage of \Se may serve as a buffer to the system.  However, \Se taken up as \selenite is not stored.  It is converted to organic \Se which is volatilized through plant transpiration processes.  \textcite{Besser1989} noted that organic \Se compounds were lost from the water column more rapidly than other \Se species.  The presence of suflates in concentrations less than \SI{80}{\milli\g\per\liter} promotes an increase in \Se bio-accumulation and concentrations greater than \SI{180}{\milli\g\per\liter} decreased \Se uptake by organisms.   

\subsection*{\Se Volitalization}
Volatilization of \Se is through two possible pathways: chemical and biological.  Chemical volatilization is very slow, as noted by \textcite{Besser1989}, when analysis of their sterile control groups reported no loss of Se. Biological pathways are again the preferred path for reduction of \selenate and \selenite to volatile organic \Se species.  A \Se volatilization study performed in California's Imperial Valley constructed wetlands showed that most of the \Se was retained in the sediments.  These sediments are the fine-textured, highly organic type noted by \textcite{Besser1989}.  Of the remaining Se, less than 1\% was accumulated in plant tissues.  This left 33-50\% of the remaining \Se unaccounted for.  \textcite{Gersberg2006} reasoned that this unaccounted for \Se was lost through volatilization.  Other constructed wetland studies have reported up to 69\%.  The constructed wetlands assessed in this study retained \Se bearing water for approximately 18 days before discharging it back to the New River.  The wetlands were planted with bulrush (\textit{Schoenoplectus californicus}), tamarisk (\textit{Tamarix spp.}), and wild grasses \parencite{Gersberg2006,johnson2009}.  

The use of tamarisk is of special note with the study in this thesis because this plant, along with Russian olive (\textit{Elaeagnus angustifolia}) is an invasive species that has overtaken much of the riparian area along the Arkansas River in Colorado \parencite{Nagler2010a}.  Tamarisk has also been shown to perform as an effect pollutant accumulator \parencite{Sorensen2009}. Table \ref{tab:SeSpeciesCharacteristics} is a summary of the common \Se species and their characteristics in an aqueous environment.

\begin{table}[!htbp]
\centering
\caption[Characteristics of \Se Species.]{Characteristics of \Se Species.}
\label{tab:SeSpeciesCharacteristics}
\begin{tabular}{ccccccc} 
\toprule
	\multirow{2}{*}{\Se Species} & Oxidation & \multirow{2}{*}{Soluble} & \multirow{2}{*}{Adsorption} & Oxidizing & Reducing & \multirow{2}{*}{Toxic} \\
		& State &  & & Conditions & Conditions & \\ 
\midrule
	Selenate & +6 & yes & weak & Present & Absent & yes \\
	Selenite & +4 & yes & strong & Present & Absent & no \\
	Selenium & 0 & no & none & Absent & Present & no \\
	Selenide & -2 & no & none & Absent & Present & yes \\
\bottomrule
\end{tabular}
\end{table}


\section{Water Balance Methods for Estimating NPS Return Flows to Streams}
\label{sec:water balance methods}
There are two basic water balance model types: regional and general.  In the US, regional models consist primarily of the Thornthwaite-Mather, Palmer, and Thomas abcd models.  The Thornthwaite-Mather model accounts for a regional water balance on a monthly accounting procedure using the mean monthly temperature, monthly total precipitation, and the latitude of the region \parencite{Thornthwaite1955}.  This model takes into account snowfall and soil storage within the region of interest.  The Palmer and Thomas abcd models are enhancements to the Thornthwaite-Mather model \parencite{Palmer1965,Thomas1983}.  These methods are not valid for the water balance models generated in this thesis which is only concerned with the water contained in the river channel.

General water balance models use the general water balance equation (Equation \ref{eq:genWater}) as a starting point and attempt to account for all gains and losses to the system.  The general model is well suited to systems where a large portion of the gains and losses are measurable or estimable.  It is also applicable to use in studying soil, groundwater, atmospheric, and other systems.  The model is not limited to natural systems as it is used in municipal and industrial mechanical system water balances.

\begin{equation}
	\label{eq:genWater}
	\Delta S = \sum gains - \sum losses
\end{equation}

\section{Mass Balance Methods for Estimating NPS Solute Loading to Streams}
\label{sec:mass balance methods}
The general form of the mass balance model is based on the water balance model with the assumption that the mass is conservative.  That is, the mass being modeled is not consumed or generated within the system boundary.  The chemical mass balance approach has been used extensively to study the in-stream reactions and sediment dynamics of multiple natural and industrial materials \parencite{Plummer1980,Christophersen1981,Elder1985,Jain1996,Latimer1988,Yuretich1988}. \textcite{Jain1996} used monitoring points along a \SI{25}{\kilo\m} stretch of river to determine the effects of multiple industrial sites discharging dissolved and suspended metal into the river.  These industries discharged into the main channel via direct addition to the river channel, addition to tributaries.  Agricultural discharge was identified as the major non-point contributor.

\textcite{mcmahon1997} calculated the mass balance of nitrogen and phosphorus in eight sub-basins to determine the importance of agricultural non-point sources to nutrient loading in a drainage basin encompassing large portions of the states of Virginia and North Carolina.  They found that the highest in-stream loads were measured in agricultural drainages with point loads contributing approximately 3\% of the total load. 

\textcite{Gersberg2006} used mass balance methods to study the unloading of \Se from the Imperial Constructed Wetlands Demonstration Project in Southern California.  This project is being used to determine the effectiveness of wetlands in remediating \Se polluted waters from agricultural runoff.  Two sub-sites were studied and removal rates of 56\% and 70\% were calculated based on a mass balance model which included monitoring the influx and out-flux of the sites and measuring the soil concentration over time.  They found that between 33\% and 50\% of the \Se was lost to volatilization. 


\section{Previous Related Studies in Colorado's Lower Arkansas River Valley}
\label{sec:previous studies}

The first significant \Se study performed by Colorado State University determined that there was a multi-variate linear relationship between in-stream dissolved \Se concentration and concentrations of \sulfate and \nitrate.  This study also verified that \Se was a significant contaminant in the Downstream Study Region (DSR) of the Lower Arkansas River Valley (LARV) with in-stream concentrations between \SIrange{1.6}{43.2}{\mgl} (median concentration of \SI{11}{\mgl}) and alluvial groundwater concentrations between \SIrange{<0.4}{166}{\mgl} (median concentration of \SI{11}{\mgl}).  They also determined that the \Se was most likely originating from shale derived soils with groundwater concentrations between \SIrange{<0.4}{3760}{\mgl}.  They performed a mass balance analysis over a one-year period and determined that approximately \SI{15.6}{\kilo\g\per\kilo\m\per\year} of \Se returned to the Arkansas River.  They estimated that \SI{1086}{\kilo\g\per\year} of \Se was discharged from the river into irrigation canals and \SI{959}{\kilo\g\per\year} of \Se was returned to the river \parencite{donnelly2005}.

In a following study, \textcite{herting2006} verified the in-stream and groundwater \Se concentrations.  They also took dissolved uranium ($ U $) samples and found a linear relationship between dissolve \Se and dissolved $ U $ in the groundwater which led them to conclude that the dissolved \Se and U originated from marine shales.  From these results, they were able to generate a map of the region that identified the location of the $ U $ and \Se rich shales in the DSR.

\textcite{Mueller2008} found that changes in stored dissolved \Se in the river channel were "a major contributing factor to the calculation of NPS loads".  They calculated the coefficient of variation (CV) of the DSR \Se loads at 0.23.  The CV was used to describe the range of uncertainty associated with the models.  They developed a stochastic model for the DSR Se mass loading and used Monte Carlo simulation techniques.  They found the stochastic mean \Se loading to the main stem of the DSR was \SI{0.028}{\kilo\g\per\kilo\m\per\day}, compared to the deterministic model mean of \SI{0.038}{\kilo\g\per\kilo\m\per\day}.

\textcite{gates2009} expanded the \Se groundwater study to include the Upstream Study Region (USR) in the LARV and found concentrations averaging about \SI{57.7}{\mgl} for the USR and \SI{33.0}{\mgl} for the DSR.  They also identified relationships between the locations of \Se concentrations and shale outcroppings, between dissolved \Se and dissolved solids in the groundwater, between dissolved \Se and $ U $, and between dissolved \Se and \nitrate.  Of particular note was their finding of the degree to which dissolved \Se depends on oxidation and inhibited reduction which indicated prospects for reducing dissolved \Se through 

In a separate study, \textcite{Miller2010} analyzed dissolved solids, dissolved \Se, and dissolved $ U $ concentrations in surface water along the main stem of the Arkansas River from the headwaters near Leadville, Colorado, to the USGS gauge in Coolidge, Kansas, which is near the Colorado-Kansas state line.  They found that the in-stream \Se concentration increase occurred upstream of Pueblo Reservoir.  The variability in dissolved \Se concentrations did not increase significantly between Avondale and Las Animas.  They did find the highest instantaneous dissolved \Se loads in the reach downstream of Fountain Creek and to Avondale.  Instantaneous loads decreased from Avondale to the Catlin Canal diversion dam and then remained fairly constant to Coolidge, Kansas.

In his Ph.D. thesis, \textcite{2010Cody} found groundwater Se concentrations at \SI{59.9}{\mgl} and \SI{33.2}{\mgl} in the USR and DSR, respectively.  He found that Se is "Strongly and significantly correlated with" many of the dissolved constituents in the groundwater.  He also confirmed the correlation between the distance from \Se and $ U $ bearing marine shales to the sample wells.  He found that \nitrate is a significant contributor to \Se dissolusion from the marine shales through oxidation processes.

\textcite{Bailey2012} created a groundwater model to determine the effect of policy changes to groundwater \Se concentrations .  They found that by reducing \nitrate loads to the aquifer, \Se loads to the Arkansas River could be greatly reduced.  Another study found correlations between \Se and specific conductivity in the two regions of the LARV discussed in this thesis.  Non-linear estimating equations using power functions were developed for \Se concentration in ground and surface water from specific conductivity \parencite{2010Cody}.

\section{Goals and Objectives of this Study}
The overall goal of this study is to estimate the magnitude, variability, and uncertainty of volumetric rates and dissolved \Se loads in non-point source (NPS) groundwater return flows to two representative reaches of Colorado's Lower Arkansas River.  The study is conducted at regional scales, i.e. along river reach lengths of tens of kilometers fed by irrigated alluvial lands with areas of thousands of hectares, and addresses daily average return flow rates and \Se loads from these lands.  The aim is to describe current conditions in the irrigated stream-aquifer system and to provide support for the development of models for predicting the prospects for reducing return flow rates and \Se loads through the use of alternative land and water best management practices.  The specific objectives of this study are as follows:

\begin{enumerate}
	\item Develop and use a deterministic model to estimate the return water flow from unaccounted for non-point sources to the main channel of the Arkansas River.
	\item Develop and use a deterministic model to estimate the return \Se mass loading to the main channel of the Arkansas River.
	\item Develop and use a stochastic model to estimate the return flow and the uncertainty of the return flow from unaccounted for non-point sources to the Arkansas River using Monte Carlo simulation methods
	\item Develop and use a stochastic model to estimate the return \Se mass loading and the uncertainty of the return \Se mass loading from unaccounted for non-point sources to the Arkansas River using Monte Carlo simulation methods
	\item Determine the sensitivity of the models to the various input variables.
\end{enumerate}

This study does not address all of the major \Se cycle processes.  Chemical and biologically assisted \Se volatilization, \Se storage and chemistry within soil pore water, and \Se transport with suspended and bed sediments is not covered.  It is our hope that the results of this study may justify studying some or all of these processes within river systems such as the Lower Arkansas River Valley.

\end{linenumbers}
\clearpage{}