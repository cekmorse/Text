
%\section{In-Stream Data}
%\label{sec:InStreamData}
%%The CDWR and USGS maintain a data bases to warehouse all of the stream gauge data gathered in the State of Colorado.  These data bases are accessible through a public web site and contain all of the recorded preliminary and validated stream gauge measurements.  The gauges on the main stem of the river are owned and operated by either the CDWR or the USGS.  Tables \ref{tab:USRlocinfo} and \ref{tab:DSRlocinfo} give the basic descriptions of the gauges in the USR and DSR, respectively.
%
%%The water quality equipment at CDWR gauges is satellite monitored by the CDWR, but only provisional data is available on their web-site.  Hydrographers in the CDWR Division 2 offices were contacted to obtain validated data, which they were able to provide.  Water quality equipment at USGS gauges is satellite monitored by the USGS.  Corrected data is available on their web-site.  All gauge sites in the main stem of the river are owned and operated by the CDWR or the USGS and are maintained and validated on a regular basis.  Gauge sites located in irrigation ditches are owned and operated by the ditch owners under the supervision of the State Engineer's Office \citep{CRS37-84}.  Since 2006, data from CDWR gauges has been reported on a near real-time basis via web interface.
%
%%The accuracy of streamflow and water quality data from the CDWR is dependent on the gauge.  If the gauge is one of their primary gauges, then the accuracy is reported in a series of annual water reports.  Streamflow data from gauges not included in the report is of lower quality.  The CDWR's primary gauge system is equivalent in quality and accuracy as the USGS's National Streamflow Information Program gauge sites \citep{USGS2007}.
%
%%Due to the cooperation between the USGS and CDWR, all gauges and water quality stations in the USR and DSR use the same units.  Table~\ref{tab:parameters} gives the available parameters and in what unit of measure the data is provided.
%%
%%\begin{table}[htbp]
%%  \centering
%%  \caption[Data parameters collected in the Lower Arkansas River Basin.]{Data parameters collected in the Lower Arkansas River Basin.}
%%    \begin{tabular}{cc}
%%    \toprule
%%    Parameter and symbol & Provided unit of measure \\
%%    &	and abbreviation \\
%%    \midrule
%%    \midrule
%%    Discharge (Q) & Cubic feet per second (cfs) \\
%%    \midrule
%%    Water temperature (T) & Degrees Celcius (C) \\
%%    \midrule
%%    Specific conductance (EC), & Decisiemens per \\
%%    temperature corrected to \ang{25}C & meter ($dS \cdot m^{-1}$) \\
%%    \midrule
%%    Stream depth (h) & Feet (ft) \\
%%    \bottomrule
%%    \end{tabular}
%%  \label{tab:parameters}
%%\end{table}
%
%%All discharge values discussed in this study will be average daily flow expressed in units of cubic meter per second (\si\{\cubic\meter\per\second}).  This is calculated by the data provider as the mean of the flow values recorded every 15 minutes through one calendar day.  Average daily water temperature is either calculated by the data provider as the average daily temperature calculated in the same manner as average daily flow, or calculated as the mean of the reported maximum and minimum water temperature in a calendar day.  Air temperature, although recorded at one gauge station, is not used in any calculation in this study.  Average daily electrical conductivity is calculated by the data provider in the same manner as average daily flow.  For the entirety of this paper, electrical conductivity is given as specific conductance (standardized to \ang{25} C) and will be referred to as EC.  No equipment, measurements or methods used in this study use any other standard or units for water conductivity.  Average daily stream depth is recorded at all stream gauge stations.  Each gauge reports depth from an arbitrary vertical datum.  These datums are tied to a nearby survey benchmarks of varying stability using normal surveying techniques and practices.  Not all of these datums are available to the public.  Therefore, stream depth is not corrected to true elevation.  Further discussion of the treatment of stream depth to calculate river volume is in section \ref{sec:RiverGeometry}.
%%
%%A few locations bear special discussion. The first is the upstream gauge site in the USR on the Arkansas R.  The discharge gauge at this site is owned and operated by the CDWR and is given the symbol ARKCATCO.  Before the October 1, 2007, which is the beginning of the 2008 water year, the USGS used data from this site combined with measurements from the gauge on the nearby Catlin Canal (CDWR's CATCANCO) to give the flow immediately upstream of the dam.  The USGS refers to this as site 07119700.  The combined flow is of no value in this study and only the average daily discharge reported for the CDWR site ARKCATCO will be used.
%%
%%The second is the site RFDRETCO.  This gauge is measuring flow returning to the Arkansas R. from the Rocky Ford Canal.  Returns are measured as part of a trans-basin agreement.  The details of this agreement are not known nor pertinent to this study.  Average daily flows from this gauge are used in this study to offset flows diverted into the Rocky Ford Canal as recorded by the RFDMANCO gauge.
%%
%%ARKROCCO is situated on the main stem of the Arkansas R.  This site only bears discussion because additional equipment has been added since the end of the data collection phase of this study.  This site is now capable of recording EC and water temperature.  Future data from this site will improve the accuracy of selenium mass transport model for the Upstream Study Region.  Only the average daily flow depth from ARKROCCO will be used in this study.
%%
%%ARKLAMCO is situated at the upstream end of the DSR.  This gauge does not have water quality instrumentation.  The closest gauge with water quality instrumentation is ARKJMRCO.  This gauge is situated at the outlet works of the John Martin Reservoir.  Both gauges are owned and operated by the USGS.  For this study, average daily flows are taken from ARKLAMCO and average daily EC and water temperature are taken from ARKJMRCO.  The location of the gauges in relation to the DSR boundary is not ideal.  This study would be better served to have water quality data collected at ARKLAMCO.  Obviously, the results would be improved with the added spatial data correlation if ARKLAMCO collected water quality data.  Since water quality data is collected at a significant distance from the DSR's upstream gauge, we anticipate an immeasurable error to be added to the estimated DSR results.  
%%
%%There is not a flow gauge situated at the downstream boundary of the DSR.  The flows from two gauges downstream of the boundary must be additively combined to estimate flow at the boundary.  These two gauges are ARKCOOKS and FRODITKS.  Both of these gauges are located in Kansas a short distance downstream from the Colorado border.  ARKCOOKS is located on the Arkansas R. near Coolidge, Kansas.  FRODITKS is located on the Frontier Ditch, in Kansas, less than a half mile from the Colorado border.   The gauge is located in Kansas, but the diversion is in Colorado, less than a quarter mile from the Kansas border.  The distance between the downstream border of the DSR and these gauges ranges approximately from 0.25 to 3.4 km.  These distances are significant and we expect additional immeasurable errors to be added to the DSR selenium mass transport model.  Water quality data is recorded at ARKCOOKS and all data is collected from the USGS web-site.
%%
%%With the continuous data, results were entered into the data base with as little human interaction as possible.  Hand entering the data or editing data points could skew results and exclude the ability for others to accurately reproduce the methods used in this study.  The data undoubtedly has errors included and great effort was made to find remove the errors, but finding and correcting all of them in an efficient, repeatable and defendable manner was not available at the time.
%%
%%Tables \ref{tab:USRvars} and \ref{tab:DSRvars} show the symbols used for each of the input variables used in the USR and DSR models, respectively.  Unless otherwise noted, all variables report the average for that parameter on the calculation date.  This table also reports the data quality used in the initial parameter estimation for the stochastic model.  Quality descriptions are included in Table~\ref{fig:USGSerror2}.
%%
%%\begin{table}[htbp]
%%  \centering
%%  \caption[List of variables used in the Upstream Study Region (USR).]{List of variables used Upstream Study Region (USR) in the water balance and selenium mass balance calculations.}
%%  \label{tab:USRvars}
%%    \begin{tabular}{lccc}
%%    \toprule
%%    \multirow{2}[0]{*}{Variable} & Model & \multirow{2}{*}{Units} & Data \\
%%    & Symbol & & Quality\\
%%    \midrule
%%    \midrule
%%    Flow at ARKCATCO & $Q_{ARKCATOC}$  & $m^3 \cdot s^{-1}$ & Good$^{23}$ \\
%%    Flow at HOLCANCO & $Q_{HOLCANCO}$  & $m^3 \cdot s^{-1}$ & Poor$^4$ \\
%%    Flow at RFDMANCO & $Q_{RFDMANCO}$  & $m^3 \cdot s^{-1}$ & Poor$^4$ \\
%%    Flow at FLSCANCO & $Q_{FLSCANCO}$  & $m^3 \cdot s^{-1}$ & Poor$^4$ \\
%%    Flow at RFDRETCO & $Q_{RFDRETCO}$  & $m^3 \cdot s^{-1}$ & Poor$^4$ \\
%%    Flow at TIMSWICO & $Q_{TIMSWICO}$  & $m^3 \cdot s^{-1}$ & Fair$^2$ \\
%%    Flow at FLYCANCO & $Q_{FLYCANCO}$  & $m^3 \cdot s^{-1}$ & Poor$^4$ \\
%%    Flow at CANSWKCO & $Q_{CANSWKCO}$  & $m^3 \cdot s^{-1}$ & Fair$^3$ \\
%%    Flow at CONDITCO & $Q_{CONDITCO}$  & $m^3 \cdot s^{-1}$ & Poor$^4$ \\
%%    Flow at HRC194CO & $Q_{HRC194CO}$  & $m^3 \cdot s^{-1}$ & Good$^3$ \\
%%    Flow at ARKLASCO & $Q_{ARKLASCO}$  & $m^3 \cdot s^{-1}$ & Good$^2$ \\
%%    EC at ARKCATCO & $EC_{ARKCATCO}$  & $dS \cdot m^{-1}$ & Good$^2$ \\
%%    Water temperature at ARKCATCO & $T_{ARKCATCO}$  & \SI{}{\degreeCelsius} & Good$^2$ \\
%%    EC at ARKLASCO & $EC_{ARKLASCO}$  & $dS \cdot m^{-1}$ & Good$^2$ \\
%%    Water temperature at ARKLASCO & $T_{ARKLASCO}$  & \SI{}{\degreeCelsius} & Good$^2$ \\
%%    Stream depth at ARKCATCO & $h_{A,0}$  & $m$ & Good$^5$ \\
%%    Stream depth in reach B	& $h_{B,0}$  & $m$ & --$^6$ \\
%%    Stream depth at ARKROCCO & $h_{C,0}$  & $m$ & Good$^5$ \\
%%    Stream depth at ARKLAJCO & $h_{D,0}$  & $m$ & Good$^5$ \\
%%    Stream depth at ARKLASCO & $h_{E,0}$  & $m$ & Good$^5$ \\
%%    Stream depth at ARKCATCO$^1$ & $h_{A,1}$  & $m$ & Good$^5$ \\
%%    Stream depth in reach B$^1$ & $h_{B,1}$  & $m$ & --$^6$ \\
%%    Stream depth at ARKROCCO$^1$ & $h_{C,1}$  & $m$ & Good$^5$ \\
%%    Stream depth at ARKLAJCO$^1$ & $h_{D,1}$  & $m$ & Good$^5$ \\
%%    Stream depth at ARKLASCO$^1$ & $h_{E,1}$  & $m$ & Good$^5$ \\
%%    Distance from a point in the main  & \multicolumn{1}{c}{\multirow{2}[0]{*}{di}} & \multirow{2}[0]{*}{$km$} & \multirow{2}[0]{*}{NA} \\
%%    stem to the upstream boundary & & & \\
%%    \bottomrule
%%    \multicolumn{4}{l}{\footnotesize $^1$ Time shifted one day prior.}\\
%%    \multicolumn{4}{l}{\footnotesize $^2$ Data quality as reported by the USGS.}\\
%%    \multicolumn{4}{l}{\footnotesize $^3$ Data quality as reported by the CDWR.}\\
%%    \multicolumn{4}{l}{\footnotesize $^4$ Assumed. Data quality not reported by the USGS or CDWR.}\\
%%    \multicolumn{4}{l}{\footnotesize $^5$ Assumed. Flow depth data quality not reported.}\\
%%    \multicolumn{4}{l}{\footnotesize $^6$ Quality uncertainty applied before calculating.  No further baseline uncertainty applied.}\\
%%    \end{tabular}
%%\end{table}
%%
%%\begin{table}[htbp]
%%  \centering
%%  \caption[List of variables used in the Downstream Study Region (DSR).]{List of variables used Downstream Study Region (DSR) in the water balance and selenium mass balance calculations.}
%%  \label{tab:DSRvars}
%%    \begin{tabular}{lccc}
%%    \toprule
%%    \multirow{2}{*}{Variable} & Model & \multirow{2}{*}{Units} & Data\\
%%    &Symbol&&Quality\\
%%    \midrule
%%    \midrule
%%    Flow at ARKLAMCO & $Q_{ARKLAMCO}$ & $dS \cdot m^{-1}$ & Good$^2$ \\
%%    Flow at BIGLAMCO & $Q_{BIGLAMCO}$ & $dS \cdot m^{-1}$ & Poor$^2$\\
%%    Flow at BUFDITCO & $Q_{BUFDITCO}$ & $dS \cdot m^{-1}$ & Poor$^4$\\
%%    Flow at WILDHOCO & $Q_{WILDHOCO}$ & $dS \cdot m^{-1}$ & Fair$^2$\\
%%    Flow at FRODITKS & $Q_{FRODITKS}$ & $dS \cdot m^{-1}$ & Poor$^4$\\
%%    Flow at ARKCOOKS & $Q_{ARKCOOKS}$ & $dS \cdot m^{-1}$ & Fair$^2$\\
%%    EC at ARKJMRCO & $EC_{ARKJMRCO}$ & $dS \cdot m^{-1}$ & Poor$^7*$\\
%%    Water temperature at ARKJMRCO & $T_{ARKJMRCO}$ & \SI{}{\degreeCelsius} & Poor$^7$\\
%%    EC at ARKCOOKS & $EC_{ARKCOOKS}$ & $dS \cdot m^{-1}$ & Good$^2$\\
%%    Water temperature at ARKCOOKS & $T_{ARKCOOKS}$ & \SI{}{\degreeCelsius} & Good$^2$\\
%%    Stream depth at ARKLAMCO & $d_{F,0}$ & $m$ & Good$^5$\\
%%    Stream depth at ARKGRACO & $d_{G,0}$ & $m$ & Good$^5$\\
%%    Stream depth at ARKLAMCO$^1$ & $d_{F,1}$ & $m$ & Good$^5$\\
%%    Stream depth at ARKGRACO$^1$ & $d_{F,1}$ & $m$ & Good$^5$\\
%%    Distance from a point in the main  & \multicolumn{1}{c}{\multirow{2}[0]{*}{di}} & \multirow{2}[0]{*}{$km$} & \multirow{2}[0]{*}{NA}\\
%%    stem to the upstream boundary &  &  \\
%%    \bottomrule
%%    \multicolumn{4}{l}{\footnotesize $^1$ Time shifted one day prior.}\\
%%    \multicolumn{4}{l}{\footnotesize $^2$ Data quality reported by the USGS.}\\
%%    \multicolumn{4}{l}{\footnotesize $^4$ Assumed. Data quality not reported by the USGS or CDWR.}\\
%%    \multicolumn{4}{l}{\footnotesize $^5$ Assumed. Flow depth data quality not reported.}\\
%%    \multicolumn{4}{l}{\footnotesize $^7$ Assumed.  Data quality reported by the USGS, but not used.}\\
%%    \end{tabular}
%%\end{table}
%%
%%Water temperature and EC data recorded at ARKJMRCO are the only input variables where the model does not use the published data quality qualifier.  This gauge is approximately 10 miles upstream of the ARKLAMCO gauge.  The distance between the gauges adds an additional level of uncertainty to the recorded data.  Data from ARKJMRCO is collected and maintained by the USGS.  Their annual reports indicate that the data quality is good for EC and water temperature data recorded at this station.  In spite of this, and due to the distance between the gauges, we will consider the EC and water temperature data recorded at this station to be of poor quality.  
%%
%%There are no major tributaries, drains, or canal returns between the ARKJMRCO and ARKLAMCO gauge sites.  Any changes in the water quality between these two gauges is mostly due to interaction with the riparian aquifer.  Some data was collected at the ARKJMRCO data in an attempt to find a relationship between the EC and water temperature at the two sites.  This effort was unsuccessful.  Water at ARKJMRCO is in a very low flow rate for most of the year.  We could not determine if the EC and water temperature were, in fact, due to water being released by John Martin Reservoir Dam or if groundwater components were a significant factor.  We speculate that there are significant groundwater interactions, but the extent of these interactions is unknown.
%
%Flow depth values, also known as stream stage, were obtained from USGS an CDWR representatives.  Flow depth is a directly measured value which is used to calculate stream flow rates by both agencies.  The CDWR publishes the raw data on its web-site as collected by its stream gauges.  This data has not been checked for errors.  The USGS does not publish the raw or sanitized data.  Representatives from both agencies were able to provide the sanitized flow depth data for all gauges and and for the required time frame.  Further discussion and complete analysis of the flow depth values are in Section \ref{sec:RiverGeometry}.
%
%Data was provided in non-consistent unit systems.  Both the U.S. Customary and S.I. systems were used in the raw data.  All data was converted to appropriate S.I. units before any calculations or analysis was performed.
%
%Gauge measurement errors are characterized by the assigned data quality.  The data qualities in Table~\ref{fig:USGSerror2} indicate the range where 95\% of the error distribution exists.  Distributions require specific, non-interchangeable, parameters as part of their definition.  Data quality error, as defined by the described ranges, is normally distributed unless otherwise specified.  Since no other distribution type has been mentioned in the found literature, it is assumed that the error associated with data quality is normally distributed.
%
%Normal distributions are characterized by the parameters mean $(\mu)$ and standard deviation $(\sigma)$.  [R] draws values from normal distributions using these two parameters, not by using the ranges described by the data quality.  The reported value is the expected value $(E)$ and the mean of the distribution.  These data quality error ranges corresponds to 1.96 times the standard deviation $(1.96\sigma)$ of the data as derived from two-tailed z-scores.  The standard deviation describes the spread of the distribution from the mean value.  This gives the derivation of the distribution's standard deviation when the distribution is characterized by a percentage of the reported value as shown in Equation \ref{eq:percentsigma}.  When the distribution is characterized by a discrete value, then Equation \ref{eq:discretesigma} is used.
%
%\begin{equation}
%	\sigma=\frac{p\cdot E}{1.96}
%	\label{eq:percentsigma}
%\end{equation}
%\begin{equation}
%	\sigma=\frac{E}{1.96}
%	\label{eq:discretesigma}
%\end{equation}
%
%\begin{tabular}{rl}
%	p =&Percent variation from the reported value.\\
%	E =&Expected value.  Reported value.\\
%\end{tabular}\\
%
%Given some extreme reported value, it is possible that the distribution of possible values may exceed the range of acceptable or possible values.  The range of acceptable or possible values is calculated from the range of reported values over the study time frame where the minimum acceptable value is one-half of the minimum reported value $(0.5\cdot \textrm{min}(E))$ and the maximum acceptable value is one and one-half of the the maximum reported value $(1.5\cdot \textrm{max}(E))$.  Truncated normal distributions with these ranges were used to generate stochastic realizations of all input variables.  The [R] function 'rtnorm' was used which uses the rejection sampling algorithms and density function as described by \citet{2013Jackson}.
%
%The distribution of the generated stochastic realizations were visually compared to the distribution of the reported values using graphs similar to Figure \ref{fig:ExampleDensity} to verify that the [R] function performed as desired.  This particular graph shows data from the analysis of the variable $Q_{ARKCATCO}$.  Similar figures for all input variables are found in Appendix \ref{App:VarDensity}.  These figures show a histogram of the reported values.  A kernel density estimate of the reported values is superimposed as a black curve over the histogram.  Kernel density estimates (KDE) are non-parametric estimates of the probability density function and are well suited to presenting continuous data sets.  Whenever reasonable, both the histogram and the KDE of the target value are presented so that the data can be more accurately represented.  The red curve is the KDE of the generated stochastic values for all time steps $(t)$ and realizations $(r)$.  The vertical lines are located at the mean value of the respective KDEs. The generated stochastic value mean and the reported value mean are nearly identical for all input variables.  The two KDEs are very comparable for most input variables.  Variability between the two KDEs are expected, especially toward the lower end as this is where most of the truncation occurs.