\chapter{Conclusion and Recommendations}
\label{chap:Conclusion}

\begin{linenumbers}
\section{Unaccounted for Return Flow Conclusions}
\label{sec:ReturnFlowConclusions}

The following list summarizes the results of the USR deterministic unaccounted for return flow model:
\begin{enumerate}
	\item Water storage changes in the main stem of the river accounted for \SI{0.000534}{\hectare\m\per\kilo\meter} (\SI{0.0374}{\acre\foot\per\mile}) loss from the main channel.
	\item Water storage changes accounted for approximately 33.6\% of the water balance model.
	\item Water flows through active gauges accounted for \SI{0.3095}{\cubic\m\per\second\per\kilo\meter} (\SI{2.42}{\cfs\per\mile}) loss from the main channel.
	\item Water flows through active gauges accounted for approximately 59\% of the water balance model.
	\item The combined effects of precipitation and evaporation accounted for \SI{0.00777}{\cubic\m\per\second\per\kilo\meter} (\SI{0.544}{\cfs\per\mile}) loss from the main channel.
	\item The combined effects of precipitation and evaporation accounted for approximately 7.4\% of the water balance model
	\item Unaccounted for return flows accounted for \SI{0.0532}{\cubic\m\per\second\per\kilo\meter} (\SI{3.73}{\cfs\per\mile}) gain to the main river channel.
\end{enumerate}

The following list summarizes the results of the USR stochastic unaccounted for return flow model:
\begin{enumerate}
	\item Water storage changes in the main stem of the river accounted for \SI{0.000282}{\hectare\m\per\kilo\meter} (\SI{0.0198}{\acre\foot\per\mile}) loss from the main channel.
	\item Water flows through active gauges accounted for \SI{0.0395}{\cubic\m\per\second\per\kilo\meter} (\SI{2.42}{\cfs\per\mile}) loss from the main channel.
	\item The combined effects of precipitation and evaporation accounted for \SI{0.00474}{\cubic\m\per\second\per\kilo\meter} (\SI{0.332}{\cfs\per\mile}) loss from the main channel.
	\item Unaccounted for return flows accounted for \SI{0.0504}{\cubic\m\per\second\per\kilo\meter} (\SI{3.53}{\cfs\per\mile}) gain to the main river channel.
\end{enumerate}

The following list summarizes the results of the DSR deterministic unaccounted for return flow model:
\begin{enumerate}
	\item Water storage changes in the main stem of the river accounted for \SI{0.000117}{\hectare\m\per\kilo\meter} (\SI{0.0082}{\acre\foot\per\mile}) loss from the main channel.
	\item Water storage changes accounted for approximately 12.7\% of the water balance model.
	\item Water flows through active gauges accounted for \SI{0.0469}{\cubic\m\per\second\per\kilo\meter} (\SI{3.29}{\cfs\per\mile}) loss from the main channel.
	\item Water flows through active gauges accounted for approximately 83.6\% of the water balance model.
	\item The combined effects of precipitation and evaporation accounted for \SI{0.00131}{\cubic\m\per\second\per\kilo\meter} (\SI{0.0918}{\cfs\per\mile}) loss from the main channel.
	\item The combined effects of precipitation and evaporation accounted for approximately 3.7\% of the water balance model
	\item Unaccounted for return flows accounted for \SI{0.0475}{\cubic\m\per\second\per\kilo\meter} (\SI{3.33}{\cfs\per\mile}) gain to the main river channel.
\end{enumerate}

The following list summarizes the results of the DSR stochastic unaccounted for return flow model:
\begin{enumerate}
	\item Water storage changes in the main stem of the river accounted for \SI{0.000119}{\hectare\m\per\kilo\meter} (\SI{0.00834}{\acre\foot\per\mile}) loss from the main channel.
	\item Water flows through active gauges accounted for \SI{0.0469}{\cubic\m\per\second\per\kilo\meter} (\SI{3.29}{\cfs\per\mile}) loss from the main channel.
	\item The combined effects of precipitation and evaporation accounted for \SI{0.00178}{\cubic\m\per\second\per\kilo\meter} (\SI{0.125}{\cfs\per\mile}) loss from the main channel.
	\item Unaccounted for return flows accounted for \SI{0.0479}{\cubic\m\per\second\per\kilo\meter} (\SI{3.36}{\cfs\per\mile}) gain to the main river channel.
\end{enumerate}

The following list enumerates the conclusions reached after analysis of the USR and DSR deterministic and stochastic models:
\begin{enumerate}
	\item Storage changes within the channel cannot be ignored.
	\item Evaporation and precipitation cannot be ignored.
	\item Both the USR and DSR deterministic water models are representative of the respective stochastic models.
\end{enumerate}
\clearpage{}

\section{Unaccounted for Return Mass Loading Conclusions}
\label{sec:ReturnMassConclusions}

The following list summarizes the results of the USR deterministic unaccounted for return mass loading model:
\begin{enumerate}
	\item Mass storage changes in the main stem of the river accounted for \SI{0.00131}{\kilo\g\per\day\per\kilo\m} (\SI{0.00289}{\pound\per\day\per\mile}) loss from the main channel.
	\item Mass storage changes accounted for approximately 28.8\% of the water balance model.
	\item Mass loadings passing through active stream gauges accounted for \SI{0.0441}{\kilo\g\per\day\per\kilo\m} (\SI{0.0972}{\pound\per\day\per\mile}) loss from the main channel.
	\item Mass loadings passing through active stream gauges accounted for approximately 71.2\% of the water balance model.
	\item Unaccounted for return mass loading accounted for \SI{0.0556}{\kilo\g\per\day\per\kilo\m} (\SI{0.123}{\pound\per\day\per\mile}) gain to the main river channel.
\end{enumerate}

The following list summarizes the results of the USR stochastic unaccounted for return mass loading model:
\begin{enumerate}
	\item Mass storage changes in the main stem of the river accounted for \SI{0.000772}{\kilo\g\per\day\per\kilo\m} (\SI{0.0017}{\pound\per\day\per\mile}) loss from the main channel.
	\item Mass loadings passing through active stream gauges accounted for \SI{0.043}{\kilo\g\per\day\per\kilo\m} (\SI{0.0948}{\pound\per\day\per\mile}) loss from the main channel.
	\item Unaccounted for return mass loading accounted for \SI{0.0543}{\kilo\g\per\day\per\kilo\m} (\SI{0.12}{\pound\per\day\per\mile}) gain to the main river channel.
\end{enumerate}


The following list summarizes the results of the DSR deterministic unaccounted for return mass loading model:
\begin{enumerate}
	\item Mass storage changes in the main stem of the river accounted for \SI{0.000302}{\kilo\g\per\day\per\kilo\m} (\SI{0.000666}{\pound\per\day\per\mile}) loss from the main channel.
	\item Mass storage changes accounted for approximately 10.6\% of the water balance model.
	\item Mass loadings passing through active stream gauges accounted for \SI{0.0734}{\kilo\g\per\day\per\kilo\m} (\SI{0.162}{\pound\per\day\per\mile}) loss from the main channel.
	\item Mass loadings passing through active stream gauges accounted for approximately 89.4\% of the water balance model.
	\item Unaccounted for return mass loading accounted for \SI{0.0819}{\kilo\g\per\day\per\kilo\m} (\SI{0.181}{\pound\per\day\per\mile}) gain to the main river channel.
\end{enumerate}

The following list summarizes the results of the DSR stochastic unaccounted for return mass loading model:
\begin{enumerate}
	\item Mass storage changes in the main stem of the river accounted for \SI{0.000346}{\kilo\g\per\day\per\kilo\m} (\SI{0.000763}{\pound\per\day\per\mile}) loss from the main channel.
	\item Mass loadings passing through active stream gauges accounted for \SI{0.0712}{\kilo\g\per\day\per\kilo\m} (\SI{0.157}{\pound\per\day\per\mile}) loss from the main channel.
	\item Unaccounted for return mass loading accounted for \SI{0.782}{\kilo\g\per\day\per\kilo\m} (\SI{0.172}{\pound\per\day\per\mile}) gain to the main river channel.
\end{enumerate}

The following list enumerates the conclusions reached after analysis of the USR and DSR deterministic and stochastic models:
\begin{enumerate}
	\item Storage changes within the channel cannot be ignored.
	\item Both the USR and DSR deterministic water models are representative of the respective stochastic models.
	\item The selenium concentration in the unaccounted for return flows is higher in the DSR than in the USR.
\end{enumerate}
\clearpage{}

\clearpage{}
\section{Unaccounted for Return Flow and Mass Loading Hypotheses}
\label{sec:FlowAndMassConclusions}

The following Hypotheses:
\begin{enumerate}
	\item The bed of John Martin Reservoir might be a significant source of selenium in the DSR.
	\item The selenium concentration in the unaccounted for return flows is significantly lower than the reported concentration from nearby test wells.  There are processes in the riparian zone of the river that are significant and should be investigated.  We suspect the primary process is biomethylation through the native and non-native plant species.
\end{enumerate}
\clearpage{}

\section{Recommendations}
\label{sec:Recommendations}

The Arkansas R. is gaining water and selenium from unknown sources.  Groundwater is most likely the largest component of these sources.  The models presented in this paper were formulated from the best available data and the most reasonable assumptions, yet there are essential pieces of information missing that can allow us to have a clearer picture of selenium transport and fate in the LARB.  Selenium volatiliztion and other transport pathways are not completely understood.  We do not know if there are spatial, temporal, or physical relationships with volatiliation.

Many assumptions were made in the analyses presented in this thesis.  Most of which are accounted for.  There are many ways to improve the study contained in these pages.  Of the concepts discussed, there are a few that deserve further study.  These include, but are not limited to, the following:
\begin{enumerate}
	\item Improve the methodology for measuring and estimating river geometry values.
	\item Improve the concentration estimating linear models for the tributaries.  These linear models showed the largest uncertainty.
	\item Improve the estimation of evaporation.  Determine a method to calculate evaporation from a river by using reference ET values calculated for locations distant from the river channel.
	\item Perform studies on selenium volitalization and boimethylation by riparian vegitation.  These contributors to selenium loss were not included in this thesis due to the lack of sufficient information to make even the most elementary of calculations.
	\item Determine a method by which the groundwater flow rate into and out of the river channel can be measured.  This will provide a measureable check to the values calculated in this thesis.
\end{enumerate}

Future studies in the LARB should include further surface water sampling as described in this thesis.  Additional data will only improve the selenium concentration estimation models.  This data may also shed light on temporal patterns not recognized at this time.

The Lower Arkansas River Basin is valuable to the State of Colorado as a source of agriculture and history.  Life and progress may appear to move slowly to those who pass through the region, but change does happen.  Changes in the region have lead to an increase in water availability to residents of the LARB in Colorado and Western Kansas.  This change has caused agriculture to spread throughout the valley.  Increase irrigation has released naturally occurring pollutants into the groundwater which returns to the river.  Understanding the interaction between the aquifer and the river with increased focus on the riparian area should help residents and water managers improve water quality in the Lower Arkansas River Basin.

\clearpage{}
\end{linenumbers}