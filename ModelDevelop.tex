\documentclass[10pt]{article}
\usepackage[usenames]{color} %used for font color
\usepackage{amssymb} %maths
\usepackage{amsmath} %maths
\usepackage[utf8]{inputenc} %useful to type directly diacritic characters
\begin{document}
\begin{align*}%\[\renewcommand{\thechapter}{5}
%\chapter{Model Development}
%\label{chap:Model Development}

%\section{Water Balance.}
%\label{sec:Water Balance}
%The water balance equation is used to determine the volume of unaccounted for water in each reach.  If we assume that all flows moving into or out of the river reach are known or can be reasonably well estimated, then the volume of water being stored during any time step is the sum of the flows being added to the river reach minus the flows being removed from the river reach.  This model accounts for flows moving through two cross sections of the river reach, the contributions from tributaries and precipitation or contributions to irrigation canals and evapo-transpiration (ET).  It also accounts for measured, known point sources.  It does not account for unknown and unmeasured surface water flows, or the flux of water between the river channel and the riparian aquifer.  The complete water balance equation form used for both study reaches was developed by expanding the basic water balance equation (\ref{eq:qbalbasic}) to include all known, measurable values.

%\begin{equation}
%	\frac{\Delta S}{\Delta t}=\sum{Q_{inflow}}-\sum{Q_{outflow}}+P-E+\sum{Q_{NPS}}
%	\label{eq:qbalbasic}
%\end{equation}
%\begin{tabular}{rl}
%$\frac{\Delta S}{\Delta t}$ =&Stored volume change between time steps $(volume \cdot time^{-1})$\\
%$\sum{Q_{inflow}}$=&Sum of measured flows entering the river section $(volume \cdot time^{-1})$\\
%$\sum{Q_{outflow}}$=&Sum of measured flows leaving the river section $(volume \cdot time^{-1})$\\
%$P$=&Precipitation in units of $volume \cdot time^{-1}$\\
%$E$=&Evaporation in units of $volume \cdot time^{-1}$\\
%$\sum{Q_{NPS}}$=&Sum of all flow rates to or from non-point sources $(volume \cdot time^{-1})$\\
%\end{tabular}\\

%The variable $\sum{Q_{UNPS}}$ is intended to capture all of the gains from and losses to the unaccounted for non-point sources and sinks and the riparian aquifer.  It is assumed that the unaccounted for non-point sources and sinks are relatively minor and intermittent.  They can include, but are not limited to, surface runoff from storm events, irrigation water returning directly from fields to the river, and groundwater.  Flows from the sum of the non-point sources $(Q_{UNPS})$ can be either added to or taken from the river reach.  The equation assumes that these flows are added to the river.  As such, positive values indicate that the river is gaining water from non-point sources and negative values indicate the river is losing water to non-point sinks.  For this study, we are assuming that the non-point source flows are dominated by groundwater interactions with the river channel.  Changing the unknown flow variable to only consider groundwater, rearranging the flow equation variables results in equation~\ref{eq:qbalbasic2}.

%\begin{equation}
%	Q_{GW}=\frac{\Delta S}{\Delta t}-\sum{Q_{inflow}}+\sum{Q_{outflow}}-P+E
%	\label{eq:qbalbasic2}
%\end{equation}

%Groundwater flows are not shown as a sum of flows like the measured inflow and outflows.  Considering them as a sum value implies that there is a way to separate them in some manner.  Groundwater contributions to the flow equation, either gains or losses, are variable along any river segment.  Grouping the variables by type produces equation~\ref{eq:qbalbasic3}, which is used as the final step of the computations.  All computations prior to this assume all storage changes, flows, precipitation, and evaporation are positive.  This step was used to help with book keeping.

%\begin{equation}
%	Q_{GW}=\frac{\Delta S}{\Delta t}-Q_{Surface}-Q_{Atmosphere}
%	\label{eq:qbalbasic3}
%\end{equation}
%Where\\
%\begin{tabular}{rl}
%$Q_{Surface} =$& $\sum{Q_{inflow}} - \sum{Q_{outflow}}$ Cumulative water flow rate.\\
%$Q_{Atmosphere} =$&$P-E$ Cumulative effects of precipitation and evaporation.\\
%\end{tabular}\\

%River section estimated stored water volume changes are taken as the sum of the values calculated for each reach within the section as shown in equation \ref{eq:qbal5}.  The calculation methodology for $\Delta V_{Reach}$ is complex and is described in section \ref{sec:RiverGeometry}.

%\begin{equation}
%	\Delta V_{Section} = \sum \Delta V_{Reach} \cdot 1~day
%	\label{eq:qbal5}
%\end{equation}
%Where\\
%\begin{tabular}{rl}
%$\Delta V_{Section} =$& Water volume change in a study region river section.  $(volume \cdot time^{-1})$\\
%$\Delta V_{Reach} =$& Water volume change in a given river reach within the river section.  $(volume)$\\
%\end{tabular}\\

%The river section total gains from tributaries and losses to canals $(Q_{Surface})$ is calculated as the sum of all gains minus the sum of all losses where the tributary flows and river section inlet flow are gains to the river and the canal flows and river section outlet flow are losses from the river as shown in equation \ref{eq:qbal4}.  

%\begin{equation}
%	Q_{Surface} = Q_{inlet} - Q_{outlet} + \sum Q_{Tributaries} - \sum Q_{Canals}
%	\label{eq:qbal4}
%\end{equation}
%Where\\
%\begin{tabular}{rl}
%$Q_{inlet} =$& Flow rate for water entering at the upstream end of a study \\
%&region river section.\\
%$\sum Q_{Tributaries} =$& Cumulative water flow rate for all tributary flows entering the \\
%&study region river section.\\
%$Q_{outlet} =$& Flow rate for water leaving at the downstream end of a study \\
%&region river section.\\
%$\sum Q_{Canals} =$& Cumulative water flow rate for all canal flows leaving the \\
%&study region river section.\\
%\end{tabular}\\

%Precipitation and evaporation values are derived from $ET_{Ref}$ and daily precipitation depth values were obtained from CoAgMet as described in section \ref{sec:data collected by other sources}.  Regional average $ET_{Ref}$ values were converted to estimated river surface evaporative loss values through the use of a variable open water evaporation coefficient as described in section \ref{sec:uncertainty of atmospheric data} .  This depth of evaporation value is then multiplied by the sum of the calculated study region river reach surface areas, as calculated in section \ref{sec:RiverGeometry}, to produce the estimated total evaporative loss for each region's river section.  Regional average depth of precipitation values are multiplied by the sum of the calculated study region river section surface areas to obtain the volume of water added to the river section due to precipitation.  This value was modified by a factor of 0.5 to reflect the estimated average storm coverage for the region as discussed in section \ref{sec:uncertainty of atmospheric data}. The atmospheric contribution to the models is taken as the sum of the modified precipitation and evaporation values with the assumption that all precipitation values are positive and all evaporation values are negative as shown in equation \ref{eq:qbal6}.

%\begin{equation}
%	Q_{Atmosphere} = 0.5 \cdot \overline{P} \cdot \sum A_{Surface} - K_{wt} \cdot \overline{ET_{Ref}} \cdot \sum A_{Surface}
%	\label{eq:qbal6}
%\end{equation}
%Where\\
%\begin{tabular}{rl}
%$Q_{Atmosphere} =$&Total water volume gained (+) or lost (-) due to atmospheric contributions \\
%&for a study region river section. Values calculated and presented as average daily flow rates. $(volume \cdot time^{-1})$\\
%$\overline{P} =$&Regional average precipitation taken as the mean of the reporting regional\\
%&rain gauges. $(depth)$\\
%$K_{wt} =$& Open water coefficient for the tall $ET_{Ref}$ equation.\\
%$\overline{ET_{Ref}} =$&Regional average $ET_{Ref}$ taken as the mean of the reporting\\
%&regional reference ET gauges.  $(depth)$\\
%$\sum A_{Surface}=$&Sum of calculated reach surface areas within a study region river \\
%&section.  $(area)$\\
%\end{tabular}\\

%All intermediate and final results are calculated in \si{\cubic\meter\per\second}.  Intermediate and final results are presented as the flow rate divided by the study region length, giving units of \si{\cubic\meter\per\second\per\kilo\meter}.  This allows the two study regions to be compared on equal footing since their lengths are significantly different.  Input variables were converted to SI units before any other calculations were performed.  Each of the preceding equations includes at least one error term which shall be discussed in the following chapters.

\section{Selenium Mass Balance.}
\label{sec:Mass Balance}
%The mass balance equation is an expansion of the flow balance equation.  Mass transport rates are given by using equation \ref{eq:mxport} which calculates an individual mass transport value at a specific location.  Combining this equation with equation \ref{eq:qbalbasic3} produces the mass balance equation used for this study.

%\begin{equation}
%	\dot{M}=QC
%	\label{eq:mxport}
%\end{equation}
%\begin{tabular}{rl}
%	$\dot{M}$ =&Mass transport $(mass \cdot time^{-1})$\\
%	$Q$=&Water flow rate $(volume \cdot time^{-1})$\\
%	$C$=&Constituent concentration $(mass \cdot volume^{-1})$\\
%\end{tabular}\\

%Combining equations \ref{eq:mxport} and \ref{eq:qbalbasic3} gives equation \ref{eq:mbal1}.  The atmospheric contribution to selenium transport is not considered in this study as discussed in the literature review.  It is known that selenium can volatilize to the atmosphere directly from the water's surface and through biological pathways.  The significance and magnitude of this contribution is unknown for the study regions included in this study.  Therefore, the atmospheric contribution to selenium transport is not included in equation \ref{eq:mbal1}. 

%\begin{equation}
%	\dot{M}_{GW}=\frac{\Delta M}{\Delta t}-\dot{M}_{Surface}
%	\label{eq:mbal1}
%\end{equation}
%\begin{tabular}{rl}
%	$\dot{M}_{GW}$ =&Mass transport to or from groundwater $(mass \cdot time^{-1})$\\
%	$\frac{\Delta M}{\Delta t}$=&River section selenium storage change $(mass \cdot time^{-1})$\\
%	$\dot{M}_{Surface}$=&Combined river section selenium surface transport rate $(mass \cdot volume^{-1})$\\
%\end{tabular}\\

%River water storage change calculations are developed and presented in chapter \ref{sec:RiverGeometry}.  Water storage change values between consecutive time steps are the basis for the calculations developed in this section.  Stored selenium mass changes between two consecutive time steps is dependent on the change in river stored water volume and the concentration of selenium in the water.  The stored selenium mass change in a given study region section is the sum of the stored selenium mass changes of the reaches within a study region river section.  Individual reach selenium storage changes are calculated using equation \ref{eq:mbal2}.

%\begin{equation}
%\frac{\Delta M}{\Delta t}=\frac{C_{Se,in}+C_{Se,out}}{2} \cdot \frac{\Delta S}{\Delta t}
%\label{eq:mbal2}
%\end{equation}
%\begin{tabular}{rl}
%$\frac{\Delta M}{\Delta t}$ =&Stored volume change between time steps $(mass \cdot time^{-1})$.\\
%$C_{Se,in}$ =& Calculated selenium concentration at the upstream end of a given\\
%&reach $(mass \cdot volume^{-1})$.\\
%$C_{Se,out}$ =& Calculated selenium concentration at the downstream end of a given\\
%&reach $(mass \cdot volume^{-1})$.\\
%$\frac{\Delta S}{\Delta t}$ =& Stored water volume change as calculated in chapter \ref{sec:RiverGeometry} $(volume \cdot time^{-1})$.\\
%\end{tabular}\\

%Ideally, the average selenium concentration should be calculated as the difference between the current calculation day stored mass and the prior calculation day stored mass.  This requires that sufficient data is available to calculate the concentrations in the two consecutive calculation days.  The actual stored water volume was not calculated and therefore the stored selenium mass could not be calculated.  The methodology shown in equation \ref{eq:mbal2} was assumed to perform as an approximation of the stored selenium mass change between two consecutive days.

%The stored water volume change is significantly larger than the average concentration.  Therefore it was assumed that small but significant changes in selenium concentrations between consecutive days would not significantly impact the stored selenium mass change.

%The combined river section selenium surface transport rate $(\dot{M}_{Surface})$ is calculated as shown in equation \ref{eq:mbal3}.  Selenium surface transport rate values for individual stream gauges are calculated as positive values regardless of whether they discharge to or receive water from the main stem of the river.

%\begin{equation}
%\dot{M}_{Surface}=\dot{M}_{inlet}-\dot{M}_{outlet}+\dot{M}_{Tributaries}-\dot{M}_{Canals}
%\label{eq:mbal3}
%\end{equation}
%\begin{tabular}{rl}
%$\dot{M}_{inlet}$ =& Selenium surface transport across the study region upstream \\
%&inlet $(mass \cdot time^{-1})$\\
%$\dot{M}_{outlet}$ =& Selenium surface transport across the study region downstream\\ 
%&outlet $(mass \cdot time^{-1})$\\
%$\dot{M}_{Tributaries}$ =& Selenium surface transport from tributaries to the river main\\
%&stem $(mass \cdot time^{-1})$\\
%$\dot{M}_{Canals}$ =& Selenium surface transport from the river main stem to\\
%&canals $(mass \cdot time^{-1})$\\
%\end{tabular}\\

%All intermediate and final results are calculated in \si{\kilo\gram\per\day}.  Intermediate and final results are presented as the flow rate divided by the study region length, giving units of \si{\kilo\gram\per\day\per\kilo\meter}.  This allows the two study regions to be compared on equal footing since their lengths are significantly different.  All input variables were converted to S.I. units before starting calculations.  Each of the variables in the preceding equations include at least one error term which shall be discussed in the following chapters.
\clearpage{}\]
%%\end{align*}
\end{document}