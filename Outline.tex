\documentclass[12pt]{article}

\usepackage[top=1in,bottom=1in,left=1in,right=1in]{geometry}
\usepackage{float}
\usepackage{outlines}
\linespread{1.5}

\begin{document}
\begin{center}
STOCHASTIC ESTIMATION OF GROUNDWATER RETURN FLOW AND DISSOLVED SELENIUM LOADING TO TWO REACHES OF THE ARKANSAS RIVER IN COLORADO
\end{center}
Title Page\\
Abstract\\
Acknowledgements\\
Table of Contents\\
List of Tables\\
List of Figures\\
\begin{outline}[enumerate]
	\1 \underline{Literature Review and Research Overview}
		\2 \underline{The Environmental Se Problem}
			\3 Describe Se, chemistry and biochem
			\3 Effects on humans w/ recorded cases
			\3 Effects on mammals (livestock) w/ recorded cases
			\3 Effects on avian and aquatic life w/ recorded cases
			\3 Se sources, both natural and anthropomorphic
			
		\2 \underline{Major Se Cycle Processes in the Environment}
			\3 Solution and dissolution
			\3 ad/absorption
			\3 volitalization and biomethylation
			\3 storage and transport in soil pore water

		\2 \underline{Water Balance Methods for Estimating NPS Return Flows to Streams}
			\3 What is the method
			\3 Recorded use cases (International, domestic, Colorado)

		\2 \underline{Mass Balance Methods for Estimating NPS Solute Loading to Streams}
			\3 What is the method
			\3 Recorded use cases (international, domestic, Colorado)

		\2 \underline{Previous Related Studies in Colorado's Lower Arkansas River Valley}
			\3 List of all known studies with main methods used and  conclusions
			
		\2 \underline{Goal and Objective of this Study}

	\1 \underline{Study Region Description}
		\2 \underline{Overview of the Lower Arkansas River Valley in Colorado}
			\3 Geographic location and limits
			\3 Description of the geology and alluvial aquifer and its connection to the stream system
			\3 Discussion of the general hydrology of the LARV
			\3 Discuss current land use, major crops, and irrigation practices
			\3 Irrigation flows induce high concentrations of salt, Se, U, and nutrients
			\3 Sources of Se in the LARV and how irrigation practices affect the mobilization of Se from sources

		\2 \underline{Upstream Study Reach and Surrounding Region}
			\3 Geographic location and limits
			\3 Study reach segmentation
			\3 Description of canals, tribs, drains
			\3 Location of existing stream gauges
			\3 Location and description of surface water quality sampling sites
			\3 Location and description of river cross section survey sites

		\2 \underline{Downstream Study Reach and Surrounding Region}
			\3 Geographic location and limits
			\3 Study reach segmentation
			\3 Description of canals, tribs, drains
			\3 Location of existing stream gauges
			\3 Location and description of surface water quality sampling sites
			\3 Location and description of river cross section survey sites

	\1 \underline{Data Collection and Compilation}
		\2 \underline{Water Quality Data Collection}
			\3 Sampling period, frequency, and quantity description
			\3 Sampling preparation description
			\3 Description of sample types taken
			\3 Description of field sampling methodology and equipment
			\3 Describe how lab results were handled

		\2 \underline{River Cross-section Geometry Survey}
			\3 Description of locations
			\3 Description of general method and equipment used
			\3 Description of data collected
			\3 Method of closing errors
			\3 converting survey data to cross section geometry

		\2 \underline{Data Compiled from Other Sources}
			\3 Flow gauge data
			\3 Continious in-situ water quality data
			\3 ET(ref) and related data ($u_2$ and $RH_{min}$)
			\3 precipitation
			
		\2 \underline{Relationships between variables}
			\3 correlations
			\3 real-world relationships

	\1 \underline{Evaluation of NPS Return Flow to the River Using a Water Balance Model}
		\2 \underline{Flow Balance Model Applied to the LARV}\\
		$Q_{U}=\frac{\Delta S}{\Delta t}-Q_{Surface}-Q_{Atmosphere} $ -- for each $\Delta t$ \\
			\3 Describe derivation of equation from 'standard' equation
			\3 Justify use of the equation in this form
			\3 Define $Q_U$ constituents
		
		\2 \underline{Stochastic and Deterministic Models}
			\3 Define uncertainty and it's sources
			\3 Define true value vs measured value
			\3 Define univariate probability distributions as estimate description of an uncertainty parameter
			\3 Define parametric and non-parametric distribution and their uses
		
		\2 \underline{River Storage Change}\\
			$\frac{\Delta S}{\Delta t}$
			\3 define the time step
			\3 define calculation of storage change\\
			$\Delta S = \frac{y_2-y_1}{2} \cdot \frac{Tw_1+Tw_2}{2}$ (as the area of trapezoid)
			\3 define flow depth ($y$) from stream gauge height ($h$)
				\4 present source document defining stream gauge height uncertainty
				\4 describe stream gauge height pre-defined parametric univariate uncertainty
			\3 define calculation of flow depth from stream gauge height based on survey data
				\4 define uncertainty of flow depth vs. stream gauge height relationship.  Based on personal observations.
				\4 present river segment flow depth results - tables and graphs
			\3 define calculation of top width from flow depth ($Tw=\beta_1 y^{\beta_2}$)
				\4 define non-linear regression used to estimate $\beta_1$ and $\beta_2$
				\4 describe residuals as model uncertainty
				\4 define method of determining parametric univariate uncertainty distribution from regression residuals
				\4 define distribution goodness-of-fit and tests (i.e. A-D, K-S, and visual)
				\4 test non-linear regression uncertainty distribution against original data
				\4 define distributions of $\beta_1$ and $\beta_2$
				\4 test uncertainty distributions against original data
				\4 present $\beta_1$ and $\beta_2$ uncertainty distributions - tables and graphs
				\4 analysis and comments on distributions
			\3 present river segment $\frac{\Delta S}{\Delta t}$ results - tables and graphs
			\3 present river reach $\frac{\Delta S}{\Delta t}$ results - tables and graphs
			\3 analysis and comments on segment and reach results

		\2 \underline{Gauged Stream Flows and Diverted Canal Flows}\\
			 $Q_{Surface} = Q_{US}-Q{DS}+Q{in}-Q_{out}$
			\3 Define the variables - which flow variable belongs to which $Q$.
			\3 Define the reported flow rate uncertainty distribution - for each flow variable
				\4 USGS and CDWR defined uncertainty
			\3 present $Q$ results for each source/sink - tables and graphs
			\3 present river segment $Q_{Surface}$ results - tables and graphs
			\3 present river reach $Q_{Surface}$ results - tables and graphs
			\3 analysis and comments on segment and reach results
		
		\2 \underline{Atmospheric Contributions to Flow Balance}\\
			$Q_{Atmosphere} = Q_{P}-Q{E}$
			\3 Define relationship between total E and ET(ref) ($E=ET_{ref} \cdot A_{river~surface}$)
			\3 Define how to convert reported ET(ref) to Evap
				\4 define the uncertainty distribution of ET(ref) as per Dr. Chavez
				\4 define the uncertainty distributions of the additional variables used to convert from ET(ref) to E
			\3 Define river surface area
				\4 State the use of $Tw$ from water storage calculations
				\4 State the use of the same uncertainty used in water storage calculations
			\3 present river segment total E results - tables and graphs
			\3 present river reach total E results - tables and graphs
			\3 define relationship between P measured at weather stations and P realized on the river's surface
			\3 define the uncertainty of measured P values
			\3 present river segment P results - tables and graphs
			\3 present river reach P results - tables and graphs
			\3 present river segment $Q_{Atmosphere}$ results - tables and graphs
			\3 present river reach $Q_{Atmosphere}$ results - tables and graphs
			\3 analysis and comments on segment and reach results
			
		\2 \underline{Results of Calculated Unaccounted for Return Flows}
			\3 Present river segment $Q_{U}$ results - tables and graphs		
			\3 Present river reach $Q_{U}$ results - tables and graphs
			\3 analysis and comments on segment and reach results

	\1 \underline{Evaluation of NPS Selenium Loading to the River Using a Mass Balance Model}
		\2 \underline{Mass balance model applied to the LARV}\\
		$\dot{M}_U = \frac{\delta M_S}{\Delta t} - \dot{M}_{Surface}$
			\3 Describe derivation of equation from standard flow balance equation and $\dot{M}=QC$
			\3 Justify use of the equation in this form
			\3 Define $\dot{M}_U$ constituents

		\2 \underline{Mass Storage Change}\\
			$\frac{\Delta M_S}{\Delta t}$
			\3 Define the time step
			\3 Define relationship between water storage change and mass storage change
			\3 \underline{Solute concentration models}
				\4 define points/locations where $C_{Se}$ is calculated
				\4 define linear regression (ordinary least squares)
				\4 define independent variables used
				\4 method used to optimize models
				\4 test optimized models 
				\4 presentation of optimized models - tables and graphs
			\3 \underline{Describing uncertainty distributions}
				\4 describe residuals as linear regression uncertainty
				\4 distribution forms tested
				\4 method used to determine best fit (goodness-of-fit)
				\4 test best fit distributions
				\4 presentation of best fit $C_{Se}$ distributions of residuals
			\3 \underline{Uncertainty of lab $C_{Se}$ values}
				\4 Define the uncertainty constituents
				\4 Data source
				\4 Calculation method
				\4 Test distribution
				\4 presentation of best fit lab $C_{Se}$ uncertainty distribution
			\3 \underline{Mass Storage Change Results}
				\4 present river sement $\frac{\Delta M_S}{\Delta t}$ results - tables and graphs
				\4 present river reach $\frac{\Delta M_S}{\Delta t}$ results - tables and graphs
				\4 analysis and comments on river segment and reach results
			
		\2 \underline{Mass Transport in Gauged Streams and Diverted Canals}\\
			$\dot{M}_{Surface}=\dot{M}_{US}-\dot{DS}_{out}+\dot{M}_{in}-\dot{M}_{out}$
			\3 Define the relationship between $Q_{Surface}$ and $\dot{M}_{Surface}$
			\3 State which solute concentration models used with which gauged flows
			\3 Present source/sink $\dot{M}$ results - tables and graphs
			\3 Present river segment $\dot{M}$ results - tables and graphs
			\3 Present river reach $\dot{M}$ results - tables and graphs
			\3 analysis and comments on river segment and reach results
		\2 \underline{Results of the Calculated Unaccounted for Return Loading}
			\3 Present river segment $\dot{M}_U$ results - tables and graphs
			\3 Present river reach $\dot{M}_U$ results - tables and graphs
			\3 Analysis and comments on river segment and reach results
		
	\1 \underline{Sensitivity Analysis}
		\2 Purpose of analysis
		\2 Scope of analysis
		\2 Method of analysis
			\3 Which variables are perturbed...by how much...and why
			\3 State use of only the deterministic models for analysis
		\2 Present analysis results - tables
		\2 Analysis and comments on analysis results			
			
	\1 \underline{Conclusion and Recommendations}
		\2 Unaccounted for return flow conclusions as supported by results
		\2 Unaccounted for return loading conclusions as supported by results
		\2 Unaccounted for flow and mass transport conclusions and hypothoses
			\3 comparison of unaccounted for Se concentrations to observed Se concentrations in surfacewater and groundwater
			\3 possible effects of bank ecology with natural remediation of Se concentrations
			\3 Discuss how results can be used to guide calibration of GW flow and mass transport models
		\2 Recommendations
			\3 additional Se samples to verify the models.
			\3 Se volitalization study on the Ark
			\3 Studies of Se sorption and chem. reduction in bed and bank sediments
			
\end{outline}
Sources cited/Bibliography\\
Appendicies
\end{document}