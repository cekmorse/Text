\documentclass[10pt]{article}
\usepackage[usenames]{color} %used for font color
\usepackage{amssymb} %maths
\usepackage{amsmath} %maths
\usepackage[utf8]{inputenc} %useful to type directly diacritic characters
\begin{document}
\[\section{Reach Stored Water Change.}
\label{sec:Reach Stored Water Change}

%The deterministic and stochastic water balance models each consist of three major components.  The flow component is the sum of all surface flows, the storage component is the sum of all storage changes, and the atmospheric component is the precipitation gains less the evaporation losses.

%Values and figures presented in this section are the storage component sub-set of the complete water balance model results.  The analyses in this section are restricted to performing a reasonability check on the results.  Results in this section are to be compared with results from the other model sub-sections from both the USR and DSR models to determine if results are acceptable.  Unacceptable results would indicate an error in the computational code or an underlying assumption.

%Figures \ref{fig:USRWaterStore} and \ref{fig:DSRWaterStore} present the sum of all river segment water storage changes in the USR and DSR, respectively, as time series plots.  The left and right sub-figures present the deterministic and stochastic time series, respectively.  The black line in the stochastic figures is the 1-D stochastic mean results with the blue band indicating the 95\% CIR.  Values in these figures and associated tables indicate the volume of water stored in the respective reach in units of \si{\cubic\meter\per\second\per\kilo\meter}.  Standardizing the results to present values in units of storage per unit length allows for comparisons to be made between the two study reaches and between water balance model components.  Positive values indicate that the reach's storage volume increase. 

%\begin{figure}[htbp]
%\centering
%	\begin{subfigure}{0.5\textwidth}
%		\centering
%		\includegraphics[width=0.9\linewidth]{"Figures/Results_DUSR/Balance Water - storage"}
%		\caption{Deterministic Model.}
%		\label{sub:USRWaterStoreD}
%	\end{subfigure}%
%	\begin{subfigure}{0.5\textwidth}
%		\centering
%		\includegraphics[width=0.9\linewidth]{"Figures/Results_USR/Balance Water - storage"}
%		\caption{Stochastic Model.}
%		\label{sub:USRWaterStoreS}
%	\end{subfigure}
%	\caption[Time series of the USR Arkansas R. water storage change contribution to the water model.]{Time series of the USR Arkansas R. water storage change contribution to the water model.}
%	\label{fig:USRWaterStore}
%\end{figure}

%\begin{figure}[htbp]
%\centering
%	\begin{subfigure}{0.5\textwidth}
%		\centering
%		\includegraphics[width=0.9\linewidth]{"Figures/Results_DDSR/Balance Water - storage"}
%		\caption{Deterministic Model.}
%		\label{sub:DSRWaterStoreD}
%	\end{subfigure}%
%	\begin{subfigure}{0.5\textwidth}
%		\centering
%		\includegraphics[width=0.9\linewidth]{"Figures/Results_DSR/Balance Water - storage"}
%		\caption{Stochastic Model.}
%		\label{sub:DSRWaterStoreS}
%	\end{subfigure}
%	\caption[Time series of the DSR Arkansas R. water storage change contribution to the water model.]{Time series of the DSR Arkansas R. water storage change contribution to the water model.}
%	\label{fig:DSRWaterStore}
%\end{figure}

%The figures show that there is a definite seasonal variation in the storage component.  Stored water changes are smallest during the cold months and are increasingly larger during the growing season.  This pattern is caused by irrigation practices in the Lower Arkansas River Basin (LARB).  Stored irrigation water is released from reservoirs upstream of the LARB during the irrigation season as demands require.  During the winter, there is very little to no irrigation occurring.  This, combined with the low precipitation and no additional flow from snow-pack runoff, allow flows in the Arkansas R. to reach base flow level when the river is at a natural equilibrium.

%This pattern is not as pronounced in the DSR.  This is most likely due to required flow rates being maintained at the Colorado-Kansas border.  With the constant required flow, John Martin Reservoir releases only enough water to meet the required flow rate.  This creates a flow regime that is nearly constant through most of the DSR. 

%Deviations are noted during peak irrigation season.  These additional volume changes are most likely due to irrigation flows returning to the Arkansas R. from fields irrigated from canal systems that procure and store water upstream of John Martin Reservoir.  The Fort Lyon Storage Canal and Fort Lyon Canal are two examples of canal systems that perform this function.  Water removed from the USR, stored, and added to the DSR is not included in this study.

%The distribution of all realizations within each time step was analyzed to determine a distribution type.  This analysis was performed to determine if the assumption that the deterministic model results were representative of the stochastic model.   Testing was performed by comparing K-S statistics for the best fit normal, log-normal, logistic, exponential, gamma, and Weibull distributions.  In the USR, 97\% of all storage component time steps best fit a normal distribution and 3\% best fit a gamma distribution.  In the DSR, 99.5\% best fit a normal distribution and 0.5\% best fit a gamma distribution.  This indicates that for both the USR and DSR, the distributions across the realizations are normal.

%Tables \ref{tab:USRWaterStore} and \ref{tab:DSRWaterStore} presents summary statistics of the deterministic and stochastic storage component changes.  Values are presented in units of \si{\cubic\meter\per\second\per\kilo\meter}.   These tables show the mean, 2.5th, and 97.5th percentile of the deterministic model and the same statistics applied to the three 1-D stochastic models.  This additional set of statistics from the 1-D stochastic 2.5th and 95.5th percentile models provide a better understanding of the extremes of the calculated values.  As before, comparison between the deterministic and stochastic models should be limited to the comparing the deterministic model to the 1-D stochastic mean model.  Also included in these two tables is the percent difference between the deterministic model and the 1-D stochastic mean model.  The mean, 2.5th, and 97.5th percentile values were calculated from the percent difference of the 1-D stochastic mean model from the deterministic model at each time step.  These values show the range of the variance from the deterministic model.

%\begin{table}[htbp]
%\centering
%\caption[USR Arkansas R. water storage change contribution to the water model.]{USR Arkansas R. water storage change contribution to the water model.  All values are in units of \si{\cubic\meter\per\second\per\kilo\meter}.}
%\label{tab:USRWaterStore}
%\begin{tabular}{c|ccc}
%	\toprule
%	Model& 2.5\% & Mean & 97.5\% \\
%	\midrule
%	\midrule
%	Deterministic&		-0.08763&	-0.0002462&	0.1047\\
%	\midrule				
%	Stochastic 2.5\%&	-0.1485&	-0.04056&	0.04926\\
%	Stochastic Mean&	-0.08959&	0.0002286&	0.1119\\ 
%	Stochastic 97.5\%&	-0.04075&	0.04078&	0.1771\\ 
%	\midrule                                        
%	\% Diff. Means &	-25.1&	76.69&	44.82      \\
%	\bottomrule
%\end{tabular}
%\end{table}

%\begin{table}[htbp]
%\centering
%\caption[DSR Arkansas R. water storage change contribution to the water model.]{DSR Arkansas R. water storage change contribution to the water model.  All values are in units of \si{\cubic\meter\per\second\per\kilo\meter}.}
%\label{tab:DSRWaterStore}
%\begin{tabular}{c|ccc}
%	\toprule
%	Model& 2.5\% & Mean & 97.5\% \\
%	\midrule
%	\midrule
%	Deterministic    &	-0.01154&	0.00007389&	0.0121\\
%	\midrule                                               
%	Stochastic 2.5\% &	-0.03297&	-0.01365&	-0.001217\\
%	Stochastic Mean  &	-0.014&	0.00008216&	0.01488\\      
%	Stochastic 97.5\%&	0.0009825&	0.01381&	0.03462\\  
%	\midrule                                               
%	\% Diff. Means&		-164.2&	9.438&	177.1\\
%	\bottomrule
%\end{tabular}
%\end{table}

%The mean of the percent difference between the deterministic and 1-D stochastic mean models is very low.  This indicates that the deterministic model is representative of the stochastic model expected value.  The high percent differences at the 2.5th and 97.5th percentile indicate that there is still a large range of uncertainty contained within the stochastic model that the deterministic model cannot replicate.  The deterministic model can be used to determine how changes can affect a reach over a span of time, but using it to estimate values for specific time steps is unwise as the differences noted at individual time steps is too large to account for.

\clearpage

\section{Reach Surface Water Flows.}
\label{sec:Reach Surface Water Flows}

%The deterministic and stochastic water balance models each consist of three major components.  The flow component is the sum of all surface flows, the storage component is the sum of all storage changes, and the atmospheric component is the precipitation gains less the evaporation losses.

%Values and figures presented in this section are the flow component sub-set of the complete water balance model results.  The analyses in this section are restricted to performing a reasonability check on the results.  Results in this section are to be compared with results from the other sub-sections from both the USR and DSR models to determine if results are acceptable.  Unacceptable results would indicate an error in the computational code or an underlying assumption.

%Figures \ref{fig:USRWaterFlow} and \ref{fig:DSRWaterFlow} present the sum of all surface flows in the USR and DSR, respectively, as time series plots.  The left and right sub-figures present the deterministic and stochastic time series, respectively.  The black line in the stochastic figures is the 1-D stochastic mean results with the blue band indicating the 95\% CIR.  Values in these figures and associated tables indicate the volume of water moving in and out of the respective reach in units of \si{\cubic\meter\per\second\per\kilo\meter}.  Standardizing the results to present values in units of storage per unit length allows for comparisons to be made between the two study reache and between water balance model components.  Positive values indicate that the reach gained water during the given time step.

%\begin{figure}[htbp]
%\centering
%	\begin{subfigure}{0.5\textwidth}
%		\centering
%		\includegraphics[width=0.9\linewidth]{"Figures/Results_DUSR/Balance Water - flow"}
%		\caption{Deterministic Model.}
%		\label{sub:USRWaterFlowD}
%	\end{subfigure}%
%	\begin{subfigure}{0.5\textwidth}
%		\centering
%		\includegraphics[width=0.9\linewidth]{"Figures/Results_USR/Balance Water - flow"}
%		\caption{Stochastic Model.}
%		\label{sub:USRWaterFlowS}
%	\end{subfigure}
%	\caption[USR Arkansas River deterministic and stochastic surface water flow balance time series.]{USR Arkansas River deterministic and stochastic surface water flow balance time series.}
%	\label{fig:USRWaterFlow}
%\end{figure}

%\begin{figure}[htbp]
%\centering
%	\begin{subfigure}{0.5\textwidth}
%		\centering
%		\includegraphics[width=0.9\linewidth]{"Figures/Results_DDSR/Balance Water - flow"}
%		\caption{Deterministic Model.}
%		\label{sub:DSRWaterFlowD}
%	\end{subfigure}%
%	\begin{subfigure}{0.5\textwidth}
%		\centering
%		\includegraphics[width=0.9\linewidth]{"Figures/Results_DSR/Balance Water - flow"}
%		\caption{Stochastic Model.}
%		\label{sub:DSRWaterFlowS}
%	\end{subfigure}
%		\caption[DSR Arkansas River deterministic and stochastic surface water flow balance time series.]{DSR Arkansas River deterministic and stochastic surface water flow balance time series.}
%	\label{fig:DSRWaterFlow}
%\end{figure}

%The figures show that there is a definite seasonal variation in the flow component.  The surface water flow balance changes from being a losing system to a gaining system during the irrigation season.  This pattern is caused by irrigation practices in the Lower Arkansas River Basin (LARB).  Excess irrigation water applied to fields runs off into tributaries which feed into the Arkansas R.

%The figures also show that the river is discharging more water than it receives.  In the USR, this matches the general observed trend where the upstream end of the reaches receives more water than the downstream ends discharges.  This does not match the observed trend in the DSR where the flows at the upstream end appear to be less than or equal to the flows at the downstream end.  In the DSR, this mismatch between the observed phenomenon and the measured values indicates that another process is a significant contributor to the flow regime.  Since the two reaches are in the same environment, it is reasonable to assume that the same phenomenon is present in the USR.  It is suspected that groundwater from the riparian aquifer is the unaccounted for source in both the USR and DSR.

%There is less uncertainty associated with the flow component than with the storage component.  This is likely due to the use of more tightly defined uncertainties for the flow component.  There is more uncertainty associated with the USR flow component than with the DSR.  This is likely due to the higher number of variables used to calculate the USR flow component.

%The distribution of all realizations within each time step was analyzed to determine a distribution type.  This analysis was performed to determine if the assumption that the deterministic model results were representative of the stochastic model.   Testing was performed by comparing K-S statistics for the best fit normal, log-normal, logistic, exponential, gamma, and Weibull distributions.  In the USR and the DSR, 100\% of all flow component time steps best fit a normal distribution. This indicates that for both the USR and DSR, the distributions across the realizations are normal. 

%Tables \ref{tab:USRWaterFlow} and \ref{tab:DSRWaterFlow} presents summary statistics of the deterministic and stochastic flow component.  Values are presented in units of \si{\cubic\meter\per\second\per\kilo\meter}.   These tables show the mean, 2.5th, and 97.5th percentile of the deterministic model and the same statistics applied to the three 1-D stochastic models.  This additional set of statistics from the 1-D stochastic 2.5th and 95.5th percentile models provide a better understanding of the extremes of the calculated values.  As before, comparison between the deterministic and stochastic models should be limited to the comparing the deterministic model to the 1-D stochastic mean model.  Also included in these two tables is the percent difference between the deterministic model and the 1-D stochastic mean model.  The mean, 2.5th, and 97.5th percentile values were calculated from the percent difference of the 1-D stochastic mean model from the deterministic model at each time step.  These values show the range of the variance from the deterministic model.

%\begin{table}[htbp]
%\centering
%\caption[USR tributary and irrigation canal contribution to the water balance model.]{USR tributary and irrigation canal contribution to the water balance model.  All values are in units of \si{\cubic\meter\per\second\per\kilo\meter}.}
%\label{tab:USRWaterFlow}
%\begin{tabular}{c|ccc}
%	\toprule
%	Model& 2.5\% & Mean & 97.5\% \\
%	\midrule
%	\midrule
%	Deterministic    & -0.1736&	-0.03952&	0.07513\\
%	\midrule
%	Stochastic 2.5\% &  -0.2481&	-0.07086&	0.02988\\
%	Stochastic Mean  &  -0.1733&	-0.03952&	0.07502\\
%	Stochastic 97.5\%&  -0.112&	-0.008186&	0.1351\\     
%	\midrule
%	\% Diff. Means &	-2.713&	0.9486&	2.385\\
%	\bottomrule
%\end{tabular}
%\end{table}

%\begin{table}[htbp]
%\centering
%\caption[DSR tributary and irrigation canal contribution to the water balance model.]{DSR tributary and irrigation canal contribution to the water balance model.  All values are in units of \si{\cubic\meter\per\second\per\kilo\meter}.}
%\label{tab:DSRWaterFlow}
%\begin{tabular}{c|ccc}
%	\toprule
%	Model& 2.5\% & Mean & 97.5\% \\
%	\midrule
%	\midrule
%	Deterministic    & -0.06635&	-0.02955&	0.01146\\
%	\midrule                                            
%	Stochastic 2.5\% & -0.07775	&-0.03799	&-0.01559\\
%	Stochastic Mean  & -0.06638	&-0.02957	&0.01099 \\
%	Stochastic 97.5\%& -0.05697	&-0.02116	&0.04833 \\
%	\midrule                                            
%	\% Diff. Means &	 -0.7374&	0.3264&	0.4883\\
%	\bottomrule
%\end{tabular}
%\end{table}

%The mean of the percent difference between the deterministic and 1-D stochastic mean models is very low.  This indicates that the deterministic model is representative of the stochastic model expected value.  The low percent differences at the 2.5th and 97.5th percentile indicate that there is very little uncertainty that the deterministic model cannot estimate.  The deterministic model can be used to determine how changes can affect a reach over a span of time and can be used to estimate values for specific time steps.
\clearpage

\section{Reach Atmospheric Contributions.}
\label{sec:Reach Atmospheric Contributions}

%The deterministic and stochastic water balance models each consist of three major components.  The flow component is the sum of all surface flows, the storage component is the sum of all storage changes, and the atmospheric component is the precipitation gains less the evaporation losses.

%Values and figures presented in this section are the atmospheric component sub-set of the complete water balance model results.  The analyses in this section are restricted to performing a reasonability check on the results.  Results in this section are to be compared with results from the other sub-sections from both the USR and DSR models to determine if results are acceptable.  Unacceptable results would indicate an error in the computational code or an underlying assumption.

%Figures \ref{fig:USRWaterAtm} and \ref{fig:DSRWaterAtm} present the precipitation gains less the evaporation losses in the USR and DSR, respectively, as time series plots.  The left and right sub-figures present the deterministic and stochastic time series, respectively.  The black line in the stochastic figures is the 1-D stochastic mean results with the blue band indicating the 95\% CIR.  Values in these figures and associated tables indicate the volume of water change due to precipitation gains and evaporation losses in units of \si{\cubic\meter\per\second\per\kilo\meter}.  Standardizing the results allows for comparisons to be made between the two study reaches and between water balance model components.  Positive values indicate that the reach gained water during the given time step.

%\begin{figure}[htbp]
%\centering
%	\begin{subfigure}{0.5\textwidth}
%		\centering
%		\includegraphics[width=0.9\linewidth]{"Figures/Results_DUSR/Balance Water - atm"}
%		\caption{Deterministic Model.}
%		\label{sub:USRWaterAtmD}
%	\end{subfigure}%
%	\begin{subfigure}{0.5\textwidth}
%		\centering
%		\includegraphics[width=0.9\linewidth]{"Figures/Results_USR/Balance Water - atm"}
%		\caption{Stochastic Model.}
%		\label{sub:USRWaterAtmS}
%	\end{subfigure}
%	\caption[USR Arkansas River deterministic and stochastic evaporation and precipitation balance time series.]{USR Arkansas River deterministic and stochastic evaporation and precipitation balance time series.}
%	\label{fig:USRWaterAtm}
%\end{figure}

%\begin{figure}[htbp]
%\centering
%	\begin{subfigure}{0.5\textwidth}
%		\centering
%		\includegraphics[width=0.9\linewidth]{"Figures/Results_DDSR/Balance Water - atm"}
%		\caption{Deterministic Model.}
%		\label{sub:DSRWaterAtmD}
%	\end{subfigure}%
%	\begin{subfigure}{0.5\textwidth}
%		\centering
%		\includegraphics[width=0.9\linewidth]{"Figures/Results_DSR/Balance Water - atm"}
%		\caption{Stochastic Model.}
%		\label{sub:DSRWaterAtmS}
%	\end{subfigure}
%		\caption[DSR Arkansas River deterministic and stochastic evaporation and precipitation balance time series.]{DSR Arkansas River deterministic and stochastic evaporation and precipitation balance time series.}
%	\label{fig:DSRWaterAtm}
%\end{figure}

%The figures show that there is a definite seasonal variation in the atmospheric component balance.  This temporal relationship follows the same pattern identified with the evaporation and precipitation time series values.  Losses are higher during the warmer months and lower during the colder months.  It should be noted that while the figures for the USR and DSR have the same pattern, the magnitude of the losses is very different.  This is due to the differences in river geometry.  The USR is wider therefore losing more water to evaporation than in the DSR.

%There figures appear to show that uncertainty with the atmospheric component is nearly equal to uncertainty with the storage component.  This isn't true as the storage component is orders of magnitude larger than the atmospheric component.  The low uncertainty associated witht the atmospheric component is due to the efforts by Dr. Cha\'{a}vez and others to characterize evaporation and precipitation uncertainty.

%The distribution of all realizations within each time step was analyzed to determine a distribution type.  This analysis was performed to determine if the assumption that the deterministic model results were representative of the stochastic model.  Testing was performed by comparing K-S statistics for the best fit normal, log-normal, logistic, exponential, gamma, and Weibull distributions.  In the USR and the DSR, 100\% of all atmospheric component time steps best fit a normal distribution. This indicates that for both the USR and DSR, the distributions across the realizations are normal. 

%Tables \ref{tab:USRWaterAtm} and \ref{tab:DSRWaterAtm} presents summary statistics of the deterministic and stochastic atmospheric component.  Values are presented in units of \si{\cubic\meter\per\second\per\kilo\meter}.   These tables show the mean, 2.5th, and 97.5th percentile of the deterministic model and the same statistics applied to the three 1-D stochastic models.  This additional set of statistics from the 1-D stochastic 2.5th and 95.5th percentile models provide a better understanding of the extremes of the calculated values.  As before, comparison between the deterministic and stochastic models should be limited to the comparing the deterministic model to the 1-D stochastic mean model.  Also included in these two tables is the percent difference between the deterministic model and the 1-D stochastic mean model.  The mean, 2.5th, and 97.5th percentile values were calculated from the percent difference of the 1-D stochastic mean model from the deterministic model at each time step.  These values show the range of the variance from the deterministic model.

%\begin{table}[htbp]
%\centering
%\caption[USR atmospheric contribution to the water balance model.]{USR atmospheric contribution to the water balance model.  All values are in units of \si{\cubic\meter\per\second\per\kilo\meter}. }
%\label{tab:USRWaterAtm}
%\begin{tabular}{c|ccc}
%	\toprule
%	Model& 2.5\% & Mean & 97.5\% \\
%	\midrule
%	\midrule
%	Deterministic    &	-0.007423&	-0.0037267&	0.0003115\\
%	\midrule                                               
%	Stochastic 2.5\% &	-0.0101&	-0.005403&	-0.0001784\\
%	Stochastic Mean  &	-0.007541&	-0.003822&	0.0003791\\ 
%	Stochastic 97.5\%&	-0.004907&	-0.002321&	0.001046\\  
%	\midrule                                               
%	\% Diff. Means&		-3.354&	-1.458&	-0.9734\\
%	\bottomrule
%\end{tabular}
%\end{table}

%\begin{table}[htbp]
%\centering
%\caption[DSR atmospheric contribution to the water balance model.]{DSR atmospheric contribution to the water balance model.  All values are in units of \si{\cubic\meter\per\second\per\kilo\meter}.}
%\label{tab:DSRWaterAtm}
%\begin{tabular}{c|ccc}
%	\toprule
%	Model& 2.5\% & Mean & 97.5\% \\
%	\midrule
%	\midrule
%	Deterministic    &	-0.002837&	-0.0008244&	0.00001137\\
%	\midrule                                              
%	Stochastic 2.5\% &	-0.004348&	-0.001562&	-0.0001332\\
%	Stochastic Mean  &	-0.003025&	-0.001006&	0.00001013\\
%	Stochastic 97.5\%&	-0.001813&	-0.0005516&	0.0001749\\ 
%	\midrule                                              
%	\% Diff. Means&		-41.32&	-28.64&	-6.203\\
%	\bottomrule
%\end{tabular}
%\end{table}

%The mean of the percent difference between the deterministic and 1-D stochastic mean models is low.  This indicates that the deterministic model is representative of the stochastic model expected value.  The low percent differences at the 2.5th and 97.5th percentile indicate that there is very little uncertainty that the deterministic model cannot estimate.  The deterministic model can be used to determine how changes can affect a reach over a span of time and can be used to estimate values for specific time steps.
\clearpage

\section{Reach Stored Selenium Change.}
\label{sec:Reach Stored Selenium Change}

%The deterministic and stochastic mass balance models each consist of two major components.  The selenium transport component is the sum of all selenium mass transport and the selenium storage component is the sum of all river segment mass storage changes.  Mass transport is the mass that is transported in or out of the river reach by tributaries or canals, respectively.  Unlike the water balance model, the atmospheric model is not included.  As discussed earlier, there is insufficient evidence to support any calculations used to estimate selenium loss to the atmosphere.

%Values and figures presented in this section are the selenium storage component sub-set of the complete mass model results.  The analyses in this section are restricted to performing a reasonability check on the results.  Results in this section are to be compared with results from the other sub-sections from both the USR and DSR models to determine if results are acceptable.  Unacceptable results would indicate an error in the computational code or an underlying assumption.

%Figures \ref{fig:USRMassStore} and \ref{fig:DSRMassStore} present the selenium storage component results in the USR and DSR, respectively, as time series plots.  The left and right sub-figures present the deterministic and stochastic time series, respectively.  The black line in the stochastic figures is the 1-D stochastic mean results with the blue band indicating the 95\% CIR.  Values in these figures and associated tables are in units of \si{\kilo\gram\per\day\per\kilo\meter}.  Standardizing the results allows for comparisons to be made between the two study reaches and between mass balance model components.  Positive values indicate that the reach gained selenium during the given time step.

%\begin{figure}[htbp]
%\centering
%	\begin{subfigure}{0.5\textwidth}
%		\centering
%		\includegraphics[width=0.9\linewidth]{"Figures/Results_DUSR/Balance Mass - Storage"}
%		\caption{Deterministic Model.}
%		\label{sub:USRMassStoreD}
%	\end{subfigure}%
%	\begin{subfigure}{0.5\textwidth}
%		\centering
%		\includegraphics[width=0.9\linewidth]{"Figures/Results_USR/Balance Mass - Storage"}
%		\caption{Stochastic Model.}
%		\label{sub:USRMassStoreS}
%	\end{subfigure}
%	\caption[USR Arkansas River deterministic and stochastic stored mass change time series.]{USR Arkansas River deterministic and stochastic stored mass change time series.}
%	\label{fig:USRMassStore}
%\end{figure}

%\begin{figure}[htbp]
%\centering
%	\begin{subfigure}{0.5\textwidth}
%		\centering
%		\includegraphics[width=0.9\linewidth]{"Figures/Results_DDSR/Balance Mass - Storage"}
%		\caption{Deterministic Model.}
%		\label{sub:DSRMassStoreD}
%	\end{subfigure}%
%	\begin{subfigure}{0.5\textwidth}
%		\centering
%		\includegraphics[width=0.9\linewidth]{"Figures/Results_DSR/Balance Mass - Storage"}
%		\caption{Stochastic Model.}
%		\label{sub:DSRMassStoreS}
%	\end{subfigure}
%		\caption[DSR Arkansas River deterministic and stochastic stored mass change time series.]{DSR Arkansas River deterministic and stochastic stored mass change time series.}
%	\label{fig:DSRMassStore}
%\end{figure}

%The figures show that there is a definite seasonal variation in the selenium storage component.  This temporal relationship follows the same pattern identified with the water balance model storage component.  There is a very strong visual relationship between the water balance model flow component and the mass balance model selenium transport component.  This is to be expected since the water balance storage component is the prime contributor to the mass balance selenium storage component.

%These figures show that uncertainty with the selenium storage component is very large.  This is to be expected since the mass balance models contain all of the uncertainty from the water balance model and the selenium concentration estimation linear models.

%The distribution of all realizations within each time step was analyzed to determine a distribution type.  This analysis was performed to determine if the assumption that the deterministic model results were representative of the stochastic model.  Testing was performed by comparing K-S statistics for the best fit normal, log-normal, logistic, exponential, gamma, and Weibull distributions.  In the USR 94\% of all atmospheric component time steps best fit a normal distribution, with the other 6\% best fit by a gamma distribution.  In the DSR 99\% of all atmospheric component time steps best fit a normal distribution, with the other 1\% best fit by a gamma distribution.  This indicates that for both the USR and DSR, the distributions across the realizations are normal with some slight skewness. 

%Tables \ref{tab:USRSeStore} and \ref{tab:DSRSeStore} presents summary statistics of the deterministic and stochastic atmospheric component.  Values are presented in units of \si{\cubic\meter\per\second\per\kilo\meter}.   These tables show the mean, 2.5th, and 97.5th percentile of the deterministic model and the same statistics applied to the three 1-D stochastic models.  This additional set of statistics from the 1-D stochastic 2.5th and 95.5th percentile models provide a better understanding of the extremes of the calculated values.  As before, comparison between the deterministic and stochastic models should be limited to the comparing the deterministic model to the 1-D stochastic mean model.  Also included in these two tables is the percent difference between the deterministic model and the 1-D stochastic mean model.  The mean, 2.5th, and 97.5th percentile values were calculated from the percent difference of the 1-D stochastic mean model from the deterministic model at each time step.  These values show the range of the variance from the deterministic model.

%\begin{table}[htbp]
%\centering
%\caption[USR river section deterministic and stochastic model selenium storage changes.]{USR river section deterministic and stochastic model selenium storage changes.  All values are in \si{\kilo\gram\per\day\per\kilo\meter}.}
%\label{tab:USRSeStore}
%\begin{tabular}{c|ccc}
%	\toprule
%	Model& 2.5\% & Mean & 97.5\% \\
%	\midrule
%	\midrule
%	Deterministic    &	-0.05792&	-0.0006277&	0.07671\\
%	\midrule                                            
%	Stochastic 2.5\% &	-0.1006&	-0.03036&	0.0348\\ 
%	Stochastic Mean  &	-0.05719&	-0.0001518&	0.07788\\
%	Stochastic 97.5\%&	-0.02281&	0.02978&	0.13\\   
%	\midrule                                            
%	\% Diff. Means&		-27.83&	-0.4654&	45.78\\
%	\bottomrule
%\end{tabular}
%\end{table}

%\begin{table}[htbp]
%\centering
%\caption[DSR river section deterministic and stochastic model selenium storage changes.]{DSR river section deterministic and stochastic model selenium storage changes.  Values are in units of \si{\kilo\gram\per\day\per\kilo\meter}.}
%\label{tab:DSRSeStore}
%\begin{tabular}{c|ccc}
%	\toprule
%	Model& 2.5\% & Mean & 97.5\% \\
%	\midrule
%	\midrule
%	Deterministic    &	-0.01058&	-0.0001905&	0.01124\\
%	\midrule                                              
%	Stochastic 2.5\% &	-0.02972&	-0.01362&	0.0001171\\
%	Stochastic Mean  &	-0.01272&	-0.0001761&	0.01265\\  
%	Stochastic 97.5\%&	0.0008917&	0.01324&	0.03121\\  
%	\midrule                                              
%	\% Diff. Means&		-139.8&	-9.668&	139.6\\
%	\bottomrule
%\end{tabular}
%\end{table}

%The mean of the percent difference between the deterministic and 1-D stochastic mean models is low, but not insignificant.  This indicates that the deterministic model is fairly representative of the stochastic model expected value.  The high percent differences at the 2.5th and 97.5th percentile indicate that there is still a large range of uncertainty contained within the stochastic model that the deterministic model cannot replicate.  The deterministic model can be used to determine how changes can affect a reach over a span of time, but using it to estimate values for specific time steps is unwise as the differences noted at individual time steps is too large to account for.
\clearpage

\section{Reach Selenium Surface Water Transport.}
\label{sec:Reach Selenium Surface Water Transport}

%The deterministic and stochastic mass balance models each consist of two major components.  The flow component is the sum of all selenium mass transport and the storage component is the sum of all river segment mass storage changes.  Mass transport is the mass that is transported in or out of the river reach by tributaries or canals, respectively.  Unlike the water balance model, the atmospheric model is not included.  As discussed earlier, there is insufficient evidence to support an estimation of selenium loss to the atmosphere.

%Values and figures presented in this section are the mass storage change component sub-set of the complete mass model results.  The analyses in this section are restricted to performing a reasonability check on the results.  Results in this section are to be compared with results from the other sub-sections from both the USR and DSR models to determine if results are acceptable.  Unacceptable results would indicate an error in the computational code or an underlying assumption.

%Figures \ref{fig:USRMassStore} and \ref{fig:DSRMassStore} present the sum of the selenium mass transport in the USR and DSR, respectively, as time series plots.  The left and right sub-figures present the determinstic and stochastic time series, respectively.  The black line in the stochastic figures is the 1-D stochastic mean results with the blue band indicating the 95\% CIR.  Values in these figures and associated tables indicate the selenium mass entering and leaving the respective reaches in units of \si{\kilo\gram\per\day\per\kilo\meter}.  Standardizing the results allows for comparisons to be made between the two study reaches and between mass balance model components.  Positive values indicate that the reach gained selenium during the given time step.

%\begin{figure}[htbp]
%\centering
%	\begin{subfigure}{0.5\textwidth}
%		\centering
%		\includegraphics[width=0.9\linewidth]{"Figures/Results_DUSR/Balance Mass - Flux"}
%		\caption{Deterministic Model.}
%		\label{sub:USRMassStoreD}
%	\end{subfigure}%
%	\begin{subfigure}{0.5\textwidth}
%		\centering
%		\includegraphics[width=0.9\linewidth]{"Figures/Results_USR/Balance Mass - Flux"}
%		\caption{Stochastic Model.}
%		\label{sub:USRMassStoreS}
%	\end{subfigure}
%	\caption[USR Arkansas River deterministic and stochastic surface water mas balance time series.]{USR Arkansas River deterministic and stochastic surface water mas balance time series.}
%	\label{fig:USRMassStore}
%\end{figure}

%\begin{figure}[htbp]
%\centering
%	\begin{subfigure}{0.5\textwidth}
%		\centering
%		\includegraphics[width=0.9\linewidth]{"Figures/Results_DDSR/Balance Mass - Flux"}
%		\caption{Deterministic Model.}
%		\label{sub:DSRMassStoreD}
%	\end{subfigure}%
%	\begin{subfigure}{0.5\textwidth}
%		\centering
%		\includegraphics[width=0.9\linewidth]{"Figures/Results_DSR/Balance Mass - Flux"}
%		\caption{Stochastic Model.}
%		\label{sub:DSRMassStoreS}
%	\end{subfigure}
%		\caption[DSR Arkansas River deterministic and stochastic surface water mas balance time series.]{DSR Arkansas River deterministic and stochastic surface water mas balance time series.}
%	\label{fig:DSRMassStore}
%\end{figure}

%The figures show that there is a definite seasonal variation in the selenium transport component.  This temporal relationship follows the same pattern identified with the water balance model flow component.  There is a visual relationship between the water balance model flow component and the mass balance model selenium transport component.  This is to be expected since the water balance storage component is the prime contributor to the mass balance selenium storage component.  This relationship is not as strong as seen between the water balance model storage component and the mass balance model selenium storage component.

%These figures show that uncertainty with the selenium storage component is very large.  This is to be expected since the mass balance models contain all of the uncertainty from the water balance model and the selenium concentration estimation linear models.  The magnitude of the flow component uncertainty is comparable to the magnitude of the storage componenet uncertainty.  This is expected since both model components use many of the same input variables with their uncertainties.

%The distribution of all realizations within each time step was analyzed to determine a distribution type.  This analysis was performed to determine if the assumption that the deterministic model results were representative of the stochastic model.  Testing was performed by comparing K-S statistics for the best fit normal, log-normal, logistic, exponential, gamma, and Weibull distributions.  In the USR 98\% of all atmospheric component time steps best fit a normal distribution, with the other 2\% best fit by a gamma distribution.  In the DSR 99.7\% of all atmospheric component time steps best fit a normal distribution, with the other 0.3\% best fit by a gamma distribution.  This indicates that for both the USR and DSR, the distributions across the realizations are normal. 

%Tables \ref{tab:USRSeFlow} and \ref{tab:DSRSeStore} presents summary statistics of the deterministic and stochastic atmospheric component.  Values are presented in units of \si{\cubic\meter\per\second\per\kilo\meter}.   These tables show the mean, 2.5th, and 97.5th percentile of the deterministic model and the same statistics applied to the three 1-D stochastic models.  This additional set of statistics from the 1-D stochastic 2.5th and 95.5th percentile models provide a better understanding of the extremes of the calculated values.  As before, comparison between the deterministic and stochastic models should be limited to the comparing the deterministic model to the 1-D stochastic mean model.  Also included in these two tables is the percent difference between the deterministic model and the 1-D stochastic mean model.  The mean, 2.5th, and 97.5th percentile values were calculated from the percent difference of the 1-D stochastic mean model from the deterministic model at each time step.  These values show the range of the variance from the deterministic model.

%\begin{table}[htbp]
%\centering
%\caption[USR river section selenium surface mass transport.]{USR river section selenium surface mass transport.  Stochastic mean values are calculated as the mean of the realizations for each time step. All values are in \si{\kilo\gram\per\day\per\kilo\meter}.}
%\label{tab:USRSeFlow}
%\begin{tabular}{c|ccc}
%	\toprule
%	Model& 2.5\% & Mean & 97.5\% \\
%	\midrule
%	\midrule
%	Deterministic    &	-0.1236&	-0.04412&	0.04339\\
%	\midrule                                           
%	Stochastic 2.5\% &	-0.1844&	-0.07846&	0.002247\\
%	Stochastic Mean  &	-0.1185&	-0.04296&	0.03445\\ 
%	Stochastic 97.5\%&	-0.07292&	-0.008939&	0.08867\\ 
%	\midrule                                           
%	\% Diff. Means&		-4.723&	4.348&	28.64\\
%	\bottomrule
%\end{tabular}
%\end{table}

%\begin{table}[htbp]
%\centering
%\caption[DSR river section selenium surface mass transport.]{DSR river section selenium surface mass transport.  Stochastic mean values are calculated as the mean of the realizations for each time step. All values are in \si{\kilo\gram\per\day\per\kilo\meter}.}
%\label{tab:DSRSeFlow}
%\begin{tabular}{c|ccc}
%	\toprule
%	Model& 2.5\% & Mean & 97.5\% \\
%	\midrule
%	\midrule
%	Deterministic    &	-0.1307	&-0.04625	&-0.01973  \\
%	\midrule                                           
%	Stochastic 2.5\% &	-0.2376&	-0.07634&	-0.03676\\
%	Stochastic Mean  &	-0.1158&	-0.0449&	-0.01902\\
%	Stochastic 97.5\%&	-0.03495&	-0.01452&	0.02444\\ 
%	\midrule                                           
%	\% Diff. Means&		-1.691&	0.6182&	16.99\\
%	\bottomrule
%\end{tabular}
%\end{table}

%The mean of the percent difference between the deterministic and 1-D stochastic mean models is very low.  This indicates that the deterministic model is representative of the stochastic model expected value.  The fairly low percent differences at the 2.5th and 97.5th percentile indicate that there is still a small but significan range of uncertainty contained within the stochastic model that the deterministic model cannot replicate.  The deterministic model can be used to determine how changes can affect a reach over a span of time.  Using it to estimate values for specific time steps is acceptable as long as the tollerance for uncertainty is acceptable.
\clearpage\]
\end{document}