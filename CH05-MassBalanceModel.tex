% !TeX root = MorseThesis.tex
% !TeX encoding = UTF-8
% !TeX spellcheck = en_US
\renewcommand{\thechapter}{5}
\chapter{Evaluation of NPS Selenium Loading to the River Using a Mass Balance Model}
\label{chap:MassBalanceModel}

\begin{linenumbers}
\section{Mass Balance Model Applied to the LARV}
\label{sec:AppliedMassModel}

The purpose of the mass balance model in this thesis is to determine the mass of unaccounted for dissolved selenium being transported into and out of the the study reaches in the LARV.  The mass transport is called the mass loading, where mass loading refers to mass entering the river channel and mass unloading refers to mass leaving the river channel.  The basic concept of the mass balance lies in Equation \ref{eq:mbal01}.
\begin{equation}
	\label{eq:mbal01}
	L=Q \cdot C
\end{equation}
\begin{tabular}{r p{5.5in}}
		Where:  \\
		$ L $ = & Mass loading (when positive) or mass unloading (when negative).\\
		$ Q $ = & Water flow rate.\\
		$ C $ = & Concentration of the dissolved constituent under investigation.\\
\end{tabular}\\

Using this equation as a basis of understanding and applying it to Equation \ref{eq:water1} in Chapter \ref{chap:WaterBalanceModel}, we arrive at a very basic equation mass balance model (Equation \ref{eq:mbal02}).
\begin{eqnarray}
	\label{eq:mbal02}
	\lefteqn{\frac{\Delta S_M}{\Delta t}=} \\
	\nonumber & &  Q_{in,US} \cdot C_{in,US} + \sum \left( Q_{in} \cdot C_{in} \right) + P \cdot C_P + R \cdot C_R  + B \cdot C_B\\
	\nonumber & & - Q_{out,DS} \cdot C{out,DS} - \sum \left( Q_{out} \cdot C_{out} \right) - E \cdot C_E - T \cdot C_T \\
	\nonumber & & - F \cdot C_F + X \cdot C_X
\end{eqnarray}\\

The $ C $ terms are the concentration for the dissolved constituent at each of the model design points.  Simplifying by using Equation \ref{eq:mbal01} results in Equation \ref{eq:mbal03}.
\begin{equation}
	\label{eq:mbal03}
	\frac{\Delta S_M}{\Delta t}= L_{in,US} + \sum L_{in} + L_P + L_R + L_B - L_{out,DS} - \sum L_{out} - L_E - L_T - L_F + L_X
\end{equation}
\begin{longtable}{r p{5.5in}}
	Where:  \\
	$ \displaystyle \frac{\Delta S_M}{\Delta t} $ = & Mass storage change in the study reach.\\
	$ L_{in,US} $ = & Mass loading to the river along the main stem of the river at the upstream end of the study reach. \\
	$ \displaystyle \sum L_{in} $ = & Sum of the mass loadings to the river from tributaries and other gauged sources.\\
	$ L_P $ = & Mass loading to the river from precipitation.\\
	$ L_R $ = & Mass loading to the river from precipitation runoff off of adjacent land.\\
	$ L_B $ = & Mass loading to the river from subsurface flow.\\
	$ L_{out,DS} $ = & Mass unloading along the main stem of the river at the downstream end of the study reach.\\
	$ \displaystyle \sum L_{out} $ = & Sum of the mass unloadings from the river to canals and other gauged sinks.\\
	$ L_E $ = & Mass unloading from the river due to evaporation.\\
	$ L_T $ = & Mass unloading from the river due to plant transpiration.\\
	$ L_F $ = & Mass unloading from the river due to infiltration into subsurface flow.\\
	$ L_X $ = & Mass loading/unloading to/from unknown sources and sinks.\\
\end{longtable}

Equation \ref{eq:mbal03} is a direct application of the two equations.  Some of the terms need to be re-defined to be more appropriate for the situation.  $ L_P $ is a possible process for many dissolved constituents, but it is unknown whether or not it occurs in any significant magnitude when the constituent is dissolved selenium.  Selenium does not naturally occur in the atmosphere.  It is transferred via biomethylation processes into the atmosphere.  This is the $ L_T $ term.  While the transport directions are the same, the $ L_E $ term is the direct volatilization of dissolved selenium species into the atmosphere.  At this time, volatilization has been found to be an insignificant factor in the transport of dissolved selenium species into the atmosphere.  

Using the same justifications as used in the previous chapter, Equation \ref{eq:mbal03} is simplified to define the mass loading from and mass loading to unaccounted for non-point source $\left( L_{UNPS} \right) $.
\begin{equation}
	\label{eq:mbal04}
	L_{UNPS} = \left( L_{out,DS} + \sum L_{out} - L_{in,US} - \sum L_{in} \right) + \frac{\Delta S_M}{\Delta t} \\ 
\end{equation}
\begin{tabular}{r p{5.5in}}
	Where: \\
	$ L_{UNPS} $ = & The sum of mass gains from non-point sources and losses to non-point sinks \\
		= &$ L_R + L_B - L_T -L_ F + L_{U,in} - L_{U,out} + L_P - L_E  L_X$.\\ 
\end{tabular}\\

This equation includes two terms not previously used.  $ L_{U,in} $ and $ L_{U,out} $ are subsets of $\displaystyle \sum L_{in} $ and $ \displaystyle \sum L_{out} $, respectively.  They are the ungauged and non-point source river reach gains and losses.
\clearpage

\section{Mass Storage Change}
\label{sec:MassStorageChange}

River water storage change calculations are developed and presented in Section \ref{sec:RiverStorageChange}.  Water storage change values between consecutive time steps are the basis for the calculations developed in this section.  Stored selenium mass changes between two consecutive time steps is dependent on the change in river stored water volume and the concentration of selenium in the water.  The stored selenium mass change in a given study region section is the sum of the stored selenium mass changes of the reaches within a study region river section.  Individual reach selenium storage changes are calculated using equation \ref{eq:mStore01}.

\begin{equation}
\frac{\Delta S_{M,i}}{\Delta t}=\frac{C_{in,i}+C_{out,i}}{2} \cdot \frac{\Delta S_i}{\Delta t}
\label{eq:mStore01}
\end{equation}
\begin{tabular}{r p{5.5in}}
	Where: \\
	$\displaystyle \frac{\Delta S_{M,i}}{\Delta t}$ = & Stored volume change in study reach segment $ i $ between time steps.\\
	$C_{in,i}$ =& Calculated dissolved selenium concentration at the upstream end of segment $ i $.\\
	$C_{out,i}$ =& Calculated dissolved selenium concentration at the downstream end of segment $ i $.\\
	$\displaystyle \frac{\Delta S_i}{\Delta t}$ =& Stored water volume change for segment $ i $ as calculated in Section \ref{sec:RiverStorageGeometry}.\\
\end{tabular}\\

Ideally, the average selenate concentration should be calculated as the difference between the current calculation day stored mass and the prior calculation day stored mass.  This requires that sufficient data is available to calculate the concentrations in the two consecutive calculation days.  The actual stored water volume was not calculated and therefore the stored selenium mass could not be calculated.  The methodology shown in equation \ref{eq:mStore01} was assumed to perform as an approximation of the stored selenium mass change between two consecutive days.

The stored water volume change is significantly larger than the average concentration.  Therefore it was assumed that small but significant changes in selenium concentrations between consecutive days would not significantly impact the stored selenium mass change.
\clearpage{}

\subsection{Solute Concentration Models}

One of the purposes of this thesis is to determine the selenium loading and unloading rate from unaccounted for sources/sinks.  Elemental selenium does not exist in an aqueous form.  The two dominant aqueous species are selenate and selnite.  Of these two, selenate is the most dominant to the extent that in most cases, selenite is immeasureable.  Therefore, throughout this thesis, selenate is used as the surrogate for all aqueous selenium species.  References in this thesis to dissolved selenium and aqueous selenium are in fact discussing aqueous selenate.  

All concentrations discussed in this thesis refer only to dissolved selenium concentration.  Therefore, the typical method of denoting concentration with the aqueous species in the subscript (i.e. $ C_{Se} $) is not going to be used in this thesis.  Instead, subscripts are being reserved to designate the location within a study region for which the concentration is used.  As such, Tables \ref{tab:concFlowStoreRelationship_USR} and \ref{tab:concFlowStoreRelationship_DSR} shows the symbolic relationship between the various gauged flow and  river segment storage changes and their associated dissolved selenium concentration for the USR and DSR, respectively.  The concentration symbols use subscripts that designate the sample location as noted in Figures \ref{map:USRSample} and \ref{map:DSRSample}.

\begin{table}[htbp]
	\centering
	\caption[USR gauged flow and river segment storage change symbolic relationship with dissolved selenium concentrations. ]{USR gauged flow and river segment storage change symbolic relationship with dissolved selenium concentrations.}
	\label{tab:concFlowStoreRelationship_USR}
	\begin{subtable}{\textwidth}
		\centering
		\subcaption*{USR gauged flow and aqueous selenium concentration relationships.}
		\begin{tabular}{c c c} 
			\toprule  
			Gauged Flow & & Concentration \\
			Symbol & & Symbol\\
			\toprule 
			$ Q_{ARKCATCO} $ & & $ C_{U163} $\\
			$ Q_{ARKLASCO} $ & & $ C_{U201} $\\
			$ Q_{CANSWKCO} $ & & $ C_{U74} $\\
			$ Q_{CONDITCO} $ & & $ C_{ARK,d=85.0} $\\
			$ Q_{FLSCANCO} $ & & $ C_{ARK,d=16.4} $\\
			$ Q_{FLYCANCO} $ & & $ C_{ARK,d=47.2} $\\
			$ Q_{HOLCANCO} $ & & $ C_{ARK,d=12.5} $\\
			$ Q_{HRC194CO} $ & & $ C_{U207} $\\
			$ Q_{RFDMANCO} $ & & $ C_{U167} $\\
			$ Q_{RFDRETCO} $ & & $ C_{U167} $\\
			$ Q_{TIMSWICO} $ & & $ C_{U60} $\\
			$ Q_{LAJWWTP} $ & & $ C_{LAJWWTP} $\\
			\bottomrule
		\end{tabular} \\
	\end{subtable}\\
	\tablevspace
	\begin{subtable}{\textwidth}
		\centering
		\subcaption*{USR river segment water volume change and aqueous selenium concentration relationship.}
		\begin{tabular}{c c c c c} 
			\toprule  
			River Segment & & US Concentration & & DS Concentration\\
			Symbol & & Symbol & & Symbol\\
			\toprule 
			$ \displaystyle \frac{\Delta S_A}{\Delta t} $ & & $ C_{U163} $ & & $ C_{ARK,d=12.5} $\\ \\
			$ \displaystyle \frac{\Delta S_B}{\Delta t} $ & & $ C_{ARK,d=12.5} $ & & $ C_{ARK,d=16.4} $\\ \\
			$ \displaystyle \frac{\Delta S_C}{\Delta t} $ & & $ C_{ARK,d=16.4} $ & & $ C_{ARK,d=47.2} $\\ \\
			$ \displaystyle \frac{\Delta S_D}{\Delta t} $ & & $ C_{ARK,d=47.2} $ & & $ C_{ARK,d=85.0} $\\ \\
			$ \displaystyle \frac{\Delta S_E}{\Delta t} $ & & $ C_{ARK,d=85.0} $ & & $ C_{U201} $\\
			\bottomrule
		\end{tabular} \\
	\end{subtable}\\
\end{table}

The concentration associated with the $ Q_{RFDRETCO} $ gauged flow is the same as that used for the $ Q_{RFDMANCO} $ gauged flow.  It was assumed that the two gauges would have the same dissolved selenium concentration due to their close proximity to each other.  The concentrations in the USR with the designation $ C_{ARK,d=x} $ are based on the concentrations for all locations in the main stem of the Arkansas River.  Since it is known that the concentration is not constant along the entire reach, a variable was needed to differentiate the concentration for the various irrigation canal diversions.  It was found that there was a slight correlation between the distance between the sample point and the upstream end of the study region and the dissolved selenium concentration.  Therefore, this distance was used and is noted in the subscript as $ x $, where $ x $ denotes the distance between the irrigation canal diversion and the upstream end of the study reach.  The La Junta WWTP does not have a stream gauge nor were selenium analyses performed by the university for this location.  Plant operators were kind enough to provide us with the total daily discharge from the plant in units of million gallons per day (mgd).  This value was converted to appropriate units as used in the previous chapter.  They also provided the monthly selenium analyses results.  These analyses were performed in accordance with State directives.  University researchers did not question the validity of the discharge or selenium concentration results.

\begin{table}[htbp]
	\centering
	\caption[DSR gauged flow and river segment storage change symbolic relationship with dissolved selenium concentrations. ]{DSR gauged flow and river segment storage change symbolic relationship with dissolved selenium concentrations.}
	\label{tab:concFlowStoreRelationship_DSR}
	\begin{subtable}{\textwidth}
		\centering
		\subcaption*{DSR gauged flow and aqueous selenium concentration relationships.}
		\begin{tabular}{c c c} 
			\toprule  
			Gauged Flow & & Concentration \\
			Symbol & & Symbol\\
			\toprule 
			$ Q_{ARKLAMCO} $ & & $ C_{D101C} $\\
			$ Q_{ARKCOOKS} $ & & $ C_{D106C} $\\
			$ Q_{BIGLAMCO} $ & & $ C_{D23} $\\
			$ Q_{BUFDITCO} $ & & $ C_{ARK,d=37.7} $\\
			$ Q_{FRODITKS} $ & & $ C_{D106C} $\\
			$ Q_{WILDHOCO} $ & & $ C_{D57} $\\
			\bottomrule
		\end{tabular} \\
	\end{subtable}\\
	\tablevspace
	\begin{subtable}{\textwidth}
		\centering
		\subcaption*{DSR river segment water volume change and aqueous selenium concentration relationship.}
		\begin{tabular}{c c c c c} 
			\toprule  
			River Segment & & US Concentration & & DS Concentration\\
			Symbol & & Symbol & & Symbol\\
			\toprule 
			$ \displaystyle \frac{\Delta S_F}{\Delta t} $ & & $ C_{D101C} $ & & $ C_{D104C} $\\ \\
			$ \displaystyle \frac{\Delta S_G}{\Delta t} $ & & $ C_{D104C} $ & & $ C_{D106C} $\\
			\bottomrule
		\end{tabular} \\
	\end{subtable}\\
\end{table}

As can be observed in Figure \ref{map:DSRSample}, the flow gauge for Buffalo Ditch ($ BUFDITCO $) has a sample location within close proximity on the channel.  The concentrations associated with this location ($ D36 $), were found to be inconsistent with concentrations at $ D104C $.  A chance encounter with the individual who owns the land immediately adjacent to the gauge location informed us that a small spring discharged from his property into Buffalo Ditch between the gauge location and the sample location.  The landowner informed us that the spring discharged at approximately \SI{0.056}{\cubic\meter\per\second} (\SI{2}{\cfs}).  He also stated that the flow rate was fairly constant throughout the year.  Since there was a significant variance between the measured concentrations at sample points $ D36 $ and $ D104C $ and since there is a known discharge into the channel with an unknown concentration, it was determined that the gauged flow $ Q_{BUFDITCO} $ would be associated with the concentration $ C_{ARK,d=37.7} $.

%\emph{IIIB - define lienar regression}\\

Dissolved selenium samples were taken as discussed in Chapter \ref{chap:data collection}.  Samples were not taken for every time step in the study time frame.  Concentration estimations for all sample locations except one were performed using linear regression.  Linear regression models are defined as models where the functions of the predictor variables are not themselves variable.  This is shown in equation \ref{eq:LinearModel}.  A regression model is considered linear if $f_i$ does not contain any fitting parameters $(\beta_i)$.
\begin{equation}
\label{eq:LinearModel}
\hat{y}=\beta_0+\beta_1 f_1+\beta_2 f_2+\ldots+\beta_n f_n
\end{equation}
\begin{tabular}{r p{5.5in}}
Where:&\\
$\hat{y}$ = & fitted or predicted value\\
$\beta_i$ = & fitting parameters\\
$f_i$ = & functions of the predictor variables $x_i$\\
\end{tabular}\\

Whenever possible, ordinary least squares regression was used to generate best fit equations with a given set of independent variables.  Pearson's r-squared value is used as an initial goodness-of-fit value so that individual linear regression models can be evaluated both independently and comparatively with regards to how well they fit the data.  R-squared values for linear models are positive, non-negative values less than one (1) with one (1) indicating a perfect fit.  R-squared values account for the percent of the dependent variable that can be accounted for in the independent variables.  The adjusted r-squared value is calculated and compared to the r-squared value.  This allows for some accounting for uncertainty when using multiple independent variables. If the two are considerably different, then the estimating model is missing an explanatory independent variable \parencite{Johnson2007}.

The f-statistic was also calculated and compared to the critical f-statistic for each estimating model. An f-statistic greater than the critical f-statistic indicates that at least one of the explanatory independent variables is linearly associated with the calculated dependent variable.  When comparing estimating models for suitability, the model with the greater f-statistic is more suitable \parencite{Johnson2007}.  The model f-statistic is returned as part of the statistics software linear regression summary.  The critical f-statistic is calculated using the f-distribution and the desired significance level and the degrees of freedom.

The significance level, $\alpha$, is closely tied to the desired confidence interval for this study.  Considering the number and source of the input variables, it was considered desirable to have all models calculated to account for 95\% of the variability.  This means that 5\% of the variability in any model can be attributed to chance.  With all models being two-tailed, the central inter-percentile range (CIR) was calculated for the range between 2.5\% and 97.5\% as a means to comprehend the daily change in variability.  This takes the 5\% unaccounted variability and distributes it to the two tails.  The $\alpha$ is not changed on account of the two-tailed nature of the models.

The two-sided p-value was used to determine if the estimating model independent variables were statistically significant.  The p-values greater than $\alpha$ indicated that the independent variable was not significant and did not contribute significantly to describe the variability of the dependent variable.  These variables were considered for removal during linear regression model optimization.

%\emph{IIIC - define independent variables used}\\



%\emph{IIID - method used to optimize models}\\

The selenium concentration field data collection effort provided excellent data for specific locations at specific times.  This data needed to be expanded to include as many intermediate locations and times as possible to provide a more complete set of results.  The methodology for determining the dissolved selenium concentration at various locations in both study region river sections is the same with only one exception.  The set of starting independent variables, or starting terms, changes depending on the specific location, but the method of reducing the equation to the final equation is the same.

Estimating equations were calculated using multivariate linear ordinary least squares regression.  The 'lm' function in the statistical software 'stats' package was used to fit linear models.   Measured selenium concentration values were initially fitted to equations that included the average daily stream flow, EC, and water temperature values for the same day as the selenium concentration sample was collected.  Equations were then refined through a process that will be discussed.

All data points were included in the linear regression even if some terms were missing.  The 'lm' function has an argument that allows the user to determine what should happen if missing data is encountered.  All analyses were performed such that data points with missing values were excluded from the analysis.  As the number of terms was reduced during the equation refinement process, these excluded data points may or may not be included in the analysis.  This allowed for the maximum number of data points without reforming the data set with each equation refinement iteration.

Determining which terms to include in each initial regression analysis began with visual analysis of an enhanced scatter-plot matrix of the selenium concentration response vector and the independent variable terms.  Figures~\ref{fig:concFullPairs_US} and \ref{fig:concFullPairs_US} contain the scatter-plot matricies for all concentration points in the USR and DSR, respectively.  The diagonal contains the variable names for the row and column.  The lower triangle shows the individual variables when plotted against each other.  Individual graphs are in appropriate unites for the investigated variables.  Flow is in units of \si{\cubic\meter\per\second}, EC is in units of \si{\deci\siemens\per\meter}, and temperature is in units of \si{\degreeCelsius}.  The upper triangle presents the Pearson correlation value for the respective variable pair.  Similar figures for the other regression analyses are included in the appendix.

\subfiguretop
\begin{landscape}
	\begin{figure}
		\begin{subfigure}{0.7\textwidth}
			\centering
			\includegraphics[width=\tableCustomSize]{"Figures/Results_USR/Stochastic/Conc Model Full PairsU163"}
			\subcaption*{$ C_{U163} $}		
		\end{subfigure}%
		\begin{subfigure}{0.7\textwidth}
			\centering
			\includegraphics[width=\tableCustomSize]{"Figures/Results_USR/Stochastic/Conc Model Full PairsU201"}
			\subcaption*{$ C_{U201} $}		
		\end{subfigure}\\
		\caption[Scatter-plot matricies of the input variables used to estimate dissolved selenium concentrations in the USR.]{Scatter-plot matricies of the input variables used to estimate dissolved selenium concentrations in the USR.  Variable names are plotted down the diagonal.  Values in the upper triangle are Pearson correlation values for the intersecting variables.  Scatter-plots for the intersecting variables are plotted in the lower triangle.  Scales are in the units for the given variable.  $C_{Se}$ is in \si{\micro\gram\per\liter}.  Q values are in \si{\cubic\meter\per\second}.  EC values are in \si{\deci\siemens\per\meter}.  T values are in \si{\degreeCelsius}.}
		\label{fig:concFullPairs_US}
	\end{figure}
\end{landscape}
\subfiguremid
\begin{landscape}
	\begin{figure}
		\begin{subfigure}{0.7\textwidth}
			\centering
			\includegraphics[width=\tableCustomSize]{"Figures/Results_USR/Stochastic/Conc Model Full PairsU167"}
			\subcaption*{$ C_{U167} $}		
		\end{subfigure}%
		\begin{subfigure}{0.7\textwidth}
			\centering			
			\includegraphics[width=\tableCustomSize]{"Figures/Results_USR/Stochastic/Conc Model Full PairsCAN"}
			\subcaption*{$ C_{U74} $}		
		\end{subfigure}\\
		\caption{Scatter-plot matricies of the input variables used to estimate dissolved selenium concentrations in the USR.}
	\end{figure}
\end{landscape}
\subfiguremid
\begin{landscape}
	\begin{figure}
		\begin{subfigure}{0.7\textwidth}
			\centering
			\includegraphics[width=\tableCustomSize]{"Figures/Results_USR/Stochastic/Conc Model Full PairsTIM"}
			\subcaption*{$ C_{60} $}		
		\end{subfigure}%
		\begin{subfigure}{0.7\textwidth}
			\centering			
			\includegraphics[width=\tableCustomSize]{"Figures/Results_USR/Stochastic/Conc Model Full PairsHRC"}
			\subcaption*{$ C_{U209} $}		
		\end{subfigure}\\
		\caption{Scatter-plot matricies of the input variables used to estimate dissolved selenium concentrations in the USR.}
	\end{figure}
\end{landscape}
\subfiguremid
\begin{landscape}
	\begin{figure}
		\begin{subfigure}{0.7\textwidth}
			\centering
			\includegraphics[width=\tableCustomSize]{"Figures/Results_USR/Stochastic/Conc Model Full PairsUDIV"}
			\subcaption*{$ C_{ARK,d=x} $}		
		\end{subfigure}%
		\begin{subfigure}{0.7\textwidth}
			\centering			
			\includegraphics[width=\tableCustomSize]{"Figures/Results_USR/Stochastic/Conc Model Full PairsWTP"}
			\subcaption*{$ C_{LAJWWTP} $}		
		\end{subfigure}\\
		\caption{Scatter-plot matricies of the input variables used to estimate dissolved selenium concentrations in the USR.}
	\end{figure}
\end{landscape}
\subfiguretop


\subfiguretop
\begin{landscape}
	\begin{figure}
		\begin{subfigure}{0.7\textwidth}
			\centering
			\includegraphics[width=\tableCustomSize]{"Figures/Results_USR/Stochastic/Conc Model Full PairsU163"}
			\subcaption*{$ C_{U163} $}		
		\end{subfigure}%
		\begin{subfigure}{0.7\textwidth}
			\centering
			\includegraphics[width=\tableCustomSize]{"Figures/Results_USR/Stochastic/Conc Model Full PairsU201"}
			\subcaption*{$ C_{U201} $}		
		\end{subfigure}\\
		\caption[Scatter-plot matricies of the input variables used to estimate dissolved selenium concentrations in the DSR.]{Scatter-plot matricies of the input variables used to estimate dissolved selenium concentrations in the DSR.  Variable names are plotted down the diagonal.  Values in the upper triangle are Pearson correlation values for the intersecting variables.  Scatter-plots for the intersecting variables are plotted in the lower triangle.  Scales are in the units for the given variable.  $C_{Se}$ is in \si{\micro\gram\per\liter}.  Q values are in \si{\cubic\meter\per\second}.  EC values are in \si{\deci\siemens\per\meter}.  T values are in \si{\degreeCelsius}.}
		\label{fig:concFullPairs_DS}
	\end{figure}
\end{landscape}
\subfiguremid
\begin{landscape}
	\begin{figure}
		\begin{subfigure}{0.7\textwidth}
			\centering
			\includegraphics[width=\tableCustomSize]{"Figures/Results_DSR/Stochastic/Conc Model Full PairsD101C"}
			\subcaption*{$ C_{D101C} $}		
		\end{subfigure}%
		\begin{subfigure}{0.7\textwidth}
			\centering			
			\includegraphics[width=\tableCustomSize]{"Figures/Results_DSR/Stochastic/Conc Model Full PairsD106C"}
			\subcaption*{$ C_{D106C} $}		
		\end{subfigure}\\
		\caption{Scatter-plot matricies of the input variables used to estimate dissolved selenium concentrations in the DSR.}
	\end{figure}
\end{landscape}
\subfiguremid
\begin{landscape}
	\begin{figure}
		\begin{subfigure}{0.7\textwidth}
			\centering
			\includegraphics[width=\tableCustomSize]{"Figures/Results_DSR/Stochastic/Conc Model Full PairsBIG"}
			\subcaption*{$ C_{D23} $}		
		\end{subfigure}%
		\begin{subfigure}{0.7\textwidth}
			\centering			
			\includegraphics[width=\tableCustomSize]{"Figures/Results_DSR/Stochastic/Conc Model Full PairsWIL"}
			\subcaption*{$ C_{D57} $}		
		\end{subfigure}\\
		\caption{Scatter-plot matricies of the input variables used to estimate dissolved selenium concentrations in the DSR.}
	\end{figure}
\end{landscape}
\subfiguremid
\begin{landscape}
	\begin{figure}
		\begin{subfigure}{0.7\textwidth}
			\centering
			\includegraphics[width=\tableCustomSize]{"Figures/Results_DSR/Stochastic/Conc Model Full PairsDDIV"}
			\subcaption*{$ C_{D23} $}		
		\end{subfigure}//
		\caption{Scatter-plot matricies of the input variables used to estimate dissolved selenium concentrations in the DSR.}
	\end{figure}
\end{landscape}
\subfiguretop

In this case and as with most other regression analyses, there are strong correlations between terms which would indicate that removing all but the EC value would give an adequate estimation.  The EC, which is related to the total salt concentration, is proportional to $C_{Se}$.  However, temperature and flow are negatively correlated, compared with EC's positive correlation with $C_{Se}$.  An increase in flow volume dilutes the total salt load and therefore $C_{Se}$.  Temperature affects the solubility of all salts.  Including temperature allows the difference in solubility constants to be expressed.  Also, flow has opposing correlations with EC and temperature.  These trends show that although EC alone can be used to estimate $C_{Se}$, it may not be not sufficient to completely describe how $C_{Se}$ reacts to the environment.

An initial regression form was used and individual terms were removed based on their individual $p$ value.  Only one variable was removed at a time and the new, reduced form was re-analyzed.  This step-wise method was performed until the $p$ value for the individual variables was less than an $\alpha$ of 0.05.  Residual standard error values, r-squared values, adjusted r-squared values, F-statistics, and p-values were collected for each analysis.  If the statistics did not appear to support the conclusion that the final, reduced form linear mode was representative of the data, then the process was re-started with a different, more complex, initial regression form.

%\emph{IIIF - present optimized models}\\\\

Tables \ref{tab:USRInitialRegression} and \ref{tab:DSRInitialRegression} present the initial regression forms for the river sections in the USR and DSR respectively. These initial regression forms presented do not include the resulting coefficients for the respective variables as they are insignificant to the results of this particular analysis and the complete model results. 

\begin{table}[htbp]
\centering
\caption{USR Selenium Concentration Initial Regression Models.}
\label{tab:USRInitialRegression}
\begin{tabular}{c l}
	\toprule
	Concentration		& Initial regression model.\\
	\toprule
	$ C_{U163} $	& $Q_{ARKCATCO} + EC_{ARKCATCO} + t_{ARKCATCO}$\\
	\\
	$ C_{U201} $	& $Q_{ARKLASCO} + EC_{ARKLASCO} + t_{ARKLASCO}$\\
	\\	
	$ C_{U167} $ & $Q_{ARKCATCO} + EC_{ARKCATCO} + Q_{ARKLASCO}$\\
					& $+ EC_{ARKLASCO} + t_{ARKLASCO}$\\
	\\	
	$ C_{U74} $		& $Q_{ARKCATCO} + EC_{ARKCATCO} + Q_{ARKLASCO}$\\ 
					& $+ EC_{ARKLASCO} + t_{ARKLASCO} + Q_{CANSWKCO}$\\
					& $+ Q_{ARKCATCO}Q_{CANSWKCO} + EC_{ARKCATCO}Q_{CANSWKCO}$\\
					& $+ Q_{ARKLASCO}Q_{CANSWKCO} + EC_{ARKLASCO}Q_{CANSWKCO}$\\
					& $+ t_{ARKLASCO}Q_{CANSWKCO}$\\
	\\
	$ C_{U60} $	& $Q_{ARKCATCO} + EC_{ARKCATCO} + Q_{ARKLASCO}$\\
					& $+ EC_{ARKLASCO} + t_{ARKLASCO} + Q_{TIMSWICO}$\\
	\\
	$ C_{U207} $& $Q_{ARKCATCO} + EC_{ARKCATCO} + Q_{ARKLASCO}$\\
					& $+ EC_{ARKLASCO} + t_{ARKLASCO} + Q_{HRC194CO}$\\
	\\	
	$ C_{ARK,d=x} $	& $Q_{ARKCATCO} + EC_{ARKCATCO} + Q_{ARKLASCO}$\\
					& $+ EC_{ARKLASCO} + t_{ARKLASCO} + d$\\
	\\	
	$ C_{LAJWWTP} $	& $\beta_{1} \cdot Q_{WTP}^{\beta_{2}}$\\
	\bottomrule \\
\end{tabular}
\end{table}

\begin{table}[htbp]
\centering
\caption{USR Selenium Concentration Initial Regression Models.}
\label{tab:DSRInitialRegression}
\begin{tabular}{c l}
	\toprule
	Concentration		& Initial regression model. \\
	\toprule
	$ C_{D101C} $	& $Q_{ARKLAMCO} + EC_{ARKJMRCO} + t_{ARKJMRCO}$\\
					& $+Q_{ARKLAMCO}^2 + Q_{ARKLAMCO}EC_{ARKJMRCO}$\\
					& $+ Q_{ARKLAMCO}t_{ARKJMRCO} + EC_{ARKJMRCO}^2$\\
					& $+ EC_{ARKJMRCO}t_{ARKJMRCO} + t_{ARKJMRCO}^2$\\
	\\
	$ C_{D106C} $	& $Q_{ARKCOOKS} + EC_{ARKCOOKS} + t_{ARKCOOKS}$\\
					& $+Q_{ARKCOOKS}^2 + Q_{ARKCOOKS}EC_{ARKCOOKS}$\\
					& $+ Q_{ARKCOOKS}t_{ARKCOOKS} + EC_{ARKCOOKS}^2$\\
					& $+ EC_{ARKCOOKS}t_{ARKCOOKS} + t_{ARKCOOKS}^2$\\
	\\
	$ C_{D23} $		& $Q_{ARKLAMCO} + EC_{ARKJMRCO} + Q_{ARKCOOKS}$\\ 
					& $+ EC_{ARKCOOKS} + t_{ARKCOOKS} + Q_{BIGLAMCO}$\\
					& $+ Q_{ARKLAMCO}Q_{BIGLAMCO} + EC_{ARKJMRCO}Q_{BIGLAMCO}$\\
					& $+ Q_{ARKCOOKS}Q_{BIGLAMCO} + EC_{ARKCOOKS}Q_{BIGLAMCO}$\\
					& $+ t_{ARKCOOKS}Q_{BIGLAMCO}$\\
	\\
	$ C_{D57} $		& $Q_{ARKLAMCO} + EC_{ARKJMRCO} + t_{ARKJMRCO} + Q_{ARKCOOKS}$\\
					& $+ EC_{ARKCOOKS} + Q_{WILDHOCO} + Q_{ARKLAMCO}^2 $\\
					& $+ Q_{ARKLAMCO} EC_{ARKJMRCO}+ Q_{ARKLAMCO} t_{ARKJMRCO} $\\
					& $+ Q_{ARKLAMCO} Q_{ARKCOOKS}+ Q_{ARKLAMCO} EC_{ARKCOOKS} $\\
					& $+ Q_{ARKLAMCO} Q_{WILDHOCO}+ EC_{ARKJMRCO}^2 $\\
					& $+ EC_{ARKJMRCO} t_{ARKJMRCO}+ EC_{ARKJMRCO} Q_{ARKCOOKS} $\\
					& $+ EC_{ARKJMRCO} EC_{ARKCOOKS}+ EC_{ARKJMRCO} Q_{WILDHOCO} $\\
					& $+ t_{ARKJMRCO}^2+ t_{ARKJMRCO} Q_{ARKCOOKS} $\\
					& $+ t_{ARKJMRCO} EC_{ARKCOOKS}+ t_{ARKJMRCO} Q_{WILDHOCO} $\\
					& $+ Q_{ARKCOOKS}^2+ Q_{ARKCOOKS} EC_{ARKCOOKS} $\\
					& $+ Q_{ARKCOOKS} Q_{WILDHOCO}+ EC_{ARKCOOKS}^2 $\\
					& $+ EC_{ARKCOOKS} Q_{WILDHOCO}+ Q_{WILDHOCO}^2$\\
	\\
	$ C_{ARK,d=x} $	& $Q_{ARKLAMCO} + EC_{ARKJMRCO} + Q_{ARKCOOKS}$ \\
					& $+ EC_{ARKCOOKS} + t_{ARKCOOKS} + d$\\
	\bottomrule \\
\end{tabular}
\end{table}

Selenium concentration at the La Junta WWTP was not able to be sufficiently estimated using linear models.  The La Junta water distribution system receives most of its water from wells.  It was not known which aquifer supplied the city.  It was assumed that the shallow river riparian aquifer was the city's water source as there are no known deep aquifers in the area.  It was also assumed that the water treatment plant processes and waste water treatment plant processes could change the selenium concentration.  The average daily flow produced by the plant is the only variable available to estimate selenium concentrations.  Visual analysis of the scatter plot similar to Figure \ref{fig:ExampleFullPairs} showed that the relationship between the plant discharge rate and selenium concentration could not be easily defined.  A non-linear model was used with the power function described in Table \ref{tab:USRInitialRegression}.  This model produced fairly reasonable results.  Better results could be obtained if average daily EC values of the plant discharge were available.

Tables \ref{tab:USRFinalRegression} and \ref{tab:DSRFinalRegression} present the final, reduced regression equations with coefficients resulting from the regression analysis.

\begin{table}[htbp]
	\centering
	\caption{USR Selenium Concentration Final Regression Models.}
	\label{tab:USRFinalRegression}
	\begin{tabular}{c l}
		\toprule
		Concentration		& Final regression model. \\
		\toprule
		$ C_{U163} $ & $-0.05106 \cdot Q_{ARKCATCO} + 4.69 \cdot EC_{ARKCATCO}$\\
		& $ -0.3063 \cdot t_{ARKCATCO} + 10.12$ \\
		$ C_{U201} $	& $-0.3138 \cdot Q_{ARKLASCO} -0.1348 \cdot t_{ARKLASCO} + 14.91$\\
		\\	
		$ C_{U167} $ & $-0.3538 \cdot Q_{ARKLASCO} -2.021 \cdot EC_{ARKLASCO}$\\
		& $ -0.3306 \cdot t_{ARKLASCO} + 21.15$\\
		\\	
		$ C_{U74} $		& $ -13.01 \cdot EC_{ARKCATCO} -1.022 \cdot Q_{ARKLASCO}-0.4132 \cdot t_{ARKLASCO}$\\ 
		& $ -16.40 \cdot Q_{CANSWKCO}-0.3138 \cdot Q_{ARKCATCO}Q_{CANSWKCO}$\\
		& $+2.0730 Q_{ARKLASCO}Q_{CANSWKCO} + 39.40$\\
		\\
		$ C_{U60} $	& $11.42 \cdot EC_{ARKCATCO} -3.160 \cdot Q_{TIMSWICO} + 4.605$\\
		\\
		$ C_{U207} $& $-17.78 \cdot Q_{HRC194CO} + 20.41$\\
		\\	
		$ C_{ARK,d=x} $	& $4.710 \cdot EC_{ARKCATCO} -0.1568 \cdot Q_{ARKLASCO}-0.1491 \cdot t_{ARKLASCO}$\\
		& $+0.0203 \cdot d + 8.0831$\\
		\\	
		$ C_{LAJWWTP} $	& $19.18 \cdot Q_{WTP}^{0.07664}$\\
		\bottomrule \\
	\end{tabular}
\end{table}

\begin{table}[htbp]
	\centering
	\caption{DSR Selenium Concentration Final Regression Models.}
	\label{tab:DSRFinalRegression}
	\begin{tabular}{c l}
		\toprule
		Concentration	& Final regression model.\\
		\toprule
		$ C_{D101C} $	& $3.429 \cdot EC_{ARKJMRCO} - 0.005623 \cdot Q_{ARKLAMCO}EC_{ARKJMRCO}$\\
		& $-0.07581 \cdot EC_{ARKJMRCO}t_{ARKJMRCO} + 6.355$\\
		\\
		$ C_{D106C} $	& $-0.37 \cdot t_{ARKCOOKS}+ 0.0627 \cdot EC_{ARKCOOKS}t_{ARKCOOKS} + 16.56$\\
		\\
		$ C_{D23} $		& $0.08951 \cdot Q_{ARKLAMCO} -0.9925 \cdot Q_{ARKCOOKS}$\\
		& $-1.376 \cdot Q_{BIGLAMCO} -0.007387 \cdot Q_{ARKLAMCO}Q_{BIGLAMCO}$\\
		& $+0.006882 \cdot Q_{ARKCOOKS}Q_{BIGLAMCO} + 36.96$\\ 
		\\
		$ C_{D57} $		& $-42.64 \cdot EC_{ARKJMRCO} -3.309 \cdot t_{ARKJMRCO}$\\
		& $-14.18 \cdot EC_{ARKCOOKS} -0.006969 \cdot Q_{ARKLAMCO}Q_{WILDHOCO}$\\
		& $+0.005895 \cdot Q_{ARKCOOKS}Q_{WILDHOCO}$\\
		& $-0.2522 \cdot Q_{WILDHOCO}EC_{ARKJMRCO}$\\
		& $+5.779 \cdot EC_{ARKJMRCO}EC_{ARKCOOKS}$\\
		& $+1.029 \cdot EC_{ARKJMRCO}t_{ARKJMRCO}$\\
		& $+0.2643 \cdot EC_{ARKCOOKS}t_{ARKJMRCO} -112.7$\\
		\\
		$ C_{ARK,d=x} $		& $-0.006576 \cdot Q_{ARKLAMCO} +1.322 \cdot EC_{ARKJMRCO}$\\
		& $+0.003794 \cdot Q_{ARKCOOKS} +0.4512 \cdot EC_{ARKCOOKS}$\\
		& $-0.07653 \cdot t_{ARKCOOKS} +0.1066 \cdot d + 5.709$\\
		\bottomrule \\
	\end{tabular}
\end{table}

Tables \ref{tab:USRSumStat} and \ref{tab:DSRSumStat} present the summary statistics from the regression analysis.  The relative standard error with the degrees of freedom (RSE, DoF), multiple R-squared, adjusted R-squared, the F-statistic with the degrees of freedom (F-statistic, DoF), and the p-value generated by the statistical software after regression analyses are presented in these tables.  All fitted linear regression equations show statistics that indicate that they fit the measured data with a fairly high degree of accuracy.

\begin{table}[htbp]
\centering
\caption{USR Selenium Concentration Regression Models Summary Statistics.}
\label{tab:USRSumStat}
\begin{tabular}{lccccc}
	\toprule
	\multirow{2}{*}{Concentration} & \multirow{2}{*}{RSE, DoF} & Multiple  & Adjusted  &   F-statistic,   & \multirow{2}{*}{p-value} \\
	                               &                           & R-squared & R-squared &       DoF        &  \\ \toprule
	$ C_{U163} $                   &        0.5923, 11         &  0.9818   &  0.9769   & 198.3, 3 and 11  &        7.4\e{-10}        \\
	$ C_{U201} $                   &        0.9467, 14         &  0.9145   &  0.9022   & 74.84, 2 and 14  &        3.3\e{-8}         \\
	$ C_{U167} $                   &        0.7016, 10         &  0.9684   &  0.9589   & 102.1, 3 and 10  &        8.4\e{-8}         \\
	$ C_{U74} $                    &        0.9868, 10         &  0.9501   &  0.9202   & 31.74, 6 and 10  &        6.0\e{-6}         \\
	$ C_{U60} $                    &         1.749, 15         &  0.8633   &   0.845   & 47.35, 2 and 15  &        3.3\e{-7}         \\
	$ C_{ARK,d=x} $                &        1.196, 130         &  0.8808   &  0.8772   & 240.2, 4 and 130 &       $ < $2.2\e{-16}        \\
	$ C_{LAJWWTP} $                &         4.153, 85         &    NA     &    NA     &        NA        &            NA            \\ \bottomrule
	\multicolumn{6}{l}{\footnotesize RSE = Residual standard error}                                                                  \\
	\multicolumn{6}{l}{\footnotesize DoF = Degrees of Freedom}
\end{tabular}
\end{table}

\begin{table}[htbp]
\centering
\caption{DSR Selenium Concentration Regression Models Summary Statistics.}
\label{tab:DSRSumStat}
\begin{tabular}{lccccc}
	\toprule
	\multirow{2}{*}{Concentration} & \multirow{2}{*}{RSE, DoF} & Multiple  & Adjusted  &   F-statistic,   & \multirow{2}{*}{p-value} \\
	                               &                           & R-squared & R-squared &       DoF        &  \\ \toprule
	$ C_{D101C} $                  &         2.553, 38         &  0.5472   &  0.5114   & 15.31, 3 and 38  &        1.1\e{-6}         \\
	$ C_{D106C} $                  &         3.534, 39         &  0.3404   &  0.3065   & 10.06, 2 and 36  &        3.0\e{-4}         \\
	$ C_{D23} $                    &         5.701, 37         &  0.5819   &  0.5254   &  10.3, 5 and 37  &        3.5\e{-6}         \\
	$ C_{D57} $                    &         2.337, 21         &  0.7278   &  0.6111   & 3.238, 9 and 21  &        2.7\e{-4}         \\
	$ C_{ARK,d=x} $                &        1.724, 156         &  0.7104   &  0.6993   & 63.78, 6 and 156 &       <2.2\e{-16}        \\ \bottomrule
	\multicolumn{6}{l}{\footnotesize RSE = Residual standard error}                                                                  \\
	\multicolumn{6}{l}{\footnotesize DoF = Degrees of Freedom}
\end{tabular}
\end{table}

Individual regression models were analyzed to determine if they were representative of the data.  Figures \ref{fig:concLmFit_US} and \ref{fig:concLmFit_DS} graphs used for each concentration point to make this analysis.  For each sub-figure, the top left panel shows the residuals plotted against the fitted values.  All points should be evenly distributed in both directions.  If the points are not symmetrical along the fitted values axis, then heteroscedasticity should be suspected.  If the points are not symmetrical about either axis, then the regression model is missing an estimating term.  The bottom left panel shows a scale-location plot of the data.  This panel shows the same information as the first panel with the exception that the square-root of the standardized residuals is used.  This reduces the effect of skewness on the analysis.  Like the first panel, the points should be symmetrical along both axes.  The top right panel shows a normal quantile-quantile (Q-Q) plot of the residuals.  It is a reasonable assumption that residuals of a linear model are normally distributed.  If the residuals are normally distributed, then the points should fit closely to the line at $y=x$.  The bottom right panel shows the residuals plotted against the leverage of the residuals.  The dashed red lines show the Cook's distance which is a measure of the influence a particular data point has on the regression model.  Points with a higher leverage have a higher influence on the model.  Points with leverage values higher than the majority of the data may be outliers.

In all panels, potential outliers are indicated by having the index number of the data printed next to the point.  If the same data points are indicated as outliers on most of the panels, then it is most likely true that they are outliers.  None of the models had the outliers removed to improve the model.  In fact, outliers were expected.  Most of the dissolved selenium samples were collected during similar times in the year.  Bias towards certain flow regimes and other factors are most likely present in the data.  The outliers represent the samples taken outside of the normal sampling season and are more representative of the extremes.  These outliers are necessary to the complete analysis and were not removed from the linear regression analyses.

\subfiguretop
\begin{landscape}
	\begin{figure}
		\begin{subfigure}{0.7\textwidth}
			\centering
			\includegraphics[width=\tableCustomSize]{"Figures/Results_USR/Stochastic/Conc Model lm-fit U163"}
			\subcaption*{$ C_{U163} $}		
		\end{subfigure}%
		\begin{subfigure}{0.7\textwidth}
			\centering
			\includegraphics[width=\tableCustomSize]{"Figures/Results_USR/Stochastic/Conc Model lm-fit U201"}
			\subcaption*{$ C_{U163} $}		
		\end{subfigure}\\
		\caption[USR selenium concentration linear model analysis graphs.]{USR selenium concentration linear model analysis graphs. The black lines indicate the stochastic model mean time-series.  The blue band indicates the 95\% central inter-percentile range (CIR) of the stochastic values.  The red dashed line in the flow depth portion of the figure indicates the reported flow depth values.}
		\label{fig:concLmFit_US}
	\end{figure}
\end{landscape}

\subfiguremid
\begin{landscape}
	\begin{figure}
		\begin{subfigure}{0.7\textwidth}
			\centering
			\includegraphics[width=\tableCustomSize]{"Figures/Results_USR/Stochastic/Conc Model lm-fit U167"}
			\subcaption*{$ C_{U167} $}		
		\end{subfigure}%
		\begin{subfigure}{0.7\textwidth}
			\centering
			\includegraphics[width=\tableCustomSize]{"Figures/Results_USR/Stochastic/Conc Model lm-fit CAN"}
			\subcaption*{$ C_{U74} $}		
		\end{subfigure}\\
		\caption{USR selenium concentration linear model analysis graphs.}
	\end{figure}
\end{landscape}

\subfiguremid
\begin{landscape}
	\begin{figure}
		\begin{subfigure}{0.7\textwidth}
			\centering
			\includegraphics[width=\tableCustomSize]{"Figures/Results_USR/Stochastic/Conc Model lm-fit TIM"}
			\subcaption*{$ C_{U60} $}		
		\end{subfigure}%
		\begin{subfigure}{0.7\textwidth}
			\centering
			\includegraphics[width=\tableCustomSize]{"Figures/Results_USR/Stochastic/Conc Model lm-fit HRC"}
			\subcaption*{$ C_{U207} $}		
		\end{subfigure}\\
		\caption{USR selenium concentration linear model analysis graphs.}
	\end{figure}
\end{landscape}

\subfiguremid
\begin{landscape}
	\begin{figure}
		\begin{subfigure}{0.7\textwidth}
			\centering
			\includegraphics[width=\tableCustomSize]{"Figures/Results_USR/Stochastic/Conc Model lm-fit UDIV"}
			\subcaption*{$ C_{ARK,d=x} $}		
		\end{subfigure}\\
		\caption{USR selenium concentration linear model analysis graphs.}
	\end{figure}
\end{landscape}

\subfiguretop
\begin{landscape}
	\begin{figure}
		\begin{subfigure}{0.7\textwidth}
			\centering
			\includegraphics[width=\tableCustomSize]{"Figures/Results_DSR/Stochastic/Conc Model lm-fit D101C"}
			\subcaption*{$ C_{D101C} $}		
		\end{subfigure}%
		\begin{subfigure}{0.7\textwidth}
			\centering
			\includegraphics[width=\tableCustomSize]{"Figures/Results_DSR/Stochastic/Conc Model lm-fit D106C"}
			\subcaption*{$ C_{D106C} $}		
		\end{subfigure}\\
		\caption[DSR selenium concentration linear model analysis graphs.]{DSR selenium concentration linear model analysis graphs. The black lines indicate the stochastic model mean time-series.  The blue band indicates the 95\% central inter-percentile range (CIR) of the stochastic values.  The red dashed line in the flow depth portion of the figure indicates the reported flow depth values.}
		\label{fig:concLmFit_DS}
	\end{figure}
\end{landscape}

\subfiguremid
\begin{landscape}
	\begin{figure}
		\begin{subfigure}{0.7\textwidth}
			\centering
			\includegraphics[width=\tableCustomSize]{"Figures/Results_DSR/Stochastic/Conc Model lm-fit BIG"}
			\subcaption*{$ C_{D23} $}		
		\end{subfigure}%
		\begin{subfigure}{0.7\textwidth}
			\centering
			\includegraphics[width=\tableCustomSize]{"Figures/Results_DSR/Stochastic/Conc Model lm-fit WIL"}
			\subcaption*{$ C_{D57} $}		
		\end{subfigure}\\
		\caption{DSR selenium concentration linear model analysis graphs.}
	\end{figure}
\end{landscape}

\subfiguremid
\begin{landscape}
	\begin{figure}
		\begin{subfigure}{0.7\textwidth}
			\centering
			\includegraphics[width=\tableCustomSize]{"Figures/Results_DSR/Stochastic/Conc Model lm-fit DDIV"}
			\subcaption*{$ C_{ARK,d=x} $}		
		\end{subfigure}\\
		\caption{DSR selenium concentration linear model analysis graphs.}
	\end{figure}
\end{landscape}
\subfiguretop

Since the dissolved selenium concentration at the La Junta WWTP is derived using a non-linear equation, the graphs used for linear models are not suitable.  In order to prevent a biased comparison between linear and non-linear models and their respective goodness-of-fit, it was determined to judge the model results for the La Junta WWTP selenium concentration on its own merits.  The purpose of this model development was to determine which model provided the lowest root mean squared error while simultaneously ensuring that heteroskedasticity was not present.  The final $ C_{LAJWWTP} $ model was analyzed with the graphs shown in Figure \ref{fig:concLajwwtpNLS}.  The upper sub-figures shows a plot of the fitted values versus the residuals.  This plot would show if there was a correlation between these two values, which is heteroskedasticity.  The lower sub-figures shows an autocorrelation analysis of the residuals.  In a well formed model, the residuals should not be correlated with themselves.  Like heteroskedasticity, this indicates that there is a factor missing in the model.  Figure \ref{fig:concLajwwtpNLS} shows that both heteroskedasticity and autocorrelation are not significant factors in the final $ C_{LAJWWTP} $ analysis.

\begin{figure}[htbp]
	\centering
	\includegraphics[width=\tableCustomSize]{"Figures/Results_USR/Stochastic/Conc Model lm-fit WTP"}
	\caption[La Junta WWTP selenium concentration residuals analysis.]{La Junta WWTP selenium concentration residuals analysis.}
	\label{fig:concLajwwtpNLS}
\end{figure}

The measured concentration values were compared to the predicted concentration values for all regression analyses.  Figures \ref{fig:concPredVMeas_US} and \ref{fig:concPredVMeas_US} show the graphs used in this comparison.  An $y=x$ line was plotted .  Points below the line show that the estimating equation under-estimated the selenium concentration.  The vertical distance between the point and the line corresponds to the estimate error, or residual.  Similar figures for the other regression analyses are presented in the appendix.

\subfiguretop
\begin{landscape}
	\begin{figure}
		\begin{subfigure}{0.7\textwidth}
			\centering
			\includegraphics[width=\tableCustomSize]{"Figures/Results_USR/Stochastic/Conc Model pred v meas U163"}
			\subcaption*{$ C_{U163} $}		
		\end{subfigure}%
		\begin{subfigure}{0.7\textwidth}
			\centering
			\includegraphics[width=\tableCustomSize]{"Figures/Results_USR/Stochastic/Conc Model pred v meas U201"}
			\subcaption*{$ C_{U201} $}		
		\end{subfigure}\\
		\caption[USR measured vs. estimated selenium concentration comparison.]{USR measured vs. estimated selenium concentration comparison.  The predicted values are calculated using the respective final regression equation.  The diagonal line has a slope of 1 passing through the origin to show the over or under estimation.  All concentrations are presented in \si{\micro\gram\per\liter}.}
		\label{fig:concPredVMeas_US}
	\end{figure}
\end{landscape}

\subfiguremid
\begin{landscape}
	\begin{figure}
		\begin{subfigure}{0.7\textwidth}
			\centering
			\includegraphics[width=\tableCustomSize]{"Figures/Results_USR/Stochastic/Conc Model pred v meas U167"}
			\subcaption*{$ C_{U167} $}		
		\end{subfigure}%
		\begin{subfigure}{0.7\textwidth}
			\centering
			\includegraphics[width=\tableCustomSize]{"Figures/Results_USR/Stochastic/Conc Model pred v meas CAN"}
			\subcaption*{$ C_{U74} $}		
		\end{subfigure}\\
		\caption{USR selenium concentration linear model analysis graphs.}
	\end{figure}
\end{landscape}

\subfiguremid
\begin{landscape}
	\begin{figure}
		\begin{subfigure}{0.7\textwidth}
			\centering
			\includegraphics[width=\tableCustomSize]{"Figures/Results_USR/Stochastic/Conc Model pred v meas TIM"}
			\subcaption*{$ C_{U60} $}		
		\end{subfigure}%
		\begin{subfigure}{0.7\textwidth}
			\centering
			\includegraphics[width=\tableCustomSize]{"Figures/Results_USR/Stochastic/Conc Model pred v meas HRC"}
			\subcaption*{$ C_{U207} $}		
		\end{subfigure}\\
		\caption{USR selenium concentration linear model analysis graphs.}
	\end{figure}
\end{landscape}

\subfiguremid
\begin{landscape}
	\begin{figure}
		\begin{subfigure}{0.7\textwidth}
			\centering
			\includegraphics[width=\tableCustomSize]{"Figures/Results_USR/Stochastic/Conc Model pred v meas UDIV"}
			\subcaption*{$ C_{ARK,d=x} $}		
		\end{subfigure}%
		\begin{subfigure}{0.7\textwidth}
			\centering
			\includegraphics[width=\tableCustomSize]{"Figures/Results_USR/Stochastic/Conc Model pred v meas WTP"}
			\subcaption*{$ C_{LAJWWTP} $}		
		\end{subfigure}\\
		\caption{USR selenium concentration linear model analysis graphs.}
	\end{figure}
\end{landscape}

\subfiguretop
\begin{landscape}
	\begin{figure}
		\begin{subfigure}{0.7\textwidth}
			\centering
			\includegraphics[width=\tableCustomSize]{"Figures/Results_DSR/Stochastic/Conc Model pred v meas D101C"}
			\subcaption*{$ C_{D101C} $}		
		\end{subfigure}%
		\begin{subfigure}{0.7\textwidth}
			\centering
			\includegraphics[width=\tableCustomSize]{"Figures/Results_DSR/Stochastic/Conc Model pred v meas D106C"}
			\subcaption*{$ C_{D106C} $}		
		\end{subfigure}\\
		\caption[DSR measured vs. estimated selenium concentration comparison.]{DSR measured vs. estimated selenium concentration comparison.  The predicted values are calculated using the respective final regression equation.  The diagonal line has a slope of 1 passing through the origin to show the over or under estimation.  All concentrations are presented in \si{\micro\gram\per\liter}.}
		\label{fig:concPredVMeas_DS}
	\end{figure}
\end{landscape}

\subfiguremid
\begin{landscape}
	\begin{figure}
		\begin{subfigure}{0.7\textwidth}
			\centering
			\includegraphics[width=\tableCustomSize]{"Figures/Results_DSR/Stochastic/Conc Model pred v meas BIG"}
			\subcaption*{$ C_{D23} $}		
		\end{subfigure}%
		\begin{subfigure}{0.7\textwidth}
			\centering
			\includegraphics[width=\tableCustomSize]{"Figures/Results_DSR/Stochastic/Conc Model pred v meas WIL"}
			\subcaption*{$ C_{D57} $}		
		\end{subfigure}\\
		\caption{DSR selenium concentration linear model analysis graphs.}
	\end{figure}
\end{landscape}

\subfiguremid
\begin{landscape}
	\begin{figure}
		\begin{subfigure}{0.7\textwidth}
			\centering
			\includegraphics[width=\tableCustomSize]{"Figures/Results_DSR/Stochastic/Conc Model pred v meas DDIV"}
			\subcaption*{$ C_{ARK,d=x} $}		
		\end{subfigure}\\
		\caption{DSR selenium concentration linear model analysis graphs.}
	\end{figure}
\end{landscape}
\subfiguretop

Each of the dissolved selenium estimating equations is accompanied by two uncertainty values as show in Equation \ref{eq:concError01}.  The uncertainty found in the regression equation $ \varepsilon_1 $ is due solely to the inability of the a regression equation to accurately describe the measured dissolved selenium concentration values.  
\begin{equation}
\label{eq:concError01}
C_x + \varepsilon_1 + \varepsilon_2
\end{equation}
\begin{tabular}{r p{5.5in}}
	Where: \\
	$ C_x $ = & A calculated concentration value.  X denotes any one of the concentration locations.\\
	$ \varepsilon_1 $ = & Uncertainty found in the regression equation.\\
	$ \varepsilon_2 $ = & Uncertainty in the measured dissolved selenium concentration lab values.\\
\end{tabular}\\

Selenium estimation calculation error, $\varepsilon_{1}$ was analyzed to determine the best fit distribution for each selenium estimating equation.  Normal and logistic distributions were fit to the regression model residuals.  Both of these distributions are unbounded and are simple to apply.  They also fit the assumption that the linear model residuals are normally distributed.  Logistic distributions were included because they are very similar to normal distributions, but with heavier tails.  The best-fit normal and logistic distributions were compared to the regression model residuals by using Kolmogorov-Smirnov, Cramer von Mises, and Anderson-Darling goodness-of-fit tests.  The results of these test determined which of either the best fit normal distribution or best fit logistic distribution described the regression model residuals.  Results from the goodness-of-fit tests are presented in Tables~\ref{tab:USRGoF} and \ref{tab:DSRGof} for the USR and DSR, respectively


\begin{table}[htbp]
  \centering
  \caption[USR selenium concentration residuals goodness-of-fit test results.]{USR selenium concentration residuals goodness-of-fit test results.  Kolmogorov-Smirnov (K-S), Cramer von Mieses (CvM), and Anderson-Darling (AD) test statistics are presented for each regression model.}
    \begin{tabular}{ccccc}
    \toprule
    \multirow{2}{*}{Concentration}&Tested & \multicolumn{3}{c}{Test Statistics} \\ \cmidrule(r{.5em}l){3-5}
    &Distribution  & K-S   & CvM   & A-D \\
    \toprule
    \multirow{2}{*}{$ C_{U163} $}			&Logistic*	&0.107	&0.022	&0.151 \\
    								&Normal		&0.116	&0.028	&0.197 \\
    \midrule
    \multirow{2}{*}{$ C_{U201} $}			&Logistic*	&0.180	&0.129	&0.766	\\
    								&Normal		&0.178	&0.140	&0.792	\\
    \midrule
    \multirow{2}{*}{$ C_{U167} $}		&Logistic*	&0.091	&0.020	&0.165	\\
    								&Normal		&0.100	&0.025	&0.192	\\
    \midrule
    \multirow{2}{*}{$ C_{U74} $}		&Logistic*	&0.136	&0.053	&0.315	\\
    								&Normal		&0.140	&0.076	&0.419	\\
    \midrule
    \multirow{2}{*}{$ C_{U60} $}		&Logistic*	&0.128	&0.076	&0.476	\\
    								&Normal		&0.161	&0.086	&0.548	\\
    \midrule
    \multirow{2}{*}{$ C_{ARK,d=x} $}		&Logistic*	&0.041	&0.032	&0.352	\\
    								&Normal		&0.109	&0.354	&2.47	\\
    \midrule
    \multirow{2}{*}{$ C_{LAJWWTP} $}	&Logistic*	&0.090	&0.163	&1.22	\\
    								&Normal		&0.130	&0.281	&1.57	\\
    \bottomrule
    \multicolumn{5}{l}{\footnotesize * = best fit distribution}\\
    \end{tabular}%
  \label{tab:USRGoF}%
\end{table}%

\begin{table}[htbp]
  \centering
  \caption[DSR selenium concentration residuals goodness-of-fit test results.]{DSR selenium concentration residuals goodness-of-fit test results.  Kolmogorov-Smirnov (K-S), Cramer von Mieses (CvM), and Anderson-Darling (AD) test statistics are presented for each regression model.}
    \begin{tabular}{lcccc}
    \toprule
    \multirow{2}{*}{Location}&Tested & \multicolumn{3}{c}{Test Statistics} \\ \cmidrule(r{.5em}l){3-5}
    &Distribution  & K-S   & CvM   & A-D \\
    \toprule
    \multirow{2}{*}{$ C_{D101C} $}			&Logistic	&0.0923	&0.038	&0.263	\\
    								&Normal*	&0.090	&0.378	&0.259	\\
    \midrule
    \multirow{2}{*}{$ C_{D106C} $}			&Logistic*	&0.082	&0.289	&0.199	\\
    								&Normal		&0.110	&0.060	&0.384	\\
    \midrule
    \multirow{2}{*}{$ C_{D23} $}		&Logistic	&0.103	&0.082	&0.508	\\
    								&Normal*	&0.092	&0.073	&0.426	\\
    \midrule
    \multirow{2}{*}{$ C_{D57} $}		&Logistic*	&0.117	&0.058	&0.340	\\
    								&Normal		&0.140	&0.093	&0.518	\\
	\midrule
    \multirow{2}{*}{$ C_{ARK,d=x} $}		&Logistic*	&0.041	&0.020	&0.112	\\
    								&Normal		&0.043	&0.050	&0.319	\\
    \bottomrule
    \multicolumn{5}{l}{\footnotesize * = best fit distribution}\\
    \end{tabular}%
  \label{tab:DSRGof}%
\end{table}%

The summaries of the fitted distributions are presented in Tables \ref{tab:USRResStat} and \ref{tab:DSRResStat}, for the USR and DSR respectively.  In all but one case, the location parameters are near zero.  The location parameter for logistic distributions and the mean parameter for normal distributions provide the same information; they describe the central tendency of the distribution.  For distributions that describe model error, the goal is to have this value near zero.  The location parameter for the La Junta WWTP selenium concentration error distribution is a significant distance from zero.  The selenium concentration lab results minimum detection level is \SI{0.4}{\micro\gram\per\liter} and the La Junta WWTP location parameter approaches this value.  This indicates that the selenium concentration estimating model for the La Junta WWTP is missing an estimating parameter.  Unfortunately, no other parameters were available for the collected data.

\begin{table}
  \caption[USR selenium concentration residual distribution summary statistics.]{USR selenium concentration residual distribution summary statistics.}
  \label{tab:USRResStat}
  \centering
    \begin{tabular}{lcccc}
    	\toprule
    	\multirow{2}{*}{Concentration} &   Best Fit   & \multirow{2}{*}{n} &         \multicolumn{2}{c}{Parameter Estimate}         \\
    	\cmidrule(r{.5em}l){4-5}       & Distribution &                    & Param. $1^{1}$ &            Param. $2^{2}$             \\ \toprule
    	$ C_{U163} $                   &   Logistic   &         15         &   5.7\e{-3}    &                0.2810                 \\
    	$ C_{U201} $                   &   Logistic   &         17         &   -3.0\e{-2}   &                0.5365                 \\
    	$ C_{U167} $                   &   Logistic   &         14         &   1.5\e{-2}    &                0.3383                 \\
    	$ C_{U74} $                    &   Logistic   &         17         &   1.7\e{-3}    &                0.4049                 \\
    	$ C_{U60} $                    &   Logistic   &         18         &   -6.3\e{-2}   &                0.8743                 \\
    	$ C_{ARK,d=x} $                &   Logistic   &        135         &   -5.2\e{-2}   &                0.5615                 \\
    	$ C_{LAJWWTP} $                &   Logistic   &         87         &   -3.8\e{-1}   &                 2.313                 \\ \bottomrule
    	\multicolumn{5}{l}{\footnotesize $^{1}$  For normal distributions, mean.  For logistic distributions, location.}            \\
    	\multicolumn{5}{l}{\footnotesize $^{2}$  For normal distributions, standard deviation.  For logistic distributions, scale.}
    \end{tabular}%
\end{table}%

\begin{table}
  \caption[DSR selenium concentration residual distribution summary statistics.]{DSR selenium concentration residual distribution summary statistics.}
  \label{tab:DSRResStat}
  \centering
    \begin{tabular}{ccccc}
    	\toprule
    	\multirow{2}{*}{Concentration} &   Best Fit   & \multirow{2}{*}{n} &         \multicolumn{2}{c}{Parameter Estimate}          \\
    	   \cmidrule(r{.5em}l){4-5}    & Distribution &                    & Param. $1^{1}$ &             Param. $2^{2}$             \\ \toprule
    	        $ C_{D101C} $          &    Normal    &         42         &  -1.7\e{-17}   &                 2.429                  \\
    	        $ C_{D106C} $          &   Logistic   &         42         &   7.5\e{-2}    &                 1.834                  \\
    	         $ C_{D23} $           &    Normal    &         43         &  -1.2\e{-16}   &                 5.289                  \\
    	         $ C_{D57} $           &   Logistic   &         31         &   -8.4\e{-3}   &                 1.026                  \\
    	       $ C_{ARK,d=x} $         &   Logistic   &        163         &   1.2\e{-2}    &                 0.9364                 \\ \bottomrule
    	\multicolumn{5}{l}{\footnotesize $^{1}$ = For normal distributions, mean.  For logistic distributions, location.}            \\
    	\multicolumn{5}{l}{\footnotesize $^{2}$ = For normal distributions, standard deviation.  For logistic distributions, scale.}
    \end{tabular}%
\end{table}%

Statistical plots of the residual distributions, shown in Figures \ref{fig:concRes-Fit_US} and \ref{fig:concRes-Fit_US}, were created to visually analyze the data distribution and determine if the chosen distribution represented the data for the USR and DSR, respectively.  These figures are diagnostic plots that are produced by statistical software.  They were not altered or customized.  The top left panel shows a histogram of the residuals.  The red curve is the chosen best-fit distribution.  The bottom left panel shows the cumulative distribution function of the chosen distribution in red.  The points represent the actual collected data.  In a well fit distribution, the points should lie on or near the fitted distribution cumulative distribution function line.  The top right panel shows the well known quantile-quantile plot.  This is a visual test for normalcy.  The points should lie on or near the line $y=x$ if the distribution is normal.  The focus on this panel is to see how well the tails fit a normal distribution.  The bottom right panel is a probability-probability plot.  This is also a visual test for normalcy and the points should lie on or near the line $y=x$ if the distribution is normal.  The focus on this panel is to see how well the center of the data fits a normal distribution.

\subfiguretop
\begin{landscape}
	\begin{figure}
		\begin{subfigure}{0.7\textwidth}
			\centering
			\includegraphics[width=\tableCustomSize]{"Figures/Results_USR/Stochastic/Conc Model res-fit U163"}
			\subcaption*{$ C_{U163} $}		
		\end{subfigure}%
		\begin{subfigure}{0.7\textwidth}
			\centering
			\includegraphics[width=\tableCustomSize]{"Figures/Results_USR/Stochastic/Conc Model res-fit U201"}
			\subcaption*{$ C_{U201} $}		
		\end{subfigure}\\
		\caption[USR upstream boundary selenium estimate residual distribution analysis.]{USR upstream boundary selenium estimate residual distribution analysis.  The top left plot is a histogram of the residuals with the estimated logistic distribution plotted over top.  The top right plot is a quantile-quantile (Q-Q).  The bottom left is a plot of the theoretical cumulative distribution function (CDF) against the empirical CDF.  The bottom right is a probability-probability plot.}
		\label{fig:concRes-Fit_US}
	\end{figure}
\end{landscape}

\subfiguremid
\begin{landscape}
	\begin{figure}
		\begin{subfigure}{0.7\textwidth}
			\centering
			\includegraphics[width=\tableCustomSize]{"Figures/Results_USR/Stochastic/Conc Model res-fit U167"}
			\subcaption*{$ C_{U167} $}		
		\end{subfigure}%
		\begin{subfigure}{0.7\textwidth}
			\centering
			\includegraphics[width=\tableCustomSize]{"Figures/Results_USR/Stochastic/Conc Model res-fit CAN"}
			\subcaption*{$ C_{U74} $}		
		\end{subfigure}\\
		\caption{USR upstream boundary selenium estimate residual distribution analysis.}
	\end{figure}
\end{landscape}

\subfiguremid
\begin{landscape}
	\begin{figure}
		\begin{subfigure}{0.7\textwidth}
			\centering
			\includegraphics[width=\tableCustomSize]{"Figures/Results_USR/Stochastic/Conc Model res-fit TIM"}
			\subcaption*{$ C_{U60} $}		
		\end{subfigure}%
		\begin{subfigure}{0.7\textwidth}
			\centering
			\includegraphics[width=\tableCustomSize]{"Figures/Results_USR/Stochastic/Conc Model res-fit HRC"}
			\subcaption*{$ C_{U207} $}		
		\end{subfigure}\\
		\caption{USR upstream boundary selenium estimate residual distribution analysis.}
	\end{figure}
\end{landscape}

\subfiguremid
\begin{landscape}
	\begin{figure}
		\begin{subfigure}{0.7\textwidth}
			\centering
			\includegraphics[width=\tableCustomSize]{"Figures/Results_USR/Stochastic/Conc Model res-fit UDIV"}
			\subcaption*{$ C_{ARK,d=x} $}		
		\end{subfigure}%
		\begin{subfigure}{0.7\textwidth}
			\centering
			\includegraphics[width=\tableCustomSize]{"Figures/Results_USR/Stochastic/Conc Model res-fit WTP"}
			\subcaption*{$ C_{LAJWWTP} $}		
		\end{subfigure}\\
		\caption{USR upstream boundary selenium estimate residual distribution analysis.}
	\end{figure}
\end{landscape}

\subfiguretop
\begin{landscape}
	\begin{figure}
		\begin{subfigure}{0.7\textwidth}
			\centering
			\includegraphics[width=\tableCustomSize]{"Figures/Results_DSR/Stochastic/Conc Model res-fit D101C"}
			\subcaption*{$ C_{D101C} $}		
		\end{subfigure}%
		\begin{subfigure}{0.7\textwidth}
			\centering
			\includegraphics[width=\tableCustomSize]{"Figures/Results_DSR/Stochastic/Conc Model res-fit D106C"}
			\subcaption*{$ C_{D106C} $}		
		\end{subfigure}\\
		\caption[DSR upstream boundary selenium estimate residual distribution analysis.]{DSR upstream boundary selenium estimate residual distribution analysis.  The top left plot is a histogram of the residuals with the estimated logistic distribution plotted over top.  The top right plot is a quantile-quantile (Q-Q).  The bottom left is a plot of the theoretical cumulative distribution function (CDF) against the empirical CDF.  The bottom right is a probability-probability plot.}
		\label{fig:concRes-Fit_DS}
	\end{figure}
\end{landscape}

\subfiguremid
\begin{landscape}
	\begin{figure}
		\begin{subfigure}{0.7\textwidth}
			\centering
			\includegraphics[width=\tableCustomSize]{"Figures/Results_DSR/Stochastic/Conc Model res-fit BIG"}
			\subcaption*{$ C_{D23} $}		
		\end{subfigure}%
		\begin{subfigure}{0.7\textwidth}
			\centering
			\includegraphics[width=\tableCustomSize]{"Figures/Results_DSR/Stochastic/Conc Model res-fit WIL"}
			\subcaption*{$ C_{D57} $}		
		\end{subfigure}\\
		\caption{DSR upstream boundary selenium estimate residual distribution analysis.}
	\end{figure}
\end{landscape}

\subfiguremid
\begin{landscape}
	\begin{figure}
		\begin{subfigure}{0.7\textwidth}
			\centering
			\includegraphics[width=\tableCustomSize]{"Figures/Results_DSR/Stochastic/Conc Model res-fit DDIV"}
			\subcaption*{$ C_{ARK,d=x} $}		
		\end{subfigure}\\
		\caption{DSR upstream boundary selenium estimate residual distribution analysis.}
	\end{figure}
\end{landscape}
\subfiguretop

A number of the chosen residual distributions do not appear to fit the data.  The residuals for the USR outlet appear to not be normally distributed.  The distributions for CANSWKCO and TIMSWICO do not appear to be good fits.  The data and distribution for HRC194CO does not have enough data to allow for a conclusive analysis.  For the most part, the lack of a good fit can be traced back to data collection.  First, for the poorly fit distributions, there was insufficient data collected.  Second, there was a tendency for the samples to be taken during the same time frame each year.  If selenium concentration is seasonal, then the samples should have been taken relatively equally spaced throughout the year to capture the seasonal variation.

These selenium concentration error distributions were not discarded due to these findings.  It was assumed that the error could be best described by the best fit distributions determined in this analysis.  Future data should be included in future analyses to develop more accurate estimation models and error distribution.

To test statistical software's ability to generate the required distribution, residuals were plotted as a histogram overlain with a kernel density estimate as shown in Figures~\ref{fig:concResHist_US} and \ref{fig:concResHist_DS} for the USR and DSR, respectively.  The black line is the kernel density estimate of the residuals.  The red line is the kernel density estimate of 5000 draws from the best fit regression model.  In spite of the earlier findings that some of the best fit selenium concentration error distributions are not good fits, all graphs visually indicated that the fitted distributions are not as poor as expected.

\subfiguretop
\begin{landscape}
	\begin{figure}
		\begin{subfigure}{0.7\textwidth}
			\centering
			\includegraphics[width=\tableCustomSize]{"Figures/Results_USR/Stochastic/Conc Model ResDist U163"}
			\subcaption*{$ C_{U163} $}		
		\end{subfigure}%
		\begin{subfigure}{0.7\textwidth}
			\centering
			\includegraphics[width=\tableCustomSize]{"Figures/Results_USR/Stochastic/Conc Model ResDist U201"}
			\subcaption*{$ C_{U201} $}		
		\end{subfigure}\\
		\caption[USR selenium estimate residual histogram.]{USR selenium estimate residual histogram.  The kernel density of the residuals is plotted over the histogram.  Similar plots for the rest of the linear model analyses are included in the appendix.}
		\label{fig:concResHist_US}
	\end{figure}
\end{landscape}

\subfiguremid
\begin{landscape}
	\begin{figure}
		\begin{subfigure}{0.7\textwidth}
			\centering
			\includegraphics[width=\tableCustomSize]{"Figures/Results_USR/Stochastic/Conc Model ResDist U167"}
			\subcaption*{$ C_{U167} $}		
		\end{subfigure}%
		\begin{subfigure}{0.7\textwidth}
			\centering
			\includegraphics[width=\tableCustomSize]{"Figures/Results_USR/Stochastic/Conc Model ResDist CAN"}
			\subcaption*{$ C_{U74} $}		
		\end{subfigure}\\
		\caption{USR selenium estimate residual histogram.}
	\end{figure}
\end{landscape}

\subfiguremid
\begin{landscape}
	\begin{figure}
		\begin{subfigure}{0.7\textwidth}
			\centering
			\includegraphics[width=\tableCustomSize]{"Figures/Results_USR/Stochastic/Conc Model ResDist TIM"}
			\subcaption*{$ C_{U60} $}		
		\end{subfigure}%
		\begin{subfigure}{0.7\textwidth}
			\centering
			\includegraphics[width=\tableCustomSize]{"Figures/Results_USR/Stochastic/Conc Model ResDist HRC"}
			\subcaption*{$ C_{U207} $}		
		\end{subfigure}\\
		\caption{USR selenium estimate residual histogram.}
	\end{figure}
\end{landscape}

\subfiguremid
\begin{landscape}
	\begin{figure}
		\begin{subfigure}{0.7\textwidth}
			\centering
			\includegraphics[width=\tableCustomSize]{"Figures/Results_USR/Stochastic/Conc Model ResDist UDIV"}
			\subcaption*{$ C_{ARK,d=x} $}		
		\end{subfigure}%
		\begin{subfigure}{0.7\textwidth}
			\centering
			\includegraphics[width=\tableCustomSize]{"Figures/Results_USR/Stochastic/Conc Model ResDist WTP"}
			\subcaption*{$ C_{LAJWWTP} $}		
		\end{subfigure}\\
		\caption{USR selenium estimate residual histogram.}
	\end{figure}
\end{landscape}

\subfiguretop
\begin{landscape}
	\begin{figure}
		\begin{subfigure}{0.7\textwidth}
			\centering
			\includegraphics[width=\tableCustomSize]{"Figures/Results_DSR/Stochastic/Conc Model ResDist D101C"}
			\subcaption*{$ C_{D101C} $}		
		\end{subfigure}%
		\begin{subfigure}{0.7\textwidth}
			\centering
			\includegraphics[width=\tableCustomSize]{"Figures/Results_DSR/Stochastic/Conc Model ResDist D106C"}
			\subcaption*{$ C_{D106C} $}		
		\end{subfigure}\\
		\caption[DSR selenium estimate residual histogram.]{DSR selenium estimate residual histogram.  The kernel density of the residuals is plotted over the histogram.  Similar plots for the rest of the linear model analyses are included in the appendix.}
		\label{fig:concResHist_DS}
	\end{figure}
\end{landscape}

\subfiguremid
\begin{landscape}
	\begin{figure}
		\begin{subfigure}{0.7\textwidth}
			\centering
			\includegraphics[width=\tableCustomSize]{"Figures/Results_DSR/Stochastic/Conc Model ResDist BIG"}
			\subcaption*{$ C_{D23} $}		
		\end{subfigure}%
		\begin{subfigure}{0.7\textwidth}
			\centering
			\includegraphics[width=\tableCustomSize]{"Figures/Results_DSR/Stochastic/Conc Model ResDist WIL"}
			\subcaption*{$ C_{D57} $}		
		\end{subfigure}\\
		\caption{DSR selenium estimate residual histogram.}
	\end{figure}
\end{landscape}

\subfiguremid
\begin{landscape}
	\begin{figure}
		\begin{subfigure}{0.7\textwidth}
			\centering
			\includegraphics[width=\tableCustomSize]{"Figures/Results_DSR/Stochastic/Conc Model ResDist DDIV"}
			\subcaption*{$ C_{ARK,d=x} $}		
		\end{subfigure}\\
		\caption{DSR selenium estimate residual histogram.}
	\end{figure}
\end{landscape}
\subfiguretop

%\emph{IIIE - test optimized models}\\

The primary purpose of the next set of analyses was to determine if the computational code and assumptions used to generate the stochastic distributions of dissolved selenium concentrations were performed correctly. The first analysis was to compare the calculated dissolved selenium concentration values to the measured results from the collected field samples.  Figure \ref{fig:BoxMUSR} is a box plot of the sampled selenium concentrations at the various sampling locations along the main stem of the river and the tributaries in the USR.  The sample locations are arranged with the upstream on the left and the downstream on the right, in order.  The value "n" above each sample location is the number of samples collected at each site.  Concentrations are measured and reported in \si{\micro\gram\per\liter} of dissolved selenium.  The boxes encompass the first to the third quartile.  The whiskers extend to 1.5 times the inter quartile range.  Blue tinted boxes indicate dissolved selenium concentrations within tributaries all other boxes are from samples collected within the main stem of the Ark R.

Concentrations for the Rocky Ford Return Ditch in the USR and Frontier Ditch in the DSR are not included.  Both ditches are assumed to have the same dissolved selenium concentration as a nearby calculated location.  The Rocky Ford Return Ditch (RFDRETCO) uses the same concentration as the Rocky Ford Ditch (RFDCANCO) as it returns water from the main ditch to the Arkansas R. less than \SI{1}{\kilo\meter} downstream of the main ditch head gate.  The Frontier Ditch (FRODITKS) diverts water near the downstream end of the DSR and uses the concentrations calculated for this point.

\begin{figure}[htbp]
\centering
	\includegraphics[width=6in]{"Figures/Results_USR/Stochastic/c BOX Measure CSe"}
	\caption[Measured Dissolved Selenium Concentrations in the USR.]{Measured Dissolved Selenium Concentrations in the USR.}
	\label{fig:BoxMUSR}
\end{figure}

Figure \ref{fig:BoxCUSR} is a box plot of the calculated estimated selenium concentration at the various key points in the USR mass balance model.  The value "n" indicates the number of steps in the time series.  The values used in the box plot are from the 1-D mean stochastic model.  The blue tinted boxes indicate calculated dissolved selenium concentrations within tributaries and the tan tinted boxes indicate calculated dissolved selenium concentrations at the irrigation canal head gates.

\begin{figure}[htbp]
\centering
	\includegraphics[width=6in]{"Figures/Results_USR/Stochastic/c BOX Estimated CSe"}
	\caption[Calculated Dissolved Selenium Concentrations in the USR.]{Calculated Dissolved Selenium Concentrations in the USR.}
	\label{fig:BoxCUSR}
\end{figure}

Figure \ref{fig:BoxMDSR} is a box plot of the measured dissolved selenium concentrations at sample points in the main stem of the river and its main tributaries in the DSR.  This plot is similar in fashion to figure \ref{fig:BoxMUSR}.

\begin{figure}[htbp]
\centering
	\includegraphics[width=6in]{"Figures/Results_DSR/Stochastic/c BOX Measure CSe"}
	\caption[Measured Dissolved Selenium Concentrations in the DSR.]{Measured Dissolved Selenium Concentrations in the DSR.}
	\label{fig:BoxMDSR}
\end{figure}

Figure \ref{fig:BoxCDSR} is a box plot of the measured dissolved selenium concentrations at sample points in the main stem of the river and its main tributaries in the DSR.  This plot is similar in fashion to figure \ref{fig:BoxCUSR}.

\begin{figure}[htbp]
\centering
	\includegraphics[width=6in]{"Figures/Results_DSR/Stochastic/c BOX Estimated CSe"}
	\caption[Calculated Dissolved Selenium Concentrations in the DSR.]{Calculated Dissolved Selenium Concentrations in the DSR.}
	\label{fig:BoxCDSR}
\end{figure}
These four figures (\ref{fig:BoxMUSR} to \ref{fig:BoxCDSR}) compare the measured dissolved selenium concentration values with the estimated values.  These figures are used along with tables \ref{tab:USRlocinfo} and \ref{tab:DSRlocinfo} in chapter \ref{chap:study regions} to make this comparison.  Study region sample locations along the Arkansas R. are not necessarily located at the same places where calculated dissolved selenium concentrations are required.

In all cases but one, the graphs support the statement that the calculated selenium concentration values are representative of the actual recorded values.  Timpas Creek (TIMSWICO) in the USR appears to be the only exception.  Here it appears that the calculated dissolved selenium concentrations are far lower than the measured values.  

This is possibly caused by three factors.  The first is the sampling frequency.  The sample results represented in figure \ref{fig:BoxMUSR} are not a uniform representation of the possible concentration values throughout a calendar year.  The sample values more heavily consider three months, March, May, and July, and either minimal consider or ignore all other months.  The second is the nature of flows within Timpas Creek.  The lower portion of the creek serves as a return flow channel for field irrigation runoff.  The selenium concentration of the runoff and the effects of other water constituents are not known.  

The second analysis was to compare the results between the deterministic model time-series and the stochastic model mean time-series results to determine if there is any unacceptable variance between the models.  The distributions of the dissolved selenium concentrations calculated for both the deterministic and stochastic models were graphically compared in Figures \ref{fig:concCSeDist_US} and \ref{fig:concCSeDist_DS} for the USR and DSR, respectively.  The histogram and the black KDE are of the calculated deterministic model values.  The red dashed KDE represents the distribution of the stochastic model mean time-series.  Similar figures for all calculated concentrations are provided in the appendix.  The third factor is the uncertainty with which dissolved selenium concentrations were calculated for Timpas Creek.

\subfiguretop
\begin{landscape}
	\begin{figure}
		\begin{subfigure}{0.7\textwidth}
			\centering
			\includegraphics[width=\tableCustomSize]{"Figures/Results_USR/Stochastic/c d&s est U163"}
			\subcaption*{$ C_{U163} $}		
		\end{subfigure}%
		\begin{subfigure}{0.7\textwidth}
			\centering
			\includegraphics[width=\tableCustomSize]{"Figures/Results_USR/Stochastic/c d&s est U201"}
			\subcaption*{$ C_{U201} $}		
		\end{subfigure}\\
		\caption[USR dissolved selenium concentration distribution analysis.]{USR dissolved selenium concentration distribution analysis.  The histogram and the black KDE are of the calculated deterministic model values.  The red dashed KDE represents the distribution of the stochastic model mean time series.}
		\label{fig:concCSeDist_US}
	\end{figure}
\end{landscape}

\subfiguremid
\begin{landscape}
	\begin{figure}
		\begin{subfigure}{0.7\textwidth}
			\centering
			\includegraphics[width=\tableCustomSize]{"Figures/Results_USR/Stochastic/c d&s est CAN"}
			\subcaption*{$ C_{U74} $}		
		\end{subfigure}%
		\begin{subfigure}{0.7\textwidth}
			\centering
			\includegraphics[width=\tableCustomSize]{"Figures/Results_USR/Stochastic/c d&s est CON"}
			\subcaption*{$ C_{ARK,d=85.0} $}		
		\end{subfigure}\\
		\caption{USR selenium concentration linear model analysis graphs.}
	\end{figure}
\end{landscape}

\subfiguremid
\begin{landscape}
	\begin{figure}
		\begin{subfigure}{0.7\textwidth}
			\centering
			\includegraphics[width=\tableCustomSize]{"Figures/Results_USR/Stochastic/c d&s est FLS"}
			\subcaption*{$ C_{ARK,d=16.4} $}		
		\end{subfigure}%
		\begin{subfigure}{0.7\textwidth}
			\centering
			\includegraphics[width=\tableCustomSize]{"Figures/Results_USR/Stochastic/c d&s est FLY"}
			\subcaption*{$ C_{ARK,d=47.2} $}		
		\end{subfigure}\\
		\caption{USR selenium concentration linear model analysis graphs.}
	\end{figure}
\end{landscape}

\subfiguremid
\begin{landscape}
	\begin{figure}
		\begin{subfigure}{0.7\textwidth}
			\centering
			\includegraphics[width=\tableCustomSize]{"Figures/Results_USR/Stochastic/c d&s est HOL"}
			\subcaption*{$ C_{ARK,d=12.5} $}		
		\end{subfigure}%
		\begin{subfigure}{0.7\textwidth}
			\centering
			\includegraphics[width=\tableCustomSize]{"Figures/Results_USR/Stochastic/c d&s est HRC"}
			\subcaption*{$ C_{207} $}		
		\end{subfigure}\\
		\caption{USR selenium concentration linear model analysis graphs.}
	\end{figure}
\end{landscape}

\subfiguremid
\begin{landscape}
	\begin{figure}
		\begin{subfigure}{0.7\textwidth}
			\centering
			\includegraphics[width=\tableCustomSize]{"Figures/Results_USR/Stochastic/c d&s est RFD"}
			\subcaption*{$ C_{U167} $}		
		\end{subfigure}%
		\begin{subfigure}{0.7\textwidth}
			\centering
			\includegraphics[width=\tableCustomSize]{"Figures/Results_USR/Stochastic/c d&s est TIM"}
			\subcaption*{$ C_{60} $}		
		\end{subfigure}\\
		\caption{USR selenium concentration linear model analysis graphs.}
	\end{figure}
\end{landscape}

%\subfiguremid
%\begin{landscape}
%	\begin{figure}
%		\begin{subfigure}{0.7\textwidth}
%			\centering
%			\includegraphics[width=\tableCustomSize]{"Figures/Results_USR/Stochastic/c d&s est WTP"}
%			\subcaption*{$ C_{LAJWWTP} $}		
%		\end{subfigure}\\
%		\caption{USR selenium concentration linear model analysis graphs.}
%	\end{figure}
%\end{landscape}



\subfiguretop
\begin{landscape}
	\begin{figure}
		\begin{subfigure}{0.7\textwidth}
			\centering
			\includegraphics[width=\tableCustomSize]{"Figures/Results_DSR/Stochastic/c d&s est D101C"}
			\subcaption*{$ C_{D101C} $}		
		\end{subfigure}%
		\begin{subfigure}{0.7\textwidth}
			\centering
			\includegraphics[width=\tableCustomSize]{"Figures/Results_DSR/Stochastic/c d&s est D106C"}
			\subcaption*{$ C_{D106C} $}		
		\end{subfigure}\\
		\caption[DSR dissolved selenium concentration distribution analysis.]{DSR dissolved selenium concentration distribution analysis.  The histogram and the black KDE are of the calculated deterministic model values.  The red dashed KDE represents the distribution of the stochastic model mean time series.}
		\label{fig:concCSeDist_DS}
	\end{figure}
\end{landscape}

\subfiguremid
\begin{landscape}
	\begin{figure}
		\begin{subfigure}{0.7\textwidth}
			\centering
			\includegraphics[width=\tableCustomSize]{"Figures/Results_DSR/Stochastic/c d&s est BIG"}
			\subcaption*{$ C_{D23} $}		
		\end{subfigure}%
		\begin{subfigure}{0.7\textwidth}
			\centering
			\includegraphics[width=\tableCustomSize]{"Figures/Results_DSR/Stochastic/c d&s est BUF"}
			\subcaption*{$ C_{ARK,d=37.7} $}
		\end{subfigure}\\
		\caption{DSR dissolved selenium concentration distribution analysis.}
	\end{figure}
\end{landscape}

\subfiguremid
\begin{landscape}
	\begin{figure}
		\begin{subfigure}{0.7\textwidth}
			\centering
			\includegraphics[width=\tableCustomSize]{"Figures/Results_DSR/Stochastic/c d&s est WIL"}
			\subcaption*{$ C_{D57} $}		
		\end{subfigure}\\
		\caption{DSR dissolved selenium concentration distribution analysis.}
	\end{figure}
\end{landscape}
\subfiguretop

These figures show that the values used for the stochastic model and the deterministic model have very similar distributions.  In some cases there are slight deviations between the two distributions at lower concentration values.  This is due to the uncertainty assigned to the stochastic concentration estimates being more noticeable at lower calculated concentrations.

Time series plots of the concentration results from both the deterministic and stochastic models were prepared to visually analyze the relationship between the dissolved selenium concentration and the calendar date as shown in Figures \ref{fig:concCSeTS_US} and \ref{fig:concCSeTS_DS} for the USR and DSR, respectively.  Two sub-figures are provided.  Sub-figure a is the deterministic model time series and sub-figure b is the stochastic model time series.  The line is the mean of the realizations and the blue band is the 97.5th CIR.

\subfiguretop
\begin{landscape}
	\begin{figure}
		$ C_{U163} $
		\begin{subfigure}{0.7\textwidth}
			\centering
			\includegraphics[width=\tableCustomSize]{"Figures/Results_USR/Deterministic/c TS U163"}
			\caption{Deterministic Model.}
		\end{subfigure}%
		\begin{subfigure}{0.7\textwidth}
			\centering
			\includegraphics[width=\tableCustomSize]{"Figures/Results_USR/Stochastic/c TS U163"}
			\caption{Stochastic Model.}
		\end{subfigure}
		\caption[USR Deterministic and stochastic model time series of dissolved selenium concentration.]{USR Deterministic and stochastic model time series of dissolved selenium concentration.  For sub-figure b, the black line is the mean of the realizations and the blue band is the 97.5th CIR.}
		\label{fig:concCSeTS_US}
	\end{figure}
\end{landscape}

\subfiguremid
\begin{landscape}
	\begin{figure}
		$ C_{U201} $
		\begin{subfigure}{0.7\textwidth}
			\centering
			\includegraphics[width=\tableCustomSize]{"Figures/Results_USR/Deterministic/c TS U201"}
			\caption{Deterministic Model.}
		\end{subfigure}%
		\begin{subfigure}{0.7\textwidth}
			\centering
			\includegraphics[width=\tableCustomSize]{"Figures/Results_USR/Stochastic/c TS U201"}
			\caption{Stochastic Model.}
		\end{subfigure}
		\caption{USR Deterministic and stochastic model time series of dissolved selenium concentration.}
	\end{figure}
\end{landscape}

\subfiguremid
\begin{landscape}
	\begin{figure}
		$ C_{U74} $
		\begin{subfigure}{0.7\textwidth}
			\centering
			\includegraphics[width=\tableCustomSize]{"Figures/Results_USR/Deterministic/c TS CAN"}
			\caption{Deterministic Model.}
		\end{subfigure}%
		\begin{subfigure}{0.7\textwidth}
			\centering
			\includegraphics[width=\tableCustomSize]{"Figures/Results_USR/Stochastic/c TS CAN"}
			\caption{Stochastic Model.}
		\end{subfigure}
		\caption{USR Deterministic and stochastic model time series of dissolved selenium concentration.}
	\end{figure}
\end{landscape}

\subfiguremid
\begin{landscape}
	\begin{figure}
		$ C_{ARK,d=85.0} $
		\begin{subfigure}{0.7\textwidth}
			\centering
			\includegraphics[width=\tableCustomSize]{"Figures/Results_USR/Deterministic/c TS CON"}
			\caption{Deterministic Model.}
		\end{subfigure}%
		\begin{subfigure}{0.7\textwidth}
			\centering
			\includegraphics[width=\tableCustomSize]{"Figures/Results_USR/Stochastic/c TS CON"}
			\caption{Stochastic Model.}
		\end{subfigure}
		\caption{USR Deterministic and stochastic model time series of dissolved selenium concentration.}
	\end{figure}
\end{landscape}

\subfiguremid
\begin{landscape}
	\begin{figure}
		$ C_{ARK,d=16.4} $
		\begin{subfigure}{0.7\textwidth}
			\centering
			\includegraphics[width=\tableCustomSize]{"Figures/Results_USR/Deterministic/c TS FLS"}
			\caption{Deterministic Model.}
		\end{subfigure}%
		\begin{subfigure}{0.7\textwidth}
			\centering
			\includegraphics[width=\tableCustomSize]{"Figures/Results_USR/Stochastic/c TS FLS"}
			\caption{Stochastic Model.}
		\end{subfigure}
		\caption{USR Deterministic and stochastic model time series of dissolved selenium concentration.}
	\end{figure}
\end{landscape}

\subfiguremid
\begin{landscape}
	\begin{figure}
		$ C_{ARK,d=47.2} $
		\begin{subfigure}{0.7\textwidth}
			\centering
			\includegraphics[width=\tableCustomSize]{"Figures/Results_USR/Deterministic/c TS FLY"}
			\caption{Deterministic Model.}
		\end{subfigure}%
		\begin{subfigure}{0.7\textwidth}
			\centering
			\includegraphics[width=\tableCustomSize]{"Figures/Results_USR/Stochastic/c TS FLY"}
			\caption{Stochastic Model.}
		\end{subfigure}
		\caption{USR Deterministic and stochastic model time series of dissolved selenium concentration.}
	\end{figure}
\end{landscape}

\subfiguremid
\begin{landscape}
	\begin{figure}
		$ C_{ARK,d=12.5} $
		\begin{subfigure}{0.7\textwidth}
			\centering
			\includegraphics[width=\tableCustomSize]{"Figures/Results_USR/Deterministic/c TS HOL"}
			\caption{Deterministic Model.}
		\end{subfigure}%
		\begin{subfigure}{0.7\textwidth}
			\centering
			\includegraphics[width=\tableCustomSize]{"Figures/Results_USR/Stochastic/c TS HOL"}
			\caption{Stochastic Model.}
		\end{subfigure}
		\caption{USR Deterministic and stochastic model time series of dissolved selenium concentration.}
	\end{figure}
\end{landscape}

\subfiguremid
\begin{landscape}
	\begin{figure}
		$ C_{U207} $
		\begin{subfigure}{0.7\textwidth}
			\centering
			\includegraphics[width=\tableCustomSize]{"Figures/Results_USR/Deterministic/c TS HRC"}
			\caption{Deterministic Model.}
		\end{subfigure}%
		\begin{subfigure}{0.7\textwidth}
			\centering
			\includegraphics[width=\tableCustomSize]{"Figures/Results_USR/Stochastic/c TS HRC"}
			\caption{Stochastic Model.}
		\end{subfigure}
		\caption{USR Deterministic and stochastic model time series of dissolved selenium concentration.}
	\end{figure}
\end{landscape}

\subfiguremid
\begin{landscape}
	\begin{figure}
		$ C_{U167} $
		\begin{subfigure}{0.7\textwidth}
			\centering
			\includegraphics[width=\tableCustomSize]{"Figures/Results_USR/Deterministic/c TS RFD"}
			\caption{Deterministic Model.}
		\end{subfigure}%
		\begin{subfigure}{0.7\textwidth}
			\centering
			\includegraphics[width=\tableCustomSize]{"Figures/Results_USR/Stochastic/c TS RFD"}
			\caption{Stochastic Model.}
		\end{subfigure}
		\caption{USR Deterministic and stochastic model time series of dissolved selenium concentration.}
	\end{figure}
\end{landscape}

\subfiguremid
\begin{landscape}
	\begin{figure}
		$ C_{U60} $
		\begin{subfigure}{0.7\textwidth}
			\centering
			\includegraphics[width=\tableCustomSize]{"Figures/Results_USR/Deterministic/c TS TIM"}
			\caption{Deterministic Model.}
		\end{subfigure}%
		\begin{subfigure}{0.7\textwidth}
			\centering
			\includegraphics[width=\tableCustomSize]{"Figures/Results_USR/Stochastic/c TS TIM"}
			\caption{Stochastic Model.}
		\end{subfigure}
		\caption{USR Deterministic and stochastic model time series of dissolved selenium concentration.}
	\end{figure}
\end{landscape}

%\subfiguremid
%\begin{landscape}
%	\begin{figure}
%		$ C_{LAJWWTP} $
%		\begin{subfigure}{0.7\textwidth}
%			\centering
%			\includegraphics[width=\tableCustomSize]{"Figures/Results_USR/Deterministic/c TS WTP"}
%			\caption{Deterministic Model.}
%		\end{subfigure}%
%		\begin{subfigure}{0.7\textwidth}
%			\centering
%			\includegraphics[width=\tableCustomSize]{"Figures/Results_USR/Stochastic/c TS WTP"}
%			\caption{Stochastic Model.}
%		\end{subfigure}
%		\caption{USR Deterministic and stochastic model time series of dissolved selenium concentration.}
%	\end{figure}
%\end{landscape}


\subfiguretop
\begin{landscape}
	\begin{figure}
		$ C_{D101C} $
		\begin{subfigure}{0.7\textwidth}
			\centering
			\includegraphics[width=\tableCustomSize]{"Figures/Results_DSR/Deterministic/c TS D101C"}
			\caption{Deterministic Model.}
		\end{subfigure}%
		\begin{subfigure}{0.7\textwidth}
			\centering
			\includegraphics[width=\tableCustomSize]{"Figures/Results_DSR/Stochastic/c TS D101C"}
			\caption{Stochastic Model.}
		\end{subfigure}
		\caption[DSR Deterministic and stochastic model time series of dissolved selenium concentration.]{DSR Deterministic and stochastic model time series of dissolved selenium concentration.  For sub-figure b, the black line is the mean of the realizations and the blue band is the 97.5th CIR.}
		\label{fig:concCSeTS_DS}
	\end{figure}
\end{landscape}

\subfiguremid
\begin{landscape}
	\begin{figure}
		$ C_{D106C} $
		\begin{subfigure}{0.7\textwidth}
			\centering
			\includegraphics[width=\tableCustomSize]{"Figures/Results_DSR/Deterministic/c TS D106C"}
			\caption{Deterministic Model.}
		\end{subfigure}%
		\begin{subfigure}{0.7\textwidth}
			\centering
			\includegraphics[width=\tableCustomSize]{"Figures/Results_DSR/Stochastic/c TS D106C"}
			\caption{Stochastic Model.}
		\end{subfigure}
		\caption{DSR Deterministic and stochastic model time series of dissolved selenium concentration.}
	\end{figure}
\end{landscape}

\subfiguremid
\begin{landscape}
	\begin{figure}
		$ C_{D23} $
		\begin{subfigure}{0.7\textwidth}
			\centering
			\includegraphics[width=\tableCustomSize]{"Figures/Results_DSR/Deterministic/c TS BIG"}
			\caption{Deterministic Model.}
		\end{subfigure}%
		\begin{subfigure}{0.7\textwidth}
			\centering
			\includegraphics[width=\tableCustomSize]{"Figures/Results_DSR/Stochastic/c TS BIG"}
			\caption{Stochastic Model.}
		\end{subfigure}
		\caption{DSR Deterministic and stochastic model time series of dissolved selenium concentration.}
	\end{figure}
\end{landscape}

\subfiguremid
\begin{landscape}
	\begin{figure}
		$ C_{ARK,d=7.7} $
		\begin{subfigure}{0.7\textwidth}
			\centering
			\includegraphics[width=\tableCustomSize]{"Figures/Results_DSR/Deterministic/c TS BUF"}
			\caption{Deterministic Model.}
		\end{subfigure}%
		\begin{subfigure}{0.7\textwidth}
			\centering
			\includegraphics[width=\tableCustomSize]{"Figures/Results_DSR/Stochastic/c TS BUF"}
			\caption{Stochastic Model.}
		\end{subfigure}
		\caption{DSR Deterministic and stochastic model time series of dissolved selenium concentration.}
	\end{figure}
\end{landscape}

\subfiguremid
\begin{landscape}
	\begin{figure}
		$ C_{U57} $
		\begin{subfigure}{0.7\textwidth}
			\centering
			\includegraphics[width=\tableCustomSize]{"Figures/Results_DSR/Deterministic/c TS WIL"}
			\caption{Deterministic Model.}
		\end{subfigure}%
		\begin{subfigure}{0.7\textwidth}
			\centering
			\includegraphics[width=\tableCustomSize]{"Figures/Results_DSR/Stochastic/c TS WIL"}
			\caption{Stochastic Model.}
		\end{subfigure}
		\caption{DSR Deterministic and stochastic model time series of dissolved selenium concentration.}
	\end{figure}
\end{landscape}
\subfiguretop

These figures show a definite cyclical pattern for concentrations within the main stem of the Arkansas R.  Although the pattern varies, it is interesting to note that higher concentrations are calculated during the colder months in all cases.

There were not any significant discrepancies between the calculated data and the measured data nor between the deterministic and stochastic models.  The results and comparison of the results are presented in tables \ref{tab:USRConcResults} and \ref{tab:DSRConcResults} for the USR and DSR, respectively.  These tables present the mean, 2.5th, and 97.5th percentile of the deterministic time series results.  These tables also present the mean, 2.5th, and 97.5th percentile of the 1-D stochastic mean time series results.  The last column provides the percent difference between the two calculated mean values.  Again, the stochastic and deterministic calculated dissolved selenium concentration resluts are not significantly different.

\subtabletop
\begin{table}[htbp]
	\centering
  \caption[USR dissolved selenium concentration results table.]{USR dissolved selenium concentration results table.  Values are in units of \si{\micro\gram\per\liter}.}
	\label{tab:USRConcResults}
	\begin{subtable}{\textwidth}
		\centering
		\subcaption*{$ C_{U163} $}
		\input{"Tables/c U163.txt"}
	\end{subtable}\\
	\tablevspace
	\begin{subtable}{\textwidth}
		\centering
		\subcaption*{$ C_{U201} $}
		\input{"Tables/c U201.txt"}
	\end{subtable}\\
\end{table}

\subtablemid
\begin{table}[htbp]
	\centering
	\caption{USR dissolved selenium concentration results table.}
	\begin{subtable}{\textwidth}
		\centering
		\subcaption*{$ C_{U74} $}
		\input{"Tables/c CAN.txt"}
	\end{subtable}\\
	\tablevspace
	\begin{subtable}{\textwidth}
		\centering
		\subcaption*{$ C_{ARK,D=85.0} $}
		\input{"Tables/c CON.txt"}
	\end{subtable}\\
\end{table}

\subtablemid
\begin{table}[htbp]
	\centering
	\caption{USR dissolved selenium concentration results table.}
	\begin{subtable}{\textwidth}
		\centering
		\subcaption*{$ C_{ARK,D=16.4} $}
		\input{"Tables/c FLS.txt"}
	\end{subtable}\\
	\tablevspace
	\begin{subtable}{\textwidth}
		\centering
		\subcaption*{$ C_{ARK,D=47.2} $}
		\input{"Tables/c FLY.txt"}
	\end{subtable}\\
\end{table}

\subtablemid
\begin{table}[htbp]
	\centering
	\caption{USR dissolved selenium concentration results table.}
	\begin{subtable}{\textwidth}
		\centering
		\subcaption*{$ C_{ARK,D=12.5} $}
		\input{"Tables/c HOL.txt"}
	\end{subtable}\\
	\tablevspace
	\begin{subtable}{\textwidth}
		\centering
		\subcaption*{$ C_{U207} $}
		\input{"Tables/c HRC.txt"}
	\end{subtable}\\
\end{table}

\subtablemid
\begin{table}[htbp]
	\centering
	\caption{USR dissolved selenium concentration results table.}
	\begin{subtable}{\textwidth}
		\centering
		\subcaption*{$ C_{U167} $}
		\input{"Tables/c RFD.txt"}
	\end{subtable}\\
	\tablevspace
	\begin{subtable}{\textwidth}
		\centering
		\subcaption*{$ C_{U60} $}
		\input{"Tables/c TIM.txt"}
	\end{subtable}\\
\end{table}

\subtablemid
\begin{table}[htbp]
	\centering
	\caption{USR dissolved selenium concentration results table.}
	\begin{subtable}{\textwidth}
		\centering
		\subcaption*{$ C_{LAJWWTP} $}
		\input{"Tables/c WTP.txt"}
	\end{subtable}\\
\end{table}

\subtabletop
\begin{table}[htbp]
	\centering
	\caption[DSR dissolved selenium concentration results table.]{DSR dissolved selenium concentration results table.  Values are in units of \si{\micro\gram\per\liter}.}
	\label{tab:DSRConcResults}
	\begin{subtable}{\textwidth}
		\centering
		\subcaption*{$ C_{D101C} $}
		\input{"Tables/c D101C.txt"}
	\end{subtable}\\
	\tablevspace
	\begin{subtable}{\textwidth}
		\centering
		\subcaption*{$ C_{D106C} $}
		\input{"Tables/c D106C.txt"}
	\end{subtable}\\
\end{table}

\subtablemid
\begin{table}[htbp]
	\centering
	\caption{DSR dissolved selenium concentration results table.}
	\begin{subtable}{\textwidth}
		\centering
		\subcaption*{$ C_{D23} $}
		\input{"Tables/c BIG.txt"}
	\end{subtable}\\
	\tablevspace
	\begin{subtable}{\textwidth}
		\centering
		\subcaption*{$ C_{ARK,d=37.7} $}
		\input{"Tables/c BUF.txt"}
	\end{subtable}\\
\end{table}

\subtablemid
\begin{table}[htbp]
	\centering
	\caption{DSR dissolved selenium concentration results table.}
	\begin{subtable}{\textwidth}
		\centering
		\subcaption*{$ C_{D57} $}
		\input{"Tables/c WIL.txt"}
	\end{subtable}\\
\end{table}

\clearpage{}


%\subsection{Uncertainty of Lab $C_{Se}$}
%\emph{V}\\

\textsc{A - defined the uncertainty constituents}\\
Any sampling methodology is subject to error from a multitude of sources.  The additional combined selenium concentration estimating error, $\varepsilon_{2}$, includes error due to variations in field sampling technique, environmental variations, and lab analysis variations.  The samples collected in this study were also subject to an additional unknown error due to environmental conditions during transport from the field to the lab.  In some cases, samples reached the lab seven days after being taken in the field.  Field technicians took great efforts to keep the samples chilled throughout the field collection process.  At the end of a field sampling trip, samples were sent in a chilled insulated container by mail to the lab.  The environmental conditions during this transport phase were not and could not be monitored.

Upon receipt at the lab, samples were stored in a refrigerator until they were analyzed.  The temperature upon receipt was not recorded by any of the labs.  The labs stored the samples in a refrigerator for a maximum of four days.  It is not known what, if any, chemistry changes within the samples from the time the samples are taken to the time they are analyzed at the lab.  It is also unknown if there is any difference due to minor variations in sampling technique.

Preferably, these error sources could be accounted for on an individual basis.  There were a number of factors that determined that this methodology would not work.  Error analysis would have to be performed for each field technician.  The total project data collection time frame spanned 10 years and included an unknown number of field technicians.  The data collection methodology previously described was not entirely adhered for the entire data collection time frame.  Not all field technicians recorded what was later considered valuable information such as date and time of sample collection, field technician name, and sampling variances.

The travel distance between the field locations and the lab is also a factor that cannot be overcome.  Preferably, the lab would be located fairly close to either the university or the study region.  At the start of the sampling time frame, there was no lab in Colorado capable of handling the required analysis with the additional physical, schedule, and fiscal requirements imposed by the project supervisor.  The additional distance made determining error due to transport time difficult to determine.

All dissolved selenium samples were treated with nitric acid to stabilize the sample for transport.  The stabilization method and acceptable sample storage duration was discussed at length with the lab before any sampling was undertaken.  The samples were additionally preserved by storing and transporting them on ice.  We were assured by the sampling lab that this additional preservation step would only serve to lengthen the time the sample would be considered viable.

The temperature variations experienced by the samples was not considered a factor during the sample collect time frame.  On hind-sight, this could have easily been performed by adding a temperature transponder to the sample container before shipping.  This technique might have brought transport temperature control issues to the technician's or the project supervisor's attention if the existed.  Unfortunately, this information is not available and we are left to assume that even though significant temperature variations may have existed, those variations did not significantly change the sample chemistry due to the applied sample preservation.

Field and lab blanks were used to determine if the samples were subjected to contamination from the environment or cross contamination from other samples.  No lab or field blanks exhibited any evidence of contamination.  Since the blanks contained only de-ionized water, they did not have any chemical or physical markers to show whether they experienced unacceptable environmental conditions.

Lab analysis errors are known to exist and the lab states these ranges.  Since the lab was USEPA certified and subscribed to USEPA proficiency testing, we can assume that the lab errors are as stated by the labs.  Verification through a different lab was not performed at any time.  Although the lab error range was known,  it was not known if that error range could be influence by variances in the sampling technique or transport environment.  

Combining all individual unaccountable errors into a single error term seemed to be the most pragmatic means to estimate the total error.  The only data available to analyze was the set of duplicate samples.  As previously discussed at least two samples per sample trip were taken as duplicates.  Duplicate samples were taken near the beginning and the end of the sampling trip.  Only the 'A' sample was used for concentration estimation.  The 'B' samples were taken to monitor for equipment malfunction, significant deviation in sampling methodology, and significant lab error.  The 'A' and 'B' samples were taken using the same equipment and transported in the same container from the sample location to the lab.  Since they experienced the same environmental conditions, it was unreasonable to assume that this method could be used to estimate the error due to extreme transport environmental conditions.  The samples were well preserved and it was assumed that temperature variations did not significantly affect the samples.  

Lab results for the 'A' and 'B' samples from both the USR and DSR were complied into a single data set.  Date, location, and all other identifying markers were removed from the data set to reduce potential bias due to prior knowledge of the individual samples.  'A' samples were assumed to be the expected value for the following samples.  

Figure \ref{fig:CSe uncertainty} shows the comparisons analyzed.  The top graph plots the 'A' and 'B' sample concentrations against the difference from the mean of the 'A' and 'B' samples.  The bottom graph plots the same data, but with respect to the percent difference from the mean of the 'A' and 'B' samples.
\begin{figure}[htbp]
	\begin{center}
		\includegraphics[width=6in]{"Figures/Results_USR/Stochastic/CSe Error"}
		\caption[Dissolved Selenium Concentration Uncertainty.]{Dissolved Selenium Concentration Uncertainty.  Each sub-figure shows the comparison of the lab reported dissolved selenium concentration and the mean for each pair of duplicate samples.}
		\label{fig:CSe uncertainty}
	\end{center}
\end{figure}

Figure \ref{fig:CSe uncertainty scatter} shows the 'A' and 'B' samples plotted against the difference and percent difference.  This analysis was performed to determine if there was a correlation between the magnitude of the concentration and the magnitude of the difference.  No such correlation was found.

\begin{figure}[htbp]
	\begin{center}
		\includegraphics[width=6in]{"Figures/Results_USR/Stochastic/CSe Error Scatter"}
		\caption[Dissolved Selenium Concentration Variation]{Dissolved Selenium Concentration Variation.}
		\label{fig:CSe uncertainty scatter}
	\end{center}
\end{figure}

The percent difference between the reported lab values and the mean duplicates was best fit with a logistic distribution.  The logistic distribution had a location parameter of -0.067 and a scale parameter of 1.807.  This comparison and distribution combination was chosen because it had the best fit when compared to others using Kolmogorov-Smirnov, Cramer-von Mises, and Anderson-Darling goodness-of-fit test statistics.  The duplicate samples had a mean difference from the mean of their respective 'A' and 'B' samples of 0\% with a standard deviation of 4.16\%.  This corresponds to 95\% of the data within $\pm$10\% of the reported value

Calculated dissolved selenium concentrations at specific sites were constrained to fit between one-half of the minimum historically reported value and 1-1/2 of the maximum historically reported value.  This range should allow for dissolved selenium concentration variations that are comparable to the values reported from the field samples.  The range allowed for variation beyond the reported concentration range to allow for the possibility that concentrations beyond the range may possibly have existed at some time.


%The first calculation, with results in the upper left panel of Figure \ref{fig:CSeError}, shows the difference of the 'B' sample from the 'A' sample.  These results appear centrally located near zero, but significantly large outliers are present.  The top right panel shows the absolute value of the difference calculated for the first panel.  Here, as expected, the best fit distributions do not fit the calculation results.  Again, a significant number of outliers are present.  The bottom left panel show the percent difference of the 'B' sample from the 'A' sample.  The outliers from the previous calculations are closer to the main body of data.  The best fit normal distribution incorporates more of the outliers within its span and the logistic distribution more closely resembles the kernel density estimate of the calculation results.  The bottom right panel shows the absolute value of the percent difference.  Again, neither the normal or logistic distributions fit the data well.
%
%Visual analysis seems to indicate that using the percent difference distribution would lead to the best characterization of the selenium sample errors.  This hypothesis was tested by using Kolmogorov-Smirnov, Cramer von Mises, and Anderson-Darling goodness-of-fit tests.  Results from these tests are presented in Table \ref{tab:CSeGoF}.  The logistic distribution of the percent difference calculation was shown to have the best fit calculation.
%
%\begin{table}
%  \caption[Selenium combined error analysis goodness-of-fit test results.]{Selenium combined error analysis goodness-of-fit test results.}
%  \label{tab:CSeGoF}
%  \centering
%    \begin{tabular}{lcccc}
%    \toprule
%    \multirow{2}{*}{Calculation} & \multirow{2}{*}{Distribution} & \multicolumn{3}{c}{Goodness-of-Fit Test Result}\\ \cline{3-5}
%     & & K-S & CvM & A-D\\
%    \midrule
%    \midrule
%    \multirow{2}{*}{Difference} & normal & 0.2876 & 2.598 & 13.48\\
%     & logistic & 0.1644 & 0.8538 & 5.225\\
%    \midrule
%    \multirow{2}{*}{Absolute Difference} & normal & 0.3114 & 3.703 & 18.80\\
%     & logistic & 0.2734 & 1.592 & 10.21\\
%    \midrule    
%    \multirow{2}{*}{Percent Difference} & normal & 0.173 & 1.150 & 6.365\\
%     & logistic & 0.1195 & 0.4139 & 2.729\\
%    \midrule     
%    \multirow{2}{*}{Absolute Percent Difference} & normal & 0.246 & 2.391 & 12.83\\
%     & logistic & 0.2260 & 1.010 & 7.698\\
%    \bottomrule
%    \end{tabular}%
%\end{table}%
%
%Given these results, the total field sampling and lab error distribution is described by a logistic distribution with a location parameter of -0.06693 and a scale parameter of 1.807.  The combined field sampling and lab error is bounded such that 95\% of the distribution lies in the range of approximately $\pm$6.6\% of the expected value.
%
%The combined field sampling and lab error is calculated independently from the selenium concentration estimation error previously described.  The estimated selenium concentration, without the estimation estimation, is taken as the expected value for the combined field sampling and lab error. 
%
%\textsc{B - data source}\\
%
%\textsc{C - calculation method}\\
%
%\textsc{D - test distribution}\\
%
%\textsc{E - present best fit lab $C_{Se}$ distribution}\\
%
%\clearpage{}
%\subsection{Mass Storage Change Results}
%\emph{VI}\\
%
%\textsc{A - present river segment results}\\
%
%Values and figures presented in this section are the results from calculations performed as described in chapter \ref{chap:Model Development} and all other precursor calculations.  The primary purpose of this analysis was to determine if the computational code and assumptions used to generate the stochastic distributions of river segment dissolved selenium mass storage changes were performed correctly.
%
%Figure \ref{fig:ExampleSeMassChange} is an example figure that shows the deterministic and stochastic time series of the mass storage change within a river segment.  This particular figure presents data for segment A in the USR.  Two sub-figures are provided.  Sub-figure a is the deterministic model time series and sub-figure b is the stochastic model time series.  The line is the mean of the realizations and the blue band is the 97.5th CIR.  Similar figures were created for all segments in both study region river reaches and are presented in appendix \ref{App:SeS}.
%
%Results in these figures and associated tables are presented in units of \si{\kilo\gram\per\day\per\kilo\meter}.  Standardizing values to mass storage per unit length allows for comparison between all river segments in both study reaches.  This also allows for comparison between the mass storage change components and the mass transport components of the mass balance models.
%
%\begin{figure}[htbp]
%\centering
%	\begin{subfigure}{0.5\textwidth}
%		\includegraphics[width=0.9\linewidth]{"Figures/Results_DUSR/f Segment A"}
%		\caption{Deterministic Model.}
%		\label{sub:ExampleDSeMassChange}
%	\end{subfigure}%
%	\begin{subfigure}{0.5\textwidth}
%		\includegraphics[width=0.9\linewidth]{"Figures/Results_USR/f Segment A"}
%		\caption{Stochastic Model.}
%		\label{sub:ExampleSSeMassChange}
%	\end{subfigure}
%	\caption[River segment deterministic and stochastic dissolved selenium mass storage change time series.]{River segment deterministic and stochastic dissolved selenium mass storage change time series.  This is an example figure presenting the results for the upstream end of the USR.  Additional figures for the other USR and DSR calculated points are provided in the appendix noted in the text.  For sub-figure b, the line is the mean of the realizations and the blue band is the 97.5th CIR.}
%	\label{fig:ExampleSeMassChange}
%\end{figure}
%
%Comparing the sub-figures provides for a visual goodness-of-fit analysis between the deterministic and stochastic models.  All USR river segment stochastic models agree with the deterministic models.  It should be noted that there is quite a fair amount of uncertainty associated with the values calculated for the stochastic model.  This is the compounding of uncertainties from the multiple input variables.  This is as expected for a complex multi-variate model.
%
%The calculated selenium storage change values for each reach were compared between the deterministic and stochastic models and are reported in table \ref{tab:ReachSeStore} for both the USR and DSR.  All selenium storage change values in the figures and tables are presented in units of \si{\kilo\gram\per\day\per\kilo\meter}.  This table is presented in a similar fashion to other comparison tables in this chapter.
%
%\begin{table}[htbp]
%\centering
%\caption[River segment deterministic and stochastic model selenium storage changes.]{River segment deterministic and stochastic model selenium storage changes.  All values are in \si{\kilo\gram\per\day\per\kilo\meter}.}
%\label{tab:ReachSeStore}
%    \begin{tabular}{l|ccc|ccc|c}
%    \toprule
%    \multirow{2}[0]{*}{Variable} & \multicolumn{3}{c}{Deterministic} & \multicolumn{3}{c}{Stochastic Mean} & \% Diff\\\cline{2-4} \cline{5-7}
%    & 2.5\% & Mean & 97.5\% & 2.5\% & Mean & 97.5\% & Mean\\
%    \midrule
%    \midrule
%	$\Delta M_A$&	-0.6816&	-0.004268&	0.7999&	-1.134&	-0.004609&	1.176&	7.99\\         
%	$\Delta M_B$&	-0.1665&	-0.00004847&	0.1995&	-0.2201&	0.002345&	0.2627&	-4940\\
%	$\Delta M_C$&	-1.3&	0.01262&	1.491&	-2.112&	0.01238&	2.278&	-1.9\\             
%	$\Delta M_D$&	-3.393&	-0.04352&	4.231&	-4.175&	-0.04125&	4.583&	-5.22\\            
%	$\Delta M_E$&	-0.9114&	-0.02073&	0.8795&	-1.356&	-0.02047&	1.362&	-1.25\\        
%	$\Delta M_F$&	-0.5934&	-0.02879&	0.4908&	-1.164&	-0.03031&	1.087&	5.28\\         
%	$\Delta M_G$&	-0.415&	-0.006064&	0.4856&	-1.247&	-0.006501&	1.26&	7.21\\             
%	\bottomrule
%	\end{tabular}
%\end{table}
%
%This table shows that the deviation between the deterministic and 1-D stochastic mean models is low, but higher than the values calculated for the individual input variables.  This is most likely indicative of the compounding of input variable uncertainties.  The percent deviation between the models for segment B in the USR is deceptive.  Both the deterministic model mean value and the 1-D stochastic mean model mean value are near zero.  This causes any deviation to appear large.
%
%\textsc{B - present river reach results}\\
%
%The deterministic and stochastic mass balance models each consist of two major components.  The selenium transport component is the sum of all selenium mass transport and the selenium storage component is the sum of all river segment mass storage changes.  Mass transport is the mass that is transported in or out of the river reach by tributaries or canals, respectively.  Unlike the water balance model, the atmospheric model is not included.  As discussed earlier, there is insufficient evidence to support any calculations used to estimate selenium loss to the atmosphere.
%
%Values and figures presented in this section are the selenium storage component sub-set of the complete mass model results.  The analyses in this section are restricted to performing a reasonability check on the results.  Results in this section are to be compared with results from the other sub-sections from both the USR and DSR models to determine if results are acceptable.  Unacceptable results would indicate an error in the computational code or an underlying assumption.
%
%Figures \ref{fig:USRMassStore} and \ref{fig:DSRMassStore} present the selenium storage component results in the USR and DSR, respectively, as time series plots.  The left and right sub-figures present the deterministic and stochastic time series, respectively.  The black line in the stochastic figures is the 1-D stochastic mean results with the blue band indicating the 95\% CIR.  Values in these figures and associated tables are in units of \si{\kilo\gram\per\day\per\kilo\meter}.  Standardizing the results allows for comparisons to be made between the two study reaches and between mass balance model components.  Positive values indicate that the reach gained selenium during the given time step.
%
%\begin{figure}[htbp]
%\centering
%	\begin{subfigure}{0.5\textwidth}
%		\centering
%		\includegraphics[width=0.9\linewidth]{"Figures/Results_DUSR/Balance Mass - Storage"}
%		\caption{Deterministic Model.}
%		\label{sub:USRMassStoreD}
%	\end{subfigure}%
%	\begin{subfigure}{0.5\textwidth}
%		\centering
%		\includegraphics[width=0.9\linewidth]{"Figures/Results_USR/Balance Mass - Storage"}
%		\caption{Stochastic Model.}
%		\label{sub:USRMassStoreS}
%	\end{subfigure}
%	\caption[USR Arkansas River deterministic and stochastic stored mass change time series.]{USR Arkansas River deterministic and stochastic stored mass change time series.}
%	\label{fig:USRMassStore}
%\end{figure}
%
%\begin{figure}[htbp]
%\centering
%	\begin{subfigure}{0.5\textwidth}
%		\centering
%		\includegraphics[width=0.9\linewidth]{"Figures/Results_DDSR/Balance Mass - Storage"}
%		\caption{Deterministic Model.}
%		\label{sub:DSRMassStoreD}
%	\end{subfigure}%
%	\begin{subfigure}{0.5\textwidth}
%		\centering
%		\includegraphics[width=0.9\linewidth]{"Figures/Results_DSR/Balance Mass - Storage"}
%		\caption{Stochastic Model.}
%		\label{sub:DSRMassStoreS}
%	\end{subfigure}
%		\caption[DSR Arkansas River deterministic and stochastic stored mass change time series.]{DSR Arkansas River deterministic and stochastic stored mass change time series.}
%	\label{fig:DSRMassStore}
%\end{figure}
%
%The figures show that there is a definite seasonal variation in the selenium storage component.  This temporal relationship follows the same pattern identified with the water balance model storage component.  There is a very strong visual relationship between the water balance model flow component and the mass balance model selenium transport component.  This is to be expected since the water balance storage component is the prime contributor to the mass balance selenium storage component.
%
%These figures show that uncertainty with the selenium storage component is very large.  This is to be expected since the mass balance models contain all of the uncertainty from the water balance model and the selenium concentration estimation linear models.
%
%The distribution of all realizations within each time step was analyzed to determine a distribution type.  This analysis was performed to determine if the assumption that the deterministic model results were representative of the stochastic model.  Testing was performed by comparing K-S statistics for the best fit normal, log-normal, logistic, exponential, gamma, and Weibull distributions.  In the USR 94\% of all atmospheric component time steps best fit a normal distribution, with the other 6\% best fit by a gamma distribution.  In the DSR 99\% of all atmospheric component time steps best fit a normal distribution, with the other 1\% best fit by a gamma distribution.  This indicates that for both the USR and DSR, the distributions across the realizations are normal with some slight skewness. 
%
%Tables \ref{tab:USRSeStore} and \ref{tab:DSRSeStore} presents summary statistics of the deterministic and stochastic atmospheric component.  Values are presented in units of \si{\cubic\meter\per\second\per\kilo\meter}.   These tables show the mean, 2.5th, and 97.5th percentile of the deterministic model and the same statistics applied to the three 1-D stochastic models.  This additional set of statistics from the 1-D stochastic 2.5th and 95.5th percentile models provide a better understanding of the extremes of the calculated values.  As before, comparison between the deterministic and stochastic models should be limited to the comparing the deterministic model to the 1-D stochastic mean model.  Also included in these two tables is the percent difference between the deterministic model and the 1-D stochastic mean model.  The mean, 2.5th, and 97.5th percentile values were calculated from the percent difference of the 1-D stochastic mean model from the deterministic model at each time step.  These values show the range of the variance from the deterministic model.
%
%\begin{table}[htbp]
%\centering
%\caption[USR river section deterministic and stochastic model selenium storage changes.]{USR river section deterministic and stochastic model selenium storage changes.  All values are in \si{\kilo\gram\per\day\per\kilo\meter}.}
%\label{tab:USRSeStore}
%\begin{tabular}{c|ccc}
%	\toprule
%	Model& 2.5\% & Mean & 97.5\% \\
%	\midrule
%	\midrule
%	Deterministic    &	-0.05792&	-0.0006277&	0.07671\\
%	\midrule                                            
%	Stochastic 2.5\% &	-0.1006&	-0.03036&	0.0348\\ 
%	Stochastic Mean  &	-0.05719&	-0.0001518&	0.07788\\
%	Stochastic 97.5\%&	-0.02281&	0.02978&	0.13\\   
%	\midrule                                            
%	\% Diff. Means&		-27.83&	-0.4654&	45.78\\
%	\bottomrule
%\end{tabular}
%\end{table}
%
%\begin{table}[htbp]
%\centering
%\caption[DSR river section deterministic and stochastic model selenium storage changes.]{DSR river section deterministic and stochastic model selenium storage changes.  Values are in units of \si{\kilo\gram\per\day\per\kilo\meter}.}
%\label{tab:DSRSeStore}
%\begin{tabular}{c|ccc}
%	\toprule
%	Model& 2.5\% & Mean & 97.5\% \\
%	\midrule
%	\midrule
%	Deterministic    &	-0.01058&	-0.0001905&	0.01124\\
%	\midrule                                              
%	Stochastic 2.5\% &	-0.02972&	-0.01362&	0.0001171\\
%	Stochastic Mean  &	-0.01272&	-0.0001761&	0.01265\\  
%	Stochastic 97.5\%&	0.0008917&	0.01324&	0.03121\\  
%	\midrule                                              
%	\% Diff. Means&		-139.8&	-9.668&	139.6\\
%	\bottomrule
%\end{tabular}
%\end{table}
%
%The mean of the percent difference between the deterministic and 1-D stochastic mean models is low, but not insignificant.  This indicates that the deterministic model is fairly representative of the stochastic model expected value.  The high percent differences at the 2.5th and 97.5th percentile indicate that there is still a large range of uncertainty contained within the stochastic model that the deterministic model cannot replicate.  The deterministic model can be used to determine how changes can affect a reach over a span of time, but using it to estimate values for specific time steps is unwise as the differences noted at individual time steps is too large to account for.
%
%\textsc{C - analysis and comments on river segment and reach results}
%
%\clearpage{}
%\section{Mass Transport in Gauged Streams and Diverted Canals}
%\label{sec:MassTransport}
%
%\emph{I - define the realtionship between $Q_{Surface}$ and $L_{Surface}$}\\
%
%The combined river section selenium surface transport rate $(\dot{M}_{Surface})$ is calculated as shown in equation \ref{eq:mbal3}.  Selenium surface transport rate values for individual stream gauges are calculated as positive values regardless of whether they discharge to or receive water from the main stem of the river.
%
%\begin{equation}
%\dot{M}_{Surface}=\dot{M}_{inlet}-\dot{M}_{outlet}+\dot{M}_{Tributaries}-\dot{M}_{Canals}
%\label{eq:mbal3}
%\end{equation}
%\begin{tabular}{rl}
%$\dot{M}_{inlet}$ =& Selenium surface transport across the study region upstream \\
%&inlet $(mass \cdot time^{-1})$\\
%$\dot{M}_{outlet}$ =& Selenium surface transport across the study region downstream\\ 
%&outlet $(mass \cdot time^{-1})$\\
%$\dot{M}_{Tributaries}$ =& Selenium surface transport from tributaries to the river main\\
%&stem $(mass \cdot time^{-1})$\\
%$\dot{M}_{Canals}$ =& Selenium surface transport from the river main stem to\\
%&canals $(mass \cdot time^{-1})$\\
%\end{tabular}\\
%
%\emph{II - state which solute concentration models are used with which gauged flows}\\
%
%\emph{III - present source/sink results}\\
%
%\emph{IV - present river segment results}\\
%
%\emph{V - present river reach results}\\
%
%The deterministic and stochastic mass balance models each consist of two major components.  The flow component is the sum of all selenium mass transport and the storage component is the sum of all river segment mass storage changes.  Mass transport is the mass that is transported in or out of the river reach by tributaries or canals, respectively.  Unlike the water balance model, the atmospheric model is not included.  As discussed earlier, there is insufficient evidence to support an estimation of selenium loss to the atmosphere.
%
%Values and figures presented in this section are the mass storage change component sub-set of the complete mass model results.  The analyses in this section are restricted to performing a reasonability check on the results.  Results in this section are to be compared with results from the other sub-sections from both the USR and DSR models to determine if results are acceptable.  Unacceptable results would indicate an error in the computational code or an underlying assumption.
%
%Figures \ref{fig:USRMassStore} and \ref{fig:DSRMassStore} present the sum of the selenium mass transport in the USR and DSR, respectively, as time series plots.  The left and right sub-figures present the determinstic and stochastic time series, respectively.  The black line in the stochastic figures is the 1-D stochastic mean results with the blue band indicating the 95\% CIR.  Values in these figures and associated tables indicate the selenium mass entering and leaving the respective reaches in units of \si{\kilo\gram\per\day\per\kilo\meter}.  Standardizing the results allows for comparisons to be made between the two study reaches and between mass balance model components.  Positive values indicate that the reach gained selenium during the given time step.
%
%\begin{figure}[htbp]
%\centering
%	\begin{subfigure}{0.5\textwidth}
%		\centering
%		\includegraphics[width=0.9\linewidth]{"Figures/Results_DUSR/Balance Mass - Flux"}
%		\caption{Deterministic Model.}
%		\label{sub:USRMassStoreD}
%	\end{subfigure}%
%	\begin{subfigure}{0.5\textwidth}
%		\centering
%		\includegraphics[width=0.9\linewidth]{"Figures/Results_USR/Balance Mass - Flux"}
%		\caption{Stochastic Model.}
%		\label{sub:USRMassStoreS}
%	\end{subfigure}
%	\caption[USR Arkansas River deterministic and stochastic surface water mas balance time series.]{USR Arkansas River deterministic and stochastic surface water mas balance time series.}
%	\label{fig:USRMassStore}
%\end{figure}
%
%\begin{figure}[htbp]
%\centering
%	\begin{subfigure}{0.5\textwidth}
%		\centering
%		\includegraphics[width=0.9\linewidth]{"Figures/Results_DDSR/Balance Mass - Flux"}
%		\caption{Deterministic Model.}
%		\label{sub:DSRMassStoreD}
%	\end{subfigure}%
%	\begin{subfigure}{0.5\textwidth}
%		\centering
%		\includegraphics[width=0.9\linewidth]{"Figures/Results_DSR/Balance Mass - Flux"}
%		\caption{Stochastic Model.}
%		\label{sub:DSRMassStoreS}
%	\end{subfigure}
%		\caption[DSR Arkansas River deterministic and stochastic surface water mas balance time series.]{DSR Arkansas River deterministic and stochastic surface water mas balance time series.}
%	\label{fig:DSRMassStore}
%\end{figure}
%
%The figures show that there is a definite seasonal variation in the selenium transport component.  This temporal relationship follows the same pattern identified with the water balance model flow component.  There is a visual relationship between the water balance model flow component and the mass balance model selenium transport component.  This is to be expected since the water balance storage component is the prime contributor to the mass balance selenium storage component.  This relationship is not as strong as seen between the water balance model storage component and the mass balance model selenium storage component.
%
%These figures show that uncertainty with the selenium storage component is very large.  This is to be expected since the mass balance models contain all of the uncertainty from the water balance model and the selenium concentration estimation linear models.  The magnitude of the flow component uncertainty is comparable to the magnitude of the storage componenet uncertainty.  This is expected since both model components use many of the same input variables with their uncertainties.
%
%The distribution of all realizations within each time step was analyzed to determine a distribution type.  This analysis was performed to determine if the assumption that the deterministic model results were representative of the stochastic model.  Testing was performed by comparing K-S statistics for the best fit normal, log-normal, logistic, exponential, gamma, and Weibull distributions.  In the USR 98\% of all atmospheric component time steps best fit a normal distribution, with the other 2\% best fit by a gamma distribution.  In the DSR 99.7\% of all atmospheric component time steps best fit a normal distribution, with the other 0.3\% best fit by a gamma distribution.  This indicates that for both the USR and DSR, the distributions across the realizations are normal. 
%
%Tables \ref{tab:USRSeFlow} and \ref{tab:DSRSeStore} presents summary statistics of the deterministic and stochastic atmospheric component.  Values are presented in units of \si{\cubic\meter\per\second\per\kilo\meter}.   These tables show the mean, 2.5th, and 97.5th percentile of the deterministic model and the same statistics applied to the three 1-D stochastic models.  This additional set of statistics from the 1-D stochastic 2.5th and 95.5th percentile models provide a better understanding of the extremes of the calculated values.  As before, comparison between the deterministic and stochastic models should be limited to the comparing the deterministic model to the 1-D stochastic mean model.  Also included in these two tables is the percent difference between the deterministic model and the 1-D stochastic mean model.  The mean, 2.5th, and 97.5th percentile values were calculated from the percent difference of the 1-D stochastic mean model from the deterministic model at each time step.  These values show the range of the variance from the deterministic model.
%
%\begin{table}[htbp]
%\centering
%\caption[USR river section selenium surface mass transport.]{USR river section selenium surface mass transport.  Stochastic mean values are calculated as the mean of the realizations for each time step. All values are in \si{\kilo\gram\per\day\per\kilo\meter}.}
%\label{tab:USRSeFlow}
%\begin{tabular}{c|ccc}
%	\toprule
%	Model& 2.5\% & Mean & 97.5\% \\
%	\midrule
%	\midrule
%	Deterministic    &	-0.1236&	-0.04412&	0.04339\\
%	\midrule                                           
%	Stochastic 2.5\% &	-0.1844&	-0.07846&	0.002247\\
%	Stochastic Mean  &	-0.1185&	-0.04296&	0.03445\\ 
%	Stochastic 97.5\%&	-0.07292&	-0.008939&	0.08867\\ 
%	\midrule                                           
%	\% Diff. Means&		-4.723&	4.348&	28.64\\
%	\bottomrule
%\end{tabular}
%\end{table}
%
%\begin{table}[htbp]
%\centering
%\caption[DSR river section selenium surface mass transport.]{DSR river section selenium surface mass transport.  Stochastic mean values are calculated as the mean of the realizations for each time step. All values are in \si{\kilo\gram\per\day\per\kilo\meter}.}
%\label{tab:DSRSeFlow}
%\begin{tabular}{c|ccc}
%	\toprule
%	Model& 2.5\% & Mean & 97.5\% \\
%	\midrule
%	\midrule
%	Deterministic    &	-0.1307	&-0.04625	&-0.01973  \\
%	\midrule                                           
%	Stochastic 2.5\% &	-0.2376&	-0.07634&	-0.03676\\
%	Stochastic Mean  &	-0.1158&	-0.0449&	-0.01902\\
%	Stochastic 97.5\%&	-0.03495&	-0.01452&	0.02444\\ 
%	\midrule                                           
%	\% Diff. Means&		-1.691&	0.6182&	16.99\\
%	\bottomrule
%\end{tabular}
%\end{table}
%
%The mean of the percent difference between the deterministic and 1-D stochastic mean models is very low.  This indicates that the deterministic model is representative of the stochastic model expected value.  The fairly low percent differences at the 2.5th and 97.5th percentile indicate that there is still a small but significan range of uncertainty contained within the stochastic model that the deterministic model cannot replicate.  The deterministic model can be used to determine how changes can affect a reach over a span of time.  Using it to estimate values for specific time steps is acceptable as long as the tollerance for uncertainty is acceptable.
%
%\emph{VI - analysis and comments on river segment and reach results}\\
%
%\clearpage{}
%\section{Results of Calculated Unaccounted for Return Loading}
%\label{sec:MassModelResults}
%
%\emph{I - present river segment results}\\
%All intermediate and final results are calculated in \si{\kilo\gram\per\day}.  Intermediate and final results are presented as the flow rate divided by the study region length, giving units of \si{\kilo\gram\per\day\per\kilo\meter}.  This allows the two study regions to be compared on equal footing since their lengths are significantly different.  All input variables were converted to S.I. units before starting calculations.  Each of the variables in the preceding equations include at least one error term which shall be discussed in the following chapters.
%
%\emph{II - present river reach results}\\
%
%Figures \ref{fig:USRMass} and \ref{fig:DSRMass} depict the final results for the DSR mass balance model.  They show the calculated average daily selenium transport rate between the aquifer and the river channel for the deterministic and stochastic models, respectively.  The blue band in figure \ref{fig:} depicts the 95\% CIR for the calculated time steps.  Positive values indicate that water is moving into the river channel from the aquifer.
%
%\begin{figure}[htbp]
%\centering
%	\begin{subfigure}{0.5\textwidth}
%		\centering
%		\includegraphics[width=0.9\linewidth]{"Figures/Results_DUSR/Balance Mass"}
%		\caption{Deterministic Model.}
%		\label{sub:USRMassD}
%	\end{subfigure}%
%	\begin{subfigure}{0.5\textwidth}
%		\centering
%		\includegraphics[width=0.9\linewidth]{"Figures/Results_USR/Balance Mass"}
%		\caption{Stochastic Model.}
%		\label{sub:USRMassS}
%	\end{subfigure}
%	\caption[Time series of the USR Arkansas R. unaccounted for selenium mass transport.]{Time series of the USR Arkansas R. unaccounted for selenium mass transport.  Positive values indicate mass is moving into the river reach.}
%	\label{fig:USRMass}
%\end{figure}
%
%\begin{figure}[htbp]
%\centering
%	\begin{subfigure}{0.5\textwidth}
%		\centering
%		\includegraphics[width=0.9\linewidth]{"Figures/Results_DDSR/Balance Mass"}
%		\caption{Deterministic Model.}
%		\label{sub:DSRMassD}
%	\end{subfigure}%
%	\begin{subfigure}{0.5\textwidth}
%		\centering
%		\includegraphics[width=0.9\linewidth]{"Figures/Results_DSR/Balance Mass"}
%		\caption{Stochastic Model.}
%		\label{sub:DSRMassS}
%	\end{subfigure}
%	\caption[Time series of the DSR Arkansas R. unaccounted for selenium mass transport.]{Time series of the DSR Arkansas R. unaccounted for selenium mass transport.  Positive values indicate mass is moving into the river reach.}
%	\label{fig:DSRMass}
%\end{figure}
%
%As anticipated with the study reach intermediate results, the model results indicate that there is seasonable variability with the transport of selenium.  The figures show that the USR and DSR receive a significant quantity of selenium from unaccounted for sources.  There is a very short period during midyear where the unaccounted for selenium transport flows are a sink for mass being lost from the Arkansas R.
%
%The distribution of all realizations within each time step was analyzed to determine a distribution type.  This analysis was performed to determine if the assumption that the deterministic model results were representative of the stochastic model.   Testing was performed by comparing K-S statistics for the best fit normal, log-normal, logistic, exponential, gamma, and Weibull distributions.  In the USR, 98\% of all storage component time steps best fit a normal distribution and 2\% best fit a gamma distribution.  In the DSR, 99.7\% best fit a normal distribution and 0.3\% best fit a gamma distribution.  This indicates that for both the USR and DSR, the distributions across the realizations are normal.  
%
%Tables \ref{tab:USRSe} and \ref{tab:DSRSe} presents summary statistics of the deterministic and stochastic storage component changes.  Values are presented in units of \si{\kilo\gram\per\day\per\kilo\meter}.   These tables show the mean, 2.5th, and 97.5th percentile of the deterministic model and the same statistics applied to the three 1-D stochastic models.  This additional set of statistics from the 1-D stochastic 2.5th and 95.5th percentile models provide a better understanding of the extremes of the calculated values.  As before, comparison between the deterministic and stochastic models should be limited to the comparing the deterministic model to the 1-D stochastic mean model.  Also included in these two tables is the percent difference between the deterministic model and the 1-D stochastic mean model.  The mean, 2.5th, and 97.5th percentile values were calculated from the percent difference of the 1-D stochastic mean model from the deterministic model at each time step.  These values show the range of the variance from the deterministic model.
%
%\begin{table}[htbp]
%\centering
%\caption[USR river section unaccounted for selenium mass transport.]{USR river section unaccounted for selenium mass transport.  Stochastic mean values are calculated as the mean of the realizations for each time step.  Positive values indicate mass is moving into the river reach.}
%\label{tab:USRSe}
%\begin{tabular}{c|ccc}
%	\toprule
%	Model& 2.5\% & Mean & 97.5\% \\
%	\midrule
%	\midrule
%	Deterministic&		-0.04191&	0.05627&	0.1424\\
%	\midrule			                               
%	Stochastic 2.5\%&	-0.1096&	0.002481&	0.06902\\
%	Stochastic Mean&	-0.03797&	0.0555&	0.1375\\     
%	Stochastic 97.5\%&	0.03027&	0.1107&	0.2174\\     
%	\midrule                                           
%	\% Diff. Means &	-11.14&	9.36&	42.13\\
%	\bottomrule
%\end{tabular}
%\end{table}
%
%\begin{table}[htbp]
%\centering
%\caption[DSR river section unaccounted for selenium mass transport.]{DSR river section unaccounted for selenium mass transport.  Stochastic mean values are calculated as the mean of the realizations for each time step.  Positive values indicate mass is moving into the river reach.}
%\label{tab:DSRSe}
%\begin{tabular}{c|ccc}
%	\toprule
%	Model& 2.5\% & Mean & 97.5\% \\
%	\midrule
%	\midrule
%	Deterministic&		0.02053&	0.05162&	0.1308\\
%	\midrule			                                
%	Stochastic 2.5\%&	-0.03585&	0.008662&	0.0297\\
%	Stochastic Mean&	0.0207&	0.04935&	0.117\\     
%	Stochastic 97.5\%&	0.04185&	0.09204&	0.2407\\
%	\midrule                                            
%	\% Diff. Means &	-5.229&	1.999&	20.44\\
%	\bottomrule
%\end{tabular}
%\end{table}
%
%The mean of the percent difference between the deterministic and 1-D stochastic mean models is very low.  This indicates that the deterministic model is representative of the stochastic model expected value.  The moderate percent differences at the 2.5th and 97.5th percentile indicate that there is still a significant range of uncertainty contained within the stochastic model that the deterministic model cannot replicate.  The deterministic model can be used to determine how changes can affect a reach over a span of time, but using it to estimate values for specific time steps is only acceptable if the tolerance for this uncertainty is acceptable.
%
%\emph{III - analysis and comments on river segment and reach results}\\
%
%When the mass equation (\ref{eq:mxport}) is transformed to calculate the concentration from the known mass transport and flow rate, as shown in equation \ref{eq:calcC}, the average concentration discharged by the unaccounted for flows into the Arkansas R. can be calculated.
%
%\begin{equation}
%	C=\frac{\dot{M}}{Q} \cdot K_{units}
%	\label{eq:calcC}
%\end{equation}
%\begin{tabular}{rl}
%Where&\\
%	$\dot{M}$ =&Mass transport $(mass \cdot time^{-1})$\\
%	$Q$=&Water flow rate $(volume \cdot time^{-1})$\\
%	$C$=&Constituent concentration $(mass \cdot volume^{-1})$\\
%	$K_{units}$=&Unit conversion factor (\si{\kilo\gram\second\per\cubic\meter\per\day} to \si{\micro\gram\per\liter}) = 11.574
%\end{tabular}\\
%
%Figures \ref{fig:USRC} and \ref{fig:DSRC} and tables \ref{tab:USRUnknownC} and \ref{tab:DSRUnknownC} present the time series results of equation \ref{eq:calcC} when applied to the USR and DSR.  The left sub-figure is the results of the deterministic model and the right sub-figure is the result of the stochastic model.  The black line in the stochastic model is the mean of all realizations for each time step.  The blue band is the 2.5th and 97.5th percentile for each time step.  Tables present the mean and the 95th CIR for the deterministic model and the three calculated 1-D stochastic models.  The percent difference values presented in the tables are the mean and 95th CIR of the daily percent differences between the deterministic and 1-D stochastic mean models.
%
%\begin{figure}[htbp]
%\centering
%	\begin{subfigure}{0.5\textwidth}
%		\centering
%		\includegraphics[width=0.9\linewidth]{"Figures/Results_DUSR/Balance C"}
%		\caption{Deterministic Model.}
%	\end{subfigure}%
%	\begin{subfigure}{0.5\textwidth}
%		\centering
%		\includegraphics[width=0.9\linewidth]{"Figures/Results_USR/Balance C"}
%		\caption{Stochastic Model.}
%	\end{subfigure}
%	\caption[Time series of the concentration of USR unaccounted for mass transport.]{Time series of the concentration of USR unaccounted for mass transport.}
%	\label{fig:USRC}
%\end{figure}
%
%\begin{figure}[htbp]
%\centering
%	\begin{subfigure}{0.5\textwidth}
%		\centering
%		\includegraphics[width=0.9\linewidth]{"Figures/Results_DDSR/Balance C"}
%		\caption{Deterministic Model.}
%	\end{subfigure}%
%	\begin{subfigure}{0.5\textwidth}
%		\centering
%		\includegraphics[width=0.9\linewidth]{"Figures/Results_DSR/Balance C"}
%		\caption{Stochastic Model.}
%	\end{subfigure}
%	\caption[Time series of the concentration of DSR unaccounted for mass transport.]{Time series of the concentration of DSR unaccounted for mass transport.}
%	\label{fig:DSRC}
%\end{figure}
%
%\begin{table}[htbp]
%\centering
%\caption[Concentration of USR unaccounted for mass transport.]{Concentration of USR unaccounted for mass transport.  Model values are in \si{\micro\gram\per\liter}.  \% Diff. Means values are the mean and 95th CIR of the daily percent differences between the deterministic and 1-D stochastic mean models}
%\label{tab:USRUnknownC}
%\begin{tabular}{c|ccc}
%	\toprule
%	Model& 2.5\% & Mean & 97.5\% \\
%	\midrule
%	\midrule
%	Deterministic&		-51.3&	11.69&	95.34\\
%	\midrule			                               
%	Stochastic 2.5\%&	-368.4&	-72.45&	9.747\\
%	Stochastic Mean&	-66.54&	22.73&	76.7\\ 
%	Stochastic 97.5\%&	6.728&	97.44&	380.8\\
%	\midrule                                           
%	\% Diff. Means &	-559.5&	-17.71&	437.4\\
%	\bottomrule
%\end{tabular}
%\end{table}
%
%\begin{table}[htbp]
%\centering
%\caption[Concentration of DSR unaccounted for mass transport.]{Concentration of DSR unaccounted for mass transport.  Model values are in \si{\micro\gram\per\liter}.  \% Diff. Means values are the mean and 95th CIR of the daily percent differences between the deterministic and 1-D stochastic mean models}
%\label{tab:DSRUnknownC}
%\begin{tabular}{c|ccc}
%	\toprule
%	Model& 2.5\% & Mean & 97.5\% \\
%	\midrule
%	\midrule
%	Deterministic&		-167&	-3.707&	98.78\\
%	\midrule			                               
%	Stochastic 2.5\%&	-781.5&	-74.51&	8.479\\
%	Stochastic Mean&	-45.31&	7.782&	100.1\\
%	Stochastic 97.5\%&	15.8&	101.8&	817.2\\
%	\midrule                                           
%	\% Diff. Means &	-36.22&	-4.362&	118.9\\
%	\bottomrule
%\end{tabular}
%\end{table}
%
%Both figures \ref{fig:USRC} and \ref{fig:DSRC} show negative concentrations, which are impossible.  The negative concentrations are due to the flow moving out of the river channel into unaccounted for sinks.  The absolute value of the negative values indicates the dissolved selenium concentration leaving the river channel.
%\clearpage
%
%Figures \ref{fig:USRCin} through \ref{fig:DSRCout} and tables \ref{tab:USRUnknownCin} through \ref{tab:DSRUnknownCout} present the unaccounted for concentration results into two sub-groups.  The distribution of concentration values within any given time step tends to span across both positive and negative values.  It was assumed that the mean value for each time step would indicate the flow direction.  Positive values indicate that flow is moving into the river channel and negative values indicate the contrary.  Statistics were taken from the inflow and outflow subsets without altering data within the time steps.  Figures showing in-flow results present values greater than zero.  Figures showing out-flow results present the absolute value of the values less than zero.
%
%\begin{figure}[htbp]
%\centering
%	\begin{subfigure}{0.5\textwidth}
%		\centering
%		\includegraphics[width=0.9\linewidth]{"Figures/Results_DUSR/Balance Cin"}
%		\caption{Deterministic Model.}
%	\end{subfigure}%
%	\begin{subfigure}{0.5\textwidth}
%		\centering
%		\includegraphics[width=0.9\linewidth]{"Figures/Results_USR/Balance Cin"}
%		\caption{Stochastic Model.}
%	\end{subfigure}
%	\caption[Time series of the concentration of USR unaccounted for river reach inflow dissolved selenium concentration.]{Time series of the concentration of USR unaccounted for river reach inflow dissolved selenium concentration.}
%	\label{fig:USRCin}
%\end{figure}
%
%\begin{figure}[htbp]
%\centering
%	\begin{subfigure}{0.5\textwidth}
%		\centering
%		\includegraphics[width=0.9\linewidth]{"Figures/Results_DDSR/Balance Cin"}
%		\caption{Deterministic Model.}
%	\end{subfigure}%
%	\begin{subfigure}{0.5\textwidth}
%		\centering
%		\includegraphics[width=0.9\linewidth]{"Figures/Results_DSR/Balance Cin"}
%		\caption{Stochastic Model.}
%	\end{subfigure}
%	\caption[Time series of the concentration of DSR unaccounted for river reach inflow dissolved selenium concentration.]{Time series of the concentration of DSR unaccounted for river reach inflow dissolved selenium concentration.}
%	\label{fig:DSRCin}
%\end{figure}
%
%\begin{table}[htbp]
%\centering
%\caption[Concentration of USR unaccounted for river reach inflow dissolved selenium concentration..]{Concentration of USR unaccounted for river reach inflow dissolved selenium concentration.  Model values are in \si{\micro\gram\per\liter}.  \% Diff. Means values are the mean and 95th CIR of the daily percent differences between the deterministic and 1-D stochastic mean models}
%\label{tab:USRUnknownCin}
%\begin{tabular}{c|ccc}
%	\toprule
%	Model& 2.5\% & Mean & 97.5\% \\
%	\midrule
%	\midrule
%	Deterministic&		2.385&	24.82&	99.24\\
%	\midrule			                               
%	Stochastic 2.5\%&	-348.3&	-58.98&	10.05\\
%	Stochastic Mean&	1.306&	37.05&	83.74\\
%	Stochastic 97.5\%&	9.989&	86.71&	362.7\\
%	\midrule                                           
%	\% Diff. Means &	-689.5&	-91.49&	91.11\\
%	\bottomrule
%\end{tabular}
%\end{table}
%
%\begin{table}[htbp]
%\centering
%\caption[Concentration of DSR unaccounted for river reach inflow dissolved selenium concentration.]{Concentration of DSR unaccounted for river reach inflow dissolved selenium concentration.  Model values are in \si{\micro\gram\per\liter}.  \% Diff. Means values are the mean and 95th CIR of the daily percent differences between the deterministic and 1-D stochastic mean models}
%\label{tab:DSRUnknownCin}
%\begin{tabular}{c|ccc}
%	\toprule
%	Model& 2.5\% & Mean & 97.5\% \\
%	\midrule
%	\midrule
%	Deterministic&		8.006&	23.44&	99.77\\
%	\midrule			                               
%	Stochastic 2.5\%&	-702.3&	-48.91&	8.514\\
%	Stochastic Mean&	7.928&	25.7&	102  \\
%	Stochastic 97.5\%&	17&	79.79&	735.2    \\
%	\midrule                                           
%	\% Diff. Means &	-31.25&	-3.703&	48.03\\
%	\bottomrule
%\end{tabular}
%\end{table}
%
%The USR and DSR deterministic model inflow dissolved concentration values are nearly equal.  This leads us to conclude that the groundwater chemistry in the two regions are very similar.  The stochastic model values are quite similar considering the range of uncertainty associated with the calculations.  The USR shows a tendency toward higher average dissolved selenium concentrations than the DSR and the DSR shows a tendency toward higher single day concentrations than in the USR.  The negative values in the figures and tables show some of the combined effects of uncertainty on the water and mass balance models.
%\clearpage
%
%\begin{figure}[htbp]
%\centering
%	\begin{subfigure}{0.5\textwidth}
%		\centering
%		\includegraphics[width=0.9\linewidth]{"Figures/Results_DUSR/Balance Cout"}
%		\caption{Deterministic Model.}
%	\end{subfigure}%
%	\begin{subfigure}{0.5\textwidth}
%		\centering
%		\includegraphics[width=0.9\linewidth]{"Figures/Results_USR/Balance Cout"}
%		\caption{Stochastic Model.}
%	\end{subfigure}
%	\caption[Time series of the concentration of USR unaccounted for river reach outflow dissolved selenium concentration.]{Time series of the concentration of USR unaccounted for river reach outflow dissolved selenium concentration.}
%	\label{fig:USRCout}
%\end{figure}
%
%\begin{figure}[htbp]
%\centering
%	\begin{subfigure}{0.5\textwidth}
%		\centering
%		\includegraphics[width=0.9\linewidth]{"Figures/Results_DDSR/Balance Cout"}
%		\caption{Deterministic Model.}
%	\end{subfigure}%
%	\begin{subfigure}{0.5\textwidth}
%		\centering
%		\includegraphics[width=0.9\linewidth]{"Figures/Results_DSR/Balance Cout"}
%		\caption{Stochastic Model.}
%	\end{subfigure}
%	\caption[Time series of the concentration of DSR unaccounted for river reach outflow dissolved selenium concentration.]{Time series of the concentration of DSR unaccounted for river reach outflow dissolved selenium concentration.}
%	\label{fig:DSRCout}
%\end{figure}
%
%\begin{table}[htbp]
%\centering
%\caption[Concentration of USR unaccounted for river reach outflow dissolved selenium concentration.]{Concentration of USR unaccounted for river reach outflow dissolved selenium concentration.  Model values are in \si{\micro\gram\per\liter}.  \% Diff. Means values are the mean and 95th CIR of the daily percent differences between the deterministic and 1-D stochastic mean models}
%\label{tab:USRUnknownCout}
%\begin{tabular}{c|ccc}
%	\toprule
%	Model& 2.5\% & Mean & 97.5\% \\
%	\midrule
%	\midrule
%	Deterministic&		849.5&	77.25&	0.657\\
%	\midrule			                               
%	Stochastic 2.5\%&	402.6&	149.2&	5.956\\ 
%	Stochastic Mean&	280&	58.8&	0.2844\\
%	Stochastic 97.5\%&	-0.9703&-158.6&	-464.8\\
%	\midrule                                           
%	\% Diff. Means &	102.7&	271.9&	913.4\\
%	\bottomrule
%\end{tabular}
%\end{table}
%
%\begin{table}[htbp]
%\centering
%\caption[Concentration of DSR unaccounted for river reach outflow dissolved selenium concentration.]{Concentration of DSR unaccounted for river reach outflow dissolved selenium concentration. Model values are in \si{\micro\gram\per\liter}.  \% Diff. Means values are the mean and 95th CIR of the daily percent differences between the deterministic and 1-D stochastic mean models}
%\label{tab:DSRUnknownCout}
%\begin{tabular}{c|ccc}
%	\toprule
%	Model& 2.5\% & Mean & 97.5\% \\
%	\midrule
%	\midrule
%	Deterministic&		1429&	382.5&	6.868\\
%	\midrule			                               
%	Stochastic 2.5\%&	1006&	441.2&	17.63 \\
%	Stochastic Mean&	1140&	248.9&	3.027 \\
%	Stochastic 97.5\%&	-6.002&	-417.4&	-997.2\\
%	\midrule               
%	\% Diff. Means &	109.4&	370.6&	1997\\
%	\bottomrule
%\end{tabular}
%\end{table}
%
%The USR and DSR river reach out flow concentrations are difficult to interpret.  There are very few values in these data sub-sets.  These values could be indicative of either the low volume of data or could be and indicator of the effects of uncertainty on the water and mass balance models.  Theoretically, these values should be near the dissolved selenium values collected in the field and calculated in the concentration estimation models.  
%
%These values do not indicate that there are significant flaws in the water and mass balance models.  There are too few values included in this data sub-set to come to this conclusion.  The figures agree with convention where there are more unaccounted for outflows during the hottest portions of the year.
%
%We can only speculate on the source of the higher dissolved selenium concentration.  Discussions others familiar with the LARB selenium issue have added to the pool of possible sources.  There is the possibility that John Martin Reservoir may be a combined source and sink.  Selenium may be dissolved from the USR, concentrated within the reservoir through evaporation of water, and discharged at a higher concentration to the DSR.  This has a couple problems as the reservoir is normally not discharging water to the DSR.  There is the possibility that the concentrated solution may be seeping through the ground under the dam and into the DSR riparian aquifer.
%
%Another possibility is that the higher concentration may be a cumulative effect of evaporation as water moves down the LARB from the USR to the DSR.  Some have even speculated that sediment transport may play a significant role.  Others have suggested that the bedrock beneath the riparian aquifer, which is the ultimate source of selenium in the LARB, may be more rich in selenium in the DSR than the USR.
%
\clearpage{}
\end{linenumbers}