% !TeX root = MorseThesis.tex
% !TeX encoding = UTF-8
% !TeX spellcheck = en_US
\renewcommand{\thechapter}{6}
\chapter{Sensitivity Analysis}
\label{chap:SensitivityAnalysis}

\begin{linenumbers}
\section{Sensitivity Analysis Purpose, Scope, and Methodology}
\label{sec:PurposeAndScope}

Sensitivity analyses serves as a means to determine the degree to which each input variable affects the model.  The results of this analyses serves to assist with future model development and refinement.  Input variables which are not found to significantly affect the complete model can be considered for removal from the model after determining if the variable is significant for a portion of the model.  A variable may be insignificant for the whole model, but may be significant for a intermediate value in the model.

For this thesis, we are only analyzing the sensitivity of the water and mass loading models to the input variables.  Many of the variables are used in multiple intermediate calculations and cannot reasonably be removed from the models.  The uncertainty range was of primary concern.  Each variable was perturbed such that the new value was at the ends of the estimated uncertainty range.  Additionally, the average daily reported flow depth values were perturbed four times, once each for uncertainty ranges of \SIlist{0.0030;0.0152;0.0304;0.0762}{\meter} (\SIlist{0.01;0.05;0.1;0.25}{\foot}).  The additional analyses were performed on the average daily reported flow depth to determine the extent of the water balance and mass loading models sensitivity to these values.  Reported average daily flow depth values are directly used to calculate water surface area used in the evaporation and precipitation calculations and to calculate river volume change.

Sensitivity analyses were only performed on the deterministic models.   Since the scope of the sensitivity analysis included testing the ends of the uncertainty distributions, two trials were run for each input variable.  One trial tests the upper bound of the uncertainty distribution and the second tests the lower bound.

\section{Sensitivity Analysis Results}
\label{sec:SAResults}

Tables \ref{tab:USRSA} and \ref{tab:DSRSA} list the results from the sensitivity analyses for both the water balance and mass loading models in the USR and DSR, respectively.  The results are given as the mean of all the calculated results in each of the two deterministic time series models and as the percent difference from the base model.  The base model is calculated without perturbing any of the input variables.  Those variables with "0" are either not significant to the model or are not part of the model.
\\
\begin{spacing}{1.0}
	\begin{center}
		\begin{longtable}{ccccc}
			\caption[USR Sensitivity Analysis Results.]{USR Sensitivity Analysis Results.  The results columns are the mean of the average daily unaccounted for water balance or mass loading, as appropriate.  The difference column indicates the percent difference between the trial result and the baseline result.  The Base trial was run without any changes to input variables.} 
			\label{tab:USRSA}
			\\ \toprule
			\multirow{3}[1]{*}{Trial} & \multicolumn{2}{c}{Water Balance} & \multicolumn{2}{c}{Mass Loading} \\
			\cmidrule(r{.5em}l){2-3} \cmidrule(r{.5em}l){4-5} & Result & Difference & Result & Difference \\
			& (\si{\cubic\meter\per\second\per\kilo\meter}) & (\%) & (\si{\kilo\gram\per\day\per\kilo\meter}) & (\%) \\ \toprule
			\endfirsthead
			\caption[]{USR Sensitivity Analysis Results. (continued)} \\ \toprule
			\multirow{3}[1]{*}{Trial} & \multicolumn{2}{c}{Water Balance} & \multicolumn{2}{c}{Mass Loading} \\
			\cmidrule(r{.5em}l){2-3} \cmidrule(r{.5em}l){4-5} & Result & Difference & Result & Difference \\
			&	(\si{\cubic\meter\per\second\per\kilo\meter}) & (\%) & (\si{\kilo\gram\per\day\per\kilo\meter}) & (\%) \\ \toprule
			\endhead
			Base (no change) & 0.0532 & - -  & 0.0556 & - - \\
			$Q_{ARKCATCO} $ +10\% & 0.0327 & -38.6 & 0.0452 & -18.6 \\
			$Q_{ARKCATCO} $ -10\% & 0.0738 & 38.6 & 0.0659 & 18.6 \\
			$EC_{ARKCATCO} $ +10\% & 0.0532 & 0 & 0.0556 & 0 \\
			$EC_{ARKCATCO} $ -10\% & 0.0532 & 0 & 0.0556 & 0 \\
			$T_{ARKCATCO} $ +5\si{\degreeCelsius} & 0.0532 & 0 & 0.0556 & 0 \\
			$T_{ARKCATCO} $ -5\si{\degreeCelsius} & 0.0532 & 0 & 0.0556 & 0 \\
			$Q_{ARKLASCO} $ +10\% & 0.0675 & 26.9 & 0.0648 & 16.7 \\
			$Q_{ARKLASCO} $ -10\% & 0.0389 & -26.9 & 0.0463 & -16.7 \\
			$EC_{ARKLASCO} $ +10\% & 0.0532 & 0 & 0.0556 & 0 \\
			$EC_{ARKLASCO} $ -10\% & 0.0532 & 0 & 0.0556 & 0 \\
			$T_{ARKLASCO} $ +5\si{\degreeCelsius} & 0.0532 & 0 & 0.0556 & 0 \\
			$T_{ARKLASCO} $ -5\si{\degreeCelsius} & 0.0532 & 0 & 0.0556 & 0 \\
			$Q_{HOLCANCO} $ +20\% & 0.0594 & 11.6 & 0.0587 & 5.63 \\
			$Q_{HOLCANCO} $ -20\% & 0.047 & -11.6 & 0.0524 & -5.63 \\
			$Q_{RFDMANCO} $ +20\% & 0.0558 & 4.85 & 0.0573 & 3.13 \\
			$Q_{RFDMANCO} $ -20\% & 0.0507 & -4.85 & 0.0538 & -3.13 \\
			$Q_{FLSCANCO} $ +20\% & 0.0551 & 3.44 & 0.0568 & 2.14 \\
			$Q_{FLSCANCO} $ -20\% & 0.0514 & -3.44 & 0.0544 & -2.14 \\
			$Q_{RFDRETCO} $ +20\% & 0.052 & -2.29 & 0.0548 & -1.46 \\
			$Q_{RFDRETCO} $ -20\% & 0.0545 & 2.29 & 0.0564 & 1.48 \\
			$Q_{TIMSWICO} $ +15\% & 0.0496 & -6.91 & 0.0537 & -3.44 \\
			$Q_{TIMSWICO} $ -15\% & 0.0569 & 6.89 & 0.0575 & 3.44 \\
			$Q_{FLYCANCO} $ +20\% & 0.0785 & 47.4 & 0.072 & 29.6 \\
			$Q_{FLYCANCO} $ -20\% & 0.028 & -47.4 & 0.0391 & -29.6 \\
			$Q_{CANSWKCO} $ +15\% & 0.0523 & -1.69 & 0.0548 & -1.4 \\
			$Q_{CANSWKCO} $ -15\% & 0.0541 & 1.69 & 0.0564 & 1.42 \\
			$Q_{CONDITCO} $ +20\% & 0.0569 & 6.8 & 0.0582 & 4.77 \\
			$Q_{CONDITCO} $ -20\% & 0.0496 & -6.8 & 0.0529 & -4.77 \\
			$Q_{HRC194CO} $ +20\% & 0.0525 & -1.43 & 0.0548 & -1.33 \\
			$Q_{HRC194CO} $ -20\% & 0.054 & 1.43 & 0.0563 & 1.33 \\
			$Q_{LAJWWTP} $ +15\% & 0.0531 & -0.263 & 0.0553 & -0.45 \\
			$Q_{LAJWWTP} $ -15\% & 0.0534 & 0.263 & 0.0558 & 0.45 \\
			$P$ +25\% & 0.0531 & -0.319 & 0.0556 & 0 \\
			$P$ -25\% & 0.0534 & 0.319 & 0.0556 & 0 \\
			$ET_{Ref}$ +0.98\si{\milli\meter} & 0.0544 & 2.24 & 0.0556 & 0 \\
			$ET_{Ref}$ -0.98\si{\milli\meter} & 0.052 & -2.24 & 0.0556 & 0 \\
			$u_{2} $ +0.5\si{\meter\per\second} & 0.0532 & -0.0939 & 0.0556 & 0 \\
			$u_{2} $ -0.5\si{\meter\per\second} & 0.0533 & 0.0939 & 0.0556 & 0 \\
			$RH_{min} $ +2\% & 0.0532 & 0 & 0.0556 & 0 \\
			$RH_{min} $ -2\% & 0.0532 & 0 & 0.0556 & 0 \\
			$\beta_{1} $ +10\% & 0.054 & 1.37 & 0.0554 & -0.234 \\
			$\beta_{1} $ -10\% & 0.0525 & -1.37 & 0.0557 & 0.234 \\
			$\beta_{2} $ +10\% & 0.0531 & -0.244 & 0.0556 & 0.018 \\
			$\beta_{2} $ -10\% & 0.0534 & 0.244 & 0.0556 & -0.018 \\
			$d_{A} $ +0.01 ft & 0.0532 & 0 & 0.0556 & 0 \\
			$d_{A} $ -0.01 ft & 0.0532 & 0 & 0.0556 & 0 \\
			$d_{B} $ +0.01 ft & 0.0532 & 0 & 0.0556 & 0 \\
			$d_{B} $ -0.01 ft & 0.0532 & 0 & 0.0556 & 0 \\
			$d_{C} $ +0.01 ft & 0.0532 & 0.0188 & 0.0556 & 0 \\
			$d_{C} $ -0.01 ft & 0.0532 & -0.0376 & 0.0556 & 0 \\
			$d_{D} $ +0.01 ft & 0.0533 & 0.0376 & 0.0556 & 0 \\
			$d_{D} $ -0.01 ft & 0.0532 & -0.0376 & 0.0556 & 0.018 \\
			$d_{E} $ +0.01 ft & 0.0532 & 0.0188 & 0.0556 & 0 \\
			$d_{E} $ -0.01 ft & 0.0532 & -0.0188 & 0.0556 & 0 \\
			$d_{A} $ +0.05 ft & 0.0533 & 0.0376 & 0.0556 & 0 \\
			$d_{A} $ -0.05 ft & 0.0532 & -0.0376 & 0.0556 & 0 \\
			$d_{B} $ +0.05 ft & 0.0532 & 0.0188 & 0.0556 & 0 \\
			$d_{B} $ -0.05 ft & 0.0532 & -0.0188 & 0.0556 & 0 \\
			$d_{C} $ +0.05 ft & 0.0533 & 0.131 & 0.0556 & 0 \\
			$d_{C} $ -0.05 ft & 0.0532 & -0.15 & 0.0556 & 0 \\
			$d_{D} $ +0.05 ft & 0.0533 & 0.15 & 0.0556 & -0.036 \\
			$d_{D} $ -0.05 ft & 0.0532 & -0.169 & 0.0556 & 0.036 \\
			$d_{E} $ +0.05 ft & 0.0533 & 0.0563 & 0.0556 & 0 \\
			$d_{E} $ -0.05 ft & 0.0532 & -0.0751 & 0.0556 & 0 \\
			$d_{A} $ +0.1 ft & 0.0533 & 0.0563 & 0.0556 & 0 \\
			$d_{A} $ -0.1 ft & 0.0532 & -0.0751 & 0.0556 & 0.018 \\
			$d_{B} $ +0.1 ft & 0.0532 & 0.0188 & 0.0556 & 0 \\
			$d_{B} $ -0.1 ft & 0.0532 & -0.0376 & 0.0556 & 0 \\
			$d_{C} $ +0.1 ft & 0.0534 & 0.244 & 0.0556 & 0 \\
			$d_{C} $ -0.1 ft & 0.0531 & -0.319 & 0.0556 & 0 \\
			$d_{D} $ +0.1 ft & 0.0534 & 0.301 & 0.0556 & -0.054 \\
			$d_{D} $ -0.1 ft & 0.053 & -0.394 & 0.0556 & 0.09 \\
			$d_{E} $ +0.1 ft & 0.0533 & 0.131 & 0.0556 & 0 \\
			$d_{E} $ -0.1 ft & 0.0532 & -0.15 & 0.0556 & 0 \\
			$d_{A} $ +0.25 ft & 0.0533 & 0.15 & 0.0556 & -0.018 \\
			$d_{A} $ -0.25 ft & 0.0538 & 0.977 & 0.0568 & 2.25 \\
			$d_{B} $ +0.25 ft & 0.0533 & 0.0563 & 0.0556 & 0 \\
			$d_{B} $ -0.25 ft & 0.0548 & 2.87 & 0.0578 & 3.98 \\
			$d_{C} $ +0.25 ft & 0.0535 & 0.563 & 0.0556 & -0.036 \\
			$d_{C} $ -0.25 ft & 0.0585 & 9.88 & 0.0623 & 12.1 \\
			$d_{D} $ +0.25 ft & 0.0536 & 0.676 & 0.0555 & -0.144 \\
			$d_{D} $ -0.25 ft & 0.0553 & 3.79 & 0.0554 & -0.378 \\
			$d_{E} $ +0.25 ft & 0.0534 & 0.282 & 0.0556 & 0 \\
			$d_{E} $ -0.25 ft & 0.0528 & -0.751 & 0.0555 & -0.216 \\
			\bottomrule
		\end{longtable}%
	\end{center}
\end{spacing}

Table \ref{tab:USRSA} shows that the USR water balance model is most sensitive to the variables $ Q_{ARKCATCO} $, $ Q_{ARKLASCO} $, $ Q_{HOLCANCO} $, and $ Q_{FLYCANCO} $ with sensitivity to other variables to a lesser extent.  Flow rate measurement uncertainty distributions for $ Q_{ARKCATCO} $ and $ Q_{ARKLASCO} $ are at the smallest possible available in the LARV.  Flow rate measurements represented by $ Q_{HOLCANCO} $ and $ Q_{FLYCANCO} $ should be improved to make the model more accurate.  The input variables $ EC_{ARKCATCO} $, $ T_{ARKCATCO} $, $ EC_{ARKLASCO} $, and $ T_{ARKLASCO} $ are not used in the water balance model and therefore show no change from the base model.  It was expected that the percent difference for the upper bound trial and the lower bound trial would have different signs but the same magnitude.  In almost all cases, this is true.  The reported average daily flow depths that were perturbed by $\pm$\SI{0.0152}{\meter} ($\pm$\SI{0.05}{\foot}) or more were not symmetrical.  This is due to the lower bound of the perturbed flow depths being less than zero, which is the lower bound of acceptable flow depths.  Since the perturbed average daily flow depths were altered such that no values were below zero, we expect that the difference between the particular trial and the base trial would not be symmetrical.

This table also shows that the USR mass loading model is most sensitive to $ Q_{ARKCATCO} $, $ Q_{ARKLASCO} $, $ Q_{HOLCANCO} $, and $ Q_{FLYCANCO} $ with sensitivity to other variables to a lesser extent.  Note that these are the same variables to which the USR water balance model is also most sensitive to.  The USR mass loading model is more sensitive to changes in flow depth than the water balance model which is evident in the difference values.  Input variables $ P $, $ ET_{Ref} $, $ u_2 $, and $ RH_{min} $ are not used in the mass loading model and therefore show no change from the base model.  As with the water balance model, the differences between the upper and lower limits for each variable are symmetrical for all variables except the reported average daily flow depths.  These are not symmetrical for the same reason as described for the water balance model.
\\
\begin{spacing}{1.0}
	\begin{center}
		\begin{longtable}{ccccc}
			\caption[DSR Sensitivity Analysis Results.]{DSR Sensitivity Analysis Results.  The results columns are the mean of the average daily unaccounted for water balance or mass loading, as appropriate.  The difference column indicates the percent difference between the trial result and the baseline result.  The Base trial was run without any changes to input variables.} \label{tab:DSRSA}  \\ \toprule
			\multirow{3}[1]{*}{Trial} & \multicolumn{2}{c}{Water Balance} & \multicolumn{2}{c}{Mass Loading} \\
			\cmidrule(r{.5em}l){2-3} \cmidrule(r{.5em}l){4-5} & Result & Difference & Result & Difference \\
			& (\si{\cubic\meter\per\second\per\kilo\meter}) & (\%) & (\si{\kilo\gram\per\day\per\kilo\meter}) & (\%) \\ \toprule
			\endfirsthead
			\caption[]{DSR Sensitivity Analysis Results. (continued)} \\ \toprule
			\multirow{3}[1]{*}{Trial} & \multicolumn{2}{c}{Water Balance} & \multicolumn{2}{c}{Mass Loading} \\
			\cmidrule(r{.5em}l){2-3} \cmidrule(r{.5em}l){4-5} & Result & Difference & Result & Difference \\
			&	(\si{\cubic\meter\per\second\per\kilo\meter}) & (\%) & (\si{\kilo\gram\per\day\per\kilo\meter}) & (\%) \\ \toprule
			\endhead
			Base (No Change) & 0.0299 & 0 & 0.0516 & 0 \\
			$Q_{ARKLASCO} $ +10\% & 0.0266 & -11.1 & 0.0501 & -3.02 \\
			$Q_{ARKLASCO} $ -10\% & 0.0332 & 11.1 & 0.0532 & 3.04 \\
			$EC_{ARKJMRCO} $ +10\% & 0.0299 & 0 & 0.0516 & 0 \\
			$EC_{ARKJMRCO} $ -10\% & 0.0299 & 0 & 0.0516 & 0 \\
			$T_{ARKJMRCO} $ +5\si{\degreeCelsius}& 0.0299 & 0 & 0.0516 & 0 \\
			$T_{ARKJMRCO} $ -5\si{\degreeCelsius}& 0.0299 & 0 & 0.0516 & 0 \\
			$Q_{ARKCOOKS} $ +10\% & 0.0378 & 26.3 & 0.0608 & 17.8 \\
			$Q_{ARKCOOKS} $ -10\% & 0.0221 & -26.3 & 0.0424 & -17.8 \\
			$EC_{ARKCOOKS} $ +10\% & 0.0299 & 0 & 0.0516 & 0 \\
			$EC_{ARKCOOKS} $ -10\% & 0.0299 & 0 & 0.0516 & 0 \\
			$T_{ARKCOOKS} $ +5\si{\degreeCelsius}& 0.0299 & 0 & 0.0516 & 0 \\
			$T_{ARKCOOKS} $ -5\si{\degreeCelsius}& 0.0299 & 0 & 0.0516 & 0 \\
			$Q_{BIGLAMCO} $ +20\% & 0.0292 & -2.41 & 0.0505 & -2.15 \\
			$Q_{BIGLAMCO} $ -20\% & 0.0307 & 2.44 & 0.0527 & 2.15 \\
			$Q_{WILDHOCO} $ +20\% & 0.0295 & -1.3 & 0.051 & -1.24 \\
			$Q_{WILDHOCO} $ -20\% & 0.0303 & 1.34 & 0.0523 & 1.26 \\
			$Q_{BUFDITCO} $ +20\% & 0.0322 & 7.75 & 0.0537 & 4.11 \\
			$Q_{BUFDITCO} $ -20\% & 0.0276 & -7.72 & 0.0495 & -4.11 \\
			$Q_{FRODITKS} $ +20\% & 0.0307 & 2.54 & 0.0525 & 1.72 \\
			$Q_{FRODITKS} $ -20\% & 0.0292 & -2.51 & 0.0507 & -1.72 \\
			$P$ +25\% & 0.0299 & -0.0334 & 0.0516 & 0 \\
			$P$ -25\% & 0.03 & 0.0668 & 0.0516 & 0 \\
			$ET_{Ref} $ +0.98 \si{\milli\meter} & 0.0301 & 0.434 & 0.0516 & 0 \\
			$ET_{Ref} $ -0.98 \si{\milli\meter} & 0.0298 & -0.401 & 0.0516 & 0 \\
			$u_{2} $ +0.5 \si{\meter\per\second} & 0.0299 & 0 & 0.0516 & 0 \\
			$u_{2} $ -0.5 \si{\meter\per\second} & 0.0299 & 0.0334 & 0.0516 & 0 \\
			$RH_{min} $ +2\% & 0.0299 & 0.0334 & 0.0516 & 0 \\
			$RH_{min} $ -2\% & 0.0299 & 0 & 0.0516 & 0 \\
			$\beta_{1} $ +10\% & 0.03 & 0.301 & 0.0516 & -0.0387 \\
			$\beta_{1} $ -10\% & 0.0298 & -0.267 & 0.0516 & 0.0387 \\
			$\beta_{2} $ +10\% & 0.0299 & -0.0334 & 0.0516 & 0.0194 \\
			$\beta_{2} $ -10\% & 0.03 & 0.0668 & 0.0516 & -0.0194 \\
			$d_{F} $ +0.01ft & 0.0299 & 0.0334 & 0.0516 & 0 \\
			$d_{F} $ -0.01ft & 0.0299 & 0 & 0.0516 & 0.0194 \\
			$d_{G} $ +0.01ft & 0.0299 & 0.0334 & 0.0516 & 0 \\
			$d_{G} $ -0.01ft & 0.0299 & 0 & 0.0516 & 0 \\
			$d_{F} $ +0.05ft & 0.03 & 0.1 & 0.0516 & -0.0387 \\
			$d_{F} $ -0.05ft & 0.0299 & -0.1 & 0.0516 & 0.0387 \\
			$d_{G} $ +0.05ft & 0.03 & 0.1 & 0.0516 & 0.0194 \\
			$d_{G} $ -0.05ft & 0.0299 & -0.0668 & 0.0516 & 0 \\
			$d_{F} $ +0.1ft & 0.03 & 0.2 & 0.0516 & -0.0581 \\
			$d_{F} $ -0.1ft & 0.0299 & -0.2 & 0.0517 & 0.0775 \\
			$d_{G} $ +0.1ft & 0.03 & 0.167 & 0.0516 & 0.0194 \\
			$d_{G} $ -0.1ft & 0.0299 & -0.134 & 0.0516 & -0.0194 \\
			$d_{F} $ +0.25ft & 0.0301 & 0.434 & 0.0515 & -0.155 \\
			$d_{F} $ -0.25ft & 0.022 & -26.4 & 0.0793 & 53.6 \\
			$d_{G} $ +0.25ft & 0.03 & 0.368 & 0.0516 & 0.0387 \\
			$d_{G} $ -0.25ft & 0.0298 & -0.368 & 0.0516 & -0.0387 \\
			\bottomrule
		\end{longtable}%
	\end{center}
\end{spacing}

Table \ref{tab:DSRSA} shows that the DSR water balance model is most sensitive to the variables $ Q_{ARKLASCO} $, $ Q_{ARKCOOKS} $, $ Q_{BUFDITCO} $, and $ d_F $ with sensitivity to other variables to a lesser extent.  Sensitivity to the $ d_F $ input variable is only significant when the flow depth is perturbed by \SI{0.0762}[-]{\meter} (\SI{0.25}[-]{\foot}).  This isn't a likely uncertainty range, but it should be noted that this is most likely due to the $ \beta_1 $ and $ \beta_2 $ shape parameters.  Flow rate measurement uncertainty distributions for $ Q_{ARKLASCO} $ and $ Q_{ARKCOOKS} $ are at the smallest possible available in the LARV.  Flow rate measurements represented by $ Q_{BUFDITCO} $ should be improved to make the model more accurate.  The input variables $ EC_{ARKJMRCO} $, $ T_{ARKJMRCO} $, $ EC_{ARKCOOKS} $, and $ T_{ARKCOOKS} $ are not used in the water balance model and therefore show no change from the base model.  It was expected that the percent difference for the upper bound trial and the lower bound trial would have different signs but the same magnitude.  In almost all cases, this is true.  The reported average daily flow depths were not symmetrical.  This is due to the same effect noted in the USR.

This table also shows that the DSR mass loading model is most sensitive to $ Q_{ARKLASCO} $, $ Q_{ARKCOOKS} $, $ Q_{BUFDITCO} $, and $ d_F $ with sensitivity to other variables to a lesser extent.  Note that these are the same variables to which the USR water balance model is also most sensitive to.  The USR mass loading model is more sensitive to changes in flow depth than the water balance model which is evident in the difference values.  Input variables $ P $, $ ET_{Ref} $, $ u_2 $, and $ RH_{min} $ are not used in the mass loading model and therefore show no change from the base model.  As with the water balance model, the differences between the upper and lower limits for each variable are symmetrical for all variables except the reported average daily flow depths.  These are not symmetrical for the same reason as described for the water balance model.

Refining reported flow rate uncertainty may improve the models but is not an acceptable solution.  Flow rate uncertainty is not solely based on measuring equipment or methodology, but also on the channel being measured.  The Arkansas River in the LARV is a sandy bed variable channel where the only method to improve measurements is to perform more frequent gauge calibration.  This is a cost that is borne by the CDWR and USGS, both of which are perpetually struggling to justify the existing stream gauge system.

Model improvement can be realistically performed by improving the characterization of the river channel.  River cross section surveys should be performed at a more frequent spatial interval along the river with each cross section tied to a benchmark.  Surveys should also be performed at the same location to determine a temporal relationship.  While more complicated to set up, this effort should be easy to routinely re-accomplish with high accuracy GPS survey equipment.

\clearpage{}
\end{linenumbers}