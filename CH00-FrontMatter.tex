\frontmatter
\begin{abstract}
The Lower Arkansas River in Colorado is a selenium impacted river.  Health effects of selinosis in livestock and aquatic animals has become a significant concern since the area was first developed for agriculture.

Two representative reaches of the Lower Arkansas River Basin in Colorado were studied to determine the contribution of return flows, including groundwater, unaccounted for surface water, and selenium(Se) loading to the Arkansas River using water and solute mass balance calculation methods.  Available stream flow and water quality data from state and federal sources and dissolved Se concentrations and in-situ properites from field samples were used to estimate Se concentrations in the main stem of the river, its tributaries, and diversions.  Relationships of Se concentrations to flow and water quality proerties were estimated using multi-variate linear regression.  Daily river volume change was calculated using surveyed river cross section geometry and flow depth data.  Daily average return flows and Se loads from groundwater and unaccounted-for surface sources were estimated as residuals in the water and solute mass balance models.  A simple deterministic sensitivity analysis was used to examine the relative effects of the individual input variables on computed return flows and loads.  Stochastic water and solute mass balance models were developed to describe the effects of uncertainty in estimated flows and loads associated with spatiotemporal variability and measurement error.  Study region water balance models resulted in unaccounted-for flow rates over the study periods in the USR and DSR of of 0.05 and 0.03 \si{\cubic\meter\per\second\per\kilo\meter}, respectively.  The mass balance models resulted in unaccounted-for mass transport rates over the study period in the USR and DSR of 0.056 and 0.049 \si{\kilo\gram\per\day\per\kilo\meter}, respectively.  Evaporation, precipitation, and channel storage changes were considered in the water balance model and found to be significant factors in the calculations of both study regions.  Channel Se mass storage changes also were found to be significant factors in the analysis for return Se loading in both study regions.  Estimated mean Se concentration of unaccounted-for return flows were found to be 37.0 and 25.7 \si{\micro\gram\per\liter} in the USR and DSR, respectively.  Uncertainty in calculated return flow and Se mass loading was found to be significant.  (Summary of major statistics --95\%CIR-- of estimated return flows and Se loading).  Reach models were found to be especially sensitive to changes in upstream and downstream boundary flow rates and to changes in river segment flow depths.

Selenium was found to be entering the main stem of the Arkansas River from unaccounted for non-point sources at a rate of \SI{0.0556}{\kilo\g\per\day\per\kilo\meter} (\SI{0.123}{\pound\per\day\per\mile}) and \SI{0.0819}{\kilo\g\per\day\per\kilo\meter} (\SI{0.181}{\pound\per\day\per\mile}) for the USR and DSR, respectively.  The water balance model was found to be sensitive to daily changes in storage volume, evaporation, and precipitation with flows entering the main stem of the river from unaccounted for non-point sources at a rate of \SI{0.0532}{\cubic\meter\per\second\per\kilo\meter} (\SI{3.73}{\cfs\per\mile}) and \SI{0.0475}{\cubic\meter\per\second\per\kilo\meter} (\SI{3.33}{\cfs\per\mile}) for the USR and DSR, respectively.  The calculated dissolved selenium concentration of non-point source return flow was found to be significantly lower than concentrations found in nearby test wells.  This leads us to conclude that there are unaccounted for activities within the riparian zone that affect dissolved selenium concentrations.
\end{abstract}

\begin{acknowledgements}
I thank God for His infinite wisdom and creativity, without which we would not be held in wonder at the world before us and my wife and daughters for supporting and motivating me throughout this research process.  Many thanks to my advisor, Dr. Timothy Gates, who has provided cruicial guidance throughout the research and writing of this work.  I also thank my committee members, Dr. Kenneth Carlson and Dr. Daniel Cooley, for taking time to review and edit this thesis.

I thank the Water Quality Control Division of the Colorado Department of Public Health and Environment, the Colorado Agricultural Experiment Station, the Southeastern Colorado Water Conservancy District, and the Lower Arkansas Valley Water Conservancy District for their financial and cooperative support.  The staffs of the Colorado Division of Water Resources, Division 2, and the U.S. Geological Survey Pueblo Southeast Colorado Office, the leadership and staff at Ward Laboratories, TestAmerica Laboratories, Olson Biochemical Laboratories at South Dakota State University, and South Dakota Agricultural Laboratories also provided crucial and appreciated insight into many of the processes addressed in this study.

Thanks also to Joe Wilmetti and Greg Steed for equipment assistance and to Dr. Eric Morway and Dr. Ryan Bailey for sharing their in-depth knowledge of the study regions.  Thanks to Leif Anderson for building this \LaTeX\ document class.  The many undergraduate students who assisted on this project also are appreciated.  Without them, I would still be collecting data.  A final thanks to all of the researchers who have come before me.

It is impossible to remember all who have assisted, and I apologize to those I have inadvertently left out.

\end{acknowledgements}

\maketitle
\clearpage
\newpage
\tableofcontents
\clearpage
\newpage
\listoftables 
\clearpage
\newpage
\listoffigures