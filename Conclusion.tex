\documentclass[10pt]{article}
\usepackage[usenames]{color} %used for font color
\usepackage{amssymb} %maths
\usepackage{amsmath} %maths
\usepackage[utf8]{inputenc} %useful to type directly diacritic characters
\begin{document}
\[\renewcommand{\thechapter}{11}
\chapter{Conclusion and Recommendations}
\label{chap:conclusion}

%The Arkansas R. is gaining water and selenium from unknown sources.  Groundwater is most likely the largest component of these sources.  The models presented in this paper were formulated from the best available data and the most reasonable assumptions, yet there are essential pieces of information missing that can allow us to have a clearer picture of selenium transport and fate in the LARB.  Selenium volatiliztion and other transport pathways are not completely understood.  We do not know if there are spatial, temporal, or physical relationships with volatiliation.

%Many assumptions were made in the analyses presented in this thesis.  Most of which are accounted for.  There are many ways to improve the study contained in these pages.  Of the concepts discussed, there are a few that deserve further study.  These include, but are not limited to, the following:
%\begin{enumerate}
%	\item Improve the methodology for measuring and estimating river geometry values.
%	\item Improve the concentration estimating linear models for the tributaries.  These linear models showed the largest uncertainty.
%	\item Improve the estimation of evaporation.  Determine a method to calculate evaporation from a river by using reference ET values calculated for locations distant from the river channel.
%	\item Perform studies on selenium volitalization and uptake by riparian vegitation.  These contributors to selenium loss were not included in this thesis due to the lack of sufficient information to make even the most elementary of calculations.
%	\item Determine a method by which the groundwater flow rate into and out of the river channel can be measured.  This will provide a measureable check to the values calculated in this thesis.
%\end{enumerate}

%Future studies in the LARB should include further surface water sampling as described in this thesis.  Additional data will only improve the selenium concentration estimation models.  This data may also shed light on temporal patterns not recognized at this time.

%The Lower Arkansas River Basin is valuable to the State of Colorado as a source of agriculture and history.  Life and progress may appear to move slowly to those who pass through the region, but change does happen.  Changes in the region have lead to an increase in water availability to residents of the LARB in Colorado and Western Kansas.  This change has caused agriculture to spread throughout the valley.  Increase irrigation has released naturally occurring pollutants into the groundwater which returns to the river.  Understanding the interaction between the aquifer and the river with increased focus on the riparian area should help residents and water managers improve water quality in the Lower Arkansas River Basin.\]
\end{document}